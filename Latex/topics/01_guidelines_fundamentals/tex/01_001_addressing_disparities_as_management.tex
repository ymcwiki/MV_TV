\section{应对主动脉瓣狭窄管理中的健康不平等:我们取得进展了吗?}
\label{sec:01_001_addressing_disparities}

% ============================================
% 文献信息
% ============================================
\subsection{文献信息}

\begin{itemize}
    \item \textbf{标题}: Addressing Disparities In Aortic Stenosis Management: Have We Made Progress?
    \item \textbf{作者}: Wayne Batchelor, MD, MHS, MBA
    \item \textbf{机构}: Inova Health System; Duke University
    \item \textbf{会议}: TCT (Transcatheter Cardiovascular Therapeutics)
    \item \textbf{PDF文件名}: addressing-disparities-in-aortic-stenosis-management-have-we-made-progress.pdf
    \item \textbf{文献类型}: 会议演讲/综述
\end{itemize}

\subsection{研究背景}

\subsubsection{TAVR的快速发展}

自2002年Alain Cribier首次实施TAVR以来,该技术经历了爆炸式增长:

\textbf{TAVR中心数量增长}(来源:STS/ACC TVT Registry Database):
\begin{itemize}
    \item 2013年:252个中心
    \item 2016年:485个中心(中危患者获批)
    \item 2019年:659个中心(低危患者获批)
    \item 2024年:850个中心
    \item \textbf{增长倍数}:3.4倍
\end{itemize}

\textbf{TAVR手术量增长}(来源:STS/ACC TVT Registry Database):
\begin{itemize}
    \item 2015年:24,647例
    \item 2019年:73,396例
    \item 2024年:106,147例
    \item \textbf{增长倍数}:4倍
\end{itemize}

\subsubsection{问题的提出}

尽管TAVR技术取得了巨大进展,但某些患者群体仍然被"落下"(left behind),存在显著的健康不平等现象。

\subsection{主要研究发现}

\subsubsection{1. 种族/族裔差异}

\textbf{TAVR患者种族构成(2015-2024)}:

\begin{table}[h]
\centering
\caption{TAVR患者种族分布变化趋势}
\label{tab:tavr_racial_demographics}
\begin{tabular}{lccc}
\toprule
\textbf{种族} & \textbf{2015} & \textbf{2020} & \textbf{2024} \\
\midrule
白人 & 94\% & 92\% & 90\% \\
黑人 & 4\% & 4\% & 4\% \\
其他 & 2\% & 4\% & 6\% \\
\bottomrule
\end{tabular}
\end{table}

\textbf{关键观察}:
\begin{itemize}
    \item 白人患者比例虽有下降(94\%→90\%),但仍占绝对多数
    \item 黑人患者比例基本未变(维持在4\%)
    \item 10年间,种族差异改善非常有限
\end{itemize}

\textbf{TAVR手术量按种族和性别分布(绝对数)}:

\begin{table}[h]
\centering
\caption{TAVR手术量按种族和性别分布}
\label{tab:tavr_volume_race_gender}
\begin{tabular}{lrrr}
\toprule
\textbf{分组} & \textbf{2015} & \textbf{2020} & \textbf{2024} \\
\midrule
白人 & 23,168 & 70,978 & 95,532 \\
黑人 & 986 & 3,086 & 4,246 \\
男性 & 13,062 & 44,746 & 61,565 \\
女性 & 11,584 & 32,402 & 44,581 \\
\bottomrule
\end{tabular}
\end{table}

\textbf{分析}:
\begin{itemize}
    \item 虽然所有群体的绝对手术量都在增加
    \item 但白人患者增长最快(4.1倍),黑人患者增长较慢(4.3倍)
    \item 性别差距持续存在:男性手术量始终高于女性
\end{itemize}

\subsubsection{差异的多因素机制}

根据Batchelor在JACC Council Perspectives 2019年发表的框架,AS治疗差异是多因素导致的:

\textbf{患者相关因素}:
\begin{itemize}
    \item 种族/族裔背景
    \item AS患病率差异
    \item 医疗可及性
    \item 农村vs城市居住地
    \item 症状认知
    \item 社会经济因素
    \item 文化信念/偏好
    \item 对医疗系统的信任/不信任
    \item 预期寿命认知
\end{itemize}

\textbf{医疗系统因素}:
\begin{itemize}
    \item 诊断转诊偏见
    \item 治疗转诊偏见
    \item 文化/语言障碍
    \item 地区性TAVR中心可及性
\end{itemize}

\textbf{疾病相关因素}:
\begin{itemize}
    \item AS严重程度
    \item 症状状态
    \item 二叶主动脉瓣
    \item 主动脉扩张
    \item 主动脉瓣反流
    \item 合并二尖瓣疾病
    \item 左心室收缩功能
\end{itemize}

\subsubsection{三种关键偏见}

\textbf{1. 监测偏见(Surveillance Bias)}

研究显示(Tanguturi, JACC Imaging 2019),以下人群接受适当超声心动图监测的可能性\textbf{更低}:
\begin{itemize}
    \item 黑人患者
    \item 女性患者
    \item 高龄患者(71-100岁)
    \item Medicaid参保患者
\end{itemize}

在42,289名VHD患者中,黑人种族的适当超声随访比值比(OR)显著降低。

\textbf{2. 治疗偏见(Treatment Bias)}

研究显示(Brennan, JAHA 2020),在控制多种混杂因素后:
\begin{itemize}
    \item \textbf{黑人患者接受TAVR的可能性比非西班牙裔白人低约25\%}
    \item 亚裔危险比(SDHR):0.70 (95\% CI: 0.62, 0.79)
    \item 黑人危险比:
    \begin{itemize}
        \item 未调整模型:0.76 (0.67, 0.85)
        \item 完全调整模型:0.74 (0.66, 0.83)
    \end{itemize}
    \item 研究纳入32,853名患者(2007-2017)
\end{itemize}

\textbf{3. 社会健康决定因素(SDOH)}

健康平等获取的7个关键领域:
\begin{enumerate}
    \item 可负担性(Affordability)
    \item 可接受性(Acceptability)
    \item 可获得性与资源(Availability \& resources)
    \item 物理可及性(Physical accessibility)
    \item 认知与需求(Awareness \& needs)
    \item 决策能力(Capacity to make decisions)
    \item 适当性(Appropriateness)
    \item 个人与文化环境(Personal \& cultural circumstances)
\end{enumerate}

\subsubsection{重要发现:TAVR结果无种族差异}

\textbf{TAVR死亡率按时间点(2015-2024)}:

\begin{table}[h]
\centering
\caption{TAVR死亡率趋势(所有种族合并)}
\label{tab:tavr_mortality}
\begin{tabular}{lccc}
\toprule
\textbf{年份} & \textbf{院内死亡率} & \textbf{30天死亡率} & \textbf{1年死亡率} \\
\midrule
2015 & 3\% & 4\% & 17\% \\
2020 & 1\% & 2\% & 11\% \\
2024 & 1\% & 2\% & 9.5\% \\
\bottomrule
\end{tabular}
\end{table}

\textbf{关键结论}:
\begin{itemize}
    \item \textbf{TAVR术后结果在不同种族/族裔间无显著差异}
    \item 院内、30天和1年死亡率在白人、黑人、亚裔、西班牙裔患者中相似
    \item 这表明\textbf{差异主要在"获得治疗"阶段,而非治疗效果本身}
\end{itemize}

\subsubsection{2. 血流动力学亚型治疗不足}

研究显示(Li SX et al, JACC 2022;79:864-77),严重症状性AS患者(N=10,795)中:

\begin{table}[h]
\centering
\caption{不同血流动力学亚型的AVR治疗率}
\label{tab:hemodynamic_subtypes_treatment}
\begin{tabular}{lcc}
\toprule
\textbf{血流动力学亚型} & \textbf{接受AVR} & \textbf{未接受AVR} \\
\midrule
高梯度-正常射血分数 (HG-NEF) & 70\% & 30\% \\
高梯度-低射血分数 (HG-LEF) & 53\% & 47\% \\
低梯度-正常射血分数 (LG-NEF) & 32\% & 68\% \\
低梯度-低射血分数 (LG-LEF) & 38\% & 62\% \\
\bottomrule
\end{tabular}
\end{table}

\textbf{关键发现}:
\begin{itemize}
    \item HG-NEF是Class I指征,但仍有30\%未接受治疗
    \item LG-NEF和LG-LEF可能符合Class IIa指征,但治疗率<40\%
    \item \textbf{总体治疗率<50\%},存在严重的治疗不足问题
\end{itemize}

\subsubsection{3. 农村地区差异}

\textbf{TAVR中心地理分布}(TVT Registry, 2025年7月):
\begin{itemize}
    \item 美国50个州 + 2个属地
    \item 总共852个TAVR中心
    \item 分布极不均匀,东部和西海岸密集,中部稀疏
\end{itemize}

\textbf{地理可及性数据}(Marquis-Gravel, JAMA Cardiology 2020):

研究纳入:
\begin{itemize}
    \item 47,527,537名Medicare患者(≥65岁)
    \item 31,098例TAVR手术
\end{itemize}

关键发现:
\begin{itemize}
    \item 仅\textbf{2.6\%}的Medicare患者居住在有TAVR中心的邮政编码区
    \item \textbf{92\%}的患者居住在有TAVR的医院转诊区域(HRR)
    \item 31,098例TAVR中,\textbf{24\%来自农村地区}
    \item \textbf{中位驾驶时间}:35分钟
    \item \textbf{驾驶时间范围}:2分钟 - 18小时(!)
\end{itemize}

\textbf{佛罗里达州研究}(Damluji, Circulation CV Outcomes 2020):

\begin{itemize}
    \item 研究期间:2011-2016
    \item 样本量:N=6,531例TAVR
\end{itemize}

关键发现:
\begin{enumerate}
    \item \textbf{TAVR使用率}:
    \begin{itemize}
        \item 高人口密度地区 vs 低人口密度地区:\textbf{7倍差异}
        \item 人口密度>750人/平方英里地区的TAVR率约为45例/10万人
        \item 人口密度<50人/平方英里地区的TAVR率约为5例/10万人
    \end{itemize}

    \item \textbf{TAVR死亡率}:
    \begin{itemize}
        \item 低人口密度地区的TAVR死亡率是高人口密度地区的\textbf{6倍}
        \item 表明农村地区患者可能就诊更晚、病情更重
    \end{itemize}
\end{enumerate}

\subsection{干预措施与解决方案}

\subsubsection{DETECT-AS试验}

\textbf{试验设计}:电子提供者通知(Electronic Provider Notification, EPN)系统

\textbf{主要结果}:
\begin{itemize}
    \item EPN组在1年时的累积AVR率:\textbf{47.8\%}
    \item 常规护理组:\textbf{37.6\%}
    \item \textbf{危险比HR 1.37 (95\%CI: 1.02-1.84), p=0.04}
\end{itemize}

\textbf{次要发现}:
\begin{itemize}
    \item 延长生存时间
    \item \textbf{减少性别和年龄在AVR实施中的差异}
    \item 女性:EPN组46.8\% vs 对照组25.9\% (OR 2.78, p<0.001)
    \item 男性:EPN组49.8\% vs 对照组45.5\% (OR 1.16, p=0.53)
    \item P值(交互作用)= 0.006,表明EPN对女性获益更大
\end{itemize}

\textbf{临床意义}:
在严重AS管理中,EPN导致更高的AVR率、更长的生存时间,并减少了性别和年龄差异。

\subsubsection{ALERT试验(进行中)}

\textbf{全称}:Addressing undertreatment and heaLth Equity in aortic stenosis and mitral regurgitation using an integrated ehR platform

\textbf{试验设计}:
\begin{itemize}
    \item 样本量:N=1,500患者
    \item 提供者:600名
    \item 医疗系统:5个
\end{itemize}

\textbf{假设}:自动化通知增加接受适当评估和治疗的患者比例

\textbf{纳入标准}:
\begin{enumerate}
    \item 严重AS
    \item 中-重度或重度二尖瓣反流
\end{enumerate}

\textbf{排除标准}:
\begin{enumerate}
    \item 年龄<18岁
    \item 既往接受过经导管或外科目标瓣膜修复/置换
    \item 超声由心脏病专家或心外科医生开具,或已在多学科心脏团队(MHT)就诊
    \item 已安排与MHT就诊或已安排经导管/外科瓣膜干预
\end{enumerate}

\textbf{干预}:
\begin{itemize}
    \item 提供者随机分配至对照组或通知组(1:1)
    \item 选定提供者:门诊心脏超声的开具提供者(若无记录则选开具人)
    \item 通知组:门诊心脏超声报告优先提供给心脏病专家和PCP
\end{itemize}

\textbf{主要终点}:
从通知发出日期(或本应发出日期)到经导管或外科瓣膜干预或MHT门诊就诊的时间(分层复合终点)

\subsection{未来方向}

\subsubsection{AI与数据分析的应用}

演讲提出了一个引人深思的问题:\textbf{AI \& Data Analytics: Good vs. Evil?}

提到的技术平台/公司:
\begin{itemize}
    \item TEMPUS
    \item egnite
    \item HeartSciences
    \item AccurKardia
\end{itemize}

\textbf{潜在应用方向}:
\begin{itemize}
    \item 利用AI识别未被诊断的AS患者
    \item 预测哪些患者可能从TAVR中获益
    \item 自动化筛查和转诊流程
    \item 减少诊断和治疗偏见
\end{itemize}

\textbf{伦理考量}:
\begin{itemize}
    \item AI可能加剧现有偏见(如果训练数据本身存在偏见)
    \item 需要确保算法公平性
    \item 透明度和可解释性要求
\end{itemize}

\subsubsection{其他正在进行的项目}

\begin{itemize}
    \item \textbf{TARGET AS}:靶向AS筛查项目
    \item \textbf{ALERT}:如上所述的临床试验
    \item \textbf{AHA-SFRN}:美国心脏协会战略重点研究网络
\end{itemize}

\subsection{结论}

\subsubsection{主要结论}

\textbf{AS治疗路径中的差异关键节点}:

\begin{enumerate}
    \item \textbf{检测}:严重瓣膜疾病的早期发现
    \item \textbf{临床识别}:症状与疾病的关联
    \item \textbf{监测影像}:适当的超声心动图随访
    \item \textbf{转诊}:转诊至手术或经导管干预
    \item \textbf{接受治疗}:实际接受AVR/TAVR
    \item \textbf{临床结果}:术后预后
\end{enumerate}

\textbf{三大差异来源}:
\begin{itemize}
    \item \textbf{种族/族裔}:黑人患者接受TAVR可能性低25\%,但术后结果相同
    \item \textbf{血流动力学亚型}:低梯度AS治疗率<40\%
    \item \textbf{农村性}:农村地区TAVR使用率低7倍,死亡率高6倍
\end{itemize}

\subsubsection{我们取得进展了吗?}

\textbf{进展方面}:
\begin{itemize}
    \item TAVR中心数量增长3.4倍
    \item TAVR手术量增长4倍
    \item 所有种族/性别的绝对手术量都在增加
    \item TAVR死亡率持续下降(1年死亡率从17\%降至9.5\%)
    \item 开展了DETECT-AS等干预试验,证明EPN有效
\end{itemize}

\textbf{仍存在的问题}:
\begin{itemize}
    \item 黑人患者比例10年几乎无变化(4\%)
    \item 黑人接受TAVR可能性仍低25\%
    \item 农村地区差距仍然巨大
    \item 低梯度AS患者治疗率<40\%
    \item 监测偏见和治疗偏见依然存在
\end{itemize}

\textbf{答案}:\textbf{取得了一些进展,但远远不够}

\subsection{临床启示}

\subsubsection{对临床实践的建议}

\begin{enumerate}
    \item \textbf{提高警惕}:
    \begin{itemize}
        \item 对所有AS患者(特别是少数族裔、女性、农村患者)进行系统性筛查
        \item 不要忽视低梯度AS患者
    \end{itemize}

    \item \textbf{实施系统性干预}:
    \begin{itemize}
        \item 考虑采用电子提供者通知(EPN)系统
        \item 建立AS患者数据库和随访系统
        \item 确保所有符合条件的患者都被转诊至心脏团队
    \end{itemize}

    \item \textbf{解决可及性问题}:
    \begin{itemize}
        \item 扩大TAVR中心覆盖范围
        \item 为农村患者提供交通支持
        \item 考虑远程医疗在筛查和随访中的应用
    \end{itemize}

    \item \textbf{文化敏感性}:
    \begin{itemize}
        \item 提供多语言医疗服务
        \item 了解不同文化背景患者的医疗偏好
        \item 建立信任关系
    \end{itemize}

    \item \textbf{教育患者和提供者}:
    \begin{itemize}
        \item 提高对AS严重性的认识
        \item 教育初级保健医生识别AS症状
        \item 向患者解释TAVR的安全性和有效性
    \end{itemize}
\end{enumerate}

\subsubsection{对研究的启示}

\begin{enumerate}
    \item 需要更多针对少数族裔和农村人群的研究
    \item 探索低梯度AS的最佳管理策略
    \item 开发和验证AI辅助诊断工具
    \item 研究社会健康决定因素的干预措施
    \item 评估不同干预措施对减少差异的效果
\end{enumerate}

\subsection{研究局限性}

\begin{enumerate}
    \item 本文献为会议演讲,数据主要来自注册研究(TVT Registry)
    \item 部分数据可能存在选择偏倚(只包括参与注册的中心)
    \item 未能完全控制所有混杂因素
    \item 某些干预措施(如ALERT)仍在进行中,尚无最终结果
    \item 主要聚焦美国数据,其他国家情况可能不同
\end{enumerate}

\subsection{个人笔记}

\subsubsection{关键数字记忆}

\begin{itemize}
    \item TAVR中心增长:252 → 850(3.4倍)
    \item TAVR手术量增长:24,647 → 106,147(4倍)
    \item 黑人患者比例:始终约4\%(无明显改善)
    \item 黑人接受TAVR可能性:比白人低25\%
    \item 低梯度AS治疗率:<40\%
    \item 农村vs城市TAVR使用率差异:7倍
    \item 农村vs城市TAVR死亡率差异:6倍
    \item DETECT-AS试验EPN效果:HR 1.37, p=0.04
\end{itemize}

\subsubsection{重要概念}

\begin{description}
    \item[Surveillance Bias] 监测偏见 - 某些人群(黑人、女性、老年人、Medicaid患者)接受适当超声监测的可能性更低
    \item[Treatment Bias] 治疗偏见 - 黑人患者接受TAVR的可能性比白人低约25\%,即使调整了其他因素
    \item[SDOH] 社会健康决定因素 - 影响健康可及性的多维度因素
    \item[EPN] 电子提供者通知 - 一种有效的系统性干预,可提高AVR率并减少性别差异
\end{description}

\subsubsection{对中国的启示}

虽然本研究聚焦美国,但对中国也有借鉴意义:
\begin{itemize}
    \item 中国城乡医疗资源差异可能更大
    \item 经济发达地区vs欠发达地区的TAVR可及性差异
    \item 不同民族、不同收入水平患者的医疗可及性
    \item 可以借鉴EPN等系统性干预措施
    \item 重视低梯度AS患者的识别和治疗
\end{itemize}

\subsubsection{值得思考的问题}

\begin{enumerate}
    \item 为什么TAVR术后结果无种族差异,但获得治疗的机会有差异?
    \begin{itemize}
        \item 答:差异主要在就医行为、诊断偏见、治疗转诊等"上游"环节
        \item 一旦接受TAVR,技术和护理质量对所有患者是相同的
    \end{itemize}

    \item 低梯度AS为何治疗率如此低?
    \begin{itemize}
        \item 诊断不确定性(需要DSE等特殊检查)
        \item 临床医生对低梯度AS认识不足
        \item 指南推荐等级相对较低(Class IIa vs Class I)
        \item 可能需要更多证据支持
    \end{itemize}

    \item AI是"Good"还是"Evil"?
    \begin{itemize}
        \item Good:可以帮助识别被遗漏的患者,减少人为偏见
        \item Evil:如果训练数据有偏见,可能固化甚至加剧现有不平等
        \item 关键:确保AI开发过程中的公平性和透明度
    \end{itemize}
\end{enumerate}

\section{应对主动脉瓣狭窄治疗不足:Target AS、DETECT AS及未来展望}

\subsection{文献信息}

\begin{itemize}
    \item \textbf{PDF文件名}: addressing-undertreatment-in-aortic-stenosis-target-as-detect-as-and-beyond.pdf
    \item \textbf{作者}: Sammy Elmariah, MD, MPH
    \item \textbf{单位}:
    \begin{itemize}
        \item Leone-Perkins Family Endowed Professor of Medicine
        \item Chief, Interventional Cardiology
        \item Director, UCSF Cardiac Catheterization Laboratory
    \end{itemize}
    \item \textbf{会议}: TCT (Transcatheter Cardiovascular Therapeutics) 2025
    \item \textbf{文献类型}: 会议演讲(17页PPT)
    \item \textbf{主题分类}: 指南与基础(Guidelines \& Fundamentals)
    \item \textbf{关键词}: AS治疗不足、Target AS、DETECT AS试验、电子医生通知、健康公平、质量改进
\end{itemize}

\subsection{研究背景}

\subsubsection{主动脉瓣狭窄的临床问题}

\begin{enumerate}
    \item \textbf{临床严重性}:
    \begin{itemize}
        \item 有症状的严重AS与高发病率和死亡率相关
        \item 如不治疗,预后极差
        \item AVR(主动脉瓣置换术)是治愈性治疗
        \item AVR可延长AS各亚型患者的生命
    \end{itemize}

    \item \textbf{治疗不足的现状}:
    \begin{itemize}
        \item AS显著治疗不足(markedly undertreated)
        \item 尤其在以下人群中:
        \begin{itemize}
            \item 女性患者
            \item 老年患者
            \item 种族/民族少数群体
        \end{itemize}
    \end{itemize}

    \item \textbf{不同血流动力学亚型的治疗率}(Li SX等,J Am Coll Cardiol. 2022):

    \begin{table}[h]
    \centering
    \caption{不同AS亚型的AVR治疗效果与治疗率}
    \begin{tabular}{lcccc}
    \toprule
    \textbf{AS亚型} & \textbf{AVR降低} & \textbf{治疗率} & \textbf{AVA适应证} \\
     & \textbf{死亡风险} & \textbf{(接受治疗)} & \textbf{级别} \\
    \midrule
    高梯度正常LVEF & 2.4倍 & 30\%未治疗 & Class I \\
    高梯度低LVEF & 3.6倍 & 47\%未治疗 & Class I \\
    低梯度正常LVEF & 1.4倍 & 67\%未治疗 & Class II \\
    低梯度低LVEF & 2.1倍 & 62\%未治疗 & Class II \\
    \bottomrule
    \end{tabular}
    \end{table}

    \item \textbf{参考文献}:
    \begin{itemize}
        \item Li SX... Elmariah S. J Am Coll Cardiol. 2022;79:864-77
        \item Crousillat DR... Elmariah S. J Am Heart Assoc. 2022;11:e025692
    \end{itemize}
\end{enumerate}

\subsection{Target: Aortic Stenosis项目}

\subsubsection{项目概述}

\begin{itemize}
    \item \textbf{发起组织}: American Heart Association (AHA) Quality Initiative
    \item \textbf{国家赞助商}: Edwards Lifesciences
    \item \textbf{项目目标}: 改善严重AS患者在AVR前的上游护理和结局
\end{itemize}

\subsubsection{AS患者护理路径(Aortic Stenosis Patient Care Pathway)}

完整的护理路径包括6个阶段:

\begin{enumerate}
    \item \textbf{Awareness}(意识)
    \item \textbf{Detection}(检测)
    \item \textbf{Diagnosis}(诊断)
    \item \textbf{Referral}(转诊)
    \item \textbf{Treatment}(治疗)
    \item \textbf{Monitoring}(监测)
\end{enumerate}

Target: AS项目覆盖所有6个阶段,而现有的手术登记数据库仅覆盖最后2个阶段(治疗和监测)。

\subsubsection{项目规模(截至2025年9月24日)}

\begin{itemize}
    \item \textbf{参与医院}: 75家医院已签约并参与
    \item \textbf{登记患者}: 12,386份患者记录
    \item \textbf{随访记录}: 47,704+次就诊记录
\end{itemize}

\subsubsection{项目价值}

\textbf{对于医疗系统}:
\begin{itemize}
    \item 基于更新指南的质量措施实施
\end{itemize}

\textbf{对于医疗服务提供者}:
\begin{itemize}
    \item 提供指南导向的最佳护理标准教育
\end{itemize}

\textbf{对于患者}:
\begin{itemize}
    \item 提高患者意识和参与度
\end{itemize}

\subsection{2025年医院认证标准}

\subsubsection{认证指标体系}

Target: Aortic Stenosis项目建立了以下关键指标(Measure Relationship Diagram):

\begin{enumerate}
    \item \textbf{主要指标(Primary Measure)- 75\%达标}:
    \begin{itemize}
        \item \textbf{指标名称}: 严重主动脉瓣狭窄的及时治疗
        \item \textbf{定义}: 具有Class I适应证的患者在初次诊断后90天内接受确定性治疗(SAVR或TAVI)的百分比
        \item \textbf{最小分母}: 至少6名患者
    \end{itemize}

    \item \textbf{次要指标(Secondary Measure)- 50\%达标}:
    \begin{itemize}
        \item \textbf{指标名称}: 无缺陷的及时诊断
        \item \textbf{定义}: 潜在严重AS超声心动图检查中,完成所有必要评估和测试以明确严重程度并确定是否存在Class I适应证的百分比
        \item \textbf{最小分母}: 至少30次超声检查
    \end{itemize}

    \item \textbf{容量标准(Volume Criteria)}:
    \begin{itemize}
        \item 2024年登记数据必须至少有40名患者才有资格获得认证
    \end{itemize}

    \item \textbf{支持性指标}(必须报告,但无达标阈值):
    \begin{itemize}
        \item 超声中关键发现的报告和总结/结论
        \item 多学科心脏瓣膜团队的评估
        \item 及时完成随访超声心动图
    \end{itemize}
\end{enumerate}

\subsubsection{详细的质量测量流程}

从超声提示中重度AS → 诊断和随访 → Class I适应证评估 → 治疗,整个流程包括:

\begin{itemize}
    \item 超声中的关键发现
    \item 确定或可能的严重AS(AVA≤1.0 cm²,或峰值流速≥4 m/s,或峰值梯度≥64 mmHg,或平均梯度≥40 mmHg)
    \item 超声总结中的临床建议
    \item AS严重程度的及时诊断(及时评估症状、LVEF、SVI和多模态检测)
    \item 如无Class I适应证 → 及时完成随访超声
    \item 如有Class I适应证 → MDT评估(仅评估接受AVR的患者)
    \item 最终测量:严重AS的及时治疗(主要指标)
\end{itemize}

\subsection{新的瓣膜性心脏病质量标准}

\subsubsection{ACC/AHA 2024年临床表现和质量措施}

\begin{enumerate}
    \item \textbf{文献来源}:
    \begin{itemize}
        \item Jneid H, et al. J Am Coll Cardiol. 2024 Apr 23;83(16):1579-1613
        \item 2024 ACC/AHA Clinical Performance and Quality Measures for Adults With Valvular and Structural Heart Disease
    \end{itemize}

    \item \textbf{核心质量指标}:
    \begin{itemize}
        \item \textbf{"准备用于公共报告和按绩效付费计划"(ready for public reporting and pay-for-performance programs)}
        \item \textbf{关键指标}: 有症状严重AS患者在诊断后90天内接受AVR的比例
    \end{itemize}

    \item \textbf{明确的临床需求}:
    \begin{quote}
    "There is a clear and unmet need for effective, low-cost, and scalable tools to bolster guideline-driven management of severe AS."

    存在明确且未满足的需求:需要有效、低成本且可扩展的工具来促进严重AS的指南导向管理。
    \end{quote}
\end{enumerate}

\subsection{DETECT AS试验}

\subsubsection{研究设计}

\begin{itemize}
    \item \textbf{试验全称}: DETECT AS Trial
    \item \textbf{ClinicalTrials.gov}: NCT05230225
    \item \textbf{研究类型}: 实用性、单盲、分组随机对照试验和质量改进项目
    \item \textbf{研究地点}: MGH(Massachusetts General Hospital)多中心学术医疗系统
    \item \textbf{研究资助}: 研究者发起,Edwards Lifesciences赞助
\end{itemize}

\subsubsection{研究方法}

\begin{enumerate}
    \item \textbf{纳入标准}:
    \begin{itemize}
        \item 经胸超声心动图(TTE)显示主动脉瓣瓣口面积(AVA)≤1.0 cm²的患者
    \end{itemize}

    \item \textbf{随机化方法}:
    \begin{itemize}
        \item 临床医生1:1随机分配
        \item 分层分配在后续患者中保持稳定
        \item 285名医生参与
        \item 945名患者入组
    \end{itemize}

    \item \textbf{干预措施}:
    \begin{itemize}
        \item \textbf{实验组}: 电子医生通知(Electronic Provider Notification, EPN)
        \item \textbf{对照组}: 常规护理(Usual Care)
    \end{itemize}

    \item \textbf{主要终点}:
    \begin{itemize}
        \item 指标超声心动图后1年内接受AVR的患者比例
    \end{itemize}

    \item \textbf{随访时间}:
    \begin{itemize}
        \item 完整1年随访
    \end{itemize}
\end{enumerate}

\subsubsection{个性化电子医生通知(Personalized EPN)}

\textbf{通知方式}:
\begin{itemize}
    \item 通过电子邮件
    \item 通过EMR(电子病历)收件箱
\end{itemize}

\textbf{基于血流动力学分型的4种个性化EPN}:

\begin{table}[h]
\centering
\caption{DETECT AS试验中的4种血流动力学分型}
\begin{tabular}{lcc}
\toprule
\textbf{血流动力学分型} & \textbf{平均梯度} & \textbf{LVEF} \\
\midrule
1. 高梯度,正常LVEF & mAVG ≥40 mmHg & LVEF ≥50\% \\
2. 高梯度,低LVEF & mAVG ≥40 mmHg & LVEF <50\% \\
3. 低梯度,正常LVEF & mAVG <40 mmHg & LVEF ≥50\% \\
4. 低梯度,低LVEF & mAVG <40 mmHg & LVEF <50\% \\
\bottomrule
\end{tabular}
\end{table}

\textbf{EPN内容}(示例):

\begin{quote}
"Hello Dr ------

Your patient, -----, recently underwent a transthoracic echocardiogram that identified severe aortic stenosis with preserved ejection fraction.

The ACC/AHA Guidelines for the Management of Valvular Heart Disease make the following recommendations which may apply to this patient:

\begin{itemize}
    \item In symptomatic patients with severe AS, AVR is indicated. (class 1 recommendation)
    \item In asymptomatic patients with severe AS and low surgical risk, AVR is reasonable when:
    \begin{itemize}
        \item AS is very severe (defined as an aortic velocity of ≥5 m/s) and there is low surgical risk, AVR is reasonable. (class 2a recommendation)
        \item An exercise test demonstrates decreased exercise tolerance or a fall in systolic blood pressure of ≥10 mmHg from baseline to peak exercise. (class 2a recommendation)
        \item Serum B-type natriuretic peptide (BNP) level is >3 times normal. (class 2a recommendation)
        \item Serial testing shows an increase in aortic velocity ≥0.3 m/s per year. (class 2a recommendation)
        \item LVEF progressively declines on at least 3 serial imaging studies reaching <60\%. (class 2b recommendation)
    \end{itemize}
\end{itemize}

Patients with severe valvular heart disease should be evaluated by a Multidisciplinary Heart Valve Team when intervention is considered. (class 1 recommendation)"
\end{quote}

\subsection{DETECT AS主要研究结果}

\subsubsection{主要结果发表}

\begin{itemize}
    \item \textbf{发表期刊}: Circulation. 2025
    \item \textbf{第一作者}: Tanguturi V
    \item \textbf{通讯作者}: Elmariah S
\end{itemize}

\subsubsection{有症状患者的AVR治疗率}

\begin{table}[h]
\centering
\caption{DETECT AS试验:有症状患者的AVR治疗率}
\begin{tabular}{lccc}
\toprule
\textbf{时间点} & \textbf{EPN组} & \textbf{常规护理组} & \textbf{统计学意义} \\
\midrule
90天(主要质量指标) & 36.9\% & 27.6\% & - \\
1年 & 60.1\% & 47.0\% & - \\
\midrule
\multicolumn{4}{l}{\textbf{因果特异性Cox模型(考虑死亡竞争风险)}} \\
\multicolumn{4}{l}{HR 1.40 (95\%CI 1.06-1.85); p=0.02} \\
\bottomrule
\end{tabular}
\end{table}

\textbf{关键发现}:
\begin{itemize}
    \item EPN显著提高了AVR治疗率
    \item 90天时,EPN组比常规护理组高9.3个百分点(36.9\% vs 27.6\%)
    \item 1年时,EPN组比常规护理组高13.1个百分点(60.1\% vs 47.0\%)
    \item 风险比(HR)1.40,提示EPN组接受AVR的可能性增加40\%
\end{itemize}

\subsubsection{性别亚组分析}

\begin{table}[h]
\centering
\caption{性别亚组的AVR治疗率(1年结局)}
\begin{tabular}{lcc}
\toprule
\textbf{亚组} & \textbf{EPN组} & \textbf{常规护理组} \\
\midrule
女性 & 46.1\% & 25.9\% \\
男性 & 48.7\% & 47.2\% \\
\bottomrule
\end{tabular}
\end{table}

\textbf{性别差异的统计学分析}:

\begin{enumerate}
    \item \textbf{女性患者:EPN vs. 常规护理}
    \begin{itemize}
        \item HR 2.10 (95\% CI 1.56-2.82); p<0.001
        \item EPN使女性患者接受AVR的可能性增加110\%
    \end{itemize}

    \item \textbf{常规护理组:女性 vs. 男性}
    \begin{itemize}
        \item HR 0.46 (95\% CI 0.35-0.62); p<0.001
        \item 常规护理中,女性接受AVR的可能性仅为男性的46\%
        \item \textbf{存在显著的性别差异}
    \end{itemize}

    \item \textbf{EPN组:女性 vs. 男性}
    \begin{itemize}
        \item HR 0.90 (95\% CI 0.70-1.16); p=0.43
        \item EPN组中,女性和男性的治疗率无显著差异
        \item \textbf{EPN消除了性别差异}
    \end{itemize}
\end{enumerate}

\textbf{关键结论}:
\begin{itemize}
    \item 常规护理中存在严重的性别不平等
    \item EPN显著改善了女性患者的治疗率
    \item EPN成功消除了性别间的治疗差异
\end{itemize}

\subsubsection{EPN延长生存时间}

\textbf{有症状患者总体}:

\begin{table}[h]
\centering
\caption{限制性平均生存时间(Restricted Mean Survival Time)}
\begin{tabular}{lccc}
\toprule
\textbf{组别} & \textbf{生存时间} & \textbf{差异} & \textbf{P值} \\
\midrule
EPN & 335天 & \multirow{2}{*}{23天} & \multirow{2}{*}{p=0.01} \\
常规护理 & 312天 & & \\
\midrule
\multicolumn{4}{l}{Log-rank检验: p=0.06} \\
\bottomrule
\end{tabular}
\end{table}

\textbf{按性别分层的生存分析}:

\begin{table}[h]
\centering
\caption{性别亚组的EPN生存获益}
\begin{tabular}{lcc}
\toprule
\textbf{性别} & \textbf{EPN延长生存时间} & \textbf{P值} \\
\midrule
女性 & 26天 & p=0.03 \\
男性 & 17天 & p=0.16 \\
\midrule
\multicolumn{3}{l}{性别亚组总体Log-rank检验: p=0.05} \\
\bottomrule
\end{tabular}
\end{table}

\textbf{关键发现}:
\begin{itemize}
    \item EPN显著延长了有症状患者的生存时间
    \item 女性患者从EPN中获得的生存获益更大且具有统计学意义
    \item 男性患者也有生存获益趋势,但未达到统计学显著性
\end{itemize}

\subsection{DETECT AS试验结论}

\textbf{总体结论}:

在AVA ≤1.0 cm²的主动脉瓣狭窄患者管理中,电子医生通知(EPN)带来了以下结果:

\begin{enumerate}
    \item \textbf{提高AVR治疗率}
    \begin{itemize}
        \item 90天和1年时的AVR率均更高
    \end{itemize}

    \item \textbf{延长生存时间}
    \begin{itemize}
        \item 有症状患者的生存时间延长23天
        \item 女性患者尤其获益(延长26天)
    \end{itemize}

    \item \textbf{减少AS管理中的健康不平等}
    \begin{itemize}
        \item 消除了性别间的治疗差异
        \item 女性患者的治疗率从25.9\%提高到46.1\%
    \end{itemize}
\end{enumerate}

\textbf{重大意义}:

\begin{quote}
"The DETECT AS Trial demonstrated the potential impact of AI-based alerts, decision support, and management tools in improving quality of care."

DETECT AS试验证明了基于AI的警报、决策支持和管理工具在改善护理质量方面的潜在影响。
\end{quote}

\subsection{2025年ASE标准化指南}

\subsubsection{指南概述}

\begin{itemize}
    \item \textbf{发布组织}: American Society of Echocardiography (ASE)
    \item \textbf{参与组织}: 20个全球超声心动图学会联合发布
    \item \textbf{文献来源}: Taub CC, et al. JASE. 2025; 38 (9) 735-774. DOI: 10.1016/J.ECHO.2025.06.001
    \item \textbf{指南名称}: Guidelines for the Standardization of Adult Echocardiography Reporting: Recommendations From the American Society of Echocardiography
\end{itemize}

\subsubsection{关键指南更新}

\begin{enumerate}
    \item \textbf{关键发现的即时沟通}:
    \begin{itemize}
        \item 包括严重AS在内的关键发现应在报告中记录
        \item 并在\textbf{数分钟内}口头告知开具检查的医生
    \end{itemize}

    \item \textbf{建议声明要求}:
    \begin{itemize}
        \item 超声心动图医生应包含关于重要AS的进一步转诊/评估的建议声明
    \end{itemize}

    \item \textbf{标准化建议语言示例}:
    \begin{quote}
    "This patient has significant aortic stenosis that, according to the current American College of Cardiology/American Heart Association/ASE valvular heart disease guidelines, may warrant treatment. As clinically appropriate, further evaluation and/or referral should be considered."

    该患者存在重要的主动脉瓣狭窄,根据当前ACC/AHA/ASE瓣膜性心脏病指南,可能需要治疗。在临床适当的情况下,应考虑进一步评估和/或转诊。
    \end{quote}
\end{enumerate}

\subsubsection{指南的变革意义}

\begin{quote}
"These guidelines may help \textbf{redefine the role of echocardiography in patient care} from passive, descriptive reporting to \textbf{active physician-guided participation in patient management}."

这些指南可能有助于重新定义超声心动图在患者护理中的角色,从被动的描述性报告转变为积极的医生引导的患者管理参与。
\end{quote}

\textbf{角色转变}:
\begin{itemize}
    \item \textbf{传统角色}: 被动、描述性报告
    \item \textbf{新角色}: 积极的、医生引导的患者管理参与
\end{itemize}

\subsection{主动监测的目标}

\subsubsection{系统化监测策略}

\textbf{总体目标}:
\begin{quote}
"To facilitate regimented surveillance and unbiased and timely evaluation and management of significant aortic stenosis."

促进系统化监测以及对重要主动脉瓣狭窄的无偏见和及时的评估和管理。
\end{quote}

\subsubsection{具体实施措施}

\begin{enumerate}
    \item \textbf{EMR集成的随访提示}:
    \begin{itemize}
        \item 电子病历集成的监测超声心动图提示
        \item 自动化随访提醒系统
    \end{itemize}

    \item \textbf{促进转诊至心脏瓣膜团队}:
    \begin{itemize}
        \item 设置有时限的"选择退出"机制
        \item 默认转诊,除非医生主动选择不转诊
    \end{itemize}

    \item \textbf{避免失访}:
    \begin{itemize}
        \item 系统化追踪患者随访
        \item 减少患者在系统中的流失
    \end{itemize}

    \item \textbf{记录未转诊原因}:
    \begin{itemize}
        \item 要求医生说明不执行转诊的原因
        \item 提高决策透明度和问责制
    \end{itemize}
\end{enumerate}

\subsubsection{EMR提示示例}

UCSF系统中的自动化EMR提示:

\begin{quote}
"Patient meets criteria for severe aortic stenosis with an EF ≤49\% and does not have a referral to the UCSF Valve Clinic or a visit in the past 90 days. Consider placing a referral below."

患者符合严重主动脉瓣狭窄标准,射血分数≤49\%,且在过去90天内没有转诊至UCSF瓣膜诊所或就诊记录。请考虑在下方放置转诊。
\end{quote}

\subsection{临床意义与讨论}

\subsubsection{AS治疗不足的现状}

\begin{enumerate}
    \item \textbf{普遍的治疗不足}:
    \begin{itemize}
        \item 即使是高梯度正常LVEF的Class I适应证患者,仍有30\%未接受治疗
        \item 低梯度亚型的治疗不足更为严重(62-67\%未治疗)
    \end{itemize}

    \item \textbf{健康不平等问题}:
    \begin{itemize}
        \item 女性患者在常规护理中接受AVR的可能性仅为男性的46\%
        \item 种族/民族少数群体存在类似的治疗差异
        \item 老年患者同样面临治疗不足
    \end{itemize}

    \item \textbf{治疗不足的后果}:
    \begin{itemize}
        \item AVR可使死亡风险降低1.4-3.6倍
        \item 不治疗导致可预防的死亡和发病
    \end{itemize}
\end{enumerate}

\subsubsection{电子医生通知(EPN)的创新价值}

\begin{enumerate}
    \item \textbf{可扩展性}:
    \begin{itemize}
        \item 基于电子病历的自动化系统
        \item 低成本、高效率
        \item 可在多个医疗系统中推广
    \end{itemize}

    \item \textbf{个性化}:
    \begin{itemize}
        \item 根据血流动力学分型提供针对性建议
        \item 引用具体的指南推荐
        \item 包含患者特定的临床信息
    \end{itemize}

    \item \textbf{及时性}:
    \begin{itemize}
        \item 超声结果出来后立即发送通知
        \item 避免诊疗延误
        \item 促进90天内的治疗决策
    \end{itemize}

    \item \textbf{促进健康公平}:
    \begin{itemize}
        \item 成功消除了性别间的治疗差异
        \item 标准化的通知减少了隐性偏见的影响
        \item 确保所有患者都能及时获得指南推荐的护理
    \end{itemize}
\end{enumerate}

\subsubsection{AI和决策支持工具的未来}

\begin{enumerate}
    \item \textbf{AI在AS管理中的应用}:
    \begin{itemize}
        \item 自动识别符合标准的患者
        \item 智能化的风险分层
        \item 预测性分析指导治疗时机
    \end{itemize}

    \item \textbf{集成化的决策支持}:
    \begin{itemize}
        \item 超声报告中嵌入指南推荐
        \item EMR中的自动化临床路径
        \item 多学科团队协作平台
    \end{itemize}

    \item \textbf{质量改进工具}:
    \begin{itemize}
        \item 实时监测护理质量指标
        \item 识别系统层面的差距
        \item 反馈和持续改进循环
    \end{itemize}
\end{enumerate}

\subsubsection{Target: AS与DETECT AS的协同作用}

\begin{enumerate}
    \item \textbf{Target: AS提供}:
    \begin{itemize}
        \item 质量测量框架
        \item 医院认证标准
        \item 全国性的数据收集平台
    \end{itemize}

    \item \textbf{DETECT AS证明}:
    \begin{itemize}
        \item 电子通知系统的有效性
        \item 改善结局的实证证据
        \item 减少健康不平等的可行途径
    \end{itemize}

    \item \textbf{共同推动}:
    \begin{itemize}
        \item 从研究到实践的转化
        \item 质量标准的实施
        \item 全国范围的护理改进
    \end{itemize}
\end{enumerate}

\subsection{研究局限性}

\begin{enumerate}
    \item \textbf{单中心研究}:
    \begin{itemize}
        \item DETECT AS在MGH医疗系统进行
        \item 可能限制结果的普遍适用性
        \item 需要多中心验证
    \end{itemize}

    \item \textbf{随机化层面}:
    \begin{itemize}
        \item 医生层面的随机化,而非患者层面
        \item 可能存在聚类效应
        \item 使用了适当的统计方法(考虑竞争风险)
    \end{itemize}

    \item \textbf{随访时间}:
    \begin{itemize}
        \item 1年随访可能不足以评估长期结局
        \item 需要更长期的生存和临床结局数据
    \end{itemize}

    \item \textbf{实施细节}:
    \begin{itemize}
        \item 演讲未详述EPN的接受率和遵从度
        \item 未说明医生对EPN的反应和反馈
        \item 系统集成的技术细节不详
    \end{itemize}

    \item \textbf{成本效益分析}:
    \begin{itemize}
        \item 未提供EPN系统的成本数据
        \item 缺乏成本效益分析
        \item 虽然强调"低成本"但无具体数据支持
    \end{itemize}
\end{enumerate}

\subsection{未来方向}

\subsubsection{研究方向}

\begin{enumerate}
    \item \textbf{扩展DETECT AS研究}:
    \begin{itemize}
        \item 多中心、多样化人群的验证研究
        \item 评估种族/民族少数群体的影响
        \item 评估农村和资源有限地区的效果
    \end{itemize}

    \item \textbf{长期随访研究}:
    \begin{itemize}
        \item 5年和10年的生存率
        \item 长期生活质量评估
        \item 再干预率和瓣膜耐久性
    \end{itemize}

    \item \textbf{优化EPN系统}:
    \begin{itemize}
        \item 测试不同的通知频率和形式
        \item 评估患者直接参与的效果
        \item 开发更精细的个性化算法
    \end{itemize}

    \item \textbf{实施研究}:
    \begin{itemize}
        \item 研究不同医疗系统中的实施障碍
        \item 评估医生行为改变的促进因素
        \item 优化EMR集成的技术方法
    \end{itemize}
\end{enumerate}

\subsubsection{临床实践改进}

\begin{enumerate}
    \item \textbf{标准化超声报告}:
    \begin{itemize}
        \item 实施2025 ASE指南推荐
        \item 所有严重AS报告包含管理建议
        \item 建立即时沟通流程
    \end{itemize}

    \item \textbf{EMR优化}:
    \begin{itemize}
        \item 集成自动化决策支持工具
        \item 建立患者追踪和随访系统
        \item 设置默认转诊路径(选择退出机制)
    \end{itemize}

    \item \textbf{多学科协作}:
    \begin{itemize}
        \item 强化心脏瓣膜团队的作用
        \item 建立标准化的会诊流程
        \item 促进超声医生、心内科、心外科的协作
    \end{itemize}

    \item \textbf{质量监测}:
    \begin{itemize}
        \item 参与Target: AS或类似登记项目
        \item 定期评估机构的质量指标
        \item 识别和解决护理差距
    \end{itemize}
\end{enumerate}

\subsubsection{政策和系统层面}

\begin{enumerate}
    \item \textbf{质量测量的推广}:
    \begin{itemize}
        \item ACC/AHA质量指标的广泛采用
        \item 纳入医院评估和认证标准
        \item 按绩效付费计划的整合
    \end{itemize}

    \item \textbf{健康公平倡议}:
    \begin{itemize}
        \item 专门针对弱势群体的项目
        \item 监测和报告健康差异
        \item 实施减少差异的干预措施
    \end{itemize}

    \item \textbf{技术基础设施}:
    \begin{itemize}
        \item 支持EMR互操作性
        \item 开发标准化的临床决策支持工具
        \item 促进数据共享和学习型医疗系统
    \end{itemize}
\end{enumerate}

\subsection{个人笔记}

\subsubsection{演讲的亮点}

\begin{enumerate}
    \item \textbf{数据驱动的质量改进}:
    \begin{itemize}
        \item 本演讲完美展示了如何使用数据识别问题(AS治疗不足)
        \item 开发干预措施(EPN系统)
        \item 通过随机对照试验验证效果
        \item 最终转化为系统性的质量改进项目(Target: AS)
    \end{itemize}

    \item \textbf{健康公平的实际解决方案}:
    \begin{itemize}
        \item 不仅识别了性别差异,还证明了消除差异的有效方法
        \item EPN成功将女性的治疗率从25.9\%提高到46.1\%
        \item 这是健康公平研究中的重要进展
    \end{itemize}

    \item \textbf{实用性和可扩展性}:
    \begin{itemize}
        \item EPN是低成本、基于现有EMR系统的干预
        \item 不需要额外的临床资源
        \item 易于在其他医疗系统中复制
    \end{itemize}

    \item \textbf{多层面的改变}:
    \begin{itemize}
        \item 指南更新(2025 ASE)
        \item 质量测量(ACC/AHA 2024)
        \item 质量改进项目(Target: AS)
        \item 临床试验证据(DETECT AS)
        \item 构成了完整的护理改进生态系统
    \end{itemize}
\end{enumerate}

\subsubsection{对中国TAVR实践的启示}

\begin{enumerate}
    \item \textbf{AS治疗不足可能更严重}:
    \begin{itemize}
        \item 中国的TAVR普及率相对较低
        \item 许多AS患者可能未被识别或未接受治疗
        \item 需要类似的质量改进项目
    \end{itemize}

    \item \textbf{超声报告标准化}:
    \begin{itemize}
        \item 可借鉴2025 ASE指南
        \item 在报告中包含管理建议
        \item 建立关键发现的即时沟通机制
    \end{itemize}

    \item \textbf{电子病历的潜力}:
    \begin{itemize}
        \item 中国医院广泛使用电子病历
        \item 可开发类似的决策支持工具
        \item 利用AI和大数据识别未治疗的患者
    \end{itemize}

    \item \textbf{关注健康公平}:
    \begin{itemize}
        \item 评估中国人群中的治疗差异
        \item 识别弱势群体(农村、低收入、女性等)
        \item 开发针对性的改进策略
    \end{itemize}
\end{enumerate}

\subsubsection{研究设计的优点}

\begin{enumerate}
    \item \textbf{实用性试验设计}:
    \begin{itemize}
        \item 在真实临床环境中进行
        \item 医生层面的随机化更接近实际实施情况
        \item 单盲设计(医生不知道被监测)减少了霍桑效应
    \end{itemize}

    \item \textbf{与质量改进项目结合}:
    \begin{itemize}
        \item DETECT AS不仅是研究,也是质量改进项目
        \item 参与的医生和医院可能从中直接受益
        \item 研究结果可立即转化为实践
    \end{itemize}

    \item \textbf{竞争风险分析}:
    \begin{itemize}
        \item 使用因果特异性Cox模型考虑死亡的竞争风险
        \item 这在AS研究中很重要,因为未治疗患者死亡率高
        \item 提供了更准确的治疗效果估计
    \end{itemize}

    \item \textbf{生存分析}:
    \begin{itemize}
        \item 不仅评估治疗率,还评估生存结局
        \item 使用限制性平均生存时间(RMST)
        \item 证明了EPN不仅改善过程指标,还改善患者结局
    \end{itemize}
\end{enumerate}

\subsubsection{值得深入研究的问题}

\begin{enumerate}
    \item \textbf{EPN的机制}:
    \begin{itemize}
        \item 为什么EPN有效?
        \item 是提高了医生意识?
        \item 还是降低了转诊的障碍?
        \item 或是标准化减少了偏见?
    \end{itemize}

    \item \textbf{性别差异的根源}:
    \begin{itemize}
        \item 为什么女性在常规护理中治疗率如此低?
        \item 是转诊不足?
        \item 还是患者拒绝?
        \item 或是其他系统性因素?
    \end{itemize}

    \item \textbf{90天指标的意义}:
    \begin{itemize}
        \item 为什么选择90天作为质量指标?
        \item 这个时间窗是基于临床证据还是可行性?
        \item 是否有患者因医学原因合理延迟治疗超过90天?
    \end{itemize}

    \item \textbf{未接受治疗的原因}:
    \begin{itemize}
        \item EPN组1年时仍有40\%患者未接受AVR
        \item 这些患者为什么没有接受治疗?
        \item 是医学禁忌、患者拒绝还是其他障碍?
    \end{itemize}
\end{enumerate}

\subsection{结论}

DETECT AS试验是AS领域的里程碑研究,证明了电子医生通知系统在改善AS管理和减少健康不平等方面的有效性。结合Target: AS质量改进项目和2025 ASE指南的更新,形成了完整的从识别问题、开发解决方案到系统性实施的框架。

关键要点:
\begin{itemize}
    \item AS显著治疗不足,尤其在女性和少数群体中
    \item 个性化EPN显著提高了AVR治疗率(HR 1.40)
    \item EPN延长了生存时间(23天,p=0.01)
    \item EPN成功消除了性别间的治疗差异
    \item 这些结果支持在更广泛的医疗系统中实施类似的决策支持工具
\end{itemize}

对临床实践的意义:
\begin{itemize}
    \item 超声报告应包含明确的管理建议
    \item 应建立自动化的患者识别和通知系统
    \item 需要系统性地监测和减少护理差异
    \item 质量测量应聚焦于过程指标(如90天内治疗率)和结局指标
\end{itemize}

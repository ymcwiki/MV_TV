\section{新指南如何影响临床实践?——主动脉瓣指南}

\subsection{文献信息}

\begin{itemize}
    \item \textbf{PDF文件名}: aortic-valve-guidelines.pdf
    \item \textbf{演讲者}: Fabien Praz, MD
    \item \textbf{单位}: Bern University Hospital, Switzerland
    \item \textbf{会议}: TCT (Transcatheter Cardiovascular Therapeutics) 2025
    \item \textbf{文献类型}: 会议演讲(18页PPT)
    \item \textbf{主要指南来源}:
    \begin{itemize}
        \item Praz F, Borger MA, Lanz J, et al., Eur Heart J. 2025 Aug 29:ehaf194
        \item ESC Congress 2025 Madrid, World Congress of Cardiology
    \end{itemize}
    \item \textbf{主题分类}: 指南与基础(Guidelines \& Fundamentals)
    \item \textbf{关键词}: ESC/EACTS指南2025、主动脉瓣反流、主动脉瓣狭窄、TAVR适应证、无症状AS、二叶瓣
\end{itemize}

\subsection{演讲概述}

本演讲系统介绍了ESC/EACTS 2025年主动脉瓣疾病管理指南的重要更新,重点包括:

\begin{enumerate}
    \item 主动脉瓣反流(Aortic Regurgitation, AR)的影像评估和管理
    \item 主动脉瓣狭窄(Aortic Stenosis, AS)TAVR适应证的扩展
    \item 无症状AS患者的早期干预
    \item 二叶主动脉瓣(Bicuspid Aortic Valve, BAV)狭窄的管理
\end{enumerate}

\subsection{主动脉瓣反流的影像评估}

\subsubsection{评估AR的多模态影像方法}

\textbf{评估标准分为三个层次}:

\begin{table}[h]
\centering
\caption{严重AR的影像学评估标准}
\begin{tabular}{lp{10cm}}
\toprule
\textbf{评估层次} & \textbf{标准} \\
\midrule
\multirow{3}{*}{\textbf{定性标准}} & • 瓣膜形态异常 \\
& • 瓣膜脱垂(Flail cusp) \\
& • 大的瓣膜合并缺损(Large coaptation defect) \\
\midrule
\multirow{4}{*}{\textbf{半定量标准}} & • 缩流颈宽度(Vena contracta)>6 mm \\
 & • 压力半时(PHT)<200 ms \\
 & • 大的中央反流束(≥65\% of LVOT diameter) \\
 & • 降主动脉全舒张期血流反转(EDV ≥20 cm/s) \\
\midrule
\multirow{3}{*}{\textbf{定量标准}} & • 有效反流瓣口面积(EROA)≥30 mm² \\
 & • 反流容量(RVol)≥60 mL/beat \\
 & • 反流分数(RF)>50\%(超声)\\
 & • 或 RF >40\%(CMR) \\
\bottomrule
\end{tabular}
\end{table}

\textbf{推荐的影像检查}:
\begin{itemize}
    \item \textbf{TTE}(经胸超声心动图):首选检查
    \item \textbf{TOE}(经食管超声心动图):半定量评估
    \item \textbf{CCT}(心脏CT):评估AR机制和主动脉扩张合并瓣叶畸形
    \item \textbf{CMR}(心脏磁共振):定量评估,RF >40\%为严重AR
\end{itemize}

\subsubsection{AR的机制评估}

评估AR的机制对选择治疗方式至关重要:

\begin{itemize}
    \item \textbf{瓣叶脱垂}(Cusp prolapse)
    \item \textbf{瓣叶回缩}(Cusp retraction)
    \item \textbf{瓣叶穿孔}(Cusp perforation)
    \item \textbf{主动脉扩张合并瓣叶畸形}(Aortic dilatation with cusp malcoaptation)
\end{itemize}

\subsection{主动脉瓣反流的管理}

\subsubsection{手术治疗的适应证}

\begin{table}[h]
\centering
\caption{严重AR的手术治疗推荐(ESC/EACTS 2025)}
\begin{tabular}{lcc}
\toprule
\textbf{推荐内容} & \textbf{推荐等级} & \textbf{证据级别} \\
\midrule
\makecell[l]{有症状严重AR患者推荐AV手术,\\不论LV功能} & I & B \\
\midrule
\makecell[l]{无症状严重AR患者推荐AV手术,\\如果LVESD >50 mm或LVESDi >25 mm/m²\\[尤其是小体型患者(BSA <1.68 m²)]\\或静息LVEF ≤50\%} & I & B \\
\midrule
\makecell[l]{无症状严重AR患者可考虑AV手术,\\如果LVESDi >22 mm/m²、\\\textbf{LVESVi >45 mL/m²}\\[尤其是小体型患者(BSA <1.68 m²)]、\\或静息LVEF ≤55\%,\\如果手术风险低} & IIb & B \\
\bottomrule
\end{tabular}
\end{table}

\textbf{新增内容标注}(标记为"REV."):
\begin{itemize}
    \item 引入了\textbf{左心室收缩末期容量指数(LVESVi)>45 mL/m²}作为手术指征
    \item 这是基于超声心动图或心脏MRI的容量截断值
    \item 强调了对小体型患者使用体表面积校正的重要性
\end{itemize}

\subsubsection{治疗方式的选择}

\begin{table}[h]
\centering
\caption{严重AR的治疗方式推荐}
\begin{tabular}{lcc}
\toprule
\textbf{推荐内容} & \textbf{推荐等级} & \textbf{证据级别} \\
\midrule
\makecell[l]{在有经验的中心,对选定的严重AR\\患者应考虑AV修复,\\当预期结果持久时} & IIa & B \\
\midrule
\makecell[l]{对于心脏团队判断为\\手术不适宜的有症状严重AR患者,\\如果解剖合适,\\可考虑TAVI治疗} & IIb & B \\
\bottomrule
\end{tabular}
\end{table}

\textbf{新增推荐}(标记为"NEW"):
\begin{itemize}
    \item \textbf{TAVI治疗AR}首次纳入指南(Class IIb)
    \item 前提条件:
    \begin{itemize}
        \item 患者有症状
        \item 心脏团队评估为手术不适宜
        \item 解剖结构适合TAVI
    \end{itemize}
    \item 这反映了AR专用瓣膜(如J-Valve)的临床经验积累
\end{itemize}

\subsection{主动脉瓣狭窄的TAVR适应证扩展}

\subsubsection{核心推荐的重大变化}

\textbf{关键变化:不再以手术风险评分为决策依据}

\begin{table}[h]
\centering
\caption{严重AS患者的干预方式推荐(ESC/EACTS 2025)}
\begin{tabular}{lcc}
\toprule
\textbf{推荐内容} & \textbf{推荐等级} & \textbf{证据级别} \\
\midrule
\makecell[l]{\textbf{≥70岁三尖瓣AS患者推荐TAVI,}\\如果解剖合适} & I & A \\
\midrule
\makecell[l]{\textbf{<70岁患者推荐SAVR,}\\如果手术风险低} & I & B \\
\midrule
\makecell[l]{所有其他主动脉生物瓣膜(BHV)候选者\\推荐SAVR或TAVI,\\根据心脏团队评估} & I & B \\
\bottomrule
\end{tabular}
\end{table}

\textbf{标记为"REV."(修订)的重要变化}:

\begin{enumerate}
    \item \textbf{年龄成为主要决策因素}:
    \begin{itemize}
        \item ≥70岁 → TAVI(Class I, Level A)
        \item <70岁 + 低手术风险 → SAVR(Class I, Level B)
        \item 说明中明确:"不考虑手术风险评分"(irrespective of the surgical risk score)
    \end{itemize}

    \item \textbf{所有其他候选者均为Class I推荐}:
    \begin{itemize}
        \item 不再区分高、中、低风险
        \item 强调心脏团队的个体化评估
        \item 包括临床、解剖和手术特征
        \item 纳入终身管理考虑和预期寿命
    \end{itemize}
\end{enumerate}

\subsubsection{决策考虑因素}

\textbf{影响SAVR vs TAVI选择的因素}:

\begin{table}[h]
\centering
\caption{SAVR与TAVI选择的考虑因素}
\begin{tabular}{lp{5cm}p{5cm}}
\toprule
\textbf{因素类别} & \textbf{倾向SAVR} & \textbf{倾向TAVI} \\
\midrule
\textbf{年龄} & <70岁 & ≥70岁 \\
\midrule
\textbf{解剖特征} & • 敌对的瓣环或LVOT钙化\newline • 二叶主动脉瓣\newline • 瓣环尺寸不适合TAVI\newline • 冠脉阻塞风险 & • 经股动脉入路适合\newline • 瓷主动脉\newline • 完整冠脉旁路移植物\newline • 严重胸廓畸形或脊柱侧弯 \\
\midrule
\textbf{伴随情况} & • 其他相关原发VHD\newline • 复杂CAD\newline • 主动脉根部或升主动脉动脉瘤\newline • 需行心肌切除的间隔肥厚 & • 合并症或增加手术风险的心脏情况\newline • 虚弱\newline • 胸部放疗后遗症 \\
\midrule
\textbf{终身管理} & \multicolumn{2}{p{11cm}}{预期重复手术选择和风险,选择方式和瓣膜类型时\newline
• Redo SAVR:再次手术风险\newline
• SAVR after TAVI:THV植入后手术风险增加\newline
• Valve-in-valve TAVI:冠脉阻塞风险,冠脉通路受损,假体-患者不匹配} \\
\bottomrule
\end{tabular}
\end{table}

\textbf{预期寿命的图示}:

指南提供了一个视觉化的生命预期曲线:
\begin{itemize}
    \item 横轴:年龄(60-70岁)
    \item 显示三种选择的耐久性限制:
    \begin{itemize}
        \item Ross手术(技术复杂性)
        \item 机械瓣(栓塞/出血风险)
        \item 生物假体SAVR或TAVI(有限耐久性)
    \end{itemize}
\end{itemize}

\subsubsection{特殊推荐}

\textbf{1. 二叶主动脉瓣(BAV)狭窄}(新增):

\begin{table}[h]
\centering
\caption{二叶瓣AS的TAVR推荐}
\begin{tabular}{lcc}
\toprule
\textbf{推荐内容} & \textbf{推荐等级} & \textbf{证据级别} \\
\midrule
\makecell[l]{对于手术风险增加的严重BAV狭窄患者,\\如果解剖合适,\\可考虑TAVI治疗} & IIb & B \\
\bottomrule
\end{tabular}
\end{table}

\textbf{标记为"NEW"的首次推荐}。

\textbf{2. 非经股动脉TAVR}(修订):

\begin{table}[h]
\centering
\caption{替代入路TAVR推荐}
\begin{tabular}{lcc}
\toprule
\textbf{推荐内容} & \textbf{推荐等级} & \textbf{证据级别} \\
\midrule
\makecell[l]{对于手术不适宜且\\经股动脉入路不可行的患者,\\应考虑非经股动脉TAVR} & IIa & B \\
\bottomrule
\end{tabular}
\end{table}

\textbf{标记为"REV."(从IIb升级到IIa)}。

\subsubsection{心脏瓣膜中心的推荐}

\begin{table}[h]
\centering
\caption{心脏瓣膜中心的要求}
\begin{tabular}{lcc}
\toprule
\textbf{推荐内容} & \textbf{推荐等级} & \textbf{证据级别} \\
\midrule
\makecell[l]{推荐在心脏瓣膜中心进行AV干预,\\该中心应:\\• 报告其本地专业知识和结局数据\\• 拥有现场介入心脏病学和心脏外科项目\\• 建立结构化的协作心脏团队} & I & C \\
\midrule
\makecell[l]{推荐干预方式基于心脏团队评估\\个体的临床、解剖和手术特征,\\\textbf{纳入终身管理考虑和预期寿命}} & I & C \\
\bottomrule
\end{tabular}
\end{table}

\subsection{无症状AS患者的早期治疗}

\subsubsection{2025指南的新推荐}

\begin{table}[h]
\centering
\caption{无症状AS患者的干预推荐(ESC/EACTS 2025)}
\begin{tabular}{lcc}
\toprule
\textbf{推荐内容} & \textbf{推荐} & \textbf{证据} \\
 & \textbf{等级} & \textbf{级别} \\
\midrule
\makecell[l]{无症状严重AS + LVEF <50\%\\(无其他原因)\\推荐干预} & I & B \\
\midrule
\makecell[l]{\textbf{无症状严重高梯度AS + LVEF ≥50\%}\\(通过正常运动试验确认,如果可行)\\\textbf{作为密切主动监测的替代方案,}\\应考虑干预,\\如果手术风险低} & \textbf{IIa} & \textbf{A} \\
\midrule
\multirow{5}{*}{\makecell[l]{无症状严重AS + LVEF ≥50\%\\如果手术风险低\\且存在以下参数之一,\\应考虑干预:}} & \multirow{5}{*}{IIa} & \multirow{5}{*}{B} \\
• 极重度AS(平均梯度≥60 mmHg或V\textsubscript{max} >5.0 m/s) & & \\
• 严重瓣膜钙化(理想由CCT评估)且V\textsubscript{max}进展≥0.3 m/s/年 & & \\
• BNP/NT-proBNP显著升高(年龄性别校正正常值3倍以上) & & \\
• LVEF <55\%(无其他原因) & & \\
\midrule
\makecell[l]{无症状严重AS\\运动试验中持续血压下降\\(>20 mmHg)\\应考虑干预} & IIa & C \\
\bottomrule
\end{tabular}
\end{table}

\textbf{最重要的新增推荐}(标记为"NEW"):

\begin{quote}
\textbf{Class IIa, Level A}推荐:对于无症状(通过正常运动试验确认)的严重高梯度AS + LVEF ≥50\%患者,作为密切主动监测的替代方案,应考虑干预,如果手术风险低。
\end{quote}

这是首次有A级证据支持无症状AS患者的早期干预。

\subsubsection{支持证据:随机对照试验}

\textbf{1. EARLY TAVR试验}

\begin{itemize}
    \item \textbf{发表}: Généreux et al. N Engl J Med 2025;392:217-27
    \item \textbf{样本量}: 901例患者
    \item \textbf{平均年龄}: 75.8岁
    \item \textbf{主要终点}: 死亡、卒中或心血管原因的非计划住院
    \item \textbf{结果}:
    \begin{itemize}
        \item HR 0.50 (95\% CI, 0.40–0.63); P<0.001
        \item 监测组26.2\%的患者在6个月内接受了瓣膜置换
        \item 5年时:TAVR组35.1\% vs 临床监测组51.2\%
    \end{itemize}
    \item \textbf{重要发现}: 早期干预显著改善结局
\end{itemize}

\textbf{2. EVOLVED试验}

\begin{itemize}
    \item \textbf{发表}: Loganathan et al. JAMA 2025;333(3):213-221
    \item \textbf{样本量}: 224例患者
    \item \textbf{平均年龄}: 73岁
    \item \textbf{主要终点}: AS相关住院
    \item \textbf{结果}:
    \begin{itemize}
        \item HR 0.79 (95\% CI, 0.44-1.43); P=0.44
        \item 未达到统计学显著性
        \item 5年时:早期干预组约23\% vs 保守治疗组约30\%
    \end{itemize}
\end{itemize}

\textbf{3. AVATAR长期随访}

\begin{itemize}
    \item \textbf{发表}: Banovic et al. Circulation. 2022;145:648–658
    \item \textbf{样本量}: 157例患者
    \item \textbf{平均年龄}: 67岁
    \item \textbf{主要终点}: 全因死亡
    \item \textbf{结果}:
    \begin{itemize}
        \item HR 0.44 (95\% CI 0.23 to 0.85); p=0.01
        \item 早期手术组死亡率显著降低
        \item 80个月时:早期手术组约20\% vs 保守治疗约38\%
    \end{itemize}
\end{itemize}

\textbf{4. RECOVERY试验}

\begin{itemize}
    \item \textbf{发表}: Kang et al. N Engl J Med 2020;382:111-9
    \item \textbf{样本量}: 145例患者
    \item \textbf{平均年龄}: 64岁
    \item \textbf{主要终点}: 手术死亡率或心血管原因死亡
    \item \textbf{结果}:
    \begin{itemize}
        \item P=0.003(Log-rank检验)
        \item P=0.003(Gray检验)
        \item 8年时:早期手术组约1\% vs 保守治疗约26\%
    \end{itemize}
\end{itemize}

\subsubsection{RCTs的局限性}

指南作者提出了对这些试验的批评性思考:

\begin{enumerate}
    \item \textbf{样本量和统计效力}:
    \begin{itemize}
        \item 部分试验样本量小且统计效力不足
        \item EVOLVED试验未达到统计学显著性
    \end{itemize}

    \item \textbf{人群选择偏倚}:
    \begin{itemize}
        \item 纳入的是\textbf{选定的年轻低风险患者}
        \item 均为\textbf{极重度主动脉瓣狭窄}
        \item 结果可能不适用于更广泛的患者群体
    \end{itemize}

    \item \textbf{保守治疗组的质量}:
    \begin{itemize}
        \item 对保守组的"密切监测"质量存疑
        \item EARLY TAVR中26.2\%的监测组患者在6个月内交叉接受治疗
        \item 真实世界的监测可能不如试验中严格
    \end{itemize}

    \item \textbf{终点定义问题}:
    \begin{itemize}
        \item EARLY TAVR将TAVR植入本身作为事件纳入复合终点
        \item 这可能夸大了早期干预的益处
    \end{itemize}
\end{enumerate}

\textbf{指南的态度}:

\begin{quote}
\textbf{Shared decision-making!}(共享决策!)
\end{quote}

尽管有RCT证据支持,指南强调:
\begin{itemize}
    \item 必须与\textbf{充分知情的患者}进行共享决策
    \item 需要考虑患者的个人偏好和目标
    \item 评估患者的合并症和虚弱状态
    \item 进行心脏团队评估
    \item 权衡早期干预和密切监测的利弊
\end{itemize}

\subsection{二叶主动脉瓣狭窄}

\subsubsection{TAVR在BAV中的挑战}

\textbf{BAV TAVR的潜在风险}:

\begin{itemize}
    \item \textbf{↑ 卒中风险}(Stroke)
    \item \textbf{↑ 瓣环破裂风险}(Annular rupture)
    \item \textbf{↑ 瓣周漏风险}(Paravalvular leak)
\end{itemize}

\subsubsection{2025指南推荐}

\begin{table}[h]
\centering
\caption{二叶瓣AS的TAVR推荐}
\begin{tabular}{lcc}
\toprule
\textbf{推荐内容} & \textbf{推荐等级} & \textbf{证据级别} \\
\midrule
\makecell[l]{对于手术风险增加的\\严重BAV狭窄患者,\\如果解剖合适,\\可考虑TAVI治疗} & IIb & B \\
\bottomrule
\end{tabular}
\end{table}

\textbf{这是新增推荐}(标记为"NEW"),首次为BAV TAVR提供指南支持。

\subsubsection{支持证据:NOTION II RCT}

虽然演讲中未提供NOTION II的详细数据,但图表显示:

\begin{itemize}
    \item \textbf{主要终点}: 死亡、卒中或再住院
    \item \textbf{随访}: 12个月
    \item \textbf{结果}:
    \begin{itemize}
        \item HR 3.8 (95\% CI, 0.8–18.5); P=0.07
        \item TAVR组风险有升高趋势,但未达统计学显著性
        \item 12个月时:TAVI组14.3\% vs 手术组3.9\%
    \end{itemize}
\end{itemize}

\textbf{解读}:
\begin{itemize}
    \item 证据尚不充分支持BAV TAVR的广泛应用
    \item 因此指南仅给予Class IIb推荐
    \item 强调"如果解剖合适"的重要性
    \item 需要更多随机对照研究
\end{itemize}

\subsection{临床实践要点(Take-home Messages)}

Fabien Praz医生总结的四大要点:

\begin{enumerate}
    \item \textbf{主动脉瓣反流}:
    \begin{quote}
    "Consider \textbf{volume} (ideally by CMR) and \textbf{modality of treatment} for aortic regurgitation (Surgery vs. TAVR)"

    考虑容量指标(理想情况下使用CMR)和AR的治疗方式(手术 vs. TAVR)
    \end{quote}
    \begin{itemize}
        \item 新增LVESVi >45 mL/m²作为手术指征
        \item TAVR治疗AR首次纳入指南(Class IIb)
    \end{itemize}

    \item \textbf{TAVR适应证扩展}:
    \begin{quote}
    "General trend of \textbf{expansion of TAVR indication} (be reasonable and consider the evidence!)"

    TAVR适应证扩展的总趋势(但要合理并考虑证据!)
    \end{quote}
    \begin{itemize}
        \item 年龄取代风险评分成为主要决策因素
        \item ≥70岁:TAVR(Class I, Level A)
        \item <70岁低风险:SAVR(Class I, Level B)
        \item 所有其他候选者:SAVR或TAVI(Class I, Level B)
        \item 强调终身管理和个体化决策
    \end{itemize}

    \item \textbf{无症状AS的早期干预}:
    \begin{quote}
    "\textbf{Wide open door} for earlier intervention in patients with \textbf{asymptomatic AS} (as a shared decision with the informed patient!)"

    为无症状AS患者的早期干预\textbf{打开了大门}(作为与充分知情患者的共享决策!)
    \end{quote}
    \begin{itemize}
        \item 首次有A级证据支持(Class IIa, Level A)
        \item 前提:严重高梯度AS、LVEF ≥50\%、低手术风险
        \item 必须通过正常运动试验确认无症状
        \item 作为密切主动监测的替代方案
        \item 强调共享决策的重要性
    \end{itemize}

    \item \textbf{二叶瓣AS}:
    \begin{quote}
    "Specific recommendation for \textbf{bicuspid aortic stenosis} (additional randomized evidence is needed)"

    二叶主动脉瓣狭窄的特定推荐(需要额外的随机对照证据)
    \end{quote}
    \begin{itemize}
        \item 首次纳入指南(Class IIb, Level B)
        \item 仅限于手术风险增加且解剖合适的患者
        \item 承认证据有限,需要更多RCT
    \end{itemize}
\end{enumerate}

\subsection{讨论与临床意义}

\subsubsection{指南更新的重大意义}

\textbf{1. 从风险评分到个体化评估的转变}

2025指南最显著的变化是决策框架的根本性转变:

\begin{itemize}
    \item \textbf{旧框架}:基于手术风险评分(STS/EuroSCORE)决定TAVR vs SAVR
    \item \textbf{新框架}:
    \begin{itemize}
        \item 年龄作为主要因素(70岁为分界点)
        \item 心脏团队综合评估个体特征
        \item 考虑终身管理策略
        \item 纳入预期寿命估计
    \end{itemize}
\end{itemize}

\textbf{理由}:
\begin{itemize}
    \item PARTNER 3和Evolut Low Risk试验证明TAVR在低风险患者中非劣效于SAVR
    \item 风险评分工具在预测个体结局方面存在局限
    \item 解剖特征、合并症和患者偏好比单一风险分数更重要
\end{itemize}

\textbf{2. 无症状AS早期干预的证据升级}

从多个小型RCT到Level A证据:

\begin{itemize}
    \item EARLY TAVR提供了最强证据(901例,HR 0.50)
    \item 但其他试验结果不一致(EVOLVED未达到显著性)
    \item 指南采取平衡态度:Class IIa而非Class I
    \item 强调共享决策而非强制性推荐
\end{itemize}

\textbf{临床影响}:
\begin{itemize}
    \item 可能改变AS的自然史认知
    \item 需要更严格的患者监测系统
    \item 增加了医疗资源的使用
    \item 对年轻患者的影响需要长期随访数据
\end{itemize}

\textbf{3. AR管理的完善}

容量指标和CMR的纳入:

\begin{itemize}
    \item LVESVi >45 mL/m²作为新的手术触发指标
    \item CMR的RF >40\%(vs 超声的>50\%)更敏感
    \item 强调多模态影像的综合应用
    \item TAVR治疗AR的首次认可(尽管是Class IIb)
\end{itemize}

\textbf{4. BAV的谨慎推荐}

尽管临床实践中BAV TAVR已较常见,指南仍保持谨慎:

\begin{itemize}
    \item 仅Class IIb推荐
    \item 强调"如果解剖合适"
    \item NOTION II数据显示风险趋势
    \item 需要更多专门针对BAV的随机对照研究
\end{itemize}

\subsubsection{临床实践的影响}

\textbf{1. 对TAVR项目的影响}:

\begin{itemize}
    \item TAVR适应证显著扩大
    \item 预期TAVR病例数将进一步增加
    \item 需要扩大TAVR项目的规模和能力
    \item 对年轻患者的长期随访变得更加重要
\end{itemize}

\textbf{2. 对心脏外科的影响}:

\begin{itemize}
    \item SAVR在<70岁低风险患者中仍是首选
    \item 强调心脏瓣膜中心需要完整的外科和介入能力
    \item 心脏团队的作用更加关键
    \item 外科医生在终身管理策略中的角色
\end{itemize}

\textbf{3. 对患者咨询的影响}:

\begin{itemize}
    \item 共享决策变得更加复杂
    \item 需要讨论更多的治疗选择
    \item 终身管理考虑需要详细解释
    \item 无症状患者的早期干预决策需要充分沟通
\end{itemize}

\textbf{4. 对随访监测的影响}:

\begin{itemize}
    \item 无症状AS患者需要更密切的监测
    \item 运动试验在评估中的作用增加
    \item CMR在AR患者中的应用增加
    \item 需要建立标准化的监测流程
\end{itemize}

\subsubsection{未解决的问题}

\textbf{1. 无症状AS早期干预的长期影响}:

\begin{itemize}
    \item 对年轻患者(60-70岁)的影响尚不清楚
    \item 早期干预是否会影响再次干预的时机和风险?
    \item 瓣膜耐久性的长期数据仍然有限
    \item 成本效益分析尚不充分
\end{itemize}

\textbf{2. BAV TAVR的适应证细化}:

\begin{itemize}
    \item 哪些BAV解剖类型最适合TAVR?
    \item 如何预测和预防并发症?
    \item 需要专门的BAV TAVR设备吗?
    \item 与三尖瓣AS相比的长期耐久性如何?
\end{itemize}

\textbf{3. AR的TAVR治疗}:

\begin{itemize}
    \item 哪些AR机制最适合TAVR?
    \item 当前设备的长期结果如何?
    \item 需要专用AR设备还是可以使用标准TAVR设备?
    \item 如何改进患者选择和手术规划?
\end{itemize}

\textbf{4. 70岁年龄分界点的合理性}:

\begin{itemize}
    \item 70岁是否适用于所有人群和种族?
    \item 生物年龄和虚弱程度是否比实际年龄更重要?
    \item 在预期寿命不同的国家是否应调整?
    \item 未来是否会根据新证据调整?
\end{itemize}

\subsection{研究局限性}

\subsubsection{演讲本身的局限性}

\begin{enumerate}
    \item \textbf{有限的细节}:
    \begin{itemize}
        \item 演讲PPT格式限制了详细讨论
        \item 许多推荐缺乏背后的详细理由
        \item 未提供完整的文献证据链
    \end{itemize}

    \item \textbf{选择性呈现}:
    \begin{itemize}
        \item 重点关注新增和修订的推荐
        \item 未涉及所有AS和AR管理方面
        \item 未讨论其他瓣膜疾病的更新
    \end{itemize}

    \item \textbf{利益冲突}:
    \begin{itemize}
        \item 演讲者与多家瓣膜公司有财务关系
        \item 虽然已披露和缓解,但可能影响呈现角度
    \end{itemize}
\end{enumerate}

\subsubsection{指南证据基础的局限性}

\begin{enumerate}
    \item \textbf{无症状AS试验}:
    \begin{itemize}
        \item 样本量相对较小
        \item 随访时间有限
        \item 高度选择的患者群体
        \item 结果不一致(EVOLVED阴性)
    \end{itemize}

    \item \textbf{BAV证据不足}:
    \begin{itemize}
        \item 主要基于观察性研究
        \item NOTION II结果不令人鼓舞
        \item 缺乏足够的随机对照证据
    \end{itemize}

    \item \textbf{AR的TAVR证据薄弱}:
    \begin{itemize}
        \item 主要来自专用设备的小型研究
        \item 长期随访数据缺乏
        \item 仅Class IIb推荐反映了证据的不确定性
    \end{itemize}

    \item \textbf{终身管理的不确定性}:
    \begin{itemize}
        \item TAVR耐久性数据仍然有限(最长15年)
        \item Valve-in-valve长期结果不明
        \item Redo TAVR的技术可行性和结局数据不足
    \end{itemize}
\end{enumerate}

\subsection{未来研究方向}

\subsubsection{迫切需要的研究}

\textbf{1. 长期耐久性研究}:

\begin{itemize}
    \item TAVR瓣膜20年以上的随访
    \item 不同瓣膜类型和世代的比较
    \item 结构性瓣膜退化的预测因素
    \item Valve-in-valve策略的长期结局
\end{itemize}

\textbf{2. 年轻患者的RCT}:

\begin{itemize}
    \item 55-70岁患者的TAVR vs SAVR研究
    \item 终身管理策略的比较
    \item 再次干预率和时机的评估
    \item 生活质量的长期比较
\end{itemize}

\textbf{3. BAV专门研究}:

\begin{itemize}
    \item 更大规模的BAV TAVR vs SAVR RCT
    \item 不同BAV类型的最佳治疗策略
    \item 并发症预测和预防的研究
    \item 专用BAV TAVR设备的开发和测试
\end{itemize}

\textbf{4. 无症状AS的长期随访}:

\begin{itemize}
    \item EARLY TAVR、EVOLVED等试验的延长随访
    \item 评估早期干预对再次干预需求的影响
    \item 成本效益分析
    \item 最佳监测策略的研究
\end{itemize}

\textbf{5. AR的TAVR研究}:

\begin{itemize}
    \item 多中心RCT比较TAVR vs 手术治疗AR
    \item 专用AR设备的进一步开发
    \item 标准TAVR设备治疗AR的优化
    \item 患者选择和手术规划的改进
\end{itemize}

\subsubsection{技术发展方向}

\textbf{1. 新一代TAVR设备}:

\begin{itemize}
    \item 改进的耐久性设计
    \item 更好的瓣周密封
    \item 适用于复杂解剖的设备
    \item 可回收和重新定位的设备
\end{itemize}

\textbf{2. 影像技术进步}:

\begin{itemize}
    \item 更精确的瓣膜尺寸选择工具
    \item 实时3D引导系统
    \item AI辅助的手术规划
    \item 更准确的耐久性预测模型
\end{itemize}

\textbf{3. 个性化医疗}:

\begin{itemize}
    \item 基于基因组学的耐久性预测
    \item 个体化的抗血栓治疗策略
    \item 精准的风险分层工具
    \item 生物标志物指导的治疗决策
\end{itemize}

\subsection{个人笔记}

\subsubsection{演讲的亮点}

\begin{enumerate}
    \item \textbf{清晰的结构}:
    \begin{itemize}
        \item 按疾病类型组织(AR、AS、BAV)
        \item 明确标注新增(NEW)和修订(REV.)内容
        \item 推荐表格简洁明了,包含推荐等级和证据级别
    \end{itemize}

    \item \textbf{平衡的观点}:
    \begin{itemize}
        \item 呈现了证据的优点和局限性
        \item 对无症状AS早期干预持谨慎态度
        \item 强调共享决策的重要性
        \item 承认BAV证据不足
    \end{itemize}

    \item \textbf{实用的总结}:
    \begin{itemize}
        \item Take-home Messages简洁有力
        \item 提供了具体的临床决策建议
        \item 强调"be reasonable and consider the evidence"
    \end{itemize}

    \item \textbf{前瞻性思考}:
    \begin{itemize}
        \item 讨论了终身管理策略
        \item 考虑了不同治疗选择的长期影响
        \item 预见了未来可能需要的证据
    \end{itemize}
\end{enumerate}

\subsubsection{与前两篇文献的联系}

本演讲与项目的前两篇文献形成了完整的AS管理图景:

\begin{enumerate}
    \item \textbf{第1篇}(Wayne Batchelor,健康不平等):
    \begin{itemize}
        \item 识别了AS治疗不足的问题
        \item 强调了系统性障碍
        \item 关注健康公平
    \end{itemize}

    \item \textbf{第2篇}(Sammy Elmariah,DETECT AS):
    \begin{itemize}
        \item 提供了改善AS治疗率的实用解决方案
        \item 证明了EPN系统的有效性
        \item 展示了减少健康不平等的可行途径
    \end{itemize}

    \item \textbf{本篇}(Fabien Praz,指南更新):
    \begin{itemize}
        \item 扩大了治疗适应证
        \item 支持更积极的治疗策略
        \item 为实施Target: AS和DETECT AS提供了指南基础
    \end{itemize}
\end{enumerate}

\textbf{综合意义}:

三篇文献共同描绘了AS管理的现状和未来:
\begin{itemize}
    \item \textbf{问题}:AS显著治疗不足,存在健康不平等
    \item \textbf{解决方案}:系统化的识别、通知和质量改进
    \item \textbf{指南支持}:扩大适应证,鼓励早期干预
\end{itemize}

如果系统性地实施,这三方面的进展可能显著改善AS患者的结局。

\subsubsection{对中国TAVR实践的启示}

\textbf{1. 指南的适用性}:

\begin{itemize}
    \item ESC指南在中国被广泛参考
    \item 但需要考虑中国特色:
    \begin{itemize}
        \item 患者更年轻(发病年龄较西方早)
        \item BAV比例可能更高
        \item 医疗资源分布不均
        \item 支付能力差异大
    \end{itemize}
\end{itemize}

\textbf{2. 70岁分界点的考虑}:

\begin{itemize}
    \item 中国人口预期寿命与欧洲相近
    \item 但"生理年龄"可能因生活方式和合并症而不同
    \item 需要中国自己的数据来验证这一分界点
\end{itemize}

\textbf{3. 无症状AS的管理}:

\begin{itemize}
    \item 早期干预的证据主要来自西方人群
    \item 中国患者的自然史可能不同
    \item 需要建立标准化的监测系统
    \item 医保支付政策需要相应调整
\end{itemize}

\textbf{4. BAV TAVR的机会}:

\begin{itemize}
    \item 中国可能有更多BAV患者
    \item 可以开展专门的BAV TAVR研究
    \item 为全球证据做出贡献
    \item 开发适合BAV的专用设备
\end{itemize}

\textbf{5. AR的TAVR治疗}:

\begin{itemize}
    \item 中国有J-Valve等专用AR设备的经验
    \item 可以为全球提供更多AR TAVR数据
    \item 这是中国在TAVR领域的潜在优势领域
\end{itemize}

\subsection{结论}

ESC/EACTS 2025主动脉瓣疾病管理指南代表了该领域的重大进展,主要体现在:

\begin{enumerate}
    \item \textbf{决策框架的转变}:从风险评分到年龄和个体化评估
    \item \textbf{适应证的扩展}:TAVR现在是≥70岁患者的首选(Class I, Level A)
    \item \textbf{早期干预的支持}:无症状AS患者首次有A级证据支持早期治疗
    \item \textbf{AR管理的完善}:引入容量指标,首次认可TAVR治疗AR
    \item \textbf{BAV的首次纳入}:尽管证据有限,但为BAV TAVR提供了指南基础
\end{enumerate}

\textbf{关键信息}:

\begin{itemize}
    \item TAVR适应证持续扩大,但需要"be reasonable and consider the evidence"
    \item 无症状AS的早期干预"wide open door",但需要共享决策
    \item 终身管理考虑在治疗选择中变得越来越重要
    \item 心脏团队评估和个体化决策是核心
    \item 需要更多研究来回答未解决的问题
\end{itemize}

\textbf{临床实践影响}:

指南将改变AS和AR的临床实践,预期带来:
\begin{itemize}
    \item TAVR病例数的进一步增加
    \item 更早期的干预时机
    \item 更复杂的患者咨询过程
    \item 更严格的随访监测要求
    \item 对长期结局数据的更大需求
\end{itemize}

医生需要与时俱进,在实施新推荐的同时,保持批判性思维,始终以患者的最佳利益为中心,进行充分知情的共享决策。

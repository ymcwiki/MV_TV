\section{当今主动脉瓣置换术:国际指南与当前趋势}

\subsection{文献信息}

\begin{itemize}
    \item \textbf{PDF文件名}: aortic-valve-replacement-today-international-guidelines-and-current-trends.pdf
    \item \textbf{演讲者}: Fabien Praz, MD
    \item \textbf{单位}: Bern University Hospital, Switzerland
    \item \textbf{会议}: CRF TCT (Transcatheter Cardiovascular Therapeutics)
    \item \textbf{主题}: 主动脉瓣置换的国际指南与当前趋势
    \item \textbf{页数}: 23页
\end{itemize}

\subsection{研究背景}

本演讲综述了TAVR领域的最新发展趋势和国际指南更新,重点关注三大趋势:
\begin{enumerate}
    \item TAVR适应证扩展与手术优化
    \item 无症状主动脉瓣狭窄患者的早期治疗
    \item 二叶主动脉瓣狭窄的TAVR治疗
\end{enumerate}

\subsection{主要内容}

\subsubsection{趋势1:TAVR适应证扩展与手术优化}

\paragraph{跨大西洋TAVR适应证比较}

\textbf{欧洲vs美国的实践差异}(Wang X et al. EuroIntervention 2024; Sharma et al. J Am Coll Cardiol 2022):
\begin{itemize}
    \item \textbf{欧洲(2021)}:18\%的65岁以下患者接受了TAVI
    \item \textbf{美国(2021)}:近50\%的65岁以下患者接受了TAVI
    \item 所有年龄组时间趋势P<0.01,显示TAVR应用持续增长
\end{itemize}

\paragraph{欧洲vs美国指南比较}(Coisne A et al. J Am Coll Cardiol. 2023;82(8):721-734)

\textbf{欧洲指南}(以年龄75岁为界):
\begin{itemize}
    \item \textbf{75岁以上}:
    \begin{itemize}
        \item SAVR:STS-Prom/EuroScore <4 (I-B)
        \item TAVR:STS-Prom/EuroScore >8或非SAVR候选者 (I-A)
    \end{itemize}
    \item \textbf{65-75岁}:
    \begin{itemize}
        \item SAVR:预期寿命>20年 (I-A)
        \item SAVR或TAVR:共同决策
    \end{itemize}
\end{itemize}

\textbf{美国指南}(以年龄65岁和80岁为界):
\begin{itemize}
    \item \textbf{80岁以上}:TAVR优先,如预期寿命<10年且有股动脉入路 (I-A)
\end{itemize}

\paragraph{严重AS患者的干预模式}(Praz F et al., Eur Heart J. 2025;ehaf194)

\textbf{最新欧洲指南推荐}(不考虑手术风险评分):

\begin{table}[h]
\centering
\begin{tabular}{lcc}
\hline
\textbf{推荐内容} & \textbf{类别} & \textbf{证据级别} \\
\hline
\textbf{年龄≥70岁},三叶瓣AS,解剖合适 & I & A \\
\textbf{年龄<70岁},手术风险低 & I & B \\
所有其他候选者,心脏团队评估 & I & B \\
\hline
\end{tabular}
\end{table}

\textbf{关键考虑因素}:
\begin{itemize}
    \item \textbf{年龄}:<70岁倾向SAVR,≥70岁倾向TAVR
    \item \textbf{解剖特征}:
    \begin{itemize}
        \item 倾向SAVR:敌对环或LVOT钙化、二叶瓣、瓣环尺寸不适合TAVR、冠脉阻塞风险
        \item 倾向TAVR:股动脉入路合适、瓷化主动脉、完整冠脉旁路移植物、严重胸畸形或脊柱侧弯
    \end{itemize}
    \item \textbf{伴随情况}:共病或心脏情况增加手术风险、虚弱、后遗症放射
    \item \textbf{终生管理}:预期再次手术风险
\end{itemize}

\paragraph{TAVR优化}

\textbf{瓣膜设计便于冠脉通路}(Angellotti D et al. JACC Adv. 2025;4(10 Pt 2):102181):

瓣膜设计的四个关键因素(COED):
\begin{itemize}
    \item \textbf{C - Coaxiality(同轴性)}:流入支架框架下铰链点与原生瓣尖的轴角
    \item \textbf{O - Orientation(对位)}:原生联合与THV联合的对齐
    \item \textbf{E - Expansion(扩张)}:THV支架框架相对装置标称直径的扩张
    \item \textbf{D - Depth(深度)}:THV支架框架底部与主动脉瓣尖的距离
\end{itemize}

\paragraph{DEDICATE RCT的优化经验}(Blankenberg S et al. N Engl J Med 2024;390:1572-1583)

\textbf{研究设计}:TAVI (N=701) vs SAVR (N=713)

\textbf{主要结果}(1年随访):
\begin{itemize}
    \item \textbf{主要终点}(死亡、卒中或心血管因素非计划住院):
    \begin{itemize}
        \item SAVR:10.0\%
        \item TAVI:5.4\%
        \item HR 0.53 (95\% CI 0.35-0.79); P<0.001
    \end{itemize}
    \item \textbf{大出血或威胁生命出血}:
    \begin{itemize}
        \item SAVR:17.2\%
        \item TAVI:4.3\%
        \item HR 0.24 (95\% CI 0.16-0.35)
    \end{itemize}
    \item \textbf{血管入路并发症}:
    \begin{itemize}
        \item SAVR:0.7\%
        \item TAVI:7.9\%
        \item HR 10.64 (95\% CI 4.84-28.94)
    \end{itemize}
\end{itemize}

\textbf{关键启示}:
\begin{itemize}
    \item TAVI的高度可重复性
    \item 需要手术标准化
    \item 干预应集中在有经验的心脏瓣膜中心
\end{itemize}

\paragraph{干预优化 - 指南建议}(Praz F et al., Eur Heart J. 2025;ehaf194)

\begin{itemize}
    \item 使用经长期耐久性证明的外科和经导管瓣膜
    \item 对有PPM风险的患者,考虑主动脉根部扩大的SAVR或植入环上经导管瓣膜
    \item 对未来TAV-in-SAV时有冠脉阻塞风险的患者,不植入无支架瓣膜或外置瓣叶瓣膜
    \item 预期未来可能干预的可行性和风险,考虑首次TAVI的相关技术方面(装置选择、裙边高度、联合对位、植入深度)
\end{itemize}

\subsubsection{趋势2:无症状AS患者的早期治疗}

\paragraph{AS干预时机}(Courtesy of Thomas Pilgrim, MD)

\textbf{治疗目标}:
\begin{itemize}
    \item 降低死亡率
    \item 减少住院
    \item 提高生活质量
    \item 终生管理
\end{itemize}

\textbf{监测风险}:
\begin{itemize}
    \item 不可逆心脏损害增加
    \item 即使AVR后死亡率仍增加
    \item 手术风险增加
\end{itemize}

\textbf{风险平衡}:"Sweet spot"(最佳时机)在早期干预和晚期干预之间

\paragraph{预防不可逆心脏损害}(Maznyczka A et al, JACC Cardiovasc Interv 2024;17(21):2502-2514)

\textbf{心脏损害进展}:
\begin{enumerate}
    \item \textbf{无心脏损害}:考虑早期AVR
    \item \textbf{左室肥厚}:心肌凋亡、弥漫纤维化、整体纵向应变下降
    \item \textbf{置换性纤维化}:MRI显示室壁晚期强化
    \item \textbf{LVEF下降/症状}:LVEF下降、左房扩大、MR≥中度、TR≥中度、肺动脉高压、右室功能障碍
\end{enumerate}

\textbf{关键发现}:心脏损害可能在症状出现前发生

\textbf{延迟干预的代价}:
\begin{itemize}
    \item 无心脏损害:基线
    \item 心脏损害1级:2倍死亡率
    \item 心脏损害2级:3倍死亡率
    \item 心脏损害3级:4倍死亡率
\end{itemize}

\paragraph{EVOLVED研究}(Loganath K et al, JAMA 2025;333(3):213-221)

\textbf{研究设计}:无症状严重AS伴心肌纤维化患者的早期AVR

\textbf{入组标准}:
\begin{itemize}
    \item N=224,平均年龄73岁,28\%女性
    \item 中位随访42个月
    \item 中位早期干预时间:5个月
\end{itemize}

\textbf{主要结果}(死亡或AS相关非计划住院):
\begin{itemize}
    \item HR 0.79 (95\% CI 0.44-1.43); P=0.44
    \item 早期干预组和保守治疗组曲线接近
\end{itemize}

\paragraph{EARLY TAVR研究}(Généreux et al. N Engl J Med 2025;392:217-27)

\textbf{研究设计}:901例患者,平均年龄75.8岁

\textbf{主要结果}:
\begin{itemize}
    \item \textbf{死亡、卒中或心血管因素非计划住院}:
    \begin{itemize}
        \item TAVR组:35.1\% (5年)
        \item 监测组:51.2\% (5年)
        \item HR 0.50 (95\% CI 0.40-0.63); P<0.001
    \end{itemize}
    \item 监测组26.2\%的患者在6个月内接受了瓣膜置换
\end{itemize}

\paragraph{AVATAR长期随访}(Banovic et al. Circulation. 2022;145:648–658)

\textbf{研究设计}:157例患者,平均年龄67岁

\textbf{主要结果}(全因死亡):
\begin{itemize}
    \item 早期手术组:20\% (80个月)
    \item 保守治疗组:~38\% (80个月)
    \item HR 0.44 (95\% CI 0.23-0.85); p=0.01
\end{itemize}

\paragraph{RECOVERY研究}(Kang et al. N Engl J Med 2020;382:111-9)

\textbf{研究设计}:145例患者,平均年龄64岁

\textbf{主要结果}(手术死亡或心血管原因死亡):
\begin{itemize}
    \item 早期手术组:1例 (8年)
    \item 保守治疗组:26例 (8年)
    \item P=0.003
\end{itemize}

\paragraph{无症状AS的干预 - 新指南推荐}(Praz F et al., Eur Heart J. 2025;ehaf194)

\begin{table}[h]
\small
\begin{tabular}{p{10cm}cc}
\hline
\textbf{推荐内容} & \textbf{类别} & \textbf{证据级别} \\
\hline
无症状严重AS且LVEF<50\%(无其他原因)推荐干预 & I & B \\
\hline
对于无症状(运动试验正常证实)严重高梯度AS且LVEF≥50\%的患者,如手术风险低,应考虑干预作为密切主动监测的替代方案 & IIa & \textbf{A (新)} \\
\hline
对于无症状严重AS且LVEF≥50\%的患者,如手术风险低且存在以下参数之一,应考虑干预: & & \\
• 极重度AS(平均梯度≥60 mmHg或Vmax >5.0 m/s) & & \\
• 严重瓣膜钙化(理想通过CCT评估)且Vmax进展≥0.3 m/s/年 & IIa & B \\
• BNP/NT-proBNP显著升高(正常范围3倍以上,重复测量确认,无其他解释) & & \\
• LVEF<55\%(无其他原因) & & \\
\hline
对于无症状严重AS且运动试验期间血压持续下降(>20 mmHg)的患者,应考虑干预 & IIa & C \\
\hline
\end{tabular}
\end{table}

\paragraph{无症状AS随机对照试验的局限性}(Praz F et al., Eur Heart J. 2025;ehaf194)

\begin{itemize}
    \item 部分研究样本量小且检验效能不足
    \item 纳入选择性人群:年轻、低风险、极重度主动脉狭窄患者
    \item 保守治疗组的密切监测质量如何?
    \item EARLY-TAVR研究将TAVR植入作为事件
\end{itemize}

\textbf{结论}:需要共同决策!

\subsubsection{趋势3:二叶主动脉瓣狭窄}

\paragraph{二叶瓣TAVR的潜在风险}

\begin{itemize}
    \item 卒中风险增加
    \item 瓣环破裂风险增加
    \item 瓣周漏风险增加
\end{itemize}

\paragraph{NOTION II RCT}(Praz F et al., Eur Heart J. 2025;ehaf194)

\textbf{主要结果}(死亡、卒中或再住院):
\begin{itemize}
    \item TAVI组:14.3\% (12个月)
    \item 外科组:3.9\% (12个月)
    \item HR 3.8 (95\% CI 0.8-18.5); P=0.07
\end{itemize}

\paragraph{二叶瓣AS新指南推荐}(Praz F et al., Eur Heart J. 2025;ehaf194)

\begin{table}[h]
\centering
\begin{tabular}{p{10cm}cc}
\hline
\textbf{推荐内容} & \textbf{类别} & \textbf{证据级别} \\
\hline
对于手术风险增加的严重二叶瓣AS患者,如解剖合适,可考虑TAVI & IIb & \textbf{B (新)} \\
\hline
\end{tabular}
\end{table}

\subsection{核心要点总结}

\begin{enumerate}
    \item \textbf{TAVR适应证扩展的总体趋势}(需合理并考虑证据!)

    \item \textbf{AVR优化}对经导管和外科干预都至关重要

    \item \textbf{应鼓励早期干预}(作为与知情患者的共同决策!)
    \begin{itemize}
        \item 新指南推荐:无症状严重高梯度AS且LVEF≥50\%的患者,如手术风险低,应考虑干预作为密切主动监测的替代方案(IIa-A)
        \item 多项RCT(EARLY TAVR、AVATAR、RECOVERY)显示早期干预可改善预后
        \item 心脏损害可能在症状出现前发生,延迟干预可导致死亡率成倍增加
    \end{itemize}

    \item \textbf{二叶主动脉瓣狭窄}需要更多(随机对照)证据
    \begin{itemize}
        \item 新指南:对于手术风险增加的严重二叶瓣AS患者,如解剖合适,可考虑TAVI(IIb-B)
        \item NOTION II RCT显示TAVI组事件率较高,但未达统计学显著差异
    \end{itemize}
\end{enumerate}

\subsection{临床意义}

\subsubsection{欧美指南差异}

\begin{itemize}
    \item 美国对年轻患者(<65岁)TAVR应用更积极
    \item 欧洲更注重年龄分层和手术风险评估
    \item 两者都强调心脏团队评估和个体化决策
\end{itemize}

\subsubsection{干预优化策略}

\begin{itemize}
    \item 选择经长期耐久性证明的瓣膜
    \item 预防PPM:主动脉根部扩大或环上瓣膜
    \item 预防未来冠脉阻塞:避免无支架或外置瓣叶瓣膜
    \item 考虑未来再干预:装置选择、植入技术
\end{itemize}

\subsubsection{无症状AS的管理范式转变}

传统观点:等待症状出现

新证据支持:
\begin{itemize}
    \item 对特定高危亚组(LVEF<50\%、极重度AS、快速进展、BNP显著升高)早期干预
    \item 密切监测的替代方案
    \item 强调共同决策和患者教育
\end{itemize}

\subsection{研究局限性}

\begin{enumerate}
    \item \textbf{无症状AS的RCT}:
    \begin{itemize}
        \item 样本量较小,检验效能不足
        \item 入组人群选择性强(年轻、低风险)
        \item 保守治疗组监测质量可能影响结果
        \item EARLY-TAVR将TAVR植入作为终点事件存在争议
    \end{itemize}

    \item \textbf{二叶瓣AS}:
    \begin{itemize}
        \item NOTION II样本量小,统计效能不足
        \item 需要更大规模RCT验证
        \item 解剖异质性大,难以统一推荐
    \end{itemize}

    \item \textbf{长期随访数据}:
    \begin{itemize}
        \item 年轻患者接受TAVR的超长期(>10年)结果尚不明确
        \item 瓣膜耐久性数据仍在积累
    \end{itemize}
\end{enumerate}

\subsection{未来方向}

\begin{enumerate}
    \item 更大规模、更长随访的RCT评估无症状AS早期干预
    \item 二叶瓣AS的TAVR技术和装置优化
    \item 个体化风险分层工具开发
    \item 瓣膜耐久性长期数据积累
    \item 优化术后管理和监测策略
\end{enumerate}

\subsection{个人笔记}

\subsubsection{指南更新亮点}

\begin{itemize}
    \item \textbf{不再强调手术风险评分}:新指南更注重年龄、解剖、伴随情况、终生管理
    \item \textbf{无症状AS证据级别提升}:从IIa-B提升至IIa-A
    \item \textbf{二叶瓣AS首次纳入}:IIb-B推荐
\end{itemize}

\subsubsection{实践启示}

\begin{enumerate}
    \item \textbf{心脏团队评估至关重要}:
    \begin{itemize}
        \item 综合考虑年龄、解剖、风险、预期寿命
        \item 不能仅依赖手术风险评分
    \end{itemize}

    \item \textbf{无症状AS患者需要密切监测}:
    \begin{itemize}
        \item 定期超声评估(梯度进展)
        \item BNP/NT-proBNP监测
        \item 运动试验(如可行)
        \item 考虑CMR评估纤维化
    \end{itemize}

    \item \textbf{共同决策}:
    \begin{itemize}
        \item 充分告知患者早期干预vs监测的利弊
        \item 考虑患者偏好、生活方式、职业
        \item 讨论长期规划(可能的再干预)
    \end{itemize}

    \item \textbf{手术优化}:
    \begin{itemize}
        \item 选择合适装置(考虑冠脉通路、未来干预)
        \item 标准化流程
        \item 集中在有经验的中心
    \end{itemize}
\end{enumerate}

\subsubsection{关键数据记忆}

\begin{itemize}
    \item DEDICATE:TAVI vs SAVR主要终点5.4\% vs 10.0\% (HR 0.53, P<0.001)
    \item EARLY TAVR:早期TAVR vs监测35.1\% vs 51.2\% (HR 0.50, P<0.001)
    \item EVOLVED:早期干预vs保守治疗HR 0.79 (P=0.44,阴性结果)
    \item NOTION II:TAVI vs手术14.3\% vs 3.9\% (HR 3.8, P=0.07)
    \item 心脏损害分级:0级→1级2倍死亡率→2级3倍→3级4倍
\end{itemize}

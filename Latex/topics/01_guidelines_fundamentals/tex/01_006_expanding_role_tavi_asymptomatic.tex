\section{无症状严重主动脉瓣狭窄的TAVI应用拓展}

\subsection{文献信息}
\begin{itemize}
    \item \textbf{标题}: Expanding Role of TAVI to Asymptomatic Severe Aortic Stenosis
    \item \textbf{作者}: Philippe Généreux, MD, FACC
    \item \textbf{机构}: Gagnon Cardiovascular Institute, Morristown Medical Center, NJ
    \item \textbf{来源}: 会议演讲资料
    \item \textbf{关键词}: 无症状主动脉瓣狭窄,TAVR,EARLY TAVR研究,急性瓣膜综合征
\end{itemize}

\subsection{研究背景}
无症状严重主动脉瓣狭窄(Asymptomatic Severe Aortic Stenosis)患者的最佳治疗时机一直存在争议。传统观点主张等待症状出现后再行主动脉瓣置换术(AVR),但近年来多项研究探索早期干预的潜在获益。

\subsection{EARLY TAVR研究}

\subsubsection{研究设计}
EARLY TAVR是一项前瞻性、多中心随机对照试验,评估无症状严重AS患者的早期TAVR治疗策略:

\textbf{纳入标准:}
\begin{itemize}
    \item 年龄≥65岁
    \item 无症状严重主动脉瓣狭窄
    \item STS评分≤10\%
    \item LVEF≥50\%
    \item 通过负荷平板运动试验确认无症状状态*
\end{itemize}

\textbf{*注:}若患者因骨科或神经系统原因无法完成运动试验,则仅通过详细的临床病史确认

\textbf{随机分组(1:1):}
\begin{itemize}
    \item \textbf{TAVR组}(N=455):经股动脉TAVR(SAPIEN 3或SAPIEN 3 Ultra瓣膜)
    \item \textbf{临床监测组}(N=446):定期临床和超声心动图随访
\end{itemize}

\textbf{主要终点:}全因死亡、任何卒中或非计划心血管住院的非分层复合终点,最少随访2年

\textbf{研究时间:}2017年3月至2021年12月入组,中位随访3.8年

\subsubsection{患者筛查与排除}
在1578例筛查患者中:
\begin{itemize}
    \item 677例(42.9\%)被排除:
    \begin{itemize}
        \item 313例(约20\%)有I类AVR适应证(被排除)
        \item 277例有症状性严重AS
        \item 213例解剖学排除标准
        \item 265例(约30\%)有IIa/IIb类AVR适应证
    \end{itemize}
    \item 最终901例患者1:1随机分配
\end{itemize}

\subsubsection{基线特征}
\textbf{人口学特征:}
\begin{itemize}
    \item 平均年龄:TAVR组76.0±6.0岁,监测组75.6±6.0岁
    \item 女性:TAVR组28.8\%,监测组33.0\%
    \item STS评分:TAVR组1.8±1.0\%,监测组1.7±1.0\%
    \item 低风险(心脏团队评估):两组均约83.5\%
\end{itemize}

\textbf{无症状确认:}
\begin{itemize}
    \item 平板运动试验:约90\%患者
    \item 仅临床病史:约10\%患者(无法完成运动试验)
    \item KCCQ评分:两组均为92.7分
    \item NT-proBNP中位数:TAVR组276 pg/mL,监测组297 pg/mL
\end{itemize}

\textbf{超声心动图基线:}
\begin{itemize}
    \item 主动脉瓣口面积(AVA):TAVR组0.9±0.2 cm²,监测组0.8±0.2 cm²
    \item 峰值流速:TAVR组4.3±0.5 m/s,监测组4.4±0.4 m/s
    \item 平均跨瓣压差:TAVR组46.5±10.1 mmHg,监测组47.3±10.6 mmHg
    \item LVEF:两组均为67.4\%
    \item ≥II级左室舒张功能不全:TAVR组42.7\%,监测组37.3\%
\end{itemize}

\subsection{主要研究结果}

\subsubsection{主要终点}
\textbf{全因死亡、卒中或非计划心血管住院复合终点:}
\begin{itemize}
    \item TAVR组:17.7\%(2年)→ 35.1\%(5年)
    \item 临床监测组:37.5\%(2年)→ 51.2\%(5年)
    \item HR 0.50 (95\% CI: 0.40-0.63), p<0.001
    \item \textbf{NNT约为6}(2年时需治疗人数)
\end{itemize}

\subsubsection{次要终点}

\textbf{1. 全因死亡:}
\begin{itemize}
    \item 2年:TAVR组2.7\% vs 监测组3.6\%
    \item 5年:TAVR组13.4\% vs 监测组13.6\%
    \item HR 0.93 (95\% CI: 0.60-1.44), p=0.74
    \item \textbf{结论:}无显著差异
\end{itemize}

\textbf{2. 卒中:}
\begin{itemize}
    \item 2年:TAVR组2.7\% vs 监测组3.7\%
    \item 5年:TAVR组5.4\% vs 监测组9.5\%
    \item HR 0.62 (95\% CI: 0.35-1.10), p=0.10
    \item \textbf{结论:}趋势降低但未达统计学显著性
\end{itemize}

\textbf{3. 非计划心血管住院:}
\begin{itemize}
    \item 2年:TAVR组15.1\% vs 监测组36.3\%
    \item 5年:TAVR组26.3\% vs 监测组46.7\%
    \item HR 0.43 (95\% CI: 0.33-0.55), p<0.001
    \item \textbf{结论:}TAVR组显著降低57\%
\end{itemize}

\textbf{4. 死亡、卒中或心衰住院:}
\begin{itemize}
    \item 2年:TAVR组6.5\% vs 监测组12.5\%
    \item 5年:TAVR组20.2\% vs 监测组27.8\%
    \item HR 0.60 (95\% CI: 0.44-0.83), p=0.002
\end{itemize}

\subsubsection{探索性分析}
\textbf{死亡、卒中、非计划心血管住院或伴有晚期征象/症状的AVR干预:}
\begin{itemize}
    \item 2年:TAVR组18.0\% vs 监测组34.3\%
    \item 5年:TAVR组35.6\% vs 监测组55.4\%
    \item HR 0.49 (95\% CI: 0.40-0.62), p<0.001
\end{itemize}

\subsection{临床监测组转化为AVR的情况}

\subsubsection{转化率和时间}
\begin{itemize}
    \item \textbf{中位转化时间:}11.1个月
    \item \textbf{累积转化率:}
    \begin{itemize}
        \item 6个月:26.2\%
        \item 1年:47.2\%
        \item 2年:71.4\%
        \item 3年:86.1\%
        \item 4年:90.4\%
        \item 5年:95.2\%
    \end{itemize}
    \item 研究结束时,仍有30例患者在监测中未转化为AVR
    \item 87.9\%的转化患者在出现AVR适应证后3个月内接受了治疗
\end{itemize}

\subsubsection{与历史队列对比}
EARLY TAVR研究的转化率几乎与Otto等人1997年的前瞻性研究(N=123)完全一致,证实了无症状严重AS的自然病程:
\begin{itemize}
    \item 2年:约70\%出现症状或死亡
    \item 3年:约75-86\%
    \item 5年:约95\%
\end{itemize}

\subsubsection{转化时的症状}
在377例因症状转化为AVR的患者中:

\textbf{最常见症状(可多选):}
\begin{itemize}
    \item 呼吸困难:83.0\%
    \item 心绞痛:24.9\%
    \item 头晕:24.7\%
    \item 疲劳:22.0\%
    \item 晕厥:7.2\%
\end{itemize}

\textbf{多重症状:}
\begin{itemize}
    \item 2种症状:34.5\%
    \item ≥3种症状:13.3\%
\end{itemize}

\textbf{症状/心衰严重程度:}
\begin{itemize}
    \item NYHA II级:70.0\%
    \item NYHA III/IV级:30.0\%
\end{itemize}

\textbf{伴随AS恶化征象:}
\begin{itemize}
    \item 峰值流速>5 m/s:22.3\%
    \item LVEF降至<50\%:4.8\%
    \item NT-proBNP较基线增加≥3倍:6.7\%
\end{itemize}

\subsection{临床表现新分类系统}

研究团队提出了一个新的AS临床表现分类系统(发表于Circulation 2025),用于描述转化为AVR时的急性程度和严重程度:

\subsubsection{稳定瓣膜综合征(Stable Valve Syndrome, SVS)}
\textbf{定义:}包括因需要其他医疗程序而转化为AVR的患者(真正无症状)

\subsubsection{进展性瓣膜综合征(Progressive Valve Syndrome, PVS)}
\textbf{定义标准(满足一项即可):}
\begin{itemize}
    \item NYHA II级
    \item 心衰药物治疗较基线增加
    \item NT-proBNP较基线和年龄特异性阈值*增加1.5-3倍
\end{itemize}

\subsubsection{急性瓣膜综合征(Acute Valve Syndrome, AVS)}
\textbf{定义标准(满足一项即可):}
\begin{itemize}
    \item NYHA III/IV级
    \item 晕厥
    \item 心房颤动
    \item 室性心律失常
    \item 心搏骤停复苏后
    \item 心衰和/或肺水肿住院
    \item LVEF降至<50\%
    \item NT-proBNP较基线和年龄特异性阈值*增加≥3倍
\end{itemize}

\textbf{*年龄特异性阈值:}≤75岁为125 pg/mL,>75岁为450 pg/mL

\subsubsection{转化患者的临床表现分布}
在388例转化为AVR的监测组患者中:
\begin{itemize}
    \item \textbf{无症状(SVS):}2.3\%
    \item \textbf{进展性征象/症状(PVS):}58.5\%
    \item \textbf{急性瓣膜综合征(AVS):}39.2\%
\end{itemize}

\textbf{重要发现:}晚期征象/症状(AVS)的比例在不同转化时间点保持一致(约34-57\%),表明延迟治疗持续带来急性失代偿的风险。

\subsection{围术期安全性}

\subsubsection{围术期结果对比(≤30天)}
\begin{table}[h]
\centering
\caption{围术期结果(Kaplan-Meier估计)}
\begin{tabular}{lcc}
\hline
\textbf{结果} & \textbf{TAVR组 (N=444)} & \textbf{监测转AVR (N=388)} \\
\hline
全因死亡 & 0.2\% & 0\% \\
心血管死亡 & 0\% & 0\% \\
卒中 & 0.9\% & 1.8\% \\
\ \ 致残性卒中 & 0\% & 1.0\% \\
新发心房颤动 & 4.5\% & 3.1\% \\
新植入永久起搏器 & 5.7\% & 8.4\% \\
大出血或威胁生命出血 & 2.5\% & 3.6\% \\
急性肾损伤 & 2.5\% & 3.4\% \\
主要血管并发症 & 1.4\% & 1.0\% \\
心肌梗死 & 0.5\% & 0.5\% \\
冠脉阻塞需干预 & 0\% & 0\% \\
\hline
\end{tabular}
\end{table}

\textbf{结论:}两组围术期安全性相似,无显著差异

\subsubsection{治疗及时性}
\begin{itemize}
    \item \textbf{TAVR组:}从随机分配到TAVR的中位时间为14天(IQR: 9-24天)
    \item \textbf{监测转AVR组:}从出现AVR适应证到转化的中位时间为32天(IQR: 18-58天)
    \item 98.2\%接受TAVR,1.8\%接受SAVR
\end{itemize}

\subsection{四项RCT的Meta分析}

Généreux等人(JACC 2025年3月)对四项无症状严重AS的RCT进行了系统回顾和Meta分析,包括EARLY TAVR、EVOLVED、AVATAR和RECOVERY研究。

\subsubsection{汇总结果}
\begin{table}[h]
\centering
\caption{Meta分析结果(早期AVR vs 临床监测)}
\begin{tabular}{lccc}
\hline
\textbf{终点} & \textbf{汇总HR (95\% CI)} & \textbf{P值} & \textbf{获益方向} \\
\hline
全因死亡 & 0.68 (0.40-1.17) & 0.17 & -- \\
心血管死亡 & 0.67 (0.35-1.29) & 0.23 & -- \\
心衰住院 & 0.28 (0.17-0.47) & <0.01 & 早期AVR \\
非计划CV/HF住院 & 0.40 (0.30-0.53) & <0.01 & 早期AVR \\
卒中 & 0.62 (0.40-0.97) & 0.03 & 早期AVR \\
\hline
\end{tabular}
\end{table}

\textbf{关键结论:}
\begin{itemize}
    \item 早期AVR显著降低心衰住院(72\%)
    \item 显著降低非计划心血管/心衰住院(60\%)
    \item 显著降低卒中(38\%)
    \item 全因死亡和心血管死亡趋势降低但未达统计学显著性
\end{itemize}

\subsection{急性瓣膜综合征的临床影响}

基于egnite数据库的两项大型真实世界研究(Structural Heart 2025和J Am Heart Assoc 2025)评估了AVR前临床表现对术后结果的影响。

\subsubsection{研究队列(Structural Heart研究)}
\begin{itemize}
    \item \textbf{总人群:}17,838例接受AVR的中重度AS患者
    \item \textbf{临床表现分布:}
    \begin{itemize}
        \item 无症状(Asymptomatic):2,504例(14.0\%)
        \item 进展性瓣膜综合征(PVS):6,116例(34.3\%)
        \item 急性瓣膜综合征(AVS):9,218例(51.7\%)
    \end{itemize}
    \item TAVR占78.6\%,SAVR占21.4\%
    \item 平均年龄:76.5±9.7岁,40.2\%为女性
\end{itemize}

\subsubsection{AVR后2年结果}
\textbf{1. 死亡率:}
\begin{itemize}
    \item 无症状(SVS):5.8\%
    \item 进展性(PVS):7.6\%
    \item 急性(AVS):17.5\%
    \item Log-rank p<0.001
    \item \textbf{AVS的校正HR:}2.2 (95\% CI: 1.8-2.6)
\end{itemize}

\textbf{2. 心衰住院:}
\begin{itemize}
    \item 无症状(SVS):11.1\%
    \item 进展性(PVS):19.0\%
    \item 急性(AVS):41.5\%
    \item Gray's test p<0.001
    \item \textbf{AVS的校正HR:}3.3 (95\% CI: 2.9-3.8)
    \item \textbf{PVS的校正HR:}1.5 (95\% CI: 1.3-1.8)
\end{itemize}

\textbf{3. 卒中/TIA:}
\begin{itemize}
    \item 1年:SVS 4.2\%,PVS 4.4\%,AVS 6.5\%
    \item Gray's test p<0.001
\end{itemize}

\subsubsection{医疗费用和资源利用(J Am Heart Assoc研究)}
基于24,075例AVR患者的分析(SVS 270例,PVS 10,195例,AVS 13,610例):

\textbf{总费用(AVR + 1年随访):}
\begin{itemize}
    \item SVS:\$146,309
    \item PVS:\$173,719(比SVS增加\$27,410,19\%,p<0.001)
    \item AVS:\$182,576(比SVS增加\$36,267,25\%,p<0.001)
\end{itemize}

\textbf{AVR住院时间:}
\begin{itemize}
    \item SVS:6.5天
    \item PVS:7.3天(增加13\%,p=0.013)
    \item AVS:8.7天(增加33\%,p<0.001)
\end{itemize}

\textbf{1年随访期间总住院次数:}
\begin{itemize}
    \item SVS:0.8次
    \item PVS:1.0次(增加29\%,p=0.019)
    \item AVS:1.2次(增加56\%,p<0.001)
\end{itemize}

\textbf{1年随访期间心衰住院次数:}
\begin{itemize}
    \item SVS:0.045次
    \item PVS:0.147次(增加117\%,p=0.018)
    \item AVS:0.219次(增加386\%,p<0.001)
\end{itemize}

\textbf{关键结论:}AVS和PVS在AVR前与术后更高的总费用、更长的住院时间、更多的全因和心衰再住院相关。

\subsection{2025 ESC/EACTS指南推荐}

2025年欧洲心脏病学会/欧洲心胸外科学会(ESC/EACTS)瓣膜性心脏病管理指南纳入了EARLY TAVR和相关Meta分析的证据。

\subsubsection{无症状严重AS患者的推荐}
\textbf{推荐1(I类,B级):}
\begin{itemize}
    \item 推荐对LVEF<50\%的无症状严重AS患者进行干预,且无其他原因
\end{itemize}

\textbf{推荐2(IIa类,A级):}
\begin{itemize}
    \item 对于无症状患者(通过正常运动试验确认,如可行),如果手术风险低,应考虑对严重高梯度AS且LVEF≥50\%的患者进行干预,\textbf{作为密切临床监测的替代方案}
\end{itemize}

\textbf{推荐3(IIa类,B级):}
\begin{itemize}
    \item 对于无症状严重AS且LVEF≥50\%的患者,如果手术风险低且存在以下参数之一,应考虑进行干预:
    \begin{itemize}
        \item 极重度AS(平均梯度≥60 mmHg或Vmax>5.0 m/s)
        \item 严重瓣膜钙化(理想情况下通过CT评估)且Vmax进展≥0.3 m/s/年
        \item BNP/NT-proBNP显著升高(超过年龄和性别校正正常范围的3倍以上,在重复测量中确认且无其他解释)
        \item LVEF<55\%且无其他原因
    \end{itemize}
\end{itemize}

\textbf{推荐4(IIa类,C级):}
\begin{itemize}
    \item 对于运动试验期间血压持续下降(>20 mmHg)的无症状严重AS患者,应考虑进行干预
\end{itemize}

\textbf{参考文献:}
\begin{itemize}
    \item 360. Généreux P等,EARLY TAVR研究(N Engl J Med 2024)
    \item 368. Généreux P等,Meta分析(J Am Coll Cardiol 2024)
\end{itemize}

\subsection{临床实践建议}

基于EARLY TAVR研究和相关证据,Généreux教授提出以下建议:

\subsubsection{给患者的建议}
\begin{itemize}
    \item 早期转诊(严重AS无症状或中度AS)
    \item 完成AS治疗的全面评估:
    \begin{itemize}
        \item 术前规划的CT扫描
        \item 牙科工作
        \item 提前计划AVR日期以避免AVS
    \end{itemize}
    \item 如果等待症状出现,应在症状出现后3个月内治疗,以降低较长等待期的死亡风险
    \item 早期干预无不良影响
    \item 及时和迅速的AVR是关键
\end{itemize}

\subsubsection{给转诊医生的建议}
\begin{itemize}
    \item 早期转诊以确保充分和预防性的术前规划
    \item 及时和迅速的AVR将节省心衰住院、卒中和AVS
    \item 及时治疗将节省心脏损害和医疗保健系统的费用
\end{itemize}

\subsection{未来展望:PROGRESS研究}

\subsubsection{研究设计}
\textbf{PROGRESS试验}旨在进一步扩展TAVR适应证至中度AS患者:

\textbf{纳入标准:}
\begin{itemize}
    \item 中度主动脉瓣狭窄伴症状或心脏损害/功能障碍
    \item 解剖适合经股动脉入路
\end{itemize}

\textbf{随机分组(1:1,750例患者):}
\begin{itemize}
    \item TAVR组(SAPIEN 3平台)
    \item 临床监测组(延迟主动脉瓣置换允许,针对出现严重AS的患者)
\end{itemize}

\textbf{主要终点:}
\begin{itemize}
    \item 全因死亡和非计划心血管住院(2年)
\end{itemize}

\textbf{随访:}每年随访,持续10年

\subsection{关键结论}

\subsubsection{EARLY TAVR研究的主要发现}
\begin{enumerate}
    \item 与临床监测相比,对无症状严重AS患者进行早期TAVR可显著降低死亡、卒中或非计划心血管住院的复合终点(HR 0.50,NNT=6)
    \item 主要获益来自于非计划心血管住院的显著减少(57\%相对降低)
    \item 全因死亡无显著差异,但复合终点和心衰住院的改善具有临床重要性
    \item 早期TAVR的安全性与等待症状后的TAVR相似
    \item 约70\%的监测患者在2年内出现症状并接受AVR,与历史自然病程一致
\end{enumerate}

\subsubsection{临床表现新分类的重要性}
\begin{enumerate}
    \item 急性瓣膜综合征(AVS)在接受AVR的AS患者中占半数以上(52-57\%)
    \item AVS是AVR后2年死亡率(HR 2.2)和心衰住院(HR 3.3)的强预测因子
    \item AVS患者的医疗费用显著增加(比无症状患者多\$36,267)
    \item 等待症状>3个月与更高的AVS风险和更差的预后相关
\end{enumerate}

\subsubsection{对AS管理未来的启示}
\begin{enumerate}
    \item \textbf{筛查}将成为检测显著(中度、重度)AS的关键
    \item \textbf{早期转诊}以确保充分和预防性的术前规划
    \item \textbf{及时和迅速的AVR}是关键;不要等待症状>3个月
    \item \textbf{终身管理}在规划首次干预时很重要
    \item \textbf{预防心脏损害}成为AS治疗的新范式
\end{enumerate}

\subsection{研究局限性}

\begin{enumerate}
    \item 主要终点为复合终点,全因死亡单独分析无显著差异
    \item 研究人群主要为低风险患者(STS评分≤10\%),高风险患者的结果可能不同
    \item 排除了大量有I类AVR适应证的患者(约20\%),实际临床实践中这些患者可能已接受治疗
    \item 中位随访3.8年,更长期的随访对于评估瓣膜耐久性很重要
    \item 主要使用SAPIEN平台,其他TAVR系统的结果可能有所不同
    \item 约30\%的患者有IIa/IIb类适应证被纳入,这可能影响结果的解释
\end{enumerate}

\subsection{临床意义}

EARLY TAVR研究及相关证据为无症状严重AS患者的管理提供了重要的循证医学证据,支持在选定的低风险患者中考虑早期TAVR作为密切监测的替代方案。这一证据已被纳入2025 ESC/EACTS指南(IIa类,A级推荐),代表了AS管理范式的重要转变——从"等待症状"到"预防心脏损害"。

急性瓣膜综合征(AVS)的概念强调了及时干预的重要性,因为延迟治疗不仅增加急性失代偿的风险,还显著增加术后并发症和医疗费用。未来的PROGRESS研究将进一步探索TAVR在中度AS患者中的作用,可能进一步扩展TAVR的适应证。

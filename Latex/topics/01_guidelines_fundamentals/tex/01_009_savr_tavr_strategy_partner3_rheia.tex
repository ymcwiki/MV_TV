\section{SAVR vs. TAVR治疗策略研究:PARTNER 3与RHEIA试验}

\subsection{文献信息}
\begin{itemize}
    \item \textbf{标题}: SAVR vs. TAVR Strategy Trials (PARTNER 3 and RHEIA): Therapeutic Strategies in Small Annulus AS
    \item \textbf{作者}: Marvin H. Eng, MD
    \item \textbf{机构}: University of Nebraska Medical Center, Vincent Miscia Chair and Section Chief of Interventional Cardiology, Structural Heart and Catheterization Lab Director
    \item \textbf{会议}: TCT (Transcatheter Cardiovascular Therapeutics)
    \item \textbf{主题}: 小瓣环主动脉瓣狭窄的治疗策略
\end{itemize}

\subsection{研究背景:女性患者的特殊挑战}

\subsubsection{早期随机试验荟萃分析}
\textit{Siontis GCM et al. EHJ 2016;37:3503-12}

\paragraph{纳入研究}
\begin{itemize}
    \item PARTNER 1A: TAVI 116/348 vs SAVR 114/351
    \item US CoreValve: TAVI 85/391 vs SAVR 99/359
    \item NOTION: TAVI 11/145 vs SAVR 14/135
    \item PARTNER 2A: TAVI 166/1011 vs SAVR 170/1021
\end{itemize}

\paragraph{总体结果}
\begin{itemize}
    \item \textbf{全因死亡}: HR 0.87 (95\% CI: 0.76-0.99), P = 0.038
    \item TAVI相比SAVR显著降低女性患者死亡率13\%
    \item 异质性检验:$\tau^2$ < 0.001, P = 0.755
\end{itemize}

\subsubsection{小瓣环AS的临床问题}
\begin{enumerate}
    \item \textbf{患者人群特征}
    \begin{itemize}
        \item 以女性为主(约60-65\%)
        \item 体型较小
        \item 主动脉瓣环直径较小
    \end{itemize}

    \item \textbf{SAVR面临的挑战}
    \begin{itemize}
        \item 瓣膜-患者不匹配(PPM)高发
        \item 术后残余压差较高
        \item 需要考虑瓣环扩大术(AE)
    \end{itemize}

    \item \textbf{临床意义}
    \begin{itemize}
        \item 影响生活质量
        \item 增加再住院风险
        \item 影响长期预后
    \end{itemize}
\end{enumerate}

\subsection{PARTNER 3研究:1年结果}

\subsubsection{性别分层分析}
\textit{Mack MJ et al. NEJM 2019;380:1695-705}

\paragraph{主要终点(死亡、卒中或再住院)}

\begin{table}[h]
\centering
\begin{tabular}{lcccc}
\hline
\textbf{亚组} & \textbf{患者数} & \textbf{TAVR} & \textbf{SAVR} & \textbf{绝对差异(95\% CI)} \\
\hline
\textbf{女性} & 292 & 13/161 (8.1\%) & 24/131 (18.5\%) & \textbf{-10.4\% (-18.3至-2.5)} \\
\textbf{男性} & 658 & 29/335 (8.7\%) & 44/323 (13.8\%) & -5.1\% (-9.9至-0.3) \\
\hline
\end{tabular}
\caption{PARTNER 3按性别分层的主要终点结果}
\end{table}

\paragraph{关键发现}
\begin{itemize}
    \item \textbf{女性获益更显著}:绝对风险降低10.4\%
    \item 男性也有获益,但幅度较小(5.1\%)
    \item 性别交互作用P值 = 0.27(无显著交互)
    \item 其他亚组分析结果一致:
    \begin{itemize}
        \item STS-PROM评分分层
        \item LVEF分层
        \item NYHA分级
        \item 房颤状态
        \item KCCQ评分
    \end{itemize}
\end{itemize}

\subsection{RHEIA试验:小瓣环AS的专门研究}

\subsubsection{研究设计}
\textit{Tchetche D et al. EHJ 2025;46:2079-2088}

\paragraph{纳入与随机化}
\begin{itemize}
    \item \textbf{筛查患者}: N=490
    \item \textbf{随机分配}: N=443
    \begin{itemize}
        \item TAVI组: N=221
        \item 外科组: N=222
    \end{itemize}
    \item \textbf{实际治疗人群}:
    \begin{itemize}
        \item TAVI: N=215
        \item Surgery: N=205
    \end{itemize}
    \item \textbf{最终植入瓣膜}:
    \begin{itemize}
        \item TAVI: N=212
        \item Surgery: N=203
    \end{itemize}
\end{itemize}

\paragraph{交叉情况}
\begin{itemize}
    \item TAVI转SAVR: 3例
    \item SAVR转TAVI: 2例
    \item 拒绝治疗并转换: TAVI组1例,SAVR组4例
\end{itemize}

\subsubsection{植入瓣膜尺寸分布}

\begin{table}[h]
\centering
\begin{tabular}{cc}
\hline
\textbf{瓣膜尺寸} & \textbf{比例} \\
\hline
19 mm & 10.9\% \\
21 mm & 34.3\% \\
23 mm & 42.8\% \\
25 mm & 10.9\% \\
27 mm & 1.0\% \\
\hline
\end{tabular}
\caption{RHEIA试验瓣膜尺寸分布(小瓣环为主)}
\end{table}

\paragraph{重要观察}
\begin{itemize}
    \item 19-21mm瓣膜占45.2\%(真正的小瓣环)
    \item 23mm瓣膜占42.8\%(中等偏小)
    \item $\geq$ 25mm仅占11.9\%
\end{itemize}

\subsection{RHEIA血流动力学结果}

\subsubsection{平均压差变化}

\paragraph{随访各时点平均压差(mmHg)}
\begin{itemize}
    \item \textbf{基线}:
    \begin{itemize}
        \item SAVR: 47.5 (N=142)
        \item TAVI: 47.8 (N=157)
        \item P = 0.68
    \end{itemize}

    \item \textbf{30天}:
    \begin{itemize}
        \item SAVR: 10.9 (N=171)
        \item TAVI: 13.6 (N=185)
        \item P < 0.001
    \end{itemize}

    \item \textbf{1年}:
    \begin{itemize}
        \item SAVR: 11.7 (N=174)
        \item TAVI: 14.3 (N=192)
        \item P < 0.001
    \end{itemize}
\end{itemize}

\subsubsection{有效瓣口面积(EOA)变化}

\paragraph{随访各时点EOA (cm\textsuperscript{2})}
\begin{itemize}
    \item \textbf{基线}:
    \begin{itemize}
        \item SAVR: 0.8 (N=132)
        \item TAVI: 0.8 (N=153)
        \item P = 0.11
    \end{itemize}

    \item \textbf{30天}:
    \begin{itemize}
        \item SAVR: 1.9 (N=163)
        \item TAVI: 1.8 (N=172)
        \item P = 0.01
    \end{itemize}

    \item \textbf{1年}:
    \begin{itemize}
        \item SAVR: 1.9 (N=159)
        \item TAVI: 1.7 (N=176)
        \item P = 0.003
    \end{itemize}
\end{itemize}

\subsubsection{血流动力学结论}
\begin{itemize}
    \item \textbf{SAVR血流动力学优势}:
    \begin{itemize}
        \item 平均压差低2-3 mmHg
        \item EOA大约0.2 cm\textsuperscript{2}
        \item 差异具有统计学意义
    \end{itemize}
    \item \textbf{但临床结局TAVR更优}(见后文)
    \item 提示:血流动力学参数非唯一预后决定因素
\end{itemize}

\subsection{RHEIA临床结局}

\subsubsection{30天超声心动图结果}

\paragraph{患者-假体不匹配(PPM)}
\begin{itemize}
    \item \textbf{总体PPM}: P = 0.14(无显著差异)
    \item \textbf{重度PPM}: P = 1.0(无差异)
    \item \textbf{压差$\geq$ 20 mmHg}:
    \begin{itemize}
        \item TAVR: 10.8\%
        \item SAVR: 2.9\%
        \item P = 0.004
    \end{itemize}
\end{itemize}

\paragraph{瓣周漏(PVL)}
\begin{itemize}
    \item \textbf{30天}:
    \begin{itemize}
        \item TAVI: 轻度14.5\%,中度0.6\%
        \item Surgery: 轻度2.8\%,中度0\%
        \item 轻度或中度PVL: P < 0.001
    \end{itemize}

    \item \textbf{1年}:
    \begin{itemize}
        \item TAVI: 轻度15.5\%,中度1.1\%
        \item Surgery: 轻度2.4\%,中度0\%
        \item 轻度或中度PVL: P < 0.001
    \end{itemize}
\end{itemize}

\subsubsection{主要终点:死亡、卒中或再住院}

\paragraph{1年Kaplan-Meier分析}
\begin{itemize}
    \item \textbf{复合终点}:
    \begin{itemize}
        \item TAVR: 8.9\%
        \item Surgery: 15.6\%
        \item \textbf{HR 0.55 [0.31-0.96]}
        \item TAVR显著优于SAVR
    \end{itemize}

    \item \textbf{死亡}(插图右上):
    \begin{itemize}
        \item TAVR: 约9\%
        \item Surgery: 约15\%
        \item P = NS(仅趋势)
    \end{itemize}
\end{itemize}

\subsubsection{终点成分分析}

\begin{table}[h]
\centering
\begin{tabular}{lccc}
\hline
\textbf{终点} & \textbf{TAVR} & \textbf{Surgery} & \textbf{统计学意义} \\
\hline
\textbf{死亡} & $\sim$2\% & $\sim$2\% & P = NS \\
\textbf{卒中} & $\sim$3\% & $\sim$3\% & P = NS \\
\textbf{再住院} & $\sim$5\% & $\sim$12\% & \textbf{HR 0.4 [0.18-0.81]} \\
\hline
\end{tabular}
\caption{RHEIA 1年终点成分分析}
\end{table}

\paragraph{关键发现}
\begin{itemize}
    \item \textbf{主要终点获益由再住院驱动}
    \item 死亡和卒中无显著差异
    \item 再住院绝对风险降低约7\%
    \item 提示TAVR主要改善中期恢复和功能状态
\end{itemize}

\subsubsection{TAVR获益的时间分布}

\paragraph{早期 vs 晚期获益}
\begin{itemize}
    \item \textbf{30天}:
    \begin{itemize}
        \item TAVR: 4.2\%
        \item Surgery: 7.4\%
        \item HR 0.56 [0.24-1.27], P = 0.2
    \end{itemize}

    \item \textbf{3个月}:
    \begin{itemize}
        \item TAVR: 约3.7\%
        \item Surgery: 约4.9\%
        \item HR 0.75 [0.30-1.90], P = 0.5
    \end{itemize}

    \item \textbf{关键观察}:
    \begin{itemize}
        \item 两组生存曲线主要在\textbf{早期分离}(0-3个月)
        \item 3个月后曲线基本平行
        \item 提示TAVR的主要获益在围手术期和早期恢复阶段
    \end{itemize}
\end{itemize}

\subsubsection{次要终点和安全性}

\begin{table}[h]
\centering
\begin{tabular}{lccc}
\hline
\textbf{终点} & \textbf{TAVR} & \textbf{SAVR} & \textbf{95\% CI} \\
\hline
主要血管并发症 & 3.3\% & 0.5\% & [0.2-5.3] \\
大出血或危及生命出血 & 6.0\% & 10.7\% & [-10.0至-0.6] \\
急性肾损伤(II/III期) & 0.9\% & 2.9\% & [-4.6至-0.6] \\
新发起搏器 & 8.8\% & 2.9\% & [1.5-10.4] \\
新发房颤 & 3.3\% & 28.8\% & [-32.2至-18.8] \\
出院回家 & 90.2\% & 49.8\% & [32.6-48.4] \\
\hline
\end{tabular}
\caption{RHEIA次要终点和安全性结果}
\end{table}

\paragraph{安全性总结}
\begin{itemize}
    \item \textbf{TAVR优势}:
    \begin{itemize}
        \item 显著降低新发房颤(3.3\% vs 28.8\%)
        \item 显著提高直接出院回家率(90.2\% vs 49.8\%)
        \item 降低大出血风险
        \item 降低急性肾损伤
    \end{itemize}

    \item \textbf{SAVR优势}:
    \begin{itemize}
        \item 主要血管并发症更少
        \item 起搏器植入率更低
    \end{itemize}

    \item \textbf{TAVR特有风险}:
    \begin{itemize}
        \item 血管并发症率3.3\%(可接受)
        \item 起搏器率8.8\%(现代瓣膜系统合理范围)
    \end{itemize}
\end{itemize}

\subsection{PARTNER 3与RHEIA联合分析}

\subsubsection{汇总研究设计}
\textit{Eltchaninoff et al. JACC Cardiovasc Interv 2025;18:1540-1553}

\begin{itemize}
    \item \textbf{总样本量}: 712例
    \begin{itemize}
        \item TAVR: N=376
        \item SAVR: N=336
    \end{itemize}
    \item \textbf{纳入研究}: PARTNER 3亚组 + RHEIA全部患者
    \item \textbf{共同特征}: 小瓣环AS患者
\end{itemize}

\subsubsection{主要复合终点}

\paragraph{1年全因死亡、卒中或再住院}
\begin{itemize}
    \item \textbf{TAVR}: 8.5\% (277/376存活)
    \item \textbf{SAVR}: 16.8\% (220/336存活)
    \item \textbf{Kaplan-Meier差异}: -8.2\%
    \item \textbf{95\% CI}: -13.1\%至-3.3\%
    \item \textbf{P值}: < 0.001
\end{itemize}

\subsubsection{终点成分详细分析}

\begin{table}[h]
\centering
\begin{tabular}{lccc}
\hline
\textbf{终点} & \textbf{TAVR} & \textbf{SAVR} & \textbf{P值} \\
\hline
\textbf{死亡} & 1.1\% & 2.1\% & P = 0.27 (NS) \\
KM差异 & \multicolumn{2}{c}{-1.0\%, 95\% CI: -2.9\%至0.8\%} & \\
\hline
\textbf{卒中} & 1.3\% & 2.4\% & P = 0.35 (NS) \\
KM差异 & \multicolumn{2}{c}{-1.2\%, 95\% CI: -3.9\%至1.4\%} & \\
\hline
\textbf{再住院} & 5.4\% & 11.9\% & \textbf{P = 0.002} \\
KM差异 & \multicolumn{2}{c}{\textbf{-6.5\%, 95\% CI: -10.7\%至-2.3\%}} & \\
\hline
\end{tabular}
\caption{联合分析1年终点成分结果}
\end{table}

\paragraph{关键结论}
\begin{itemize}
    \item \textbf{再住院是主要获益驱动因素}
    \item 死亡和卒中无统计学差异但趋势有利于TAVR
    \item 再住院绝对风险降低6.5\%(NNT=15)
    \item 结果与RHEIA单独分析一致
\end{itemize}

\subsection{SAVR患者再住院的预测因素}

\subsubsection{PARTNER 2和3联合分析:2年预测模型}
\textit{Carter-Storch et al. JACC:Advances 2024;3:100853}

\paragraph{全人群预测因素}
\begin{itemize}
    \item 年龄(每年): HR 1.04 [1.02-1.07]
    \item 房颤: HR 2.06 [1.58-2.72]
    \item COPD: HR 1.47 [1.10-1.97]
    \item LVEF(每\%): HR 0.98 [0.96-1.00]
    \item 平均压差(每mmHg): HR 0.99 [0.98-1.00]
\end{itemize}

\paragraph{男性亚组}
\begin{itemize}
    \item 年龄(每年): HR 1.05 [1.02-1.08]
    \item 房颤: HR 2.45 [1.72-3.49]
    \item COPD: HR 1.58 [1.10-2.28]
    \item 平均压差(每mmHg): HR 0.98 [0.96-1.00]
\end{itemize}

\paragraph{女性亚组}
\begin{itemize}
    \item \textbf{房颤: HR 1.67 [1.07-2.60]}
    \item 其他因素未达到统计学显著性
\end{itemize}

\subsubsection{女性SAVR患者的特殊风险:低流量}

\paragraph{低流量对女性SAVR预后的影响}
\begin{itemize}
    \item \textbf{SAVR组}:
    \begin{itemize}
        \item 低流量: 24.5\%(2年全因死亡或心衰再住院)
        \item 正常流量: 15.8\%
        \item \textbf{HR 1.65 [1.04-2.60]}, P = 0.0307
    \end{itemize}

    \item \textbf{TAVR组}:
    \begin{itemize}
        \item 低流量: 13.2\%
        \item 正常流量: 14.4\%
        \item HR 0.92 [0.52-1.63], P = 0.7767
    \end{itemize}
\end{itemize}

\paragraph{临床意义}
\begin{itemize}
    \item 低流量在女性SAVR患者中是强预后因素
    \item TAVR消除了低流量的负面影响
    \item 可能机制:
    \begin{itemize}
        \item 避免体外循环对心肌的额外损伤
        \item 更好的瓣环-瓣膜匹配
        \item 减少术后房颤
    \end{itemize}
\end{itemize}

\subsection{瓣膜-动脉阻抗(Valvulo-Arterial Impedance)}

\subsubsection{概念与计算}
\textit{Hahn RT et al. JACC Cardiovasc Imaging 2025;18:625-640}

\paragraph{定义}
\begin{itemize}
    \item $Z_{va} = \frac{MG + SBP}{SVI}$
    \item MG: 平均压差(mmHg)
    \item SBP: 收缩压(mmHg)
    \item SVI: 每搏输出量指数(ml/m\textsuperscript{2})
    \item 单位: mm Hg/mL/m\textsuperscript{2}
\end{itemize}

\paragraph{生理意义}
\begin{itemize}
    \item 反映左室射血的总阻力
    \item 结合瓣膜阻抗和血管阻抗
    \item 影响左室后负荷
    \item 与心室-动脉耦合相关
\end{itemize}

\subsubsection{PARTNER 3的5年Z\textsubscript{va}数据}

\paragraph{Z\textsubscript{va}随时间变化}
\begin{itemize}
    \item \textbf{基线}:
    \begin{itemize}
        \item Surgery: 4.76
        \item TAVR: 4.72
    \end{itemize}

    \item \textbf{1个月}:
    \begin{itemize}
        \item Surgery: 3.92
        \item TAVR: 3.73
    \end{itemize}

    \item \textbf{12个月至60个月}:
    \begin{itemize}
        \item Surgery: 约3.76-3.83
        \item TAVR: 约3.51-3.68
        \item \textbf{5年比较P = 0.007}
    \end{itemize}
\end{itemize}

\subsubsection{Z\textsubscript{va}与预后关系}

\paragraph{5年死亡、卒中或再住院}
\begin{itemize}
    \item \textbf{高Z\textsubscript{va}组} ($\geq$ 4 mm Hg/mL/m\textsuperscript{2}): 27.7\%
    \item \textbf{低Z\textsubscript{va}组} (< 4 mm Hg/mL/m\textsuperscript{2}): 21.8\%
    \item \textbf{5年HR [95\% CI] = 1.34 [1.02-1.76]}
    \item \textbf{P = 0.038}
\end{itemize}

\paragraph{Z\textsubscript{va}与治疗方式}
\begin{itemize}
    \item SAVR患者Z\textsubscript{va}持续较高
    \item 可能原因:
    \begin{itemize}
        \item 瓣膜支架结构占据更多空间
        \item 相对较小的有效瓣口面积
        \item 术后血管顺应性变化
    \end{itemize}
    \item 高Z\textsubscript{va}预测更差的长期预后
\end{itemize}

\subsection{右心功能与RV-PA耦合}

\subsubsection{RV-PA耦合的定义与测量}
\textit{Hahn RT et al. JACC Cardiovasc Imaging 2025;18:625-640}

\paragraph{测量指标}
\begin{itemize}
    \item \textbf{TAPSE/PASP比值}
    \item TAPSE: 三尖瓣环收缩期位移(mm)
    \item PASP: 肺动脉收缩压(mmHg)
    \item \textbf{良好耦合}: TAPSE/PASP $\geq$ 0.50 mm/mmHg
    \item \textbf{受损耦合}: TAPSE/PASP < 0.50 mm/mmHg
\end{itemize}

\subsubsection{PARTNER 3中RV-PA耦合受损的发生率}

\paragraph{TAPSE/PASP < 0.50的患者比例}
\begin{itemize}
    \item \textbf{基线}:
    \begin{itemize}
        \item TAVR: 26.3\%
        \item Surgery: 29.3\%
        \item P = 0.454
    \end{itemize}

    \item \textbf{30天}:
    \begin{itemize}
        \item TAVR: 25.3\%
        \item Surgery: 72.2\%
        \item P < 0.001
    \end{itemize}

    \item \textbf{1年}:
    \begin{itemize}
        \item TAVR: 19.7\%
        \item Surgery: 54.5\%
        \item P < 0.001
    \end{itemize}

    \item \textbf{5年}:
    \begin{itemize}
        \item TAVR: 33.9\%
        \item Surgery: 59.1\%
        \item P < 0.001
    \end{itemize}
\end{itemize}

\subsubsection{RV-PA耦合与生存}

\paragraph{5年死亡、卒中或再住院}
\begin{itemize}
    \item \textbf{受损耦合组} (T/P < 0.50): 25.7\%
    \item \textbf{良好耦合组} (T/P $\geq$ 0.50): 18.9\%
    \item \textbf{5年HR [95\% CI] = 1.45 [1.03-2.06]}
    \item \textbf{P = 0.033}
\end{itemize}

\paragraph{临床意义}
\begin{itemize}
    \item SAVR后RV-PA耦合受损率显著增加
    \item 可能机制:
    \begin{itemize}
        \item 体外循环对右心功能的影响
        \item 术后肺动脉压力变化
        \item 术后房颤导致的右心负荷增加
        \item 术后三尖瓣反流加重
    \end{itemize}
    \item 受损的RV-PA耦合是独立预后因素
    \item TAVR避免了这些不利影响
\end{itemize}

\subsection{PARTNER 3的5年血流动力学稳定性}

\subsubsection{平均压差和AVA的长期趋势}

\paragraph{平均压差 (mmHg)}
\begin{itemize}
    \item \textbf{基线}:
    \begin{itemize}
        \item Surgery: 49.4
        \item TAVR: 48.3
    \end{itemize}

    \item \textbf{1个月}:
    \begin{itemize}
        \item Surgery: 11.2
        \item TAVR: 12.7
    \end{itemize}

    \item \textbf{5年}:
    \begin{itemize}
        \item Surgery: 11.7
        \item TAVR: 12.8
        \item \textbf{5年比较P < 0.001}
    \end{itemize}
\end{itemize}

\paragraph{有效瓣口面积 (cm\textsuperscript{2})}
\begin{itemize}
    \item \textbf{基线}:
    \begin{itemize}
        \item Surgery: 0.77
        \item TAVR: 0.77
    \end{itemize}

    \item \textbf{1个月}:
    \begin{itemize}
        \item Surgery: 1.74
        \item TAVR: 1.79
    \end{itemize}

    \item \textbf{5年}:
    \begin{itemize}
        \item Surgery: 1.82
        \item TAVR: 1.87
        \item \textbf{5年比较P = 0.895}
    \end{itemize}
\end{itemize}

\paragraph{关键观察}
\begin{itemize}
    \item 两种瓣膜5年血流动力学\textbf{高度稳定}
    \item 压差保持在11-13 mmHg范围
    \item AVA保持在1.7-1.9 cm\textsuperscript{2}范围
    \item 无结构性瓣膜退化(SVD)的证据
    \item SAVR仍有轻微血流动力学优势,但无临床意义
\end{itemize}

\subsection{瓣环扩大术(Annular Enlargement)的局限性}

\subsubsection{STS数据库分析}
\textit{Hawkins et al. Ann Thorac Surg 2019;108:1752-60}

\paragraph{瓣环扩大术的应用}
\begin{itemize}
    \item \textbf{总体使用率}: 约5\%(NET rate)
    \item 适应症:预计严重PPM
    \item 常用技术:
    \begin{itemize}
        \item Manouguian技术
        \item Nicks技术
        \item 瓣环后扩大
    \end{itemize}
\end{itemize}

\paragraph{瓣环扩大术的效果与风险}

\begin{table}[h]
\centering
\begin{tabular}{lccc}
\hline
\textbf{结局} & \textbf{AE组} & \textbf{无AE组} & \textbf{P值} \\
\hline
重度PPM & $\sim$5\% & $\sim$2\% & P = 0.024 \\
手术死亡率 & $\sim$5\% & $\sim$2.5\% & P = 0.059 \\
主要并发症 & $\sim$18\% & $\sim$14\% & P = 0.016 \\
\hline
\end{tabular}
\end{table}

\paragraph{多变量分析结果}
\begin{itemize}
    \item \textbf{手术死亡率}: OR 2.0 (P = 0.016)
    \item \textbf{主要并发症}: OR 1.4 (P = 0.016)
    \item 瓣环扩大术增加手术风险
    \item 对PPM的预防效果有限
\end{itemize}

\subsubsection{临床决策考量}
\begin{itemize}
    \item 瓣环扩大术\textbf{增加手术风险}
    \item 对改善预后的证据不足
    \item 在小瓣环患者中,TAVR可能是更好选择
    \item 避免了瓣环扩大的需求和相关风险
\end{itemize}

\subsection{再住院的驱动因素总结}

\subsubsection{SAVR女性患者再住院的主要因素}

\begin{enumerate}
    \item \textbf{新发房颤}
    \begin{itemize}
        \item SAVR: 28.8\% vs TAVR: 3.3\%(RHEIA)
        \item 女性预测因素:HR 1.67 [1.07-2.60]
        \item 可能机制:
        \begin{itemize}
            \item 心房直接手术创伤
            \item 心包炎症反应
            \item 体外循环相关损伤
            \item 围手术期电解质紊乱
        \end{itemize}
        \item 后续影响:
        \begin{itemize}
            \item 症状性心律失常
            \item 心衰恶化
            \item 卒中风险增加
            \item 抗凝治疗需求
        \end{itemize}
    \end{itemize}

    \item \textbf{低流量状态}
    \begin{itemize}
        \item 女性SAVR中低流量HR: 1.65 [1.04-2.60]
        \item 反映:
        \begin{itemize}
            \item 慢性心肌损伤
            \item 右心功能障碍
            \item 潜在的小左室腔
        \end{itemize}
        \item SAVR可能加重:
        \begin{itemize}
            \item 体外循环对已受损心肌的额外打击
            \item 术后瓣膜-患者不匹配
            \item 残余高压差增加心肌负荷
        \end{itemize}
    \end{itemize}

    \item \textbf{瓣膜-动脉阻抗增加}
    \begin{itemize}
        \item SAVR的Z\textsubscript{va}持续高于TAVR
        \item 代表:
        \begin{itemize}
            \item 瓣膜阻抗 + 血管系统阻抗的总和
            \item 左室射血的总阻力
        \end{itemize}
        \item 高Z\textsubscript{va}后果:
        \begin{itemize}
            \item 左室后负荷增加
            \item 心肌耗氧量增加
            \item 左室重构延迟或不完全
            \item 5年HR 1.34 [1.02-1.76]
        \end{itemize}
    \end{itemize}

    \item \textbf{受损的RV-PA耦合}
    \begin{itemize}
        \item SAVR后受损率显著高于TAVR
        \item 反映:
        \begin{itemize}
            \item 右心功能障碍
            \item 肺血管阻力异常
            \item 肺动脉高压
        \end{itemize}
        \item 机制:
        \begin{itemize}
            \item 体外循环对右心的直接影响
            \item 术后三尖瓣反流加重
            \item 房颤导致的右房压力升高
            \item 左室舒张功能异常传导至肺循环
        \end{itemize}
        \item 受损RV-PA耦合:5年HR 1.45 [1.03-2.06]
    \end{itemize}

    \item \textbf{其他因素}
    \begin{itemize}
        \item 年龄(每年HR 1.04-1.05)
        \item COPD(HR 1.47-1.58)
        \item 延长的恢复期和康复时间
        \item 出院至康复机构而非回家
    \end{itemize}
\end{enumerate}

\subsubsection{TAVR如何避免这些风险}

\begin{itemize}
    \item \textbf{避免体外循环}:
    \begin{itemize}
        \item 减少全身炎症反应
        \item 避免心肌缺血-再灌注损伤
        \item 保护右心功能
        \item 减少神经认知功能障碍
    \end{itemize}

    \item \textbf{减少房颤发生}:
    \begin{itemize}
        \item 无心房直接创伤
        \item 无心包切开
        \item 围手术期应激更小
    \end{itemize}

    \item \textbf{更快恢复}:
    \begin{itemize}
        \item 无胸骨切开
        \item 住院时间更短
        \item 直接出院回家率更高(90\% vs 50\%)
        \item 更早恢复日常活动
    \end{itemize}

    \item \textbf{血流动力学优化}:
    \begin{itemize}
        \item 虽然压差稍高,但可接受
        \item 低Z\textsubscript{va}
        \item 更好的RV-PA耦合保存
    \end{itemize}
\end{itemize}

\subsection{患者-假体不匹配与性别差异}

\subsubsection{STS数据库:女性与S-PPM}
\textit{Aalaei-Andabili SH et al. Innovations 2019;14:243-250}\\
\textit{Nam et al. Ann Thorac Surg 2025;120:478-86}

\paragraph{SAVR中S-PPM的特征}
\begin{itemize}
    \item \textbf{女性占S-PPM的64\%}
    \item \textbf{总患者}: N=82 (17.7\%)
    \item \textbf{瓣膜尺寸$\leq$ 23 mm}: 54例 (65.9\%)
    \item \textbf{术后平均压差}: 14.3 $\pm$ 8.2 mmHg
    \item \textbf{$\leq$ 23 mm瓣膜压差}: 16.3 $\pm$ 9.1 mmHg
    \item \textbf{> 23 mm瓣膜压差}: 10.6 $\pm$ 3.6 mmHg
    \item \textbf{术后轻度压差}: 73例 (89.0\%)
    \item \textbf{术后中/重度压差}: 9例 (11.0\%)
\end{itemize}

\paragraph{临床意义}
\begin{itemize}
    \item 女性因体型较小更易发生S-PPM
    \item 小瓣膜($\leq$ 23 mm)是主要问题
    \item 即使SAVR,小瓣环患者仍有较高残余压差
    \item TAVR在小瓣环中相对血流动力学劣势减小
\end{itemize}

\subsection{综合临床决策框架}

\subsubsection{小瓣环AS患者的治疗选择}

\paragraph{支持TAVR的因素}
\begin{enumerate}
    \item \textbf{患者特征}:
    \begin{itemize}
        \item 女性
        \item 预计小瓣环(< 23 mm外科瓣膜)
        \item 低流量状态
        \item 基线房颤或房颤高危
        \item 虚弱或功能状态较差
        \item 年龄较大
    \end{itemize}

    \item \textbf{临床证据}:
    \begin{itemize}
        \item 1年复合终点降低(HR 0.55-0.87)
        \item 再住院显著减少
        \item 新发房颤减少(3\% vs 29\%)
        \item 更快恢复和回家
        \item 避免体外循环相关并发症
    \end{itemize}

    \item \textbf{血流动力学考量}:
    \begin{itemize}
        \item 压差仅高2-3 mmHg(临床意义小)
        \item 5年稳定性良好
        \item 低Z\textsubscript{va}
        \item 保存RV-PA耦合
    \end{itemize}
\end{enumerate}

\paragraph{支持SAVR的因素}
\begin{enumerate}
    \item \textbf{患者特征}:
    \begin{itemize}
        \item 年轻患者(< 65岁)
        \item 预期寿命很长(> 20年)
        \item 主动脉根部需要同时手术
        \item 二叶瓣畸形伴升主动脉扩张
        \item 需要同期冠脉搭桥
    \end{itemize}

    \item \textbf{技术因素}:
    \begin{itemize}
        \item TAVR解剖学禁忌(严重钙化、二叶瓣等)
        \item 血管通路困难
        \item 极小瓣环可能需要瓣环扩大
    \end{itemize}

    \item \textbf{血流动力学}:
    \begin{itemize}
        \item 压差稍低(但临床获益未体现)
        \item EOA稍大(但预后未改善)
    \end{itemize}
\end{enumerate}

\subsubsection{个体化决策路径}

\begin{figure}[h]
\centering
\begin{tikzpicture}[node distance=2cm]
% Decision tree style
\tikzstyle{decision} = [rectangle, draw, fill=blue!20, text width=6em, text centered, rounded corners, minimum height=3em]
\tikzstyle{block} = [rectangle, draw, fill=green!20, text width=8em, text centered, minimum height=3em]
\tikzstyle{line} = [draw, -latex']

% Nodes
\node [decision] (start) {小瓣环AS};
\node [decision, below of=start, node distance=3cm] (female) {女性?};
\node [block, left of=female, node distance=4cm] (tavr1) {优先考虑TAVR};
\node [decision, right of=female, node distance=4cm] (age) {年龄 > 75?};
\node [block, above right of=age, node distance=3cm] (tavr2) {TAVR};
\node [decision, below right of=age, node distance=3cm] (anatomy) {解剖适合?};
\node [block, right of=anatomy, node distance=3cm] (tavr3) {考虑TAVR};
\node [block, below of=anatomy, node distance=2.5cm] (savr) {考虑SAVR};

% Lines
\path [line] (start) -- (female);
\path [line] (female) -- node[above] {是} (tavr1);
\path [line] (female) -- node[above] {否} (age);
\path [line] (age) -- node[above] {是} (tavr2);
\path [line] (age) -- node[above] {否} (anatomy);
\path [line] (anatomy) -- node[above] {是} (tavr3);
\path [line] (anatomy) -- node[right] {否} (savr);
\end{tikzpicture}
\caption{小瓣环AS患者决策流程示意}
\end{figure}

\subsection{关键信息总结}

\begin{tcolorbox}[colback=blue!5!white,colframe=blue!75!black,title=小瓣环AS治疗策略核心要点]
\begin{enumerate}
    \item \textbf{PARTNER 3亚组分析}:
    \begin{itemize}
        \item 女性TAVR获益更显著:-10.4\% vs 男性-5.1\%
        \item 1年死亡/卒中/再住院:女性TAVR 8.1\% vs SAVR 18.5\%
    \end{itemize}

    \item \textbf{RHEIA试验}:
    \begin{itemize}
        \item 专门针对小瓣环患者(19-23mm占88\%)
        \item 1年主要终点:TAVR 8.9\% vs SAVR 15.6\%, HR 0.55 [0.31-0.96]
        \item 获益主要由再住院驱动(HR 0.4 [0.18-0.81])
    \end{itemize}

    \item \textbf{联合分析} (N=712):
    \begin{itemize}
        \item 1年复合终点:TAVR 8.5\% vs SAVR 16.8\%
        \item 绝对差异-8.2\% (95\% CI: -13.1至-3.3\%), P<0.001
        \item 再住院减少6.5\% (P=0.002)
    \end{itemize}

    \item \textbf{TAVR优势}:
    \begin{itemize}
        \item 新发房颤:3.3\% vs 28.8\%
        \item 出院回家:90.2\% vs 49.8\%
        \item 大出血更少:6.0\% vs 10.7\%
    \end{itemize}

    \item \textbf{SAVR劣势原因}:
    \begin{itemize}
        \item 高房颤率(女性HR 1.67)
        \item 低流量女性预后差(HR 1.65)
        \item 高瓣膜-动脉阻抗(5年HR 1.34)
        \item 受损RV-PA耦合(5年HR 1.45)
    \end{itemize}

    \item \textbf{血流动力学}:
    \begin{itemize}
        \item SAVR压差低2-3 mmHg(P<0.001),但临床意义有限
        \item 5年稳定性:两组均优秀,无SVD证据
        \item TAVR的Z\textsubscript{va}更低(预后更好)
    \end{itemize}

    \item \textbf{临床建议}:
    \begin{itemize}
        \item \textcolor{red}{\textbf{女性小瓣环AS:优先考虑TAVR}}
        \item 低流量女性:TAVR消除不良预后
        \item 瓣环扩大术风险高(死亡OR 2.0),应避免
        \item 血流动力学差异小,临床结局差异大
    \end{itemize}
\end{enumerate}
\end{tcolorbox}

\subsection{未来研究方向}

\begin{enumerate}
    \item \textbf{长期随访}:
    \begin{itemize}
        \item RHEIA 5-10年结果
        \item 联合分析的长期数据
        \item SVD发生率比较
    \end{itemize}

    \item \textbf{机制研究}:
    \begin{itemize}
        \item 房颤与再住院的因果关系
        \item RV-PA耦合恢复的预测因素
        \item 性别差异的分子机制
    \end{itemize}

    \item \textbf{优化策略}:
    \begin{itemize}
        \item TAVR瓣膜设计改进(减少PVL和起搏器)
        \item 房颤预防策略在SAVR中的价值
        \item 个体化瓣膜选择算法
    \end{itemize}

    \item \textbf{卫生经济学}:
    \begin{itemize}
        \item TAVR vs SAVR成本效益
        \item 再住院成本分析
        \item 生活质量调整寿命年(QALY)
    \end{itemize}
\end{enumerate}

\subsection{参考文献}
\begin{enumerate}
    \item Eng MH. SAVR vs. TAVR Strategy Trials (PARTNER 3 and RHEIA): Therapeutic Strategies in Small Annulus AS. TCT Conference Presentation.
    \item Siontis GCM, et al. Transcatheter aortic valve implantation vs. surgical aortic valve replacement for treatment of symptomatic severe aortic stenosis: an updated meta-analysis. \textit{Eur Heart J}. 2016;37:3503-3512.
    \item Mack MJ, et al. Transcatheter Aortic-Valve Replacement with a Balloon-Expandable Valve in Low-Risk Patients. \textit{N Engl J Med}. 2019;380:1695-1705.
    \item Tchetche D, et al. Transcatheter versus surgical aortic valve replacement in patients with small aortic annuli: a multicentre randomized clinical trial (RHEIA trial). \textit{Eur Heart J}. 2025;46:2079-2088.
    \item Eltchaninoff H, et al. Pooled Analysis of Transcatheter Versus Surgical Aortic Valve Replacement in Small Aortic Annuli From the RHEIA and PARTNER 3 Trials. \textit{JACC Cardiovasc Interv}. 2025;18:1540-1553.
    \item Carter-Storch R, et al. Predictors of Death or Rehospitalization After Aortic Valve Replacement in Low-Surgical-Risk Patients. \textit{JACC: Advances}. 2024;3:100853.
    \item Hahn RT, et al. Hemodynamic and Prognostic Implications of Transcatheter and Surgical Aortic Valve Replacement in Low-Risk Patients: 5-Year Outcomes From the PARTNER 3 Randomized Trial. \textit{JACC Cardiovasc Imaging}. 2025;18:625-640.
    \item Aalaei-Andabili SH, et al. Patient-Prosthesis Mismatch After Aortic Valve Replacement: Impact of Age and Sex. \textit{Innovations}. 2019;14:243-250.
    \item Nam K, et al. Patient-Prosthesis Mismatch After Surgical Aortic Valve Replacement: Analysis From the Society of Thoracic Surgeons Adult Cardiac Surgery Database. \textit{Ann Thorac Surg}. 2025;120:478-486.
    \item Hawkins RB, et al. Impact of Annular Enlargement on Long-Term Outcomes After Aortic Valve Replacement. \textit{Ann Thorac Surg}. 2019;108:1752-1760.
\end{enumerate}

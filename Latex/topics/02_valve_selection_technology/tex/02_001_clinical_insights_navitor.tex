\section{Clinical Insights from Navitor}
\label{sec:02_001_clinical_insights_navitor}

\subsection{文献信息}

\begin{itemize}
    \item \textbf{标题}:Clinical Insights from Navitor
    \item \textbf{作者}:Santiago Garcia, MD
    \item \textbf{会议}:TCT 2025 (Transcatheter Cardiovascular Therapeutics)
    \item \textbf{研究类型}:真实世界经验 - Navitor上市后研究(PAS)
    \item \textbf{数据来源}:TVT Registry
\end{itemize}

\subsection{研究背景}

Navitor瓣膜系统是由Abbott公司开发的自膨胀式经导管主动脉瓣。本研究通过TVT Registry收集商业使用数据,是首个报告Navitor Vision瓣膜真实世界经验的研究,也是迄今为止最大规模的Navitor患者队列。

\subsection{研究设计}

\textbf{Navitor上市后研究(PAS)设计:}
\begin{itemize}
    \item \textbf{设计}:通过TVT Registry收集商业使用数据
    \item \textbf{临床中心}:美国198个中心的2,958名患者
    \item \textbf{入组时间}:2023年1月至2024年12月
    \item \textbf{人群}:三叶主动脉瓣形态的原生TAVR患者
    \item \textbf{瓣膜类型}:Navitor Classic (n=2,237) 和 Navitor Vision (n=721)
\end{itemize}

\textbf{主要分析指标:}
\begin{itemize}
    \item 真实世界经验数据(Navitor Classic和Vision)
    \item 技术成功率
    \item 30天死亡或卒中
\end{itemize}

\subsection{基线特征}

\textbf{患者特征 (N=2,958):}
\begin{itemize}
    \item 年龄:81.4 ± 7.6岁
    \item 女性:62.4\%
    \item BMI:28.6 ± 6.9 kg/m²
    \item STS评分:6.6 ± 5.4
    \item NYHA III/IV级:57.7\%
    \item LVEF:58.5 ± 11.0\%
\end{itemize}

\textbf{病史:}
\begin{itemize}
    \item 房颤/房扑:39.7\%
    \item 传导阻滞:32.5\%
    \item 既往CABG:11.0\%
    \item 既往PCI:29.4\%
    \item 糖尿病:38.4\%
    \item 外周动脉疾病:17.1\%
    \item 永久起搏器:12.1\%
\end{itemize}

\subsection{手术结果}

\textbf{技术成功率 (离开手术室时):}
\begin{itemize}
    \item \textbf{全部Navitor}:97.9\%
    \item Navitor Classic:97.9\%
    \item Navitor Vision:98.1\%
\end{itemize}

\textbf{手术指标:}
\begin{itemize}
    \item 需要第二个瓣膜:0.1\%
    \item 手术死亡率:0.1\%
    \item 转开胸手术:0.1\%
    \item 主要血管并发症:1.7\%
\end{itemize}

\subsection{30天安全性结果}

\begin{table}[h]
\centering
\caption{30天安全性事件}
\begin{tabular}{lccc}
\toprule
\textbf{结局} & \textbf{全部Navitor} & \textbf{Classic} & \textbf{Vision} \\
 & \textbf{(N=2,958)} & \textbf{(n=2,237)} & \textbf{(n=721)} \\
\midrule
全因死亡或卒中 & 5.2\% & 5.2\% & 4.9\% \\
全因死亡 & 2.8\% & 2.9\% & 2.5\% \\
卒中 & 2.7\% & 2.6\% & 2.7\% \\
威胁生命或大出血 & 0.7\% & 0.8\% & 0.6\% \\
主要血管并发症 & 1.8\% & 1.9\% & 1.5\% \\
主动脉瓣再干预 & 0.3\% & 0.4\% & 0.1\% \\
\textbf{新起搏器} & \textbf{17.8\%} & \textbf{17.8\%} & \textbf{17.8\%} \\
\quad 无基线传导阻滞 & 14.1\% & 14.5\% & 12.3\% \\
\bottomrule
\end{tabular}
\end{table}

\subsection{起搏器植入与中心经验}

\textbf{重要发现:}超过20例植入后,起搏器植入率显著下降
\begin{itemize}
    \item \textbf{≤20例经验}:
    \begin{itemize}
        \item Navitor Classic:19.4\%
        \item Navitor Vision:19.1\%
    \end{itemize}
    \item \textbf{>20例经验}:
    \begin{itemize}
        \item Navitor Classic:13.5\% (p=0.0039)
        \item Navitor Vision:10.6\% (p=0.047)
    \end{itemize}
\end{itemize}

\subsection{超声心动图结果}

\textbf{血流动力学 (出院至30天):}
\begin{itemize}
    \item 基线平均跨瓣压差:39.9 mmHg
    \item 出院平均压差:7.6 mmHg
    \item 30天平均压差:7.3 mmHg
    \item 基线有效瓣口面积:0.75 cm²
    \item 出院EOA:2.17 cm²
    \item 30天EOA:1.98 cm²
\end{itemize}

\textbf{瓣周漏 (PVL):}
\begin{itemize}
    \item \textbf{出院}:无/微量 80.6\%,轻度 18.4\%,中度 0.9\%,重度 0.1\%
    \item \textbf{30天}:无/微量 75.6\%,轻度 22.6\%,中度 1.6\%,重度 0.2\%
\end{itemize}

\subsection{4年结果(Navitor IDE研究)}

\textbf{瓣膜耐久性(4年):}
\begin{itemize}
    \item 生物瓣膜功能障碍(BVD):5.9\%
    \item 中度HSVD:0\%
    \item 非结构性瓣膜退化:1.7\%
    \begin{itemize}
        \item 重度PPM:1.7\%
        \item 重度PVL:0\%
    \end{itemize}
    \item 感染性心内膜炎:4.2\%
    \item 临床瓣膜血栓:0\%
    \item \textbf{生物瓣膜失败}:\textbf{0\%}
    \item 重度HSVD:0\%
    \item 主动脉瓣再干预:0\%
    \item 瓣膜相关死亡:0\%
\end{itemize}

\subsection{关键结论}

\begin{enumerate}
    \item \textbf{首次报告Navitor Vision商业使用经验}
    \item \textbf{最大Navitor患者队列} (N=2,958)
    \item \textbf{高技术成功率}和30天低临床事件率
    \item \textbf{优秀的瓣膜血流动力学}和低PVL率
    \item \textbf{起搏器植入率随经验增加显著下降}
    \item \textbf{4年瓣膜失败率为0\%},显示出色的耐久性
\end{enumerate}

\subsection{临床意义}

\begin{itemize}
    \item \textbf{真实世界验证}:在大规模商业使用中验证了Navitor瓣膜的安全性和有效性
    \item \textbf{学习曲线}:20例经验后起搏器率显著降低,强调了操作技术的重要性
    \item \textbf{Vision技术}:Navitor Vision展示了与Classic相似的良好结果
    \item \textbf{优秀血流动力学}:单数字平均压差(7-8 mmHg)和大有效瓣口面积
    \item \textbf{极低PVL}:中重度PVL率<2\%
    \item \textbf{长期耐久性}:4年0\%瓣膜失败率具有重要临床价值
\end{itemize}

\subsection{研究局限性}

\begin{itemize}
    \item 非随机化真实世界数据
    \item 随访时间相对较短(30天)
    \item 缺乏与其他瓣膜系统的直接比较
    \item TVT Registry数据的固有局限性
\end{itemize}

\textit{文献来源:Garcia S. Real World Experience with the Navitor Valve in US Patients. New York Valves, June 2025.}

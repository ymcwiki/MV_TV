\section{Navitor Vision瓣膜设计}
\label{sec:02_002_navitor_vision_valve_design}

\subsection{文献信息}

\begin{itemize}
    \item \textbf{标题}:Navitor Vision Valve Design
    \item \textbf{作者}:Deepak Talreja, MD
    \item \textbf{会议}:TCT 2025 (Transcatheter Cardiovascular Therapeutics)
    \item \textbf{主题}:Navitor Vision瓣膜系统的设计特点和植入技术
\end{itemize}

\subsection{Navitor Vision瓣膜设计特点}

\subsubsection{核心设计要素}

\textbf{1. 主动密封袖带 (Active-Sealing Cuff)}
\begin{itemize}
    \item 与心动周期同步
    \item 主动密封以减少瓣周漏
    \item 独特的聚乙烯外层织物袖带设计
\end{itemize}

\textbf{2. 大细胞设计 (Large Cell Design)}
\begin{itemize}
    \item 最小化冠脉阻塞风险
    \item 改善冠脉通路和血流
    \item 便于瓣中瓣后冠脉介入
\end{itemize}

\textbf{3. 瓣环内小叶 (Intra-Annular Leaflets)}
\begin{itemize}
    \item 植入过程中立即功能
    \item 持续的血流动力学稳定性
    \item 减少低位植入风险
\end{itemize}

\textbf{4. 一致的径向力 (Consistent Radial Force)}
\begin{itemize}
    \item 扩张、锚定、稳定和密封
    \item 优化的支架设计
    \item 适用于各种瓣环大小
\end{itemize}

\textbf{5. 三个放射性不透标记 (Three Radiopaque Markers)}
\begin{itemize}
    \item 清晰显示3mm植入深度
    \item 便于精确定位
    \item 实时可视化反馈
\end{itemize}

\textbf{6. 瓣环治疗范围}
\begin{itemize}
    \item 19-30 mm瓣环直径
    \item 五种瓣膜尺寸:23、25、27、29、35 mm
\end{itemize}

\subsubsection{内部材料改进}

\textbf{新型聚乙烯织物材料:}
\begin{itemize}
    \item 保持较低的外形
    \item 提高耐久性
    \item 改善密封性能
\end{itemize}

\textbf{外层PVL特征:}
\begin{itemize}
    \item 新型聚乙烯外层织物袖带
    \item 专门设计以减少瓣周漏率和严重程度
    \item 增加密封区域
\end{itemize}

\subsection{FlexNav输送系统}

\subsubsection{系统特点}

\textbf{1. 一体化鞘管 (Integrated Sheath)}
\begin{itemize}
    \item 14F和15F当量
    \item 低插入轮廓
    \item 亲水涂层减少插入力
\end{itemize}

\textbf{2. 人体工程学手柄 (Ergonomic Handle)}
\begin{itemize}
    \item 直观的释放轮
    \item 可视化释放指示器
    \item 触觉锁定机制
\end{itemize}

\textbf{3. 柔性胶囊、轴和鼻锥 (Flexible Components)}
\begin{itemize}
    \item 可穿越常规至迂曲解剖
    \item 创伤最小的鼻锥
    \item 增强操控性
\end{itemize}

\textbf{4. 稳定层 (Stability Layer)}
\begin{itemize}
    \item 实现精确和稳定的释放
    \item 减少瓣膜移位
    \item 改善植入可预测性
\end{itemize}

\subsection{Navitor Cusp Overlap植入技术}

\subsubsection{术前准备:有效的球囊预扩张}

\textbf{关键要点:}
\begin{itemize}
    \item \textbf{球囊选择}:直径不超过周长衍生的瓣环直径
    \item \textbf{球囊位置}:更偏主动脉侧而非心室侧
    \item \textbf{目的}:最小化与传导系统的相互作用
\end{itemize}

\subsubsection{步骤1:初始FlexNav系统定位}

\textbf{在瓣尖重叠视图(Cusp Overlap View)中:}
\begin{itemize}
    \item 消除视差(parallax)
    \item 将支架内缘与瓣环对齐
\end{itemize}

\subsubsection{步骤2:瓣膜释放}

\textbf{释放过程:}
\begin{itemize}
    \item 稳定释放,允许瓣膜下降至3mm
    \item 在瓣尖重叠视图中:如果在瓣膜开放期间且瓣环完全接触前出现视差
    \begin{itemize}
        \item 暂停释放
        \item 移至LAO视图
        \item 消除视差
    \end{itemize}
    \item 使用导丝和/或FlexNav调整同轴对齐
    \item 检查支架是否完全开放,如存在问题需解决
    \item 在瓣尖重叠和LAO视图中确认3mm目标深度
\end{itemize}

\subsubsection{步骤3:瓣膜释放}

\textbf{最终释放:}
\begin{itemize}
    \item 保持FlexNav中性或略微前推
    \item 使FlexNav居中,将导丝拉至心室中部位置
    \item 继续释放瓣膜,确认所有3个卡扣完全脱离
    \item 拉回导丝使鼻锥居中,缓慢撤出FlexNav输送系统
    \item 通过主动脉根部造影确认最终位置
\end{itemize}

\subsection{关键技术要点:识别不完全支架开放}

\subsubsection{80\%释放的瓣膜示例}

\textbf{在LAO视图中的发现:}
\begin{itemize}
    \item 支架与瓣环之间存在间隙
    \item 不完全支架开放
    \item 4-5个接触点之间的空隙
\end{itemize}

\subsubsection{无效球囊预扩张的后果}

\textbf{支架开放对比:}
\begin{itemize}
    \item \textbf{预期直径}:完整的圆形开放
    \item \textbf{较小直径}:椭圆形,不完全支架开放
    \item \textbf{原因}:钙化未充分预处理
\end{itemize}

\subsubsection{RAO/CAU投影的价值}

\textbf{重要性:}
\begin{itemize}
    \item LAO投影可能掩盖不完全开放
    \item RAO/CAU投影更好地显示支架开放情况
    \item 建议在释放过程中使用多个投影角度
\end{itemize}

\subsection{临床意义}

\begin{enumerate}
    \item \textbf{主动密封技术}
    \begin{itemize}
        \item 与心动周期同步的密封机制
        \item 显著降低瓣周漏发生率
        \item 改善长期临床结果
    \end{itemize}

    \item \textbf{大细胞设计优势}
    \begin{itemize}
        \item 降低冠脉阻塞风险
        \item 便于未来冠脉介入
        \item 适合复杂解剖结构
    \end{itemize}

    \item \textbf{瓣环内位置}
    \begin{itemize}
        \item 即时血流动力学稳定性
        \item 减少传导系统干扰
        \item 降低起搏器需求
    \end{itemize}

    \item \textbf{技术要求}
    \begin{itemize}
        \item 充分的球囊预扩张至关重要
        \item 多投影角度监测植入过程
        \item 及时识别和纠正不完全开放
    \end{itemize}

    \item \textbf{学习曲线}
    \begin{itemize}
        \item Cusp Overlap技术需要培训
        \item 经验积累可降低并发症
        \item 标准化操作流程的重要性
    \end{itemize}
\end{enumerate}

\subsection{操作建议}

\textbf{术前规划:}
\begin{itemize}
    \item CT测量瓣环大小和钙化分布
    \item 选择合适的球囊大小
    \item 确定最佳植入深度
\end{itemize}

\textbf{术中监测:}
\begin{itemize}
    \item 使用多个投影角度
    \item 实时评估支架开放情况
    \item 及时调整位置和对齐
\end{itemize}

\textbf{术后评估:}
\begin{itemize}
    \item 主动脉根部造影确认位置
    \item 超声评估血流动力学
    \item 检查瓣周漏和传导异常
\end{itemize}

\subsection{与其他瓣膜的比较}

\textbf{Navitor Vision的独特优势:}
\begin{itemize}
    \item 主动密封技术(区别于被动密封)
    \item 瓣环内设计(vs瓣环上设计)
    \item 大细胞支架(改善冠脉通路)
    \item 低外形输送系统
    \item 一体化鞘管设计
\end{itemize}

\textit{文献来源:Talreja D. Navitor Vision Valve Design. TCT 2025.}

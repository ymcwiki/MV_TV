\section{VANTAGE试验:Navitor瓣膜在低中危患者中的30天和1年结果}
\label{sec:02_003_vantage_trial_navitor}

\subsection{文献信息}

\begin{itemize}
    \item \textbf{标题}:Thirty-Day and One-Year Outcomes of the Navitor TAVR System in Patients with Low or Intermediate Risk - The VANTAGE Trial
    \item \textbf{作者}:Stephen Worthley, MD
    \item \textbf{会议}:TCT 2025
    \item \textbf{研究类型}:前瞻性、单臂、多中心、上市前临床试验
    \item \textbf{注册号}:NCT04788888
\end{itemize}

\subsection{研究背景}

Navitor经导管主动脉瓣植入(TAVI)系统最初获批用于治疗高危或极高危手术风险的症状性重度主动脉瓣狭窄患者,基于Navitor IDE研究的结果。随着TAVR从高危患者向低中危患者适应证的拓展,VANTAGE试验旨在评估Navitor TAVI系统在低或中危手术风险的症状性重度AS患者中的安全性和性能。

\subsection{研究设计}

\subsubsection{试验概述}

\textbf{VANTAGE试验设计:}
\begin{itemize}
    \item \textbf{设计}:前瞻性、单臂、多中心、上市前试验
    \item \textbf{地点}:欧洲、澳大利亚和以色列的40个中心
    \item \textbf{随访}:10年
    \item \textbf{入组}:原生瓣434例(低危203例,中危231例)
    \item \textbf{瓣中瓣}:目前正在入组(最多100例)
\end{itemize}

\subsubsection{原生瓣队列纳入排除标准}

\textbf{主要纳入标准:}
\begin{itemize}
    \item 症状性重度原生主动脉瓣狭窄
    \item 30天手术死亡风险<7\%
    \begin{itemize}
        \item 中危:≥3\%且<7\%
        \item 低危:<3\%
    \end{itemize}
    \item AVA ≤ 1.0 cm² 或 iEOA ≤ 0.6 cm²/m²
    \item 同时满足:平均压差≥40 mmHg 或峰值速度≥4.0 m/s 或DVI≤0.25
\end{itemize}

\textbf{主要排除标准:}
\begin{itemize}
    \item 先天性单叶瓣/二叶瓣形态
    \item 既往任何位置的人工心脏瓣膜
    \item 混合型瓣膜病变伴主要反流
    \item 重度二尖瓣/三尖瓣反流
\end{itemize}

\subsubsection{研究终点}

\textbf{主要安全终点:}
\begin{itemize}
    \item 12个月全因死亡或致命性卒中/致残性卒中 (N=262)
    \item 性能目标:97.5\%上置信界<11.3\%
\end{itemize}

\textbf{主要有效性终点:}
\begin{itemize}
    \item 30天中度或以上瓣周漏 (N=434)
    \item 性能目标:97.5\%上置信界<6.6\%
\end{itemize}

\subsection{研究组织}

\textbf{研究负责人(PIs):}
\begin{itemize}
    \item Nicolas van Mieghem, Erasmus University Medical Center, 荷兰
    \item Stephen Worthley, Macquarie University Hospital, 澳大利亚
\end{itemize}

\textbf{指导委员会:}
\begin{itemize}
    \item Francesco Bedogni, Policinico San Donato, 意大利
    \item Maurizio Taramasso, HerzZentrum Hirslanden, 瑞士
    \item Didier Tchetche, Clinique Pasteur, 法国
\end{itemize}

\textbf{参与中心:}
\begin{itemize}
    \item 11个国家的36个中心
    \item 主要分布:欧洲、澳大利亚、以色列
\end{itemize}

\subsection{基线特征}

\begin{table}[h]
\centering
\caption{基线特征和病史}
\begin{tabular}{lcc}
\toprule
\textbf{变量} & \textbf{低危 (N=203)} & \textbf{中危 (N=231)} \\
\midrule
年龄(岁) & 75.1 ± 3.2 & 79.1 ± 3.7 \\
女性 & 46.8\% & 52.8\% \\
STS-PROM (\%) & 1.5 ± 0.5 & 2.6 ± 1.2 \\
EuroSCORE II (\%) & 1.4 ± 0.6 & 2.1 ± 1.2 \\
总衰弱评分(0-4) & 0.4 ± 0.5 & 0.9 ± 0.8 \\
冠心病 & 21.7\% & 28.1\% \\
房颤 & 14.3\% & 23.4\% \\
一度AVB & 3.4\% & 7.8\% \\
LBBB & 2.5\% & 6.1\% \\
RBBB & 5.4\% & 8.7\% \\
永久起搏器 & 3.9\% & 6.5\% \\
主动脉瓣口面积(cm²) & 0.7 ± 0.2 & 0.7 ± 0.2 \\
平均压差(mmHg) & 48.8 ± 9.9 & 47.7 ± 11.2 \\
LVEF (\%) & 59.9 ± 6.7 & 60.9 ± 7.6 \\
\bottomrule
\end{tabular}
\end{table}

\subsection{手术特征和技术成功率}

\begin{table}[h]
\centering
\caption{手术特征}
\begin{tabular}{lccc}
\toprule
\textbf{手术特征} & \textbf{低危} & \textbf{中危} & \textbf{总计} \\
 & \textbf{(N=203)} & \textbf{(N=231)} & \textbf{(N=434)} \\
\midrule
清醒镇静 & 58.1\% & 60.2\% & 59.2\% \\
经股动脉入路 & 99.5\% & 100.0\% & 99.8\% \\
预扩张 & 88.7\% & 92.6\% & 90.8\% \\
后扩张 & 33.7\% & 32.0\% & 32.8\% \\
无再鞘 & 48.3\% & 42.4\% & 45.1\% \\
住院时间(天) & 3.6 ± 1.9 & 3.7 ± 2.9 & 3.6 ± 2.5 \\
\textbf{技术成功率} & \textbf{97.5\%} & \textbf{96.5\%} & \textbf{97.0\%} \\
额外Navitor瓣膜 & 0.5\% & 0.9\% & 0.7\% \\
其他TAV/转SAVR & 1.0\% & 0.0\% & 0.5\% \\
\bottomrule
\end{tabular}
\end{table}

\textbf{植入瓣膜尺寸分布:}
\begin{itemize}
    \item 23 mm: 7\%
    \item 25 mm: 29\%
    \item 27 mm: 30\%
    \item 29 mm: 29\%
    \item 35 mm: 5\%
\end{itemize}

\subsection{主要研究终点}

\subsubsection{主要安全终点 - 成功达到}

\begin{table}[h]
\centering
\caption{主要安全终点 (12个月)}
\begin{tabular}{lcccc}
\toprule
\textbf{终点} & \textbf{观察率} & \textbf{97.5\% UCB} & \textbf{性能目标} & \textbf{P值} \\
\midrule
全因死亡或致命性卒中/ & & & & \\
致残性卒中 (N=262) & 2.3\% & 5.0\% & 11.3\% & <0.0001 \\
\bottomrule
\end{tabular}
\end{table}

\textbf{组成部分(KM率):}
\begin{itemize}
    \item 低危组 (N=123): 0.8\%
    \item 中危组 (N=139): 3.6\%
    \item 总计 (N=262): 2.3\%
\end{itemize}

\textbf{各组成部分:}
\begin{itemize}
    \item 全因死亡: 0.8\% (0\% 低危, 1.4\% 中危)
    \item 致命性卒中/致残性卒中: 1.5\% (0.8\% 低危, 2.2\% 中危)
\end{itemize}

\subsubsection{主要有效性终点 - 成功达到}

\begin{table}[h]
\centering
\caption{主要有效性终点 (30天)}
\begin{tabular}{lcccc}
\toprule
\textbf{终点} & \textbf{观察率} & \textbf{97.5\% UCB} & \textbf{性能目标} & \textbf{P值} \\
\midrule
中度或以上PVL (N=434) & 0.0\% & 0.9\% & 6.6\% & <0.0001 \\
\bottomrule
\end{tabular}
\end{table}

\subsection{30天安全性结果}

\begin{table}[h]
\centering
\caption{30天安全性事件 (N=434)}
\begin{tabular}{lccc}
\toprule
\textbf{事件} & \textbf{低危} & \textbf{中危} & \textbf{总计} \\
 & \textbf{(N=203)} & \textbf{(N=231)} & \textbf{(N=434)} \\
\midrule
全因死亡 & 0.0\% (0) & 0.9\% (2) & 0.5\% (2) \\
心血管死亡 & 0.0\% (0) & 0.9\% (2) & 0.5\% (2) \\
瓣膜相关死亡 & 0.0\% (0) & 0.4\% (1) & 0.2\% (1) \\
全部卒中 & 1.0\% (2) & 2.2\% (5) & 1.6\% (7) \\
致命性卒中/致残性卒中 & 0.5\% (1) & 1.3\% (3) & 0.9\% (4) \\
TIA & 0.0\% (0) & 1.7\% (4) & 0.9\% (4) \\
3/4期急性肾损伤 & 0.0\% (0) & 0.9\% (2) & 0.5\% (2) \\
3/4型出血 & 2.5\% (5) & 4.8\% (11) & 3.7\% (16) \\
主要血管并发症 & 3.0\% (6) & 5.6\% (13) & 4.4\% (19) \\
主要心脏结构并发症 & 0.5\% (1) & 0.9\% (2) & 0.7\% (3) \\
心肌梗死 & 0.0\% (0) & 2.6\% (6) & 1.4\% (6) \\
主动脉瓣再干预 & 0.5\% (1) & 0.4\% (1) & 0.5\% (2) \\
\textbf{起搏器植入(无起搏器)} & \textbf{15.9\% (31)} & \textbf{21.3\% (46)} & \textbf{18.7\% (77)} \\
心血管再住院 & 6.4\% (13) & 7.8\% (18) & 7.1\% (31) \\
\bottomrule
\end{tabular}
\end{table}

\subsection{12个月安全性结果}

\begin{table}[h]
\centering
\caption{12个月安全性事件 (N=262, KM率)}
\begin{tabular}{lccc}
\toprule
\textbf{事件} & \textbf{低危} & \textbf{中危} & \textbf{总计} \\
 & \textbf{(N=123)} & \textbf{(N=139)} & \textbf{(N=262)} \\
\midrule
全因死亡 & 0.0\% (0) & 1.4\% (2) & 0.8\% (2) \\
心血管死亡 & 0.0\% (0) & 1.4\% (2) & 0.8\% (2) \\
瓣膜相关死亡 & 0.0\% (0) & 0.7\% (1) & 0.4\% (1) \\
全部卒中 & 1.6\% (2) & 3.6\% (5) & 2.7\% (7) \\
致命性卒中/致残性卒中 & 0.8\% (1) & 2.2\% (3) & 1.5\% (4) \\
TIA & 1.6\% (2) & 2.9\% (4) & 2.3\% (6) \\
心肌梗死 & 0.8\% (1) & 1.4\% (2) & 1.1\% (3) \\
需要干预的冠脉阻塞 & 0.0\% (0) & 0.0\% (0) & 0.0\% (0) \\
需要时成功冠脉通路 & 100.0\% (2/2) & NA & 100.0\% (2/2) \\
主动脉瓣再干预 & 0.0\% (0) & 0.0\% (0) & 0.0\% (0) \\
心血管再住院 & 13.0\% (16) & 18.7\% (26) & 16.0\% (42) \\
人工瓣膜心内膜炎 & 0.0\% (0) & 0.7\% (1) & 0.4\% (1) \\
\bottomrule
\end{tabular}
\end{table}

\subsection{瓣周漏和血流动力学表现}

\subsubsection{瓣周漏}

\begin{table}[h]
\centering
\caption{瓣周漏分级}
\begin{tabular}{lcc}
\toprule
\textbf{PVL等级} & \textbf{30天 (N=403)} & \textbf{12个月 (N=238)} \\
\midrule
无/微量 & 86.4\% & 84.0\% \\
轻度 & 13.6\% & 16.0\% \\
中度 & 0\% & 0\% \\
重度 & 0\% & 0\% \\
\bottomrule
\end{tabular}
\end{table}

\subsubsection{血流动力学表现}

\textbf{跨瓣压差变化:}
\begin{itemize}
    \item 基线:42.9 mmHg
    \item 出院:8.3 mmHg
    \item 30天:7.7 mmHg
    \item 12个月:8.0 mmHg
\end{itemize}

\textbf{有效瓣口面积(EOA):}
\begin{itemize}
    \item 基线:0.7 cm²
    \item 出院:1.9 cm²
    \item 30天:1.9 cm²
    \item 12个月:1.9 cm²
\end{itemize}

\subsection{功能状态改善}

\subsubsection{NYHA心功能分级}

\textbf{I/II级比例:}
\begin{itemize}
    \item 基线 (N=434): 71.3\% I级, 26.6\% II级, 2.1\% III级
    \item 30天 (N=429): 70.0\% I级, 26.5\% II级, 3.5\% III级
    \item 12个月 (N=255): 75.8\% I级, 24.2\% II级, 0\% III/IV级
    \item \textbf{12个月96.5\%患者NYHA I/II级}
\end{itemize}

\subsubsection{KCCQ生活质量评分}

\textbf{KCCQ-OS评分变化:}
\begin{itemize}
    \item 基线 (N=432): 71.0分
    \item 30天 (N=424): 87.8分
    \item 12个月 (N=253): 86.2分
    \item \textbf{改善Δ+15.2分}(临床显著改善)
\end{itemize}

\subsection{研究局限性}

\begin{enumerate}
    \item 单臂试验,无法与其他TAVR瓣膜直接比较
    \item 筛选过程可能引入固有的选择偏倚
    \begin{itemize}
        \item 总筛选失败率:33.2\%
        \item 解剖筛选失败率:14.7\%
    \end{itemize}
    \item 12个月结果目前仅适用于前262名患者
\end{enumerate}

\subsection{研究结论}

\begin{enumerate}
    \item Navitor瓣膜在治疗症状性重度主动脉瓣狭窄的低或中危手术风险患者时,在30天和12个月表现出良好的安全性和性能

    \item \textbf{主要安全和有效性终点均成功达到} (p<0.0001)
    \begin{itemize}
        \item 12个月复合终点仅2.3\%
        \item 30天中重度PVL率0\%
    \end{itemize}

    \item 瓣环内瓣膜设计提供优异的血流动力学表现
    \begin{itemize}
        \item 残余压差为个位数(7.7 mmHg)
        \item 30天轻度PVL<14\%
        \item 12个月无中重度PVL
    \end{itemize}

    \item 这些结果支持Navitor瓣膜在低中危患者中的适应证扩展
\end{enumerate}

\subsection{临床意义}

\begin{itemize}
    \item \textbf{适应证扩展}:支持Navitor瓣膜用于低中危AS患者
    \item \textbf{优秀安全性}:极低的死亡率和卒中率
    \item \textbf{优异血流动力学}:单数字压差和大EOA
    \item \textbf{极低PVL}:无中重度PVL,轻度PVL率低
    \item \textbf{功能改善}:NYHA分级和KCCQ评分显著改善
    \item \textbf{冠脉通路}:需要时100\%成功冠脉通路
    \item \textbf{低再干预率}:12个月主动脉瓣再干预率0\%
\end{itemize}

\textit{文献来源:Worthley S, et al. Thirty-Day and One-Year Outcomes of the Navitor TAVR System in Patients with Low or Intermediate Risk: The VANTAGE Trial. TCT 2025.}

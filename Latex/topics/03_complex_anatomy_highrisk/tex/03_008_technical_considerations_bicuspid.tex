\section{二叶主动脉瓣狭窄TAVR的技术要点}
\label{sec:03_008_technical_considerations_bicuspid}

\subsection{文献信息}

\begin{itemize}
    \item \textbf{标题}: Technical Considerations for TAVR in Bicuspid AS
    \item \textbf{作者}: Giuseppe Tarantini, MD, PhD
    \item \textbf{单位}: Director of Interventional Cardiology Unit, University of Padova
    \item \textbf{会议}: CRF TCT (Transcatheter Cardiovascular Therapeutics)
    \item \textbf{主要参考文献}: Tarantini G, Fabris T. Circ Cardiovasc Interv. 2021 Jul;14(7):e009827
    \item \textbf{利益冲突}: 顾问费/酬金来自Abbott Laboratories, Boston Scientific, Edwards Lifesciences, Medtronic, GADA, Microport, SMT(已缓解)
\end{itemize}

\subsection{研究背景}

二叶主动脉瓣(BAV)是最常见的先天性心脏病,发病率约为1-2\%。随着TAVR技术的发展和适应证扩大,越来越多的BAV患者接受TAVR治疗。然而,BAV的解剖学特点使其TAVR面临独特挑战:

\subsubsection{BAV的解剖学特点}

\begin{itemize}
    \item 瓣叶融合形成嵴(raphe)
    \item 瓣环椭圆形,非圆形
    \item 钙化分布不均匀
    \item 升主动脉扩张常见
    \item 左室流出道形态变化
\end{itemize}

\subsubsection{TAVR的技术挑战}

\begin{itemize}
    \item 瓣膜尺寸选择困难
    \item 瓣周漏风险增加
    \item 瓣膜扩张不全或不对称
    \item 主动脉根部损伤风险
    \item 冠状动脉受压风险
\end{itemize}

本讲座系统阐述了BAV TAVR的四大技术要点:表型分析、尺寸选择、定位策略和术后优化。

\subsection{主要研究发现}

\subsubsection{手术流程概览}

BAV TAVR的成功需要系统化的四步骤方法:

\begin{enumerate}
    \item \textbf{表型分析(Phenotyping)}: 确定BAV类型和解剖特征
    \item \textbf{尺寸选择(Sizing)}: 选择合适的瓣膜尺寸
    \item \textbf{定位策略(Positioning)}: 确定瓣膜释放位置
    \item \textbf{术后优化(Optimization)}: 评估和优化瓣膜扩张
\end{enumerate}

\subsubsection{一、表型分析(Phenotyping)}

\paragraph{BAV分类系统}

目前存在两种主要的BAV分类系统:

\begin{table}[h]
\centering
\caption{BAV形态学分类系统比较}
\label{tab:bav_classification_systems}
\begin{tabular}{p{3cm}p{5cm}p{6cm}}
\toprule
\textbf{分类系统} & \textbf{依据} & \textbf{主要分型} \\
\midrule
Sievers分类 & 外科观察为基础 & Type 0(无嵴型): L-L, A-P \newline Type 1(单嵴型): L-R, L-N, R-N \newline Type 2(双嵴型): L-R/R-N, L-R/L-N \\
\midrule
Jilaihawy分类 & CT影像为基础 & Type 0: 无嵴型(lat和ap亚型) \newline Type 1: 单嵴型(L-R, R-N, N-L亚型) \\
\bottomrule
\end{tabular}
\end{table}

\paragraph{CT扫描表型评估}

基于CT扫描的表型分析应包括:
\begin{itemize}
    \item 瓣叶融合类型和嵴的位置
    \item 瓣环形态(圆形 vs 椭圆形)
    \item 钙化分布模式
    \item 主动脉根部各层面测量
    \item 冠状动脉开口位置
\end{itemize}

\subsubsection{二、尺寸选择(Sizing)}

\paragraph{测量参数}

作者提出了两种互补的测量方法:

\begin{table}[h]
\centering
\caption{虚拟基底环(VBR)vs虚拟嵴环(VRR)测量参数}
\label{tab:vbr_vrr_parameters}
\begin{tabular}{p{5cm}p{5cm}p{4cm}}
\toprule
\textbf{测量参数} & \textbf{虚拟基底环(VBR)} & \textbf{虚拟嵴环(VRR)} \\
\midrule
测量平面 & 瓣环水平 & 瓣上结构(嵴水平) \\
\midrule
面积(Area) & 628 mm² & -- \\
周长(Perimeter) & 89.8 mm & 78.7 mm \\
最小直径 & 27.2 mm & -- \\
最大直径 & 28.4 mm & -- \\
平均直径 & 27.8 mm & -- \\
周长推导直径 & 28.6 mm & 25.1 mm \\
瓣间距离 & -- & 28.0 mm \\
嵴前间距 & -- & 20.5 mm \\
嵴长度 & -- & 测量 \\
\midrule
应用 & Type 0 BAV主要依据 & Type 1和2 BAV参考 \\
\bottomrule
\end{tabular}
\end{table}

\paragraph{基于BAV类型的尺寸选择策略}

\begin{table}[h]
\centering
\caption{不同BAV类型的尺寸选择策略}
\label{tab:sizing_strategy_by_type}
\begin{tabular}{p{3cm}p{4cm}p{4cm}p{4cm}}
\toprule
\textbf{BAV类型} & \textbf{共主导型} & \textbf{瓣环主导型} & \textbf{嵴主导型} \\
\midrule
Type 0 \newline (三瓣联合型) & VBR = VRR \newline → VBR sizing & VRR > VBR \newline → VBR sizing & 不适用 \\
\midrule
Type 1和2 \newline (有嵴型) & VBR = VRR \newline → VBR sizing & VRR > VBR \newline → VBR sizing & VBR > VRR \newline → VRR sizing \\
\bottomrule
\end{tabular}
\end{table}

关键原则:
\begin{itemize}
    \item \textbf{共主导型和瓣环主导型}: 使用VBR sizing(瓣环水平定径)
    \item \textbf{嵴主导型}: 使用VRR sizing(瓣上水平定径)
    \item 需要多参数综合评估
    \item Type 0 BAV的VBR测量可能具有挑战性
\end{itemize}

\paragraph{球囊瓣膜成形术测试}

球囊瓣膜成形术可作为功能性测试,评估瓣上结构:

\begin{itemize}
    \item \textbf{瓣上 vs 瓣环束腰(Waisting)}
    \begin{itemize}
        \item VRR束腰 + 嵴前扩张 → 提示VRR sizing
        \item VBR密封 + 对称扩张 → 提示VBR sizing
    \end{itemize}
    \item \textbf{对称 vs 不对称扩张}
    \begin{itemize}
        \item 对称扩张 → 倾向VBR sizing
        \item 不对称扩张 → 需谨慎选择尺寸
    \end{itemize}
\end{itemize}

\subsubsection{三、瓣膜类型选择(THV Type Choice)}

\begin{table}[h]
\centering
\caption{Evolut vs Sapien在BAV中的优劣势对比}
\label{tab:thv_comparison_bav}
\begin{tabular}{p{3.5cm}p{5.5cm}p{5.5cm}}
\toprule
\textbf{特性} & \textbf{Evolut系列 (自膨胀瓣)} & \textbf{Sapien系列 (球扩瓣)} \\
\midrule
\multicolumn{3}{l}{\textit{优势}} \\
\midrule
可重新定位/回收 & \checkmark & -- \\
主动脉根部损伤风险 & 较低 & 较高 \\
瓣上结构适应性 & 环形瓣上瓣膜 & -- \\
血流动力学 & 更好 & -- \\
径向支撑力 & -- & 更高 \\
形态维持 & -- & 保持圆形 \\
瓣周漏 & -- & 更低 \\
冠状动脉通路 & -- & 冠脉友好 \\
\midrule
\multicolumn{3}{l}{\textit{劣势}} \\
\midrule
瓣周漏 & 更多 & -- \\
起搏器植入 & 风险更高 & -- \\
冠状动脉通路 & 受损 & -- \\
主动脉根部损伤 & -- & 风险更高 \\
主动脉夹层 & -- & 风险更高 \\
\midrule
\multicolumn{3}{l}{\textit{监管状态}} \\
\midrule
CE认证 & 2020年6月获BAV适应证 & BAV注意事项已从IFU移除 \\
\bottomrule
\end{tabular}
\end{table}

\subsubsection{四、定位策略(Positioning)}

\paragraph{瓣环水平 vs 瓣上水平定位}

定位策略应基于尺寸选择方法:

\begin{itemize}
    \item \textbf{基于VBR sizing}: 着陆区在瓣环平面(annular plane)
    \item \textbf{基于VRR sizing}: 着陆区在嵴平面(raphe plane)
\end{itemize}

\paragraph{瓣上定位的细微差别}

对于自膨胀瓣(SEV),瓣上定位有两种策略:

\begin{table}[h]
\centering
\caption{球扩瓣(BEV) vs 自膨胀瓣(SEV)定位考量}
\label{tab:bev_sev_positioning}
\begin{tabular}{p{3cm}p{5cm}p{6cm}}
\toprule
\textbf{定位类型} & \textbf{主要风险} & \textbf{适用情况} \\
\midrule
BEV瓣上定位 & THV脱位风险 \newline 瓣膜-患者不匹配(降尺寸) & 需要精确控制 \newline 避免过度瓣上定位 \\
\midrule
SEV瓣上定位 & 冠状动脉通路受损 \newline Redo-TAVR可行性降低 & 可接受适度瓣上定位 \newline 需平衡即刻效果和长期考虑 \\
\bottomrule
\end{tabular}
\end{table}

\subsubsection{五、术后优化(Optimization)}

\paragraph{THV扩张评估}

术后需系统评估瓣膜扩张情况:

\begin{enumerate}
    \item \textbf{透视评估}
    \begin{itemize}
        \item LAO/CRAN和RAO/CAU投照角度
        \item 评估瓣膜形态的对称性
        \item 不同瓣膜类型的表现差异
    \end{itemize}

    \item \textbf{有创血流动力学评估}
    \begin{itemize}
        \item 测量跨瓣压差
        \item 高残余压差(如40 mmHg)提示扩张不全
    \end{itemize}

    \item \textbf{超声心动图评估}
    \begin{itemize}
        \item 瓣周漏程度
        \item 瓣膜扩张不对称性
    \end{itemize}
\end{enumerate}

\paragraph{后扩张策略}

针对自膨胀瓣的后扩张建议:

\begin{table}[h]
\centering
\caption{Evolut系列瓣膜的后扩张球囊选择}
\label{tab:post_dilatation_strategy}
\begin{tabular}{p{3cm}p{3cm}p{3cm}p{4cm}}
\toprule
\textbf{Evolut尺寸} & \textbf{适用瓣环范围} & \textbf{瓣架束腰} & \textbf{球囊尺寸范围} \\
\midrule
23 mm & 18-20 mm & 20 mm & 下限: VBR或VRR最小直径 \newline 上限: 瓣架束腰直径 \\
26 mm & 20-23 mm & 22 mm & 18-20 mm至20-23 mm \\
29 mm & 23-26 mm & 23 mm & 20-23 mm至23-26 mm \\
34 mm & 26-30 mm & 24 mm & 23-26 mm至26-30 mm \\
\bottomrule
\end{tabular}
\end{table}

后扩张指征:
\begin{itemize}
    \item 扩张不全伴高残余压差
    \item 不对称扩张伴显著瓣周漏
    \item 嵴型BAV的局部束腰
\end{itemize}

后扩张注意事项:
\begin{itemize}
    \item 球囊尺寸不应超过瓣架束腰直径
    \item 避免过度扩张导致主动脉根部损伤
    \item 对于球扩瓣,后扩张风险更高
\end{itemize}

\subsection{结论}

Giuseppe Tarantini提出了BAV TAVR的系统化技术框架,核心要点包括:

\begin{enumerate}
    \item \textbf{BAV并非单一实体}
    \begin{itemize}
        \item 无嵴型和功能性BAV的TAVR类似经典三叶瓣
        \item 有嵴型BAV需要个体化策略
    \end{itemize}

    \item \textbf{个体化THV尺寸选择和定位是嵴型BAV的规则}
    \begin{itemize}
        \item VBR vs VRR sizing的选择
        \item 瓣环水平 vs 瓣上水平定位
        \item 基于BAV形态学特征的决策
    \end{itemize}

    \item \textbf{推荐术后评估和优化}
    \begin{itemize}
        \item 多模态评估瓣膜扩张
        \item 必要时进行后扩张
        \item 优化即刻和长期结局
    \end{itemize}
\end{enumerate}

\subsection{临床启示}

\subsubsection{术前规划要点}

\begin{enumerate}
    \item \textbf{详细的CT评估}
    \begin{itemize}
        \item 明确BAV分型(Sievers和Jilaihawy分类)
        \item 同时测量VBR和VRR参数
        \item 评估钙化分布和嵴的位置
        \item 评估冠状动脉高度和解剖
        \item 评估升主动脉直径
    \end{itemize}

    \item \textbf{制定个体化策略}
    \begin{itemize}
        \item 根据共主导/瓣环主导/嵴主导模式选择sizing方法
        \item 选择合适的THV类型(SEV vs BEV)
        \item 规划定位深度
        \item 准备后扩张球囊
    \end{itemize}
\end{enumerate}

\subsubsection{术中技术要点}

\begin{enumerate}
    \item \textbf{球囊瓣膜成形术}
    \begin{itemize}
        \item 作为功能性测试评估瓣上结构
        \item 观察束腰位置和扩张对称性
        \item 必要时调整sizing和定位策略
    \end{itemize}

    \item \textbf{THV释放}
    \begin{itemize}
        \item 对于SEV,利用可重新定位特性
        \item 确保着陆区与sizing策略一致
        \item 避免过深或过浅释放
    \end{itemize}

    \item \textbf{术后优化}
    \begin{itemize}
        \item 常规透视评估(LAO/CRAN和RAO/CAU)
        \item 测量跨瓣压差
        \item 超声评估瓣周漏
        \item 必要时进行后扩张
    \end{itemize}
\end{enumerate}

\subsubsection{不同BAV亚型的策略}

\begin{table}[h]
\centering
\caption{不同BAV亚型的推荐TAVR策略}
\label{tab:tavr_strategy_by_subtype}
\begin{tabular}{p{3cm}p{5cm}p{6cm}}
\toprule
\textbf{BAV亚型} & \textbf{推荐策略} & \textbf{特殊考虑} \\
\midrule
Type 0无嵴型 & 类似三叶瓣方法 \newline VBR sizing \newline 瓣环水平定位 & VBR测量可能困难 \newline 需仔细识别瓣环平面 \\
\midrule
Type 0三瓣联合型 & 评估VBR和VRR \newline 多数为VBR sizing & 可能需要更大尺寸 \newline 注意升主动脉扩张 \\
\midrule
Type 1嵴主导型 & VRR sizing \newline 瓣上水平定位 & 后扩张可能性高 \newline 注意嵴位置的束腰 \\
\midrule
Type 1瓣环主导型 & VBR sizing \newline 瓣环水平定位 & 嵴对扩张影响较小 \\
\midrule
Type 2 & VBR sizing为主 \newline 个体化评估 & 解剖复杂 \newline 可能需要更大尺寸 \\
\bottomrule
\end{tabular}
\end{table}

\subsection{研究局限性}

\begin{enumerate}
    \item \textbf{技术框架的局限性}
    \begin{itemize}
        \item 基于单中心经验和文献综述
        \item 缺乏前瞻性随机对照研究验证
        \item VBR vs VRR sizing的优劣缺乏直接比较
        \item 长期结局数据有限
    \end{itemize}

    \item \textbf{测量方法的挑战}
    \begin{itemize}
        \item VRR测量在某些情况下困难(如Type 0 BAV)
        \item 嵴的识别和定位可能存在观察者间差异
        \item 共主导/瓣环主导/嵴主导的界定标准不够明确
        \item 缺乏自动化测量工具
    \end{itemize}

    \item \textbf{THV选择的证据}
    \begin{itemize}
        \item SEV vs BEV在BAV中的对比数据主要来自观察性研究
        \item 缺乏新一代THV(如SAPIEN 3 Ultra, Evolut FX)的BAV专门数据
        \item Edwards已移除BAV注意事项,但临床证据基础不如三叶瓣充分
    \end{itemize}

    \item \textbf{定位策略的细化}
    \begin{itemize}
        \item 瓣上定位的具体深度缺乏量化标准
        \item BEV vs SEV的最佳定位深度可能不同
        \item 定位对长期结局(如冠状动脉通路、redo-TAVR)的影响尚不明确
    \end{itemize}

    \item \textbf{后扩张的决策}
    \begin{itemize}
        \item 后扩张指征和时机缺乏明确标准
        \item 球囊尺寸选择主要基于经验
        \item 后扩张的获益-风险平衡需要更多研究
        \item 不同THV类型的后扩张策略可能不同
    \end{itemize}

    \item \textbf{未涵盖的特殊情况}
    \begin{itemize}
        \item 大瓣环BAV(>30 mm)的策略
        \item 合并升主动脉显著扩张的处理
        \item 冠状动脉异常或低位开口的应对
        \item Valve-in-valve(BioProsthetic BAV)的考虑
    \end{itemize}
\end{enumerate}

\subsection{个人笔记}

\subsubsection{Tarantini方法的核心理念}

Giuseppe Tarantini作为BAV TAVR领域的领先专家,其方法体现了几个重要理念:

\begin{enumerate}
    \item \textbf{BAV异质性的深刻认识}
    \begin{itemize}
        \item 强调"BAV is NOT a MONOLITHIC ENTITY"
        \item 区分功能性BAV(接近三叶瓣)和解剖性BAV(真正的挑战)
        \item 认识到不同BAV亚型需要不同策略
    \end{itemize}

    \item \textbf{系统化的四步骤方法}
    \begin{itemize}
        \item Phenotyping → Sizing → Positioning → Optimization
        \item 每一步都有明确的技术要点和决策依据
        \item 强调术前规划和术中灵活调整相结合
    \end{itemize}

    \item \textbf{基于解剖的sizing策略}
    \begin{itemize}
        \item VBR vs VRR的概念创新性地将瓣环和瓣上结构分开考虑
        \item 共主导/瓣环主导/嵴主导的分类简化了决策
        \item 多参数评估而非依赖单一指标
    \end{itemize}
\end{enumerate}

\subsubsection{VBR vs VRR sizing的实践思考}

这一sizing策略的核心在于识别瓣膜的"有效着陆区":

\begin{itemize}
    \item \textbf{VBR sizing}适用于瓣环可提供充分支撑的情况
    \begin{itemize}
        \item 瓣环相对圆形
        \item 嵴的束腰效应不显著
        \item 类似于经典三叶瓣sizing
    \end{itemize}

    \item \textbf{VRR sizing}适用于嵴主导束腰的情况
    \begin{itemize}
        \item 嵴形成明显的瓣上束腰
        \item 瓣环过大可能导致瓣膜在瓣上水平受压
        \item 需要接受较小的瓣膜尺寸以适应瓣上结构
    \end{itemize}
\end{itemize}

实践中的挑战:
\begin{itemize}
    \item 共主导情况最常见,但VBR=VRR并非绝对等值
    \item 需要结合球囊瓣膜成形术的功能测试
    \item CT测量的准确性和可重复性至关重要
\end{itemize}

\subsubsection{THV类型选择的个人观点}

文献对比了Evolut和Sapien在BAV中的应用,但选择应个体化:

\begin{enumerate}
    \item \textbf{倾向Evolut(SEV)的情况}
    \begin{itemize}
        \item 解剖复杂、sizing不确定性高
        \item 需要术中调整定位
        \item 瓣环较大但瓣上束腰明显
        \item 主动脉根部钙化较重
    \end{itemize}

    \item \textbf{倾向Sapien(BEV)的情况}
    \begin{itemize}
        \item 解剖接近三叶瓣的BAV
        \item 瓣周漏风险高(如钙化不足)
        \item 需要保留冠状动脉通路
        \item 考虑未来redo-TAVR的可行性
    \end{itemize}

    \item \textbf{实际应用}
    \begin{itemize}
        \item Evolut在欧洲获得CE认证后在BAV中应用更广
        \item 可重新定位特性在复杂解剖中优势明显
        \item 但Sapien在经验丰富的中心同样可取得良好结果
    \end{itemize}
\end{enumerate}

\subsubsection{定位策略的关键点}

瓣膜定位深度是影响结局的关键因素:

\begin{table}[h]
\centering
\caption{不同定位深度的影响}
\label{tab:implantation_depth_effects}
\begin{tabular}{p{3cm}p{5cm}p{5cm}}
\toprule
\textbf{定位深度} & \textbf{潜在优势} & \textbf{潜在风险} \\
\midrule
较深(瓣环或瓣下) & 更好的锚定 \newline 降低脱位风险 \newline 更好的瓣周密封 & 传导阻滞风险增加 \newline 瓣膜血流动力学可能受影响 \newline LVOT阻塞风险(极少) \\
\midrule
适中(瓣环水平) & 平衡的锚定和血流动力学 \newline 标准定位 & 需要准确识别瓣环平面 \\
\midrule
较浅(瓣上) & 更好的血流动力学 \newline 降低传导阻滞 \newline 便于冠状动脉通路 & 脱位风险增加 \newline 瓣膜-患者不匹配 \newline 影响redo-TAVR \\
\bottomrule
\end{tabular}
\end{table}

\subsubsection{后扩张的艺术与科学}

后扩张在BAV TAVR中特别重要,但需要谨慎:

\begin{itemize}
    \item \textbf{明确指征}
    \begin{itemize}
        \item 透视显示明显扩张不全或不对称
        \item 跨瓣压差>20 mmHg(对于自膨胀瓣)
        \item 中度以上瓣周漏
    \end{itemize}

    \item \textbf{球囊选择}
    \begin{itemize}
        \item 不超过瓣架束腰直径(表\ref{tab:post_dilatation_strategy})
        \item 考虑使用较小球囊多次扩张
        \item 非顺应性球囊优于半顺应性球囊
    \end{itemize}

    \item \textbf{风险控制}
    \begin{itemize}
        \item 避免对球扩瓣进行后扩张(主动脉损伤风险高)
        \item 在嵴型BAV中,局部后扩张可能优于全周扩张
        \item 扩张后需重新评估压差和瓣周漏
    \end{itemize}
\end{itemize}

\subsubsection{未来研究方向}

\begin{enumerate}
    \item \textbf{标准化和自动化}
    \begin{itemize}
        \item 开发自动化的VBR和VRR测量工具
        \item 建立sizing算法和决策支持系统
        \item 利用人工智能优化BAV表型分析
    \end{itemize}

    \item \textbf{新一代THV}
    \begin{itemize}
        \item 评估专门为BAV设计的THV
        \item 研究不同THV设计对BAV结局的影响
        \item 探索可调节瓣膜在BAV中的应用
    \end{itemize}

    \item \textbf{长期结局}
    \begin{itemize}
        \item BAV TAVR的耐久性数据
        \item Redo-TAVR的可行性和结局
        \item 升主动脉病变的进展
    \end{itemize}

    \item \textbf{影像技术}
    \begin{itemize}
        \item 术中融合影像的应用
        \item 3D打印模型在术前规划中的价值
        \item 实时超声或CT引导的THV释放
    \end{itemize}
\end{enumerate}

\subsubsection{与其他专家方法的比较}

Tarantini的方法与其他BAV TAVR专家(如Jilaihawy, Chen等)相比:

\begin{itemize}
    \item \textbf{优势}
    \begin{itemize}
        \item 系统化的四步骤框架易于学习和应用
        \item VBR vs VRR的概念清晰
        \item 强调术后优化,而非仅关注释放技术
    \end{itemize}

    \item \textbf{可能的改进}
    \begin{itemize}
        \item 共主导/瓣环主导/嵴主导的量化标准可进一步细化
        \item 不同THV类型的策略差异可更明确
        \item 可纳入更多的临床结局数据支持
    \end{itemize}
\end{itemize}

\subsubsection{临床实践建议}

基于Tarantini的框架,建议BAV TAVR的实践者:

\begin{enumerate}
    \item 投入充分时间进行术前CT分析
    \item 熟练掌握VBR和VRR的测量技术
    \item 积累不同BAV亚型的经验
    \item 准备多种THV尺寸和后扩张球囊
    \item 建立多学科团队讨论复杂病例
    \item 系统收集数据评估自身中心的结局
\end{enumerate}

Tarantini的技术框架代表了当前BAV TAVR的最佳实践,但仍在不断演进。随着经验积累和技术进步,未来可能会有更精细的个体化策略。

\section{极端复杂解剖下生物瓣膜破裂联合套索技术的瓣中瓣TAVR}
\label{sec:03_015_viv_navitor_small_annulus}

% ============================================
% 文献信息
% ============================================
\subsection{文献信息}

\begin{itemize}
    \item \textbf{标题}: Valve-in-Valve TAVR with Bioprosthetic Valve Fracture and Snaring Technique in Extremely Challenging Anatomy
    \item \textbf{作者}: Ju Han Kim, MD, PhD; Seok Oh, MD, PhD
    \item \textbf{机构}: International St. Mary's Hospital, Catholic Kwandong University (韩国)
    \item \textbf{会议}: TCT 2025 (Transcatheter Cardiovascular Therapeutics)
    \item \textbf{发表}: Korean Circulation Journal. 2025 Sep;55(9):855-857
    \item \textbf{PDF文件名}: 03\_015\_viv\_navitor\_small\_annulus.pdf
    \item \textbf{文献类型}: 病例报告/会议演讲
\end{itemize}

\subsection{研究背景}

\subsubsection{病例介绍}

本病例报告了一例技术极其复杂的瓣中瓣TAVR(VIV-TAVR)手术,患者具有双重极端挑战性解剖特征。

\textbf{患者基本信息}:
\begin{itemize}
    \item 88岁韩国女性患者
    \item 既往接受主动脉瓣置换术(AVR)
    \item 既往植入21mm Carpentier-Edwards PERIMOUNT Magna Ease生物瓣膜退化
    \item 合并症:高血压
\end{itemize}

\textbf{术前评估}:
\begin{itemize}
    \item \textbf{超声心动图}:
    \begin{itemize}
        \item LVEF 57.3\%
        \item 重度主动脉瓣狭窄(AS)
        \item 中度主动脉瓣反流(AR)
        \item 生物瓣膜功能障碍
    \end{itemize}

    \item \textbf{血流动力学参数}:
    \begin{itemize}
        \item 有效瓣口面积(EOA):0.75 cm²
        \item 平均跨瓣压差:54.0 mmHg
        \item 主动脉瓣峰值流速:4.63 m/s
        \item 提示显著的瓣膜-患者不匹配(PPM)
    \end{itemize}
\end{itemize}

\subsubsection{极端解剖挑战}

\textbf{两大极端解剖特征}:

\begin{enumerate}
    \item \textbf{超小瓣环}(Small Annulus):
    \begin{itemize}
        \item 外科生物瓣膜标称尺寸:21mm
        \item CT测量真实内径:仅19.0mm
        \item 属于极小瓣环范畴
    \end{itemize}

    \item \textbf{极端水平主动脉成角}(Extreme Horizontal Angulation):
    \begin{itemize}
        \item 升主动脉水平倾斜角度:97°
        \item 严重影响器械输送和对位
        \item 解剖扭曲程度罕见
    \end{itemize}
\end{enumerate}

\textbf{临床挑战}:
\begin{itemize}
    \item 小瓣环限制瓣膜选择和尺寸
    \item 极端成角增加输送失败风险
    \item 既往瓣膜框架可能干扰新瓣膜通过
    \item 需要在血流动力学优化和技术可行性之间权衡
\end{itemize}

\subsection{主要手术策略与技术}

\subsubsection{瓣膜选择的决策}

\textbf{小瓣环的瓣膜选择困境}:

\begin{table}[h]
\centering
\caption{小瓣环VIV-TAVR瓣膜选择考虑}
\label{tab:valve_selection_small_viv}
\begin{tabular}{p{3cm}p{5.5cm}p{5.5cm}}
\toprule
\textbf{瓣膜类型} & \textbf{优势} & \textbf{劣势} \\
\midrule
瓣上型 & \multicolumn{1}{p{5.5cm}}{• 更大的有效瓣口面积 \newline • 更低的残余压差 \newline • 小瓣环的常规首选} & \multicolumn{1}{p{5.5cm}}{• 输送系统通常较硬 \newline • 极端成角可能无法输送 \newline • 对位困难} \\
\midrule
瓣内型 & \multicolumn{1}{p{5.5cm}}{• 输送系统更灵活 \newline • 低轮廓设计 \newline • 更易通过扭曲解剖} & \multicolumn{1}{p{5.5cm}}{• 有效瓣口面积相对较小 \newline • 可能残余压差较高 \newline • 小瓣环非首选} \\
\bottomrule
\end{tabular}
\end{table}

\textbf{本病例的选择}:
\begin{itemize}
    \item 选用23mm Navitor瓣膜(Abbott,瓣内型球囊扩张瓣膜)
    \item \textbf{主要考虑}:FlexNav输送系统的灵活性和低轮廓设计
    \item 判断器械输送成功是首要挑战,超过追求最优血流动力学
    \item 计划通过生物瓣膜破裂(BVF)优化瓣口面积,补偿瓣内型瓣膜的劣势
\end{itemize}

\subsubsection{生物瓣膜破裂策略}

\textbf{BVF时机选择}:

\begin{table}[h]
\centering
\caption{VIV-TAVR中生物瓣膜破裂时机的对比}
\label{tab:bvf_timing}
\begin{tabular}{p{3cm}p{5.5cm}p{5.5cm}}
\toprule
\textbf{时机} & \textbf{优势} & \textbf{风险} \\
\midrule
TAVR后破裂 & \multicolumn{1}{p{5.5cm}}{• 当前共识推荐 \newline • 降低急性主动脉瓣反流风险 \newline • 已有瓣膜支撑保护} & \multicolumn{1}{p{5.5cm}}{• 破裂可能不充分 \newline • 已部署瓣膜限制扩张空间} \\
\midrule
TAVR前破裂 & \multicolumn{1}{p{5.5cm}}{• 确保小而僵硬的外科瓣膜框架完全扩张 \newline • 改善扭曲解剖中的器械输送性 \newline • 为新瓣膜创造更大空间} & \multicolumn{1}{p{5.5cm}}{• 短暂的主动脉瓣反流 \newline • 破裂后未立即植入的风险} \\
\bottomrule
\end{tabular}
\end{table}

\textbf{本病例选择}:
\begin{itemize}
    \item 采用\textbf{TAVR前破裂}策略
    \item 理由:确保19mm的小而僵硬的外科瓣膜框架完全扩张
    \item 改善严重成角主动脉中的瓣膜输送性
\end{itemize}

\subsubsection{手术步骤与套索技术}

\textbf{手术关键步骤}:

\begin{enumerate}
    \item \textbf{脑保护装置}:
    \begin{itemize}
        \item 放置SENTINEL脑栓塞保护装置(Boston Scientific)
        \item 预防手术相关脑栓塞
    \end{itemize}

    \item \textbf{生物瓣膜破裂}:
    \begin{itemize}
        \item 使用球囊对既往21mm生物瓣膜进行破裂
        \item 目标:充分扩张瓣环,优化后续瓣膜植入空间
    \end{itemize}

    \item \textbf{瓣膜输送尝试与失败}:
    \begin{itemize}
        \item 尝试输送23mm Navitor瓣膜
        \item \textbf{输送失败}:极端成角主动脉和既往瓣膜框架干扰
        \item 瓣膜无法通过并到达目标位置
    \end{itemize}

    \item \textbf{同侧套索技术(Ipsilateral Snare Technique)}:
    \begin{itemize}
        \item 使用25mm Amplatz Goose Neck套索(Medtronic)
        \item 套索预装在FlexNav输送系统上
        \item 套索和瓣膜输送系统同时进入升主动脉
        \item \textbf{关键操作}:通过轻柔的外部牵引力实现同轴对位
        \item 成功引导瓣膜通过退化的生物瓣膜
    \end{itemize}

    \item \textbf{瓣膜部署}:
    \begin{itemize}
        \item 23mm Navitor瓣膜成功部署
        \item 无并发症
        \item 位置和扩张良好
    \end{itemize}
\end{enumerate}

\subsection{结果}

\subsubsection{血流动力学结果}

\textbf{术前与术后对比}:

\begin{table}[h]
\centering
\caption{VIV-TAVR前后血流动力学参数对比}
\label{tab:hemodynamic_outcomes_viv}
\begin{tabular}{lccc}
\toprule
\textbf{参数} & \textbf{术前} & \textbf{术后} & \textbf{改善幅度} \\
\midrule
有效瓣口面积(EOA) & 0.75 cm² & 1.63 cm² & +117\% \\
平均跨瓣压差 & 54.0 mmHg & 11.7 mmHg & -78\% \\
主动脉瓣峰值流速 & 4.63 m/s & 2.40 m/s & -48\% \\
\bottomrule
\end{tabular}
\end{table}

\textbf{结果分析}:
\begin{itemize}
    \item \textbf{显著的血流动力学改善}:
    \begin{itemize}
        \item EOA增加一倍以上(0.75 → 1.63 cm²)
        \item 平均压差降低78\%(54.0 → 11.7 mmHg)
        \item 峰值流速显著降低(4.63 → 2.40 m/s)
    \end{itemize}

    \item \textbf{瓣内型瓣膜的优异表现}:
    \begin{itemize}
        \item 尽管选用瓣内型Navitor,仍实现了优异的血流动力学结果
        \item 生物瓣膜破裂策略有效扩大了有效瓣口面积
        \item 证明在极端解剖下,技术可行性优先的策略是正确的
    \end{itemize}

    \item \textbf{解决了瓣膜-患者不匹配}:
    \begin{itemize}
        \item 术前存在严重PPM(EOA 0.75 cm²)
        \item 术后EOA 1.63 cm²,已无显著PPM
        \item 残余压差在可接受范围内
    \end{itemize}
\end{itemize}

\subsubsection{手术安全性}

\begin{itemize}
    \item 瓣膜部署成功,无并发症
    \item 无主动脉瓣反流
    \item 无传导阻滞或其他即刻并发症
    \item 套索技术安全有效
\end{itemize}

\subsection{结论}

\subsubsection{主要结论}

\begin{enumerate}
    \item \textbf{极端复杂解剖VIV-TAVR可行}:
    \begin{itemize}
        \item 即使在19mm超小瓣环和97°极端成角的情况下
        \item 通过综合策略可以实现成功
        \item 需要个体化的手术规划
    \end{itemize}

    \item \textbf{瓣膜选择需平衡多重因素}:
    \begin{itemize}
        \item 血流动力学优化(倾向瓣上型)
        \item 技术可行性(可能需要瓣内型)
        \item 在极端解剖下,输送成功是首要目标
    \end{itemize}

    \item \textbf{生物瓣膜破裂时机应个体化}:
    \begin{itemize}
        \item 传统推荐术后破裂
        \item 极端解剖(小瓣环+严重成角)可考虑术前破裂
        \item 目标:改善输送性和确保充分扩张
    \end{itemize}

    \item \textbf{套索技术是有效的救援策略}:
    \begin{itemize}
        \item 当标准输送失败时
        \item 同侧套索技术可实现同轴对位
        \item 使瓣膜通过极端扭曲的解剖结构
    \end{itemize}

    \item \textbf{生物瓣膜破裂联合套索技术的可行性}:
    \begin{itemize}
        \item 两种技术的联合应用是安全的
        \item 可作为解剖极端挑战性病例的备选方案
        \item 扩展了VIV-TAVR的适应症范围
    \end{itemize}
\end{enumerate}

\subsection{临床启示}

\subsubsection{对极端挑战性VIV-TAVR的指导}

\textbf{术前评估与规划}:

\begin{enumerate}
    \item \textbf{全面的影像评估}:
    \begin{itemize}
        \item 精确测量既往外科瓣膜的真实内径
        \item 评估主动脉成角和扭曲程度
        \item 评估钙化分布和瓣膜框架特征
        \item 模拟器械输送路径
    \end{itemize}

    \item \textbf{多方案准备}:
    \begin{itemize}
        \item 准备多种瓣膜类型和尺寸
        \item 准备救援器械(套索、额外导丝等)
        \item 制定主要策略和备选方案
        \item 预见可能的技术困难
    \end{itemize}

    \item \textbf{个体化决策框架}:
    \begin{itemize}
        \item 评估"输送成功"vs"血流动力学最优"的优先级
        \item 极端解剖下,输送成功可能是限制性因素
        \item 瓣内型瓣膜+BVF可能优于无法输送的瓣上型瓣膜
    \end{itemize}
\end{enumerate}

\textbf{技术要点}:

\begin{enumerate}
    \item \textbf{生物瓣膜破裂技术}:
    \begin{itemize}
        \item 标准:推荐术后破裂
        \item 特殊情况(超小+僵硬+严重成角):考虑术前破裂
        \item 需要心脏外科团队备台
        \item 快速应对可能的急性反流
    \end{itemize}

    \item \textbf{套索辅助输送}:
    \begin{itemize}
        \item 适应症:标准输送失败的极端扭曲解剖
        \item 同侧套索技术较对侧更简便
        \item 套索预装在输送系统上
        \item 轻柔牵引,避免过度用力
        \item 实现同轴对位和顺利通过
    \end{itemize}

    \item \textbf{器械选择}:
    \begin{itemize}
        \item Navitor瓣膜:FlexNav系统灵活性好
        \item Amplatz Goose Neck套索:标准救援工具
        \item 脑保护装置:复杂操作时推荐使用
    \end{itemize}
\end{enumerate}

\subsubsection{适用人群识别}

\textbf{需要考虑这些复杂策略的患者}:

\begin{itemize}
    \item \textbf{解剖特征}:
    \begin{itemize}
        \item 超小外科瓣膜(≤21mm,真实内径<20mm)
        \item 升主动脉严重成角(>80-90°)
        \item 水平主动脉或极端扭曲
        \item 僵硬的外科瓣膜框架
    \end{itemize}

    \item \textbf{临床考虑}:
    \begin{itemize}
        \item 高龄,外科手术风险极高
        \item 严重PPM导致症状明显
        \item 生物瓣膜退化伴血流动力学恶化
        \item 无其他治疗选择
    \end{itemize}

    \item \textbf{技术考虑}:
    \begin{itemize}
        \item 预期标准输送可能失败
        \item 有经验的团队和完善的设备
        \item 心脏外科团队备台支持
    \end{itemize}
\end{itemize}

\subsubsection{对亚洲人群的特殊意义}

\begin{enumerate}
    \item \textbf{亚洲患者的解剖特点}:
    \begin{itemize}
        \item 体型普遍较小,小瓣环更常见
        \item 既往外科手术常植入较小瓣膜(19-21mm)
        \item VIV-TAVR中PPM风险特别高
        \item 本韩国病例具有代表性
    \end{itemize}

    \item \textbf{技术策略的适用性}:
    \begin{itemize}
        \item 亚洲患者VIV-TAVR更可能遇到输送困难
        \item 需要更灵活的瓣膜选择策略
        \item 生物瓣膜破裂技术尤其重要
        \item 救援技术(套索)需熟练掌握
    \end{itemize}

    \item \textbf{血流动力学优化}:
    \begin{itemize}
        \item 小体型患者对残余压差更敏感
        \item 即使瓣内型瓣膜,通过BVF也可能实现良好结果
        \item 本例术后EOA 1.63 cm²证明可行性
    \end{itemize}
\end{enumerate}

\subsection{研究局限性}

\subsubsection{病例报告的固有局限性}

\begin{enumerate}
    \item \textbf{单一病例经验}:
    \begin{itemize}
        \item 仅为一例病例报告
        \item 无法提供该策略的系统性数据
        \item 成功率、并发症发生率未知
        \item 需要更大样本量的研究
    \end{itemize}

    \item \textbf{缺乏长期随访}:
    \begin{itemize}
        \item 仅报告即刻手术结果
        \item 瓣膜耐久性未知
        \item 长期血流动力学表现未知
        \item 生物瓣膜破裂对长期结果的影响不明
    \end{itemize}

    \item \textbf{缺乏对照}:
    \begin{itemize}
        \item 无法比较不同策略的优劣
        \item 无法确定哪个技术细节最关键
        \item 术前vs术后BVF的直接比较缺失
        \item 套索技术的必要性无对照验证
    \end{itemize}
\end{enumerate}

\subsubsection{技术局限性}

\begin{enumerate}
    \item \textbf{操作者依赖}:
    \begin{itemize}
        \item 套索技术需要高水平技术
        \item 学习曲线陡峭
        \item 可能不是所有中心都能实施
        \item 需要充足的设备和团队支持
    \end{itemize}

    \item \textbf{风险未充分阐述}:
    \begin{itemize}
        \item 术前BVF的反流风险
        \item 套索技术的血管损伤风险
        \item 复杂操作的脑栓塞风险
        \item 失败时的备选方案未讨论
    \end{itemize}

    \item \textbf{适应症界定不明确}:
    \begin{itemize}
        \item 何种程度的成角需要套索技术?
        \item 何时选择术前而非术后BVF?
        \item 何种情况下放弃VIV-TAVR选择外科手术?
        \item 缺乏量化的决策标准
    \end{itemize}
\end{enumerate}

\subsubsection{报告局限性}

\begin{enumerate}
    \item \textbf{技术细节不够详尽}:
    \begin{itemize}
        \item 套索技术的具体操作步骤
        \item 牵引力的大小和方向
        \item 失败后的再尝试策略
        \item 关键的荧光图像角度
    \end{itemize}

    \item \textbf{决策过程简化}:
    \begin{itemize}
        \item 为何最终选择23mm Navitor
        \item 是否考虑过其他瓣膜
        \item 如何权衡风险获益
        \item 患者和家属的参与程度
    \end{itemize}
\end{enumerate}

\subsection{个人笔记}

\subsubsection{关键数字记忆}

\begin{itemize}
    \item 患者年龄:88岁
    \item 既往瓣膜:21mm Carpentier-Edwards PERIMOUNT Magna Ease
    \item 真实瓣环内径:19mm(CT测量)
    \item 主动脉成角:97°(极端水平倾斜)
    \item 术前EOA:0.75 cm²,平均压差54.0 mmHg
    \item 术后EOA:1.63 cm²(+117\%),平均压差11.7 mmHg(-78\%)
    \item 使用瓣膜:23mm Navitor(Abbott)
    \item 套索尺寸:25mm Amplatz Goose Neck(Medtronic)
\end{itemize}

\subsubsection{重要概念}

\begin{description}
    \item[瓣中瓣TAVR(VIV-TAVR)] 在退化的外科生物瓣膜内植入经导管瓣膜,避免再次开胸手术

    \item[超小瓣环] 本例真实内径仅19mm,是VIV-TAVR中最具挑战性的解剖之一

    \item[极端水平成角] 97°的主动脉成角极为罕见,严重影响器械输送和对位

    \item[生物瓣膜破裂(BVF)] 使用高压球囊破裂既往外科瓣膜框架,扩大植入空间,改善血流动力学

    \item[术前vs术后BVF] 传统推荐术后BVF以降低反流风险,但本例采用术前BVF以改善极端解剖下的输送性

    \item[同侧套索技术] 从同一血管入路使用套索辅助瓣膜输送,实现同轴对位和通过困难解剖的救援策略

    \item[瓣上型vs瓣内型瓣膜选择] 小瓣环通常首选瓣上型以优化血流动力学,但极端扭曲解剖可能需要更灵活的瓣内型瓣膜

    \item[技术可行性优先] 在极端复杂解剖下,确保器械成功输送可能比追求最优血流动力学设计更重要
\end{description}

\subsubsection{值得思考的问题}

\begin{enumerate}
    \item \textbf{术前BVF的风险获益如何权衡?}
    \begin{itemize}
        \item 术前破裂可能导致短暂的严重主动脉瓣反流
        \item 如何快速评估患者能否耐受?
        \item 心脏外科团队的备台是否必须?
        \item 何种情况下术前BVF是合理的?
    \end{itemize}

    \item \textbf{套索技术的学习曲线和推广}:
    \begin{itemize}
        \item 需要何种培训和经验?
        \item 是否应该作为VIV-TAVR中心的必备技能?
        \item 如何在模拟环境中练习?
        \item 首次尝试时的安全保障?
    \end{itemize}

    \item \textbf{瓣膜选择的决策算法}:
    \begin{itemize}
        \item 如何量化"输送困难"的风险?
        \item 何种程度的成角需要优先考虑输送系统灵活性?
        \item CT模拟能否预测输送成功率?
        \item 如何在术前制定决策树?
    \end{itemize}

    \item \textbf{为何瓣内型瓣膜实现了如此好的结果?}
    \begin{itemize}
        \item 术后EOA 1.63 cm²相当优异
        \item BVF的贡献有多大?
        \item 是否与Navitor瓣膜的特殊设计有关?
        \item 这种结果是否可复制?
    \end{itemize}

    \item \textbf{该策略在何种人群中最有价值?}
    \begin{itemize}
        \item 亚洲小体型患者
        \item 既往植入小瓣膜(19-21mm)的年轻患者
        \item 主动脉严重扭曲的解剖变异
        \item 需要制定适应症标准
    \end{itemize}

    \item \textbf{与其他救援策略的比较}:
    \begin{itemize}
        \item 套索技术vs其他导丝技巧
        \item 何时考虑外科转换?
        \item 多种救援策略的组合
        \item 成功率和并发症比较
    \end{itemize}

    \item \textbf{长期结果的预期}:
    \begin{itemize}
        \item BVF是否影响瓣膜耐久性?
        \item 破裂的外科瓣膜框架是否增加并发症?
        \item 瓣内型瓣膜在小瓣环中的长期表现
        \item 是否需要更密切的随访?
    \end{itemize}
\end{enumerate}

\subsubsection{临床实践建议}

\textbf{极端复杂VIV-TAVR的准备清单}:

\begin{enumerate}
    \item \textbf{术前准备}:
    \begin{itemize}
        \item 详细CT评估:真实内径、成角、钙化
        \item 3D打印模型或CT模拟输送
        \item 多学科讨论(介入、外科、影像)
        \item 准备多种瓣膜和救援器械
        \item 心脏外科团队备台
    \end{itemize}

    \item \textbf{术中策略}:
    \begin{itemize}
        \item 脑保护装置(复杂操作时强烈推荐)
        \item BVF技术(根据解剖决定时机)
        \item 套索等救援器械随时可用
        \item 灵活调整策略,不拘泥于计划
    \end{itemize}

    \item \textbf{术后随访}:
    \begin{itemize}
        \item 即刻超声评估血流动力学
        \item 密切监测瓣膜位置和功能
        \item 长期随访瓣膜耐久性
        \item 积累经验完善方案
    \end{itemize}
\end{enumerate}

\subsubsection{对学习和教学的启示}

\begin{enumerate}
    \item \textbf{复杂病例的教学价值}:
    \begin{itemize}
        \item 展示真实世界的技术挑战
        \item 救援策略的重要性
        \item 个体化决策的必要性
        \item 多种技术的联合应用
    \end{itemize}

    \item \textbf{技能培训重点}:
    \begin{itemize}
        \item VIV-TAVR特有技术(BVF等)
        \item 救援器械的熟练使用
        \item 极端解剖的术前评估
        \item 快速应变和决策能力
    \end{itemize}

    \item \textbf{中心能力建设}:
    \begin{itemize}
        \item 建立复杂病例团队
        \item 配备完整的救援器械库
        \item 心脏外科紧密合作
        \item 积累经验持续改进
    \end{itemize}
\end{enumerate}

\subsubsection{研究方向建议}

\begin{enumerate}
    \item \textbf{需要的研究}:
    \begin{itemize}
        \item BVF时机(术前vs术后)的比较研究
        \item 套索辅助技术的系统性研究
        \item 极端解剖VIV-TAVR的登记研究
        \item 长期结果和瓣膜耐久性数据
    \end{itemize}

    \item \textbf{技术改进方向}:
    \begin{itemize}
        \item 更灵活的瓣膜输送系统
        \item 小瓣环专用瓣膜设计
        \item 术前CT模拟预测输送成功率
        \item 标准化的救援策略流程
    \end{itemize}
\end{enumerate}

\section{复杂解剖和既往治疗主动脉中TAVR的风险降低策略}
\label{sec:03_017_derisk_challenging_anatomy}

\subsection{文献信息}

\begin{itemize}
    \item \textbf{标题}: How Do I Derisk TAVR in Challenging or Previously Treated Aortic Anatomy?
    \item \textbf{中文标题}: 如何在具有挑战性或既往治疗过的主动脉解剖结构中降低TAVR风险?
    \item \textbf{作者}: Oscar Mendiz MD. MSCAI
    \item \textbf{机构}: Favaloro Foundation University Hospital, Interventional Cardiologist Department, Buenos Aires, Argentina
    \item \textbf{会议}: CRF TCT (Transcatheter Cardiovascular Therapeutics) - Favaloro Cardiovascular Symposium
    \item \textbf{PDF文件名}: 03\_017\_derisk\_challenging\_anatomy.pdf
    \item \textbf{文献类型}: 会议演讲幻灯片
\end{itemize}

\subsection{研究背景}

随着TAVR技术的革命性发展,其适应证已从高危患者扩展至低危患者和复杂解剖结构。本演讲旨在系统性地介绍在面对挑战性或既往治疗过的主动脉解剖结构时,如何通过先进的影像学技术和手术策略来降低TAVR的风险。

\subsubsection{学习目标}

\begin{itemize}
    \item 识别TAVR的关键解剖学挑战
    \item 理解既往治疗过解剖结构的特殊考虑
    \item 探索先进影像学和手术策略以降低TAVR风险
\end{itemize}

\subsubsection{常见解剖学障碍}

TAVR成功的主要解剖学障碍包括:

\begin{enumerate}
    \item \textbf{小瓣环/瓣中瓣(VIV)}: 患者-假体不匹配(PPM)风险、冠状动脉阻塞风险
    \item \textbf{大瓣环}: 瓣周漏(PVL)风险、瓣膜脱位风险
    \item \textbf{极度成角主动脉(水平主动脉)}: 导管操作困难、THV位置错误
    \item \textbf{重度钙化}:
    \begin{itemize}
        \item 瓣环/LVOT钙化: 瓣环破裂、PVL、传导阻滞风险
        \item 主动脉壁/弓部钙化: 卒中风险
    \end{itemize}
    \item \textbf{瓦氏窦浅/冠状动脉距离短}: 高冠状动脉阻塞风险
    \item \textbf{胸主动脉或腹主动脉瘤}: 主动脉破裂或栓塞风险
    \item \textbf{既往主动脉治疗(EVAR-TEVAR-移植物)}: 血管并发症
    \item \textbf{股动脉小且钙化}: 血管并发症
\end{enumerate}

\subsection{主要研究发现}

\subsubsection{克服困难的策略}

\begin{table}[h]
\centering
\caption{复杂解剖TAVR的风险降低策略}
\label{tab:derisk_strategies}
\begin{tabular}{p{5cm}p{9cm}}
\toprule
\textbf{策略} & \textbf{具体措施} \\
\midrule
通路管理 & 替代通路(锁骨下、直接主动脉)用于挑战性股动脉解剖;精细闭合技术 \\
\midrule
瓣膜选择 & 选择最佳瓣膜类型(球囊扩张式vs.自膨胀式)、瓣上vs.瓣内设计、裙边高度、植入深度 \\
\midrule
冠状动脉保护 & 主动支架植入、烟囱支架、BASILICA(生物瓣膜主动脉瓣叶故意撕裂预防医源性冠状动脉阻塞)技术 \\
\midrule
瓣环准备 & 预扩张(在某些解剖结构中最小化或避免)vs.重度钙化瓣叶的碎石术等先进技术 \\
\bottomrule
\end{tabular}
\end{table}

\subsubsection{替代血管通路}

\paragraph{经腋动脉TAVR}

当股动脉通路不适合时,经腋动脉入路成为重要的替代选择。演讲展示了一例81岁男性患者的病例:

\begin{table}[h]
\centering
\caption{经腋动脉TAVR病例血管测量}
\label{tab:transaxillary_measurements}
\begin{tabular}{lc}
\toprule
\textbf{血管部位} & \textbf{直径(mm)} \\
\midrule
腹主动脉远端(AIPD) & 11.6 \\
腹主动脉左端(AIED) & 11.6 \\
腹主动脉近端(AIPI) & 11.6 \\
腹主动脉左内(AIEI) & 9.7 \\
腹股沟动脉(AFI) & 14.2 \\
升主动脉(ASCI) & 11.4 / 8.5 \\
\bottomrule
\end{tabular}
\end{table}

该病例使用18Fr鞘管植入EvoluteR 29瓣膜,手术成功完成。术后10天因泌尿系统感染出院,无血管并发症,心电图显示左束支传导阻滞(LBBB)。1个月随访:无呼吸困难,无胸痛,超声心动图无反流,平均跨瓣压差8 mmHg。

\paragraph{经颈动脉TAVR}

经颈动脉通路是另一种重要的替代方案。演讲展示了CT和超声引导下的颈动脉评估技术,测量颈动脉直径约6.14-6.26 mm,确保通路的可行性和安全性。

手术步骤包括:
\begin{itemize}
    \item 超声和CT引导下评估颈动脉
    \item 直接穿刺颈动脉
    \item 植入瓣膜
    \item 精细血管修复技术
\end{itemize}

术后结果良好,出院时无血管并发症,心电图显示LBBB,1个月随访无反流,平均压差8 mmHg。

\subsubsection{改善血管通路技术}

\paragraph{血管内碎石术(IVL)用于钙化股动脉}

对于严重钙化的股动脉,演讲介绍了使用血管内碎石术(Intravascular Lithotripsy, IVL)的创新方法:

\begin{itemize}
    \item 使用IVL球囊 7.0×60mm
    \item 在TAVR通路准备前进行股动脉钙化修饰
    \item 显著改善血管通路的安全性
    \item 减少血管并发症发生率
\end{itemize}

该技术通过声波能量破碎钙化斑块,使血管更具弹性,便于大口径鞘管的安全植入。

\subsubsection{极度成角主动脉的处理}

演讲展示了一例主动脉成角60°的病例:

\begin{table}[h]
\centering
\caption{极度成角主动脉病例影像学参数}
\label{tab:angulated_aorta}
\begin{tabular}{lc}
\toprule
\textbf{参数} & \textbf{测量值} \\
\midrule
LAO角度 & 3° \\
尾侧角度 & 6° \\
主动脉成角度数 & 60° \\
RAO角度 & 68° \\
尾侧角度 & 15° \\
左冠状动脉高度 & 13.6 mm \\
右冠状动脉高度 & 13 mm \\
\bottomrule
\end{tabular}
\end{table}

\textbf{处理策略}:
\begin{itemize}
    \item 优先选择球囊扩张式瓣膜(BE THV),具有更好的定位控制
    \item 使用特殊的导管操作技术
    \item 精确的影像学引导
    \item 避免过度预扩张,减少瓣环损伤风险
\end{itemize}

\subsubsection{主动脉狭窄合并腹主动脉疾病}

\paragraph{TAVR(VIV)联合腹主动脉PTA}

演讲展示了主动脉瓣狭窄合并腹主动脉疾病的复杂病例:

\begin{table}[h]
\centering
\caption{VIV TAVR联合腹主动脉介入病例瓣环参数}
\label{tab:viv_aortic_pla}
\begin{tabular}{lc}
\toprule
\textbf{参数} & \textbf{测量值} \\
\midrule
瓣环最小直径 & 21.5 mm \\
瓣环最大直径 & 23.6 mm \\
瓣环平均直径 & 22.6 mm \\
瓣环衍生直径 & 22.3 mm \\
瓣环周径衍生直径 & 22.5 mm \\
瓣环面积 & 389.5 mm² \\
瓣环周径 & 70.5 mm \\
左冠状动脉高度 & 6.5 mm \\
垂直平面距离 & 7.1 mm \\
\bottomrule
\end{tabular}
\end{table}

腹主动脉多处测量显示严重钙化狭窄,直径范围7.0-12.2 mm。手术策略包括:
\begin{enumerate}
    \item 先行腹主动脉球囊扩张(PTA)
    \item 完成VIV TAVR
    \item 确保血管通路安全
    \item 术后密切监测
\end{enumerate}

\paragraph{TAVR联合EVAR}

对于同时存在主动脉瓣狭窄和腹主动脉瘤的患者,联合手术成为必要选择:

\begin{table}[h]
\centering
\caption{TAVR+EVAR联合手术病例参数}
\label{tab:tavr_evar_combined}
\begin{tabular}{lc}
\toprule
\textbf{参数} & \textbf{测量值} \\
\midrule
LAO角度 & 0° \\
尾侧角度 & 2° \\
RAO角度 & 69° \\
尾侧角度 & 27° \\
腹主动脉瘤直径 & 81.3 mm \\
腹主动脉瘤面积 & 815 mm² \\
\bottomrule
\end{tabular}
\end{table}

手术策略:
\begin{itemize}
    \item 可选择分期手术或同期手术
    \item 本例采用先TAVR后EVAR的策略
    \item 使用相同股动脉通路
    \item 减少患者手术次数和总体风险
\end{itemize}

\subsubsection{既往EVAR患者的TAVR}

对于既往接受过EVAR治疗的患者,进行TAVR面临特殊挑战:
\begin{itemize}
    \item 血管通路受限于既往移植物
    \item 需要仔细评估移植物的完整性
    \item 可能需要替代通路
    \item 导管操作可能更加困难
\end{itemize}

\subsubsection{血管闭合后再次穿刺}

演讲展示了经皮血管闭合后再次穿刺的技术要点:

\begin{itemize}
    \item 使用超声引导精确定位
    \item 避开既往闭合装置的位置
    \item 评估血管壁的完整性
    \item 选择合适的穿刺点
    \item 使用小号导丝和扩张器逐步扩张
\end{itemize}

影像学显示:
\begin{itemize}
    \item 可以识别既往闭合装置
    \item 评估局部钙化情况
    \item 确认穿刺点的安全性
    \item 术后验证闭合效果
\end{itemize}

\subsection{结论}

本演讲系统性地介绍了在复杂或既往治疗过的主动脉解剖结构中降低TAVR风险的策略。主要结论包括:

\begin{enumerate}
    \item \textbf{多模态影像学评估至关重要}: CTA是术前规划的基础,可精确测量血管直径、评估钙化程度、规划通路策略

    \item \textbf{替代通路扩展了TAVR的适应证}: 经腋动脉和经颈动脉通路为股动脉通路不适合的患者提供了安全有效的选择

    \item \textbf{创新技术改善血管通路}: 血管内碎石术等技术可以安全地修饰钙化血管,提高通路的可行性

    \item \textbf{复杂解剖需要个体化策略}: 极度成角主动脉、小瓣环、大瓣环等特殊解剖需要针对性的瓣膜选择和植入技术

    \item \textbf{联合手术可行且安全}: TAVR可与EVAR、腹主动脉PTA等手术联合进行,减少患者总体风险

    \item \textbf{既往血管干预不是禁忌证}: 既往EVAR或血管闭合的患者仍可安全接受TAVR,但需要更仔细的规划
\end{enumerate}

\subsection{临床启示}

\subsubsection{对临床实践的指导}

\begin{enumerate}
    \item \textbf{术前评估}:
    \begin{itemize}
        \item 所有TAVR候选者必须进行高质量CTA评估
        \item 评估应包括主动脉瓣、主动脉根部、整个主动脉及股动脉
        \item 对于复杂解剖,应考虑MDT讨论
    \end{itemize}

    \item \textbf{通路规划}:
    \begin{itemize}
        \item 首选股动脉通路,但应有替代通路的准备
        \item 腋动脉和颈动脉是安全有效的替代选择
        \item 对于严重钙化股动脉,考虑使用IVL预处理
    \end{itemize}

    \item \textbf{瓣膜选择}:
    \begin{itemize}
        \item 根据解剖特点选择球囊扩张式或自膨胀式瓣膜
        \item 极度成角主动脉优选球囊扩张式瓣膜
        \item 考虑瓣膜设计对冠状动脉的影响
    \end{itemize}

    \item \textbf{联合手术}:
    \begin{itemize}
        \item 对于合并腹主动脉疾病的患者,评估联合治疗的可行性
        \item 制定明确的手术顺序和应急预案
        \item 平衡各项手术的风险和收益
    \end{itemize}
\end{enumerate}

\subsubsection{对未来研究的启示}

\begin{enumerate}
    \item \textbf{技术创新}:
    \begin{itemize}
        \item 开发更小口径的瓣膜输送系统
        \item 改进瓣膜设计,适应更广泛的解剖变异
        \item 探索新的血管准备技术
    \end{itemize}

    \item \textbf{影像学进展}:
    \begin{itemize}
        \item 开发更精确的影像学评估工具
        \item 整合AI技术辅助术前规划
        \item 改进术中影像引导技术
    \end{itemize}

    \item \textbf{临床研究}:
    \begin{itemize}
        \item 建立复杂解剖TAVR的注册登记研究
        \item 比较不同替代通路的长期结果
        \item 评估联合手术的最佳策略
    \end{itemize}
\end{enumerate}

\subsection{研究局限性}

\begin{enumerate}
    \item \textbf{病例展示性质}: 本演讲主要通过病例展示介绍技术,缺乏系统的临床研究数据支持

    \item \textbf{单中心经验}: 所展示的病例来自单一中心,可能存在选择偏倚,结果的普遍适用性需要更多研究验证

    \item \textbf{缺乏长期随访}: 演讲中展示的病例多为短期随访结果(1个月),长期疗效和并发症尚不明确

    \item \textbf{学习曲线考虑}: 替代通路和复杂技术需要专门的培训和经验积累,不同中心的成功率可能存在差异

    \item \textbf{成本效益未评估}: 未讨论这些复杂技术和联合手术的成本效益,这在临床决策中也是重要考虑因素

    \item \textbf{并发症详细数据缺失}: 虽然展示了成功病例,但未系统报告并发症发生率和处理策略
\end{enumerate}

\subsection{个人笔记}

\subsubsection{关键数字记忆}

\begin{itemize}
    \item \textbf{60°}: 极度成角主动脉的角度,这种情况下需要特殊的瓣膜选择和植入技术
    \item \textbf{81.3 mm}: TAVR+EVAR病例中腹主动脉瘤的直径,提示需要联合治疗
    \item \textbf{18Fr}: 经腋动脉植入EvoluteR 29所需鞘管大小
    \item \textbf{6.14-6.26 mm}: 经颈动脉通路的颈动脉直径范围
    \item \textbf{7.0×60mm}: IVL球囊规格,用于钙化股动脉的预处理
    \item \textbf{8 mmHg}: 替代通路TAVR术后1个月的平均跨瓣压差,显示良好的血流动力学结果
\end{itemize}

\subsubsection{重要概念}

\begin{enumerate}
    \item \textbf{替代通路的重要性}:
    \begin{itemize}
        \item 经腋动脉和经颈动脉通路是股动脉通路的有效替代
        \item 需要专门的培训和经验
        \item 结果与股动脉通路相当
    \end{itemize}

    \item \textbf{血管内碎石术(IVL)}:
    \begin{itemize}
        \item 创新的钙化修饰技术
        \item 可应用于股动脉通路准备
        \item 提高复杂钙化病例的成功率
    \end{itemize}

    \item \textbf{BASILICA技术}:
    \begin{itemize}
        \item 预防VIV TAVR中冠状动脉阻塞的创新方法
        \item 通过故意撕裂生物瓣膜瓣叶实现
        \item 扩展了TAVR在高风险冠状动脉阻塞病例中的应用
    \end{itemize}

    \item \textbf{联合手术策略}:
    \begin{itemize}
        \item TAVR可与EVAR、腹主动脉PTA联合进行
        \item 需要careful的术前规划和手术顺序安排
        \item 可减少患者的手术次数和总体风险
    \end{itemize}

    \item \textbf{球囊扩张式vs.自膨胀式瓣膜选择}:
    \begin{itemize}
        \item 极度成角主动脉优选球囊扩张式
        \item 球囊扩张式提供更精确的定位控制
        \item 自膨胀式在某些大瓣环病例中可能更合适
    \end{itemize}
\end{enumerate}

\subsubsection{值得思考的问题}

\begin{enumerate}
    \item \textbf{如何选择最佳通路}:
    \begin{itemize}
        \item 在何种情况下应该首选替代通路而不是尝试股动脉通路?
        \item 不同替代通路(腋动脉vs.颈动脉)的选择标准是什么?
        \item IVL预处理是否应该成为钙化股动脉的常规策略?
    \end{itemize}

    \item \textbf{联合手术的时机}:
    \begin{itemize}
        \item TAVR和EVAR应该同期进行还是分期进行?
        \item 如何平衡两项手术的风险?
        \item 哪些患者不适合联合手术?
    \end{itemize}

    \item \textbf{复杂病例的学习曲线}:
    \begin{itemize}
        \item 如何系统化地培训替代通路技术?
        \item 新手术者需要完成多少例才能达到熟练程度?
        \item 如何在保证患者安全的前提下推广这些技术?
    \end{itemize}

    \item \textbf{长期结果}:
    \begin{itemize}
        \item 替代通路TAVR的长期疗效如何?
        \item 联合手术患者的长期生存和生活质量如何?
        \item 是否存在特定的远期并发症?
    \end{itemize}

    \item \textbf{技术普及性}:
    \begin{itemize}
        \item 这些复杂技术在非高容量中心的可行性如何?
        \item 如何建立转诊网络确保复杂病例得到适当治疗?
        \item 成本效益如何影响技术的推广应用?
    \end{itemize}

    \item \textbf{未来发展方向}:
    \begin{itemize}
        \item 瓣膜设计如何进一步优化以适应复杂解剖?
        \item AI和机器学习能否辅助术前规划和风险评估?
        \item 哪些新技术可能进一步降低复杂病例的风险?
    \end{itemize}
\end{enumerate}
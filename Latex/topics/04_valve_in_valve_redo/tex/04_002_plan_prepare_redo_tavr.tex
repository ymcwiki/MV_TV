\section{如何规划和准备再次TAVR(Redo-TAVR)?}
\label{sec:04_002_plan_prepare_redo_tavr}

% ============================================
% 文献信息
% ============================================
\subsection{文献信息}

\begin{itemize}
    \item \textbf{标题}: How Do I Plan and Prepare for Redo TAVR?
    \item \textbf{作者}: Oskar Angerås, MD PhD
    \item \textbf{机构}: Sahlgrenska University Hospital, Gothenburg, Sweden
    \item \textbf{会议}: TCT (Transcatheter Cardiovascular Therapeutics)
    \item \textbf{PDF文件名}: how-do-i-plan-and-prepare-for-redo-tavr.pdf
    \item \textbf{文献类型}: 会议演讲/教学讲座
    \item \textbf{利益冲突披露}:
    \begin{itemize}
        \item 研究支持:Abbott, Medtronic
        \item 咨询/讲课费:Abbott, Medtronic, Meril, Novo Nordisk
        \item 股权:Texray
    \end{itemize}
\end{itemize}

\subsection{研究背景}

\subsubsection{Redo-TAVR的现状}

随着TAVR技术的普及和患者年龄趋势的年轻化,瓣膜失败后的再次干预逐渐成为临床实践中需要面对的重要问题。本演讲基于瑞典SWEDEHEART注册研究的数据和临床经验,探讨如何系统性地规划和准备redo-TAVR手术。

\textbf{欧洲redo-TAVR现状}(基于SWEDEHEART数据):

\begin{itemize}
    \item Redo-TAVI在欧洲仍然相对罕见
    \item 从2008年至2024年,瑞典TAVR手术比例逐年增加
    \item 2008年TAVR比例约1\%,到2024年已达到约6.5\%
    \item TAVI-TAVI(瓣中瓣)手术比例开始出现,但仍然较少(粉色部分)
    \item 患者平均年龄在80岁左右,过去十年基本稳定
\end{itemize}

\subsubsection{核心观察}

\textbf{瓣膜"比患者活得更长"现象}:

在当前的临床实践中,绝大多数接受TAVR的患者在其生命周期内不会经历瓣膜失败,即:
\begin{itemize}
    \item 瓣膜的耐久性通常超过患者的预期寿命
    \item 这意味着\textbf{在选择初始瓣膜时,耐久性是最重要的考虑因素}
    \item 特别是对于年轻患者(<65岁),初始瓣膜选择更加关键
    \item 需要在初始TAVR时就考虑未来可能的redo-TAVR或外科手术选择
\end{itemize}

\subsection{Redo-TAVR的规划原则}

\subsubsection{风险优先级体系}

在规划redo-TAVR时,必须按照以下优先级评估和管理风险:

\begin{enumerate}
    \item \textbf{冠状动脉闭塞}(Coronary occlusion)
    \begin{itemize}
        \item \textcolor{red}{术中风险!}
        \item 最高优先级,直接威胁手术安全性
        \item 需要术前详细评估冠状动脉闭塞风险
        \item 可能需要预防性冠状动脉保护策略
    \end{itemize}

    \item \textbf{冠状动脉通路}(Coronary access)
    \begin{itemize}
        \item 术后风险
        \item 影响患者未来PCI治疗的可能性
        \item 需要确保redo-TAVR后仍能进行冠状动脉介入
    \end{itemize}

    \item \textbf{长期耐久性}(Long-time durability)
    \begin{itemize}
        \item 长期术后风险
        \item 影响第二次瓣膜的使用寿命
        \item 选择合适的瓣膜型号和尺寸
    \end{itemize}
\end{enumerate}

\subsubsection{了解初始瓣膜的关键要素}

在规划redo-TAVR之前,必须充分了解患者的初始瓣膜特征:

\textbf{必须掌握的信息}:
\begin{itemize}
    \item \textbf{联合线位置}(Commissures):确定瓣叶的解剖位置
    \item \textbf{新裙边}(Neo-skirt):预测术后新裙边的高度和范围
    \item \textbf{支架框架网格}(Stent frame cells):了解支架的结构特征
    \item \textbf{定位标志}(Landmarks):用于第二个瓣膜的精确定位
\end{itemize}

\subsection{初始瓣膜分类系统}

\subsubsection{经导管主动脉瓣分类与标志}

根据JACC Cardiovascular Interventions (Vol. 17, No. 14, 2024)发表的分类系统,不同类型的TAVR瓣膜有不同的特征和定位标志:

\textbf{框架高度分类}:
\begin{itemize}
    \item \textbf{短框架}(Short Frame):Sapien XT/Sapien 3, Myval, Lotus, Portico/Navilor, CoreValve/Evolut
    \item \textbf{高框架}(Tall Frame):ACURATE, Allegra
\end{itemize}

\textbf{TAV设计特征}:
\begin{itemize}
    \item \textbf{球囊扩张式}(Balloon-Expandable):Sapien系列
    \item \textbf{自膨胀式}(Self-Expanding):CoreValve/Evolut, ACURATE
    \item \textbf{机械扩张式}(Mechanical-expanding):Myval, Lotus, Portico/Navilor
\end{itemize}

\textbf{不同瓣膜的定位标志}:

\begin{table}[h]
\centering
\caption{不同TAVR瓣膜的关键定位标志}
\label{tab:tavr_valve_landmarks}
\begin{tabular}{p{3cm}p{5cm}p{6cm}}
\toprule
\textbf{瓣膜类型} & \textbf{瓣叶顶部/最低点标志} & \textbf{重要荧光镜标志} \\
\midrule
Sapien XT/3 & 联合线标签顶部 / 联合线标签上方2-4mm & 联合线标签顶部 \\
Myval & 联合线标签顶部 / 流入上方2-4mm & 联合线标签顶部 \\
Lotus & "调节叉"顶部 / 流入 & 流入和"调节叉"底部 \\
Portico/Navilor & 联合线标签底部 / Node 1 & 联合线标签底部 \\
CoreValve/Evolut & Node 6 (Evolut 23) / Node 3 & Node 3\&5 (Evolut 23) \\
ACURATE & 联合线标签底部 / 上冠底部 & 上冠底部 \& Node 1 \\
Allegra & Node 5 / Node 3 & Node 1至Node 5 \\
\bottomrule
\end{tabular}
\end{table}

\subsubsection{瓣膜兼容性矩阵}

不同瓣膜之间进行valve-in-valve的兼容性:

\begin{itemize}
    \item \textbf{Sapien 3}:与所有瓣膜兼容
    \item \textbf{Myval}:与所有瓣膜兼容
    \item \textbf{Navitor}:与Lotus兼容,与Portico/Navilor和Allegra正在研究中
    \item \textbf{Evolut}:与Sapien 3兼容,与Lotus和Allegra正在研究中
    \item \textbf{ACURATE}:与Sapien 3兼容,与Lotus、CoreValve/Evolut和Allegra\textcolor{red}{不兼容}
    \item \textbf{Allegra}:与Sapien 3和Myval兼容,与其他瓣膜正在研究中
\end{itemize}

\subsection{关键测量指标}

\subsubsection{核心测量参数定义}

使用\textbf{心电门控心脏CT}(ECG gated cardiac CT)进行以下关键测量:

\begin{enumerate}
    \item \textbf{冠状动脉风险平面}(Coronary risk plane)
    \begin{itemize}
        \item 定义:从支架框架底部到冠状动脉开口底部的距离
        \item 临床意义:评估redo-TAVR时冠状动脉闭塞的风险
        \item 测量方法:在矢状面重建图像上测量垂直距离
    \end{itemize}

    \item \textbf{新裙边平面}(Neoskirt plane)
    \begin{itemize}
        \item 定义:从支架框架底部到预测的完整新裙边高度
        \item 临床意义:预测redo-TAVR后新裙边可能覆盖的范围
        \item 重要性:新裙边可能阻碍冠状动脉通路
    \end{itemize}

    \item \textbf{STJ平面}(ST-junction plane)
    \begin{itemize}
        \item 定义:从支架框架底部到窦管交界的高度
        \item 临床意义:评估瓣膜定位与解剖结构的关系
    \end{itemize}

    \item \textbf{VTC}(Valve to Coronary distance)
    \begin{itemize}
        \item 定义:瓣膜到冠状动脉的距离
        \item 临床意义:直接评估冠状动脉闭塞风险
        \item 关键阈值:通常需要>4mm才相对安全
    \end{itemize}

    \item \textbf{VTSTJ}(Valve to ST-Junction distance)
    \begin{itemize}
        \item 定义:瓣膜到窦管交界的距离(\textbf{实际上是面积!})
        \item 测量方式:在横断面上测量
        \item 临床意义:评估冠状动脉窦的空间
    \end{itemize}
\end{enumerate}

\subsubsection{CT测量实践}

\textbf{测量技术要点}:

\begin{itemize}
    \item 使用心电门控CT确保图像质量
    \item 在收缩期和舒张期都进行测量
    \item 矢状面重建用于垂直距离测量
    \item 横断面用于VTC和VTSTJ测量
    \item 三维重建帮助理解空间关系
\end{itemize}

\textbf{关键测量平面可视化}:

左图显示球囊扩张式瓣膜(如Sapien),右图显示自膨胀式瓣膜(如Evolut):
\begin{itemize}
    \item 蓝色箭头:冠状动脉风险平面的垂直距离
    \item 粉色/紫色阴影区域:新裙边预测覆盖范围
    \item 虚线:STJ平面
    \item VTA和VTC:横向距离测量
    \item VTSTJ:在流出道水平测量
\end{itemize}

\subsection{临床病例分析}

\subsubsection{病例1:73岁女性,合并冠心病}

\textbf{患者信息}:
\begin{itemize}
    \item 年龄:73岁,女性
    \item 初始瓣膜:Evolut 26 mm失败
    \item 合并症:冠状动脉疾病
\end{itemize}

\textbf{术前评估}(CT测量):
\begin{itemize}
    \item 左冠状动脉高度:26.3 mm
    \item 右冠状动脉高度:27.6 mm
    \item 瓣膜窦最大高度:35.1 mm
    \item \textbf{LCA距离(VTC)}:7.9 mm
    \item \textbf{RCA距离(VTC)}:8.7 mm
\end{itemize}

\textbf{治疗策略}:
\begin{itemize}
    \item 选择:\textbf{Navitor 25 mm} valve-in-valve
    \item 理由:VTC距离充足(>7mm),冠状动脉闭塞风险低
    \item 结果:手术成功,无冠状动脉闭塞
\end{itemize}

\subsubsection{病例2:76岁女性}

\textbf{患者信息}:
\begin{itemize}
    \item 年龄:76岁,女性
    \item 初始瓣膜:Evolut 29 mm失败
\end{itemize}

\textbf{术前评估}(CT测量):
\begin{itemize}
    \item 右冠状动脉高度:15.7 mm
    \item \textbf{VTC距离较短,冠状动脉闭塞风险较高}
\end{itemize}

\textbf{治疗策略}:
\begin{itemize}
    \item 选择:\textbf{MyVal 24.5 mm} valve-in-valve
    \item 关键策略:\textbf{预防性冠状动脉保护}
    \item 方法:在RCA和LCA预先植入导丝和保护装置
    \item 结果:成功完成手术,冠状动脉通畅
\end{itemize}

\textbf{技术要点}:
\begin{itemize}
    \item 手术过程中保持冠状动脉导丝在位
    \item 使用荧光镜监测瓣膜定位与冠状动脉的关系
    \item 如发生冠状动脉受压,可立即进行冠状动脉支架植入
\end{itemize}

\subsubsection{病例3:65岁男性,复杂病史}

\textbf{患者信息}:
\begin{itemize}
    \item 年龄:65岁,男性
    \item 初始瓣膜:Evolut 26 mm失败
    \item 既往手术史:Mitroflow 23(外科生物瓣)
    \item 既往介入史:左主干冠状动脉支架
\end{itemize}

\textbf{术前评估}(CT测量):
\begin{itemize}
    \item 左冠状动脉高度:17.4 mm
    \item \textbf{VTC距离极短}
    \item 左主干支架可能影响冠状动脉血流
\end{itemize}

\textbf{治疗决策}:
\begin{itemize}
    \item 选择:\textbf{外科手术}(而非redo-TAVR)
    \item 理由:
    \begin{itemize}
        \item \textcolor{red}{冠状动脉闭塞风险极高}
        \item 左主干支架存在,redo-TAVR可能导致支架变形或血流受阻
        \item 未来需要额外介入(PCI和可能的第三次瓣膜干预)的风险高
    \end{itemize}
\end{itemize}

\textbf{外科手术结果}:
\begin{itemize}
    \item 手术风险:冠状动脉闭塞风险
    \item 术后并发症严重:
    \begin{itemize}
        \item 呼吸机支持:10天
        \item ICU住院时间:15天
        \item 总体恢复过程漫长
    \end{itemize}
    \item 切除的瓣膜显示:复杂的多层结构(Mitroflow + Evolut)
\end{itemize}

\textbf{病例启示}:
\begin{itemize}
    \item 并非所有瓣膜失败都适合redo-TAVR
    \item 冠状动脉解剖是决策的关键因素
    \item 多次瓣膜干预增加了复杂性
    \item 外科手术虽然创伤大,但在某些高风险病例中仍是更安全的选择
    \item 年轻患者(65岁)需要考虑长期策略
\end{itemize}

\subsection{主要研究发现}

\subsubsection{Redo-TAVR的风险分层}

基于临床经验和上述病例,redo-TAVR的风险可以分层如下:

\textbf{低风险}(适合redo-TAVR):
\begin{itemize}
    \item VTC >8 mm
    \item 冠状动脉高度>15 mm
    \item 无复杂冠状动脉疾病
    \item 初始瓣膜框架高度适中
\end{itemize}

\textbf{中风险}(可考虑redo-TAVR + 冠状动脉保护):
\begin{itemize}
    \item VTC 4-8 mm
    \item 冠状动脉高度12-15 mm
    \item 可能的冠状动脉受压
    \item 需要预防性冠状动脉保护策略
\end{itemize}

\textbf{高风险}(考虑外科手术):
\begin{itemize}
    \item VTC <4 mm
    \item 冠状动脉高度<12 mm
    \item 既往冠状动脉支架(特别是左主干)
    \item 多层瓣膜结构
    \item 需要未来多次干预的年轻患者
\end{itemize}

\subsubsection{冠状动脉保护策略}

当决定进行中风险redo-TAVR时,可采用以下保护策略:

\begin{enumerate}
    \item \textbf{预防性导丝保护}
    \begin{itemize}
        \item 在redo-TAVR前在RCA和LCA放置导丝
        \item 准备好球囊和支架
        \item 如发生冠状动脉受压,可立即介入
    \end{itemize}

    \item \textbf{Chimney技术}(烟囱技术)
    \begin{itemize}
        \item 在TAVR同时或之后立即植入冠状动脉支架
        \item 支架从主动脉根部延伸至冠状动脉
        \item 保证冠状动脉血流通畅
    \end{itemize}

    \item \textbf{BASILICA技术}
    \begin{itemize}
        \item 术前裂开瓣叶以创造冠状动脉血流通道
        \item 适用于某些特定瓣膜类型
        \item 技术要求高
    \end{itemize}
\end{enumerate}

\subsubsection{瓣膜选择策略}

在redo-TAVR中,瓣膜选择需要考虑:

\textbf{短框架 vs 高框架}:
\begin{itemize}
    \item 短框架瓣膜(如Sapien, Myval):较少影响冠状动脉
    \item 高框架瓣膜(如Evolut):可能增加冠状动脉闭塞风险
\end{itemize}

\textbf{球囊扩张 vs 自膨胀}:
\begin{itemize}
    \item 球囊扩张式:定位更精确,可控性强
    \item 自膨胀式:适应性好,但可能向上迁移
\end{itemize}

\textbf{新裙边高度预测}:
\begin{itemize}
    \item 不同瓣膜的新裙边高度不同
    \item 需要预测新裙边是否会覆盖冠状动脉开口
    \item 选择新裙边较低的瓣膜可能更安全
\end{itemize}

\subsection{结论}

\subsubsection{核心要点总结}

\textbf{Take-home Messages}:

\begin{enumerate}
    \item \textbf{初始瓣膜耐久性至关重要}
    \begin{itemize}
        \item 避免redo-TAVR的最佳策略是选择耐久性好的初始瓣膜
        \item 对年轻患者尤为重要
        \item 需要平衡当前手术适应性与长期耐久性
    \end{itemize}

    \item \textbf{术前规划极其重要}
    \begin{itemize}
        \item 必须进行详细的CT评估
        \item 了解初始瓣膜的所有特征
        \item 预测redo-TAVR的潜在风险
        \item 制定应急预案(冠状动脉保护、备用方案)
    \end{itemize}

    \item \textbf{风险优先级}
    \begin{itemize}
        \item 第一优先:避免冠状动脉闭塞(immediate safety)
        \item 第二优先:保证未来冠状动脉通路(future intervention)
        \item 第三优先:长期耐久性(long-term outcomes)
    \end{itemize}
\end{enumerate}

\subsubsection{决策算法}

\textbf{Redo-TAVR决策流程}:

\begin{enumerate}
    \item 评估初始瓣膜失败原因(结构性退化 vs 非结构性)
    \item 进行详细CT评估(VTC, 冠状动脉高度, 新裙边预测)
    \item 根据冠状动脉风险分层:
    \begin{itemize}
        \item 低风险 → 标准redo-TAVR
        \item 中风险 → redo-TAVR + 冠状动脉保护
        \item 高风险 → 考虑外科手术或其他治疗
    \end{itemize}
    \item 选择合适的第二个瓣膜(考虑兼容性、框架高度、新裙边)
    \item 制定详细手术计划和应急方案
    \item 多学科团队讨论(Heart Team)
\end{enumerate}

\subsection{临床启示}

\subsubsection{对临床实践的启示}

\begin{enumerate}
    \item \textbf{初始TAVR时的前瞻性思考}
    \begin{itemize}
        \item 对于年轻患者,选择瓣膜时要考虑未来可能的redo-TAVR
        \item 选择框架高度适中、新裙边较低的瓣膜
        \item 避免过度高位植入
        \item 记录详细的瓣膜信息和解剖数据
    \end{itemize}

    \item \textbf{建立标准化评估流程}
    \begin{itemize}
        \item 所有redo-TAVR候选者都应进行心电门控CT
        \item 使用标准化测量方案(VTC, VTSTJ, 冠状动脉高度等)
        \item 建立本中心的风险分层标准
        \item 记录和学习每个病例
    \end{itemize}

    \item \textbf{多学科团队合作}
    \begin{itemize}
        \item 影像科:高质量CT和精确测量
        \item 介入心脏病学:redo-TAVR技术和冠状动脉保护
        \item 心脏外科:高风险病例的外科备选方案
        \item 共同决策,个体化治疗
    \end{itemize}

    \item \textbf{技术储备}
    \begin{itemize}
        \item 掌握冠状动脉保护技术(导丝保护、Chimney、BASILICA等)
        \item 熟悉不同瓣膜的特性和兼容性
        \item 准备完善的应急设备和团队
        \item 必要时考虑转诊至高容量中心
    \end{itemize}
\end{enumerate}

\subsubsection{未来研究方向}

\begin{itemize}
    \item 长期随访数据:不同瓣膜组合的耐久性
    \item 冠状动脉保护技术的有效性和安全性
    \item AI辅助的CT测量和风险预测
    \item 新一代瓣膜设计(考虑redo-TAVR友好性)
    \item 最佳瓣膜组合的前瞻性研究
    \item Valve-in-valve血流动力学的长期影响
\end{itemize}

\subsection{研究局限性}

\begin{enumerate}
    \item 本演讲基于单中心经验和SWEDEHEART注册数据,可能存在选择偏倚
    \item Redo-TAVR仍然是相对罕见的操作,长期数据有限
    \item 不同瓣膜组合的系统性比较数据缺乏
    \item 冠状动脉闭塞风险的阈值(如VTC cutoff)尚无统一标准
    \item 病例数量有限,难以进行大规模统计分析
    \item 随访时间相对较短,长期结果尚不明确
    \item 技术快速发展,现有经验可能需要不断更新
\end{enumerate}

\subsection{个人笔记}

\subsubsection{关键数字记忆}

\textbf{风险评估阈值}:
\begin{itemize}
    \item VTC >8 mm:低风险
    \item VTC 4-8 mm:中风险,需要冠状动脉保护
    \item VTC <4 mm:高风险,考虑外科手术
    \item 冠状动脉高度 >15 mm:相对安全
    \item 冠状动脉高度 <12 mm:高风险
\end{itemize}

\textbf{病例数据}:
\begin{itemize}
    \item 病例1:73岁女性,VTC = 7.9/8.7 mm,成功redo-TAVR(Navitor 25mm)
    \item 病例2:76岁女性,VTC较短,redo-TAVR + 冠状动脉保护(MyVal 24.5mm)
    \item 病例3:65岁男性,VTC极短 + 左主干支架,选择外科手术,术后ICU 15天
\end{itemize}

\textbf{欧洲现状}:
\begin{itemize}
    \item Redo-TAVI比例仍然较低(<1\%)
    \item 患者平均年龄约80岁
    \item 瓣膜通常"比患者活得更长"
\end{itemize}

\subsubsection{重要概念}

\begin{description}
    \item[VTC (Valve to Coronary distance)] 瓣膜到冠状动脉距离,redo-TAVR最关键的测量指标,直接决定冠状动脉闭塞风险

    \item[VTSTJ (Valve to ST-Junction)] 瓣膜到窦管交界距离,实际上是在横断面测量的面积,评估冠状动脉窦的空间

    \item[Neoskirt plane] 新裙边平面,预测redo-TAVR后新裙边可能覆盖的高度,可能影响冠状动脉通路

    \item[Coronary risk plane] 冠状动脉风险平面,从支架框架底部到冠状动脉开口底部的距离

    \item[Chimney technique] 烟囱技术,一种冠状动脉保护策略,在TAVR同时植入冠状动脉支架以保证血流

    \item[Valve-in-Valve compatibility] 瓣中瓣兼容性,不同瓣膜组合的可行性,Sapien 3和Myval与大多数瓣膜兼容

    \item[Index valve] 初始瓣膜,第一次植入的TAVR瓣膜,其耐久性和特征决定了未来redo-TAVR的可行性
\end{description}

\subsubsection{临床决策要点}

\textbf{何时选择redo-TAVR}:
\begin{itemize}
    \item 瓣膜结构性退化(SVD)
    \item VTC距离充足(>4mm,最好>8mm)
    \item 冠状动脉解剖适宜
    \item 患者适合介入治疗
    \item 初始瓣膜与第二个瓣膜兼容
\end{itemize}

\textbf{何时选择外科手术}:
\begin{itemize}
    \item VTC极短(<4mm)
    \item 复杂冠状动脉疾病(特别是左主干支架)
    \item 多层瓣膜结构
    \item 年轻患者需要长期策略
    \item 患者外科风险可接受
\end{itemize}

\textbf{何时使用冠状动脉保护}:
\begin{itemize}
    \item VTC在边缘范围(4-8mm)
    \item 新裙边可能覆盖冠状动脉开口
    \item 冠状动脉开口位置低
    \item 高框架瓣膜valve-in-valve
    \item 有冠心病史,未来需要PCI
\end{itemize}

\subsubsection{对中国临床实践的启示}

\begin{enumerate}
    \item \textbf{准备迎接redo-TAVR时代}
    \begin{itemize}
        \item 中国TAVR起步较晚,但发展迅速
        \item 随着低危患者扩展,年轻患者增多
        \item 未来5-10年将面临更多redo-TAVR需求
        \item 需要提前建立标准化评估和治疗流程
    \end{itemize}

    \item \textbf{初始瓣膜选择的长远考虑}
    \begin{itemize}
        \item 不仅考虑当前手术成功率
        \item 重视瓣膜耐久性数据
        \item 选择国际认可、长期随访数据充分的瓣膜
        \item 避免过度追求新技术而忽视长期结果
    \end{itemize}

    \item \textbf{建立多学科协作模式}
    \begin{itemize}
        \item 心脏影像科的CT评估能力至关重要
        \item 介入与外科需要紧密合作
        \item 建立Heart Team决策机制
        \item 复杂病例考虑转诊至高容量中心
    \end{itemize}

    \item \textbf{技术培训和能力建设}
    \begin{itemize}
        \item 培养标准化CT测量技能
        \item 学习冠状动脉保护技术
        \item 熟悉不同瓣膜特性和兼容性
        \item 建立应急预案和团队
    \end{itemize}
\end{enumerate}

\subsubsection{值得进一步思考的问题}

\begin{enumerate}
    \item \textbf{如何平衡初始瓣膜选择的多个因素?}
    \begin{itemize}
        \item 当前手术成功率 vs 长期耐久性
        \item 解剖适配性 vs redo-TAVR友好性
        \item 成本效益 vs 临床结果
        \item 需要个体化评估和决策
    \end{itemize}

    \item \textbf{VTC的安全阈值到底是多少?}
    \begin{itemize}
        \item 文献报道从4mm到8mm不等
        \item 可能与瓣膜类型、新裙边高度相关
        \item 需要更多循证医学证据
        \item 建立个体化风险预测模型
    \end{itemize}

    \item \textbf{冠状动脉保护的最佳策略是什么?}
    \begin{itemize}
        \item 预防性保护 vs 救援性介入
        \item Chimney vs BASILICA vs 其他技术
        \item 成本效益如何?
        \item 需要前瞻性随机对照研究
    \end{itemize}

    \item \textbf{外科手术在redo时代的角色?}
    \begin{itemize}
        \item 哪些患者绝对适合外科而非redo-TAVR?
        \item 如何改进外科技术降低并发症?
        \item Ross手术等年轻患者替代方案?
        \item 需要重新评估外科在现代瓣膜治疗中的地位
    \end{itemize}

    \item \textbf{如何设计更"redo-friendly"的瓣膜?}
    \begin{itemize}
        \item 较低的框架高度
        \item 更少的新裙边形成
        \item 清晰的定位标志
        \item 联合线对准功能
        \item 未来瓣膜设计需要考虑这些因素
    \end{itemize}
\end{enumerate}

\subsubsection{关键表格总结}

\textbf{Redo-TAVR风险分层}:

\begin{table}[h]
\centering
\caption{Redo-TAVR风险分层与治疗策略}
\label{tab:redo_tavr_risk_stratification}
\begin{tabular}{p{2.5cm}p{4cm}p{4cm}p{4cm}}
\toprule
\textbf{风险等级} & \textbf{VTC距离} & \textbf{冠状动脉高度} & \textbf{推荐策略} \\
\midrule
低风险 & >8 mm & >15 mm & 标准redo-TAVR \\
中风险 & 4-8 mm & 12-15 mm & Redo-TAVR + 冠状动脉保护 \\
高风险 & <4 mm & <12 mm & 考虑外科手术 \\
\bottomrule
\end{tabular}
\end{table}

\textbf{三个优先级}:

\begin{table}[h]
\centering
\caption{Redo-TAVR规划的优先级体系}
\label{tab:redo_tavr_priorities}
\begin{tabular}{clp{8cm}}
\toprule
\textbf{优先级} & \textbf{风险类型} & \textbf{临床意义} \\
\midrule
1 & 冠状动脉闭塞 & 术中风险,直接威胁手术安全,可能导致心肌梗死和死亡 \\
2 & 冠状动脉通路 & 术后风险,影响未来PCI治疗的可能性 \\
3 & 长期耐久性 & 长期风险,影响第二次瓣膜的使用寿命 \\
\bottomrule
\end{tabular}
\end{table}

\section{如何治疗失败的Evolut瓣膜:病例示例}
\label{sec:04_003_treat_failed_evolut}

% ============================================
% 文献信息
% ============================================
\subsection{文献信息}

\begin{itemize}
    \item \textbf{标题}: How I Treat a Failed Evolut: Case Example
    \item \textbf{作者}: Adnan K. Chhatriwalla, MD, FACC
    \item \textbf{机构}: Saint Luke's Mid America Heart Institute; University of Missouri - Kansas City
    \item \textbf{会议/期刊}: 会议演讲材料
    \item \textbf{PDF文件名}: how-i-treat-a-failed-evolut.pdf
    \item \textbf{文献类型}: 会议演讲/病例分享/手术技术指导
\end{itemize}

\subsection{研究背景}

\subsubsection{瓣中瓣(Valve-in-Valve)治疗需求}

随着TAVR手术的普及和患者生存期延长,TAVR瓣膜失败后的再次干预成为日益重要的临床问题。Evolut系列瓣膜(Medtronic公司的自膨胀瓣膜)失败后,可以选择:
\begin{itemize}
    \item 外科手术瓣膜置换
    \item 瓣中瓣(TAV in TAV)TAVR手术
\end{itemize}

\textbf{Sapien in Evolut的特殊性}:
\begin{itemize}
    \item Sapien系列(Edwards公司)为球囊扩张式瓣膜
    \item Evolut系列为自膨胀式瓣膜
    \item 两种不同机制的瓣膜组合带来独特的技术挑战
\end{itemize}

\subsubsection{关键临床问题}

\textbf{1. 冠状动脉阻塞风险}

TAV in TAV手术的主要并发症之一是冠状动脉阻塞,原因包括:
\begin{itemize}
    \item 原始瓣膜瓣叶被推向主动脉窦
    \item 第二个瓣膜位置过高
    \item 新瓣膜支架遮挡冠状动脉开口
    \item Neoskirt(新裙边)形成导致的间隙封闭
\end{itemize}

\textbf{2. 瓣膜功能优化}

需要在以下目标间取得平衡:
\begin{itemize}
    \item 治疗原有瓣膜狭窄或反流
    \item 避免新的瓣周漏
    \item 保持足够的有效瓣口面积
    \item 避免瓣膜位置过高影响冠状动脉
\end{itemize}

\subsection{主要研究发现}

\subsubsection{Sapien in Evolut的手术策略}

\textbf{手术目标(Procedural Goals)}:

\begin{enumerate}
    \item \textbf{避免冠状动脉阻塞(Avoid Coronary Obstruction)}
    \begin{itemize}
        \item 这是最关键的安全目标
        \item 需要术前详细评估冠状动脉高度
        \item 确定安全的植入位置
    \end{itemize}

    \item \textbf{保留冠状动脉通路(Preserve Coronary Access)}
    \begin{itemize}
        \item 为将来可能的冠脉介入保留通路
        \item 考虑患者长期随访需求
    \end{itemize}

    \item \textbf{确保适当的Sapien功能(Ensure suitable Sapien function)}
    \begin{itemize}
        \item 充分的瓣口面积
        \item 低跨瓣压差
        \item 最小的瓣周漏和反流
    \end{itemize}
\end{enumerate}

\textbf{初始考虑(Initial Considerations)}:

根据原始瓣膜失败机制的不同,策略有所差异:

\begin{itemize}
    \item \textbf{主动脉瓣狭窄(AS)为主}:
    \begin{itemize}
        \item 第二个瓣膜的位置需要治疗Evolut的狭窄部分
        \item 需要覆盖Evolut瓣叶最狭窄的区域
        \item 植入位置相对较低,以确保瓣叶被充分压制
    \end{itemize}

    \item \textbf{主动脉瓣反流(AR)为主}:
    \begin{itemize}
        \item 对Evolut瓣叶的关注较少
        \item 因为没有梗阻需要管理
        \item 重点是密封反流,可以植入位置相对灵活
    \end{itemize}
\end{itemize}

\subsubsection{Neoskirt高度的概念与影响}

\textbf{Neoskirt定义}:

完全密封的neoskirt(新裙边)从原始CV/Evolut THV瓣膜的\textbf{流入端}延伸至SAPIEN瓣膜的\textbf{流出端}。

\textbf{Neoskirt的临床意义}:
\begin{itemize}
    \item Neoskirt高度影响冠状动脉阻塞风险
    \item 过高的neoskirt可能遮挡冠状动脉开口
    \item Neoskirt高度由Sapien植入位置决定
\end{itemize}

\textbf{Evolut瓣膜的Node系统}:

Evolut瓣膜有编号的节点(Node)用于定位:
\begin{itemize}
    \item Node 1:流入端(最低位置)
    \item Node 2-5:中间位置
    \item Node 6:流出端(最高位置)
    \item Node 3:瓣叶最窄处(Nadir of leaflet)
\end{itemize}

\subsubsection{Akodad等人研究的关键数据}

本演讲引用了重要研究:\textit{Balloon-Expandable Valve for Treatment of Evolut Valve Failure: Implications on Neoskirt Height and Leaflet Overhang}(JACC Intv 2022; 15:368-377)

\textbf{研究设计}:

使用\textbf{体外模拟实验},测试不同尺寸组合的Sapien 3在Evolut中的表现:
\begin{itemize}
    \item 20mm S3 in 23mm Evolut R
    \item 23mm S3 in 26mm Evolut R
    \item 26mm S3 in 29mm Evolut R
    \item 29mm S3 in 34mm Evolut R
\end{itemize}

每种组合测试S3流出端对准Evolut的三个不同Node(Node 4, 5, 6)。

\textbf{主要测量指标}:
\begin{enumerate}
    \item \textbf{Neoskirt高度}:从Evolut流入端到Sapien流出端的距离
    \item \textbf{瓣叶悬垂(Leaflet Overhang)}:Evolut瓣叶超出Sapien支架的百分比
    \item \textbf{Evolut尺寸变化}:植入Sapien后Evolut支架的径向增加
\end{enumerate}

\textbf{Neoskirt高度和瓣叶悬垂详细数据}:

\begin{table}[h]
\centering
\caption{不同Sapien 3植入位置的Neoskirt高度和瓣叶悬垂}
\label{tab:neoskirt_leaflet_overhang}
\begin{tabular}{lcccccc}
\toprule
\multirow{2}{*}{\textbf{瓣膜组合}} & \multicolumn{2}{c}{\textbf{Node 4}} & \multicolumn{2}{c}{\textbf{Node 5}} & \multicolumn{2}{c}{\textbf{Node 6}} \\
\cmidrule(lr){2-3} \cmidrule(lr){4-5} \cmidrule(lr){6-7}
 & Neoskirt & 悬垂 & Neoskirt & 悬垂 & Neoskirt & 悬垂 \\
 & (mm) & (\%) & (mm) & (\%) & (mm) & (\%) \\
\midrule
20mm S3 in 23mm Evolut R & 16.3 & 90 & 20.7 & 32 & 23.9 & 0 \\
23mm S3 in 26mm Evolut R & 17.1 & 90 & 21.0 & 49 & 23.4 & 9 \\
26mm S3 in 29mm Evolut R & 18.3 & 90 & 20.6 & 39 & 24.7 & 3 \\
29mm S3 in 34mm Evolut R & 19.9 & 94 & 23.0 & 32 & 27.0 & 2 \\
\bottomrule
\end{tabular}
\end{table}

\textbf{Evolut尺寸增加数据}:

\begin{table}[h]
\centering
\caption{植入Sapien 3后Evolut瓣膜的径向扩张}
\label{tab:evolut_expansion}
\begin{tabular}{lccc}
\toprule
\textbf{瓣膜组合} & \textbf{Node 4} & \textbf{Node 5} & \textbf{Node 6} \\
\midrule
20mm S3 in 23mm Evolut R & N6: 0.0mm, N5: 0.5mm, & N6: 0.5mm, N5: 0.3mm, & N6: 0.3mm, N5: 0.7mm, \\
 & N4: 0.0mm, N3: 0.2mm & N4: 0.3mm, N3: 0.3mm & N4: 0.8mm, N3: 0.6mm \\
\midrule
23mm S3 in 26mm Evolut R & N6: 0.7mm, N5: 0.5mm, & N6: 0.8mm, N5: 0.2mm, & N6: 0.7mm, N5: 0.7mm, \\
 & N4: 0.3mm, N3: 0.5mm & N4: 0.1mm, N3: 0.8mm & N4: 0.8mm, N3: 0.8mm \\
\midrule
26mm S3 in 29mm Evolut R & N6: 0.7mm, N5: 0.3mm, & N6: 1.3mm, N5: 1.3mm, & N6: 1.2mm, N5: 1.1mm, \\
 & N4: 0.1mm, N3: 1.7mm & N4: 1.1mm, N3: 1.7mm & N4: 1.0mm, N3: 1.0mm \\
\midrule
29mm S3 in 34mm Evolut R & N6: 0.8mm, N5: 1.3mm, & N6: 1.5mm, N5: 1.5mm, & N6: 2.3mm, N5: 2.3mm, \\
 & N4: 1.1mm, N3: 1.6mm & N4: 1.2mm, N3: 1.5mm & N4: 1.5mm, N3: 1.5mm \\
\bottomrule
\end{tabular}
\end{table}

\textbf{关键发现1}:\textbf{植入位置与Neoskirt高度和瓣叶悬垂的关系}

\begin{itemize}
    \item \textbf{S3流出端对准Node 4(较低位置)}:
    \begin{itemize}
        \item Neoskirt高度最低(16.3-19.9mm)
        \item 瓣叶悬垂最大(90-94\%)
        \item 冠状动脉阻塞风险相对较低
        \item 但瓣叶大量悬垂可能影响血流动力学
    \end{itemize}

    \item \textbf{S3流出端对准Node 6(较高位置)}:
    \begin{itemize}
        \item Neoskirt高度最高(23.9-27.0mm)
        \item 瓣叶悬垂最小(0-9\%)
        \item 血流动力学更好
        \item 但冠状动脉阻塞风险增加
    \end{itemize}

    \item \textbf{S3流出端对准Node 5(中间位置)}:
    \begin{itemize}
        \item 平衡选择
        \item Neoskirt高度中等(20.6-23.0mm)
        \item 瓣叶悬垂中等(32-49\%)
    \end{itemize}
\end{itemize}

\textbf{关键发现2}:\textbf{较低的植入位置可减少neoskirt高度达7.6mm}

这对冠状动脉位置较低的患者具有重要临床意义。

\textbf{关键发现3}:\textbf{植入Sapien导致Evolut尺寸增加}

\begin{itemize}
    \item S3植入会导致Evolut半径增加0-2.5mm
    \item 这种现象在Evolut in Evolut中\textbf{不会}出现
    \item 尺寸增加与球囊扩张的机制有关
\end{itemize}

\subsubsection{Redo-TAVR Sapien 3的血流动力学性能}

\textbf{研究来源}:Sellers S, Meier D, Nigade A, et al. TCT-396 Calcification Patterns in TAVR Explants: Informing Durability and Implications for Reintervention. JACC. 2023 Oct, 82 (17\_Supplement) B158.

\textbf{四种瓣膜组合的详细血流动力学数据}:

\begin{table}[h]
\centering
\caption{Redo-TAVR Sapien 3血流动力学表现}
\label{tab:redo_tavr_hemodynamics}
\begin{tabular}{lccccccccc}
\toprule
\multirow{2}{*}{\textbf{组合}} & \multicolumn{3}{c}{\textbf{EOA (cm²)}} & \multicolumn{2}{c}{\textbf{MG (mmHg)}} & \multicolumn{2}{c}{\textbf{PV (m/s)}} & \textbf{RF} \\
\cmidrule(lr){2-4} \cmidrule(lr){5-6} \cmidrule(lr){7-8} \cmidrule(lr){9-9}
 & Pre & Post & ISO & Pre & Post & Pre & Post & Post (\%) \\
\midrule
20mm S3 in & 0.82 & 1.17 & 0.95 & 56.3 & 28.5 & 5.0 & 3.4 & 7.9 \\
23mm Evolut R & & & & & & & & \\
\midrule
26mm S3 in & 1.10 & 2.16 & 1.60 & 32.7 & 9.5 & 3.8 & 1.9 & 18.9 \\
29mm CoreValve & & & & & & & & \\
\midrule
26mm S3 in & 0.85 & 2.07 & 1.60 & 41.4 & 10.2 & 4.6 & 1.9 & 12.3 \\
29mm Evolut PRO & & & & & & & & \\
\midrule
29mm S3 in & 0.66 & 2.54 & 2.10 & 76.6 & 6.9 & 6.2 & 1.6 & 25.8* \\
34mm Evolut R & & & & & & & & \\
\bottomrule
\end{tabular}
\end{table}

\textbf{注释}:
\begin{itemize}
    \item EOA = 有效瓣口面积(Effective Orifice Area)
    \item MG = 平均跨瓣压差(Mean Gradient)
    \item PV = 峰值流速(Peak Velocity)
    \item RF = 反流分数(Regurgitant Fraction)
    \item ISO accepted = ISO标准接受的最小EOA值
    \item * = 29mm S3 in 34mm Evolut R在Node 6位置时RF为25.8\%(>20\%,需要额外研究)
\end{itemize}

\textbf{血流动力学表现分析}:

\begin{enumerate}
    \item \textbf{有效瓣口面积(EOA)显著改善}:
    \begin{itemize}
        \item 所有组合的术后EOA均达到或超过ISO标准
        \item 20mm S3: 0.82 → 1.17 cm²(增加43\%)
        \item 26mm S3 in CoreValve: 1.10 → 2.16 cm²(增加96\%)
        \item 26mm S3 in Evolut PRO: 0.85 → 2.07 cm²(增加144\%)
        \item 29mm S3: 0.66 → 2.54 cm²(增加285\%)
    \end{itemize}

    \item \textbf{平均跨瓣压差(MG)显著降低}:
    \begin{itemize}
        \item 20mm S3: 56.3 → 28.5 mmHg(降低49\%)
        \item 26mm S3 in CoreValve: 32.7 → 9.5 mmHg(降低71\%)
        \item 26mm S3 in Evolut PRO: 41.4 → 10.2 mmHg(降低75\%)
        \item 29mm S3: 76.6 → 6.9 mmHg(降低91\%)
    \end{itemize}

    \item \textbf{峰值流速(PV)明显下降}:
    \begin{itemize}
        \item 所有组合术后流速均<3.5 m/s
        \item 最大的改善见于29mm S3组合(6.2 → 1.6 m/s)
    \end{itemize}

    \item \textbf{反流分数总体可接受}:
    \begin{itemize}
        \item 三种组合RF <20\%(可接受范围)
        \item 仅29mm S3 in 34mm Evolut R的RF为25.8\%,需要谨慎
        \item 这提示大尺寸瓣膜组合可能需要优化植入位置
    \end{itemize}
\end{enumerate}

\textbf{重要观察}:\textbf{Evolut/CoreValve瓣叶固定现象}

当瓣叶悬垂<40\%且瓣叶已钙化时:
\begin{itemize}
    \item 瓣叶被固定在打开位置
    \item 在整个心动周期中保持静止(运动<10\%)
    \item 这种"固定打开"状态不会明显影响血流动力学
    \item 前提是悬垂不超过40\%
\end{itemize}

\textbf{临床启示}:
\begin{itemize}
    \item 除了在34mm Evolut R中反流分数>20\%的29mm S3在Node 6位置外
    \item \textbf{所有测试的植入位置血流动力学功能均可接受}
    \item 这为临床医生提供了较大的选择空间
    \item 可以根据冠状动脉解剖灵活调整植入位置
\end{itemize}

\subsubsection{Sapien 3支架变形现象}

\textbf{研究发现}:

Sapien 3在Redo-TAVR后出现支架变形(Frame Deformation)很常见,表现为:
\begin{itemize}
    \item 支架未充分扩张(Under-expansion)
    \item 支架直径小于标称直径
    \item 不同节点(Frame Nodes)的扩张程度不一
\end{itemize}

\textbf{影响因素}:

\begin{enumerate}
    \item \textbf{钙化位置}:
    \begin{itemize}
        \item 原始瓣膜钙化分布影响Sapien扩张
        \item 不对称钙化导致不对称扩张
    \end{itemize}

    \item \textbf{TAV尺寸}:
    \begin{itemize}
        \item 较大尺寸组合更容易出现变形
        \item 与Evolut和Sapien尺寸匹配度相关
    \end{itemize}
\end{enumerate}

\textbf{临床意义}:

尽管存在支架变形,但血流动力学功能仍可接受,说明:
\begin{itemize}
    \item 适度的支架变形不影响瓣膜功能
    \item 重要的是瓣叶功能而非支架的完美圆形
    \item 术后评估应关注血流动力学参数而非仅看影像学形态
\end{itemize}

\subsection{病例展示}

\subsubsection{病例基本信息}

\textbf{患者特征}:
\begin{itemize}
    \item \textbf{年龄/性别}:58岁,男性
    \item \textbf{左心室射血分数(LVEF)}:15\%(严重心功能不全)
    \item \textbf{慢性肾脏病(CKD)}
    \item \textbf{肺动脉高压(Pulmonary HTN)}
\end{itemize}

\textbf{TAVR病史}:
\begin{itemize}
    \item 2019年在外院接受34mm Evolut瓣膜植入
    \item 目前表现为\textbf{中度主动脉瓣狭窄(AS)}和\textbf{中度主动脉瓣反流(AI)}
\end{itemize}

\textbf{当前临床状况}:
\begin{itemize}
    \item \textbf{心源性休克}(Cardiogenic shock)
    \item 需要\textbf{正性肌力药物}(Inotropes)支持
    \item 需要\textbf{主动脉内球囊反搏(IABP)}支持
\end{itemize}

\subsubsection{临床挑战}

\textbf{主要问题}:

患者需要左心室辅助装置(LVAD)治疗严重心衰,但存在以下担忧:
\begin{itemize}
    \item LVAD植入后主动脉瓣关闭不全会加重
    \item 持续的AI会影响LVAD效果
    \item 需要在LVAD植入前或同时处理瓣膜问题
\end{itemize}

\textbf{治疗方案选择}:

\textbf{能否进行联合LVAD + TAV in TAV手术?}

这是一个复杂的临床决策,需要考虑:
\begin{enumerate}
    \item 患者能否耐受复合手术
    \item TAV in TAV的技术可行性
    \item 冠状动脉阻塞风险
    \item 术后血流动力学优化
\end{enumerate}

\subsubsection{TAV in TAV手术规划}

使用\textbf{TAV in TAV APP}(Redo TAVR KRUTSCH应用程序)进行详细术前规划:

\textbf{Step 1: Index TAV测量}

\begin{itemize}
    \item \textbf{瓣膜类型}:Medtronic Evolut FX
    \item \textbf{瓣膜尺寸}:34mm
    \item \textbf{瓣膜高度}:45mm
    \item \textbf{瓣膜直径}:34mm
    \item \textbf{内裙高度(Inner Skirt Height)}:14mm
    \item \textbf{原生主动脉瓣环周长}:81.7-94.2mm
\end{itemize}

\textbf{Step 2: 识别冠状动脉风险平面}

通过CT测量冠状动脉开口位置:
\begin{itemize}
    \item \textbf{RCA(右冠状动脉)开口}:位于Node 4水平
    \item \textbf{LCA(左冠状动脉)开口}:位于Node 4水平
    \item \textbf{冠状动脉风险平面(CRP)}:Node 4
\end{itemize}

\textbf{Step 3: 选择第二个TAV设备}

考虑因素:
\begin{itemize}
    \item Evolut FX和Evolut PRO+标记为"USE WITH CAUTION"(谨慎使用)
    \item 不建议Evolut in Evolut(同类瓣膜组合)
    \item \textbf{选择}:SAPIEN 3(球囊扩张式瓣膜)
\end{itemize}

\textbf{Step 4: 选择NSP(New Stent Position)和评估NSP/CRP}

NSP = 新支架植入位置

选项:
\begin{itemize}
    \item Node 6(较高位置)
    \item Node 5(中等位置)
    \item Node 4(较低位置)
    \item Node 3(仅用于AR,对准瓣叶最窄处)
\end{itemize}

\textbf{本例选择}:\textbf{Node 6}

理由:
\begin{itemize}
    \item 减少瓣叶悬垂,改善血流动力学
    \item 冠状动脉位于Node 4,Node 6植入有一定安全距离
    \item 需要详细评估冠状动脉风险
\end{itemize}

\textbf{Step 5: 第二个TAV尺寸选择}

测量Index TAV各节点面积:
\begin{itemize}
    \item Node 6: 418 mm²
    \item Node 5: 411 mm²
    \item Node 4: 413 mm²
    \item Node 3: 479 mm²
    \item Node 2: 570 mm²
    \item Node 1: 656 mm²
\end{itemize}

选择Node 3-6的四个测量值计算:
\begin{itemize}
    \item \textbf{平均面积}:430.3 mm²
    \item \textbf{面积导出直径(Area Derived Diameter)}:23.4 mm
\end{itemize}

根据尺寸表和测量值:
\begin{itemize}
    \item \textbf{选择瓣膜}:26mm SAPIEN 3
    \item 26mm S3标称面积适合430.3 mm²的平均测量值
\end{itemize}

\textbf{Step 6: 冠状动脉风险评估}

\textbf{CT测量虚拟瓣膜对冠状动脉(VTA)参数}:

\begin{itemize}
    \item \textbf{NSP}(新支架位置):Node 6
    \item \textbf{RCA和LCA冠状动脉开口}:均在Node 4
    \item NSP在RCA和LCA上方(Above)
    \item NSP在STJ下方(Below)
\end{itemize}

\textbf{VTA测量值}:
\begin{itemize}
    \item \textbf{VTSTJ}(虚拟到STJ距离):N/A
    \item \textbf{VTAoS}(虚拟到主动脉窦距离):未输入
    \item \textbf{VTC-RCA}(虚拟到RCA距离):3.9 mm
    \item \textbf{VTC-LCA}(虚拟到LCA距离):5.3 mm
\end{itemize}

\textbf{Step 7: 总结报告}

\textbf{手术计划总结}:
\begin{itemize}
    \item \textbf{Index TAV}:34mm Evolut FX
    \item \textbf{Second TAV}:26mm SAPIEN 3
    \item \textbf{失败机制}:AS + AR(主动脉瓣狭窄合并反流)
    \item \textbf{CRP}(冠状动脉风险平面):Node 4
    \item \textbf{NSP}(新支架位置):Node 6
    \item \textbf{平均面积}:430.3 mm²
\end{itemize}

\textbf{冠状动脉风险评估结果}:

\begin{itemize}
    \item \textbf{RCA VTC}:3.9 mm(\textcolor{orange}{黄色警告})
    \item \textbf{LCA VTC}:5.3 mm(\textcolor{green}{绿色安全})
    \item \textbf{总体风险}:\textbf{中等冠状动脉风险}(Intermediate risk to coronaries)
\end{itemize}

\textbf{警告信息}:
\begin{quote}
\textit{Caution: Consider coronary protection if in doubt}
(警告:如有疑虑,考虑冠状动脉保护)
\end{quote}

\subsubsection{手术执行}

演讲展示了实际手术过程的透视影像,显示:
\begin{itemize}
    \item 成功植入26mm SAPIEN 3瓣膜
    \item 瓣膜位置良好
    \item 冠状动脉通畅
\end{itemize}

\subsection{结论}

\subsubsection{TAV in TAV手术的核心要点}

\textbf{1. 彻底的解剖学分析是成功的基础}

TAV in TAV的可行性评估需要彻底分析:
\begin{itemize}
    \item \textbf{主动脉根部解剖}(Aortic root anatomy)
    \begin{itemize}
        \item 主动脉窦的大小和形态
        \item 主动脉瓣环大小
        \item STJ(窦管交界)高度和直径
    \end{itemize}

    \item \textbf{冠状动脉解剖}(Coronary anatomy)
    \begin{itemize}
        \item 冠状动脉开口高度
        \item 冠状动脉与瓣膜的距离
        \item 主动脉窦的宽度和深度
    \end{itemize}

    \item \textbf{原始瓣膜特征}
    \begin{itemize}
        \item 瓣膜类型、尺寸、位置
        \item 瓣膜失败机制(AS vs AR)
        \item 瓣膜钙化程度和分布
        \item 瓣叶运动情况
    \end{itemize}
\end{itemize}

\textbf{2. 冠状动脉风险评估并不复杂}

虽然看似复杂,但评估冠状动脉阻塞风险的算法实际上是系统化和可操作的:
\begin{itemize}
    \item 使用标准化的CT测量方法
    \item 应用TAV in TAV专用计算工具(如Redo TAVR APP)
    \item 遵循明确的测量步骤和风险分层标准
    \item 关键参数包括:
    \begin{itemize}
        \item VTC(Virtual to Coronary)距离
        \item Neoskirt高度
        \item 瓣叶悬垂程度
        \item STJ高度
    \end{itemize}
\end{itemize}

\textbf{3. 精确定位技术是手术成功的关键}

促进TAV in TAV手术成功的因素:
\begin{itemize}
    \item \textbf{术前规划工具}:
    \begin{itemize}
        \item 3D CT重建
        \item TAV in TAV专用APP
        \item 虚拟植入模拟
    \end{itemize}

    \item \textbf{术中定位技术}:
    \begin{itemize}
        \item 高质量透视成像
        \item 多角度透视评估
        \item 可重定位瓣膜系统的优势
        \item 球囊扩张式瓣膜便于精确控制
    \end{itemize}

    \item \textbf{冠状动脉保护措施}:
    \begin{itemize}
        \item 必要时预置导丝
        \item 准备冠状动脉支架
        \item BASILICA等预防技术
    \end{itemize}
\end{itemize}

\subsubsection{Sapien in Evolut的特殊考虑}

\textbf{优势}:
\begin{enumerate}
    \item \textbf{可重定位性}:Sapien 3可在释放前调整位置
    \item \textbf{精确扩张}:球囊扩张机制允许精确控制
    \item \textbf{可预测性}:支架扩张程度可预测
    \item \textbf{良好的血流动力学}:大部分组合表现优异
\end{enumerate}

\textbf{挑战}:
\begin{enumerate}
    \item \textbf{支架变形}:常见但通常不影响功能
    \item \textbf{瓣叶悬垂}:需要平衡冠脉风险和血流动力学
    \item \textbf{大尺寸组合的反流}:29mm S3 in 34mm Evolut需谨慎
    \item \textbf{Neoskirt高度}:影响冠脉阻塞风险
\end{enumerate}

\textbf{推荐策略}:
\begin{itemize}
    \item 优先考虑Node 5或Node 6植入位置(取决于冠脉高度)
    \item 瓣叶悬垂<40\%时血流动力学通常可接受
    \item VTC距离<4mm需要特别警惕
    \item 必要时准备冠状动脉保护措施
\end{itemize}

\subsection{临床启示}

\subsubsection{对临床实践的指导}

\textbf{1. 术前评估流程}

建立标准化的TAV in TAV术前评估流程:
\begin{enumerate}
    \item \textbf{高质量CT扫描}:
    \begin{itemize}
        \item 心脏门控CT
        \item 多时相采集
        \item 薄层扫描(<1mm)
    \end{itemize}

    \item \textbf{详细测量}:
    \begin{itemize}
        \item 原始瓣膜的所有节点面积和周长
        \item 冠状动脉开口位置和高度
        \item STJ高度和直径
        \item 主动脉窦尺寸
    \end{itemize}

    \item \textbf{使用专用工具}:
    \begin{itemize}
        \item Redo TAVR APP或类似软件
        \item 虚拟植入模拟
        \item 自动化风险评估
    \end{itemize}

    \item \textbf{多学科讨论}:
    \begin{itemize}
        \item 介入心脏病医生
        \item 心脏外科医生
        \item 影像科医生
        \item 麻醉医生
    \end{itemize}
\end{enumerate}

\textbf{2. 瓣膜选择原则}

根据原始瓣膜类型选择合适的第二个瓣膜:

\begin{itemize}
    \item \textbf{Evolut失败}:
    \begin{itemize}
        \item 首选球囊扩张式瓣膜(如Sapien系列)
        \item 避免Evolut in Evolut(需谨慎使用)
        \item 考虑可重定位的优势
    \end{itemize}

    \item \textbf{尺寸选择}:
    \begin{itemize}
        \item 基于原始瓣膜中部节点(Node 3-6)的平均面积
        \item 参考尺寸表和ISO标准
        \item 避免过大或过小
    \end{itemize}
\end{itemize}

\textbf{3. 植入位置策略}

根据失败机制和冠脉解剖选择植入位置:

\begin{table}[h]
\centering
\caption{不同临床情况下的植入位置推荐}
\label{tab:implantation_strategy}
\begin{tabular}{lll}
\toprule
\textbf{临床情况} & \textbf{推荐位置} & \textbf{理由} \\
\midrule
AS为主 + 冠脉高 & Node 5-6 & 充分覆盖狭窄,冠脉相对安全 \\
AS为主 + 冠脉低 & Node 4-5 & 减少Neoskirt高度,避免冠脉阻塞 \\
AR为主 + 冠脉高 & Node 6 & 最佳血流动力学,瓣叶悬垂最少 \\
AR为主 + 冠脉低 & Node 4-5 & 平衡密封效果和冠脉安全 \\
AS+AR混合 & Node 5-6 & 综合考虑,倾向较高位置 \\
\bottomrule
\end{tabular}
\end{table}

\textbf{4. 冠状动脉保护策略}

根据风险分层制定保护措施:

\begin{itemize}
    \item \textbf{低风险}(VTC >5mm):
    \begin{itemize}
        \item 常规手术
        \item 术中密切监测
    \end{itemize}

    \item \textbf{中等风险}(VTC 4-5mm):
    \begin{itemize}
        \item 预置冠脉导丝(如本例RCA VTC=3.9mm)
        \item 准备冠脉支架
        \item 考虑冠脉造影
    \end{itemize}

    \item \textbf{高风险}(VTC <4mm):
    \begin{itemize}
        \item 强烈考虑BASILICA(瓣叶故意裂开术)
        \item 预置冠脉导丝和保护装置
        \item 准备紧急支架
        \item 考虑嵌合技术(Chimney/Snorkel)
    \end{itemize}
\end{itemize}

\textbf{5. 特殊临床场景}

\textbf{LVAD候选者合并瓣膜失败}(如本例):

考虑因素:
\begin{itemize}
    \item LVAD植入会加重主动脉瓣反流
    \item 联合手术vs分期手术的选择
    \item 患者血流动力学稳定性
    \item 手术风险评估
\end{itemize}

策略选择:
\begin{itemize}
    \item 血流动力学稳定:可考虑先TAV in TAV,后LVAD
    \item 血流动力学不稳定:可考虑联合手术
    \item ECMO支持下进行TAV in TAV
    \item 详细的术前规划至关重要
\end{itemize}

\subsubsection{对研究的启示}

\textbf{需要进一步研究的问题}:

\begin{enumerate}
    \item \textbf{长期耐久性}:
    \begin{itemize}
        \item TAV in TAV的长期结果(>5年)
        \item 第二个瓣膜的衰败模式
        \item 第三次干预(TAV in TAV in TAV)的可行性
    \end{itemize}

    \item \textbf{不同瓣膜组合的比较}:
    \begin{itemize}
        \item Sapien in Evolut vs Evolut in Evolut
        \item 不同代际瓣膜的组合
        \item 最佳尺寸匹配策略
    \end{itemize}

    \item \textbf{优化植入技术}:
    \begin{itemize}
        \item 新型植入装置的评估
        \item 影像引导技术的改进
        \item 人工智能辅助规划
    \end{itemize}

    \item \textbf{冠脉保护技术}:
    \begin{itemize}
        \item BASILICA的适应症和时机
        \item 新型冠脉保护装置
        \item 预防性vs救援性干预
    \end{itemize}

    \item \textbf{特殊人群}:
    \begin{itemize}
        \item 年轻患者的TAV in TAV策略
        \item 二叶瓣患者
        \item 合并其他瓣膜病变
        \item 心衰患者(如LVAD候选者)
    \end{itemize}
\end{enumerate}

\subsection{研究局限性}

\begin{enumerate}
    \item \textbf{演讲形式的局限性}:
    \begin{itemize}
        \item 本文献为会议演讲材料,非正式出版论文
        \item 缺乏详细的方法学描述
        \item 未提供统计学分析
    \end{itemize}

    \item \textbf{单中心经验}:
    \begin{itemize}
        \item 仅代表Saint Luke's的经验
        \item 可能存在中心特异性偏倚
        \item 操作者经验和技术差异
    \end{itemize}

    \item \textbf{病例数量}:
    \begin{itemize}
        \item 仅展示单个病例
        \item 无法评估手术成功率
        \item 缺乏并发症数据
    \end{itemize}

    \item \textbf{随访数据缺失}:
    \begin{itemize}
        \item 未提供术后随访结果
        \item 不清楚长期血流动力学表现
        \item 未知LVAD植入情况和结果
    \end{itemize}

    \item \textbf{引用研究的局限}:
    \begin{itemize}
        \item Akodad研究为体外实验,非真实临床数据
        \item Sellers研究为摘要形式,信息有限
        \item 缺乏大规模临床研究支持
    \end{itemize}
\end{enumerate}

\subsection{个人笔记}

\subsubsection{关键数字记忆}

\textbf{Neoskirt高度范围}:
\begin{itemize}
    \item Node 4: 16.3-19.9 mm(最低)
    \item Node 5: 20.6-23.0 mm(中等)
    \item Node 6: 23.9-27.0 mm(最高)
    \item \textbf{最大差异}:7.6 mm(临床显著)
\end{itemize}

\textbf{瓣叶悬垂百分比}:
\begin{itemize}
    \item Node 4: 90-94\%(最大)
    \item Node 5: 32-49\%(中等)
    \item Node 6: 0-9\%(最小)
    \item \textbf{临界值}:<40\%时血流动力学可接受
\end{itemize}

\textbf{血流动力学改善}:
\begin{itemize}
    \item EOA增加:43-285\%
    \item MG降低:49-91\%
    \item 反流分数:大部分<20\%(除29mm S3 in 34mm Evolut: 25.8\%)
\end{itemize}

\textbf{冠状动脉风险阈值}:
\begin{itemize}
    \item VTC >5 mm:低风险(绿色)
    \item VTC 4-5 mm:中等风险(黄色)
    \item VTC <4 mm:高风险(红色)
\end{itemize}

\textbf{病例关键数据}:
\begin{itemize}
    \item Index TAV: 34mm Evolut FX
    \item Second TAV: 26mm SAPIEN 3
    \item NSP: Node 6
    \item RCA VTC: 3.9 mm(中等风险)
    \item LCA VTC: 5.3 mm(低风险)
    \item 平均面积: 430.3 mm²
\end{itemize}

\subsubsection{重要概念}

\begin{description}
    \item[Neoskirt(新裙边)] 从原始瓣膜流入端到新瓣膜流出端形成的完全密封区域,高度影响冠脉阻塞风险

    \item[Leaflet Overhang(瓣叶悬垂)] 原始瓣膜瓣叶超出新瓣膜支架的百分比,<40\%时血流动力学功能通常可接受

    \item[Node System] Evolut瓣膜的节点编号系统(Node 1-6),用于精确定位和手术规划

    \item[VTC (Virtual to Coronary)] 虚拟瓣膜到冠状动脉的距离,评估冠脉阻塞风险的关键参数

    \item[CRP (Coronary Risk Plane)] 冠状动脉风险平面,即冠状动脉开口所在的Evolut节点水平

    \item[NSP (New Stent Position)] 新支架植入位置,根据临床情况选择Node 4/5/6

    \item[ISO Accepted EOA] ISO标准接受的最小有效瓣口面积,用于评估瓣膜血流动力学是否合格

    \item[BASILICA] Bioprosthetic or native Aortic Scallop Intentional Laceration to prevent Iatrogenic Coronary Artery obstruction,预防性瓣叶裂开术

    \item[TAV in TAV APP] 专用于瓣中瓣手术规划的移动应用程序,如Redo TAVR KRUTSCH
\end{description}

\subsubsection{临床实践要点}

\textbf{术前规划的"三步走"}:
\begin{enumerate}
    \item \textbf{明确失败机制}:AS vs AR vs 混合,决定植入策略
    \item \textbf{评估冠脉风险}:测量VTC,制定保护方案
    \item \textbf{优化植入位置}:平衡血流动力学和冠脉安全
\end{enumerate}

\textbf{手术成功的"三要素"}:
\begin{enumerate}
    \item \textbf{精准测量}:高质量CT和详细测量
    \item \textbf{合理选择}:适当的瓣膜类型和尺寸
    \item \textbf{精确植入}:使用影像引导和可重定位技术
\end{enumerate}

\textbf{冠脉保护的"三级预防"}:
\begin{enumerate}
    \item \textbf{一级预防}:术前充分评估,选择安全的植入位置
    \item \textbf{二级预防}:中高风险患者预置导丝,准备支架
    \item \textbf{三级预防}:必要时BASILICA或嵌合技术
\end{enumerate}

\subsubsection{与中国临床实践的关联}

\textbf{相似之处}:
\begin{itemize}
    \item 中国TAVR手术量快速增长,瓣中瓣需求增加
    \item 面临相似的技术挑战(冠脉阻塞、瓣膜选择等)
    \item 需要标准化的术前评估和手术规划
\end{itemize}

\textbf{中国特色考虑}:
\begin{itemize}
    \item 中国患者主动脉根部尺寸可能较小
    \item 瓣膜可及性和成本考虑
    \item 可能需要针对亚洲人群的尺寸数据
    \item 术前规划工具的本地化和可及性
\end{itemize}

\textbf{可借鉴的经验}:
\begin{itemize}
    \item 建立TAV in TAV的标准化评估流程
    \item 引进或开发术前规划软件工具
    \item 培训团队掌握精确植入技术
    \item 建立多学科协作机制
    \item 积累中国人群的瓣中瓣数据
\end{itemize}

\subsubsection{值得思考的问题}

\begin{enumerate}
    \item \textbf{为什么瓣叶悬垂<40\%时血流动力学可接受?}
    \begin{itemize}
        \item 钙化瓣叶被固定在打开位置
        \item 不会在心动周期中明显运动
        \item 不造成显著的流出道梗阻
        \item 但>40\%时可能开始影响血流
    \end{itemize}

    \item \textbf{为什么不推荐Evolut in Evolut?}
    \begin{itemize}
        \item 两个自膨胀瓣膜的相互作用不可预测
        \item 可能的过度扩张或扩张不足
        \item 球囊扩张式瓣膜更易于精确控制
        \item 但标记为"USE WITH CAUTION"而非绝对禁忌
    \end{itemize}

    \item \textbf{29mm S3 in 34mm Evolut为何反流较高?}
    \begin{itemize}
        \item 可能的尺寸不匹配
        \item 较大的瓣膜间隙
        \item 支架变形更明显
        \item 提示大尺寸组合需要更仔细的评估
    \end{itemize}

    \item \textbf{LVAD患者为何需要处理AI?}
    \begin{itemize}
        \item LVAD导致主动脉瓣长期关闭
        \item 关闭不全导致血液反流回左心室
        \item 降低LVAD的有效血流输出
        \item 可能影响LVAD疗效和患者预后
    \end{itemize}

    \item \textbf{未来TAV in TAV in TAV可行吗?}
    \begin{itemize}
        \item 理论上可能,但空间有限
        \item 冠脉阻塞风险极高
        \item 可能需要新型瓣膜设计
        \item 这是年轻TAVR患者面临的重要问题
    \end{itemize}
\end{enumerate}

\subsubsection{个人总结}

这篇演讲提供了TAV in TAV,特别是Sapien in Evolut的实用指导:

\textbf{最重要的takeaway}:
\begin{enumerate}
    \item TAV in TAV不是"简单的第二次TAVR",而是需要精心规划的复杂手术
    \item 术前评估比术中技术更重要
    \item 使用专用工具可以简化复杂的评估过程
    \item 冠脉风险可以通过系统化方法有效评估和管理
    \item 不同植入位置的选择是平衡冠脉安全和血流动力学的艺术
\end{enumerate}

\textbf{临床应用建议}:
\begin{itemize}
    \item 开始进行TAV in TAV前,必须掌握详细的CT评估技术
    \item 强烈推荐使用Redo TAVR APP或类似工具
    \item 从低风险病例(冠脉高、AR为主)开始积累经验
    \item 建立标准化的术前讨论和决策流程
    \item 准备冠脉保护措施,即使风险评估为低或中等
\end{itemize}

\section{瓣中瓣TAVR的术前规划:外科生物瓣膜衰败的实用深度解析}
\label{sec:04_005_procedural_planning_viv_tavr}

% ============================================
% 文献信息
% ============================================
\subsection{文献信息}

\begin{itemize}
    \item \textbf{标题}: Procedural Planning for ViV TAVR: Surgical Bioprosthetic Valve Failure: A Practical Deep Dive
    \item \textbf{作者}: Chad Kliger, MD, MS
    \item \textbf{机构}:
    \begin{itemize}
        \item Associate Professor of Cardiology and Cardiothoracic Surgery, Hofstra School of Medicine
        \item Director, Structural Heart and Valve Center, Lenox Hill Hospital, Northwell Health
    \end{itemize}
    \item \textbf{会议}: TCT (Transcatheter Cardiovascular Therapeutics)
    \item \textbf{PDF文件名}: procedural-planning-for-viv-tavr.pdf
    \item \textbf{文献类型}: 会议演讲/教学讲座
    \item \textbf{利益冲突声明}: 无相关财务关系披露
\end{itemize}

% ============================================
% 研究背景
% ============================================
\subsection{研究背景}

\subsubsection{ViV TAVR的临床需求}

随着TAVR技术的广泛应用和早期SAVR患者的瓣膜衰败,瓣中瓣(Valve-in-Valve, ViV)TAVR成为越来越重要的治疗选择。外科生物假体瓣膜(Surgical Bioprosthetic Valve)的衰败是ViV TAVR的主要适应症。

\subsubsection{术前规划的重要性}

ViV TAVR的术前规划与天然瓣膜TAVR有本质区别:
\begin{itemize}
    \item \textbf{天然瓣膜TAVR}:关注天然瓣环、天然瓣叶、血管通路
    \item \textbf{ViV TAVR}:关注缝合环/TAVR支架(THV sizing)、假体瓣叶(冠脉阻塞风险)、血管通路(不变)
\end{itemize}

\subsubsection{CTA的核心地位}

\textbf{核心理念}:\textit{基于CTA的个体化ViV TAVR算法 = 优化结果}

CTA为介入医师/外科医生提供了术前规划的完整路径,涵盖:
\begin{enumerate}
    \item \textbf{瓣膜选择}(Valve Selection)
    \item \textbf{血管通路评估}(Access Assessment)
\end{enumerate}

% ============================================
% 研究方法
% ============================================
\subsection{研究方法}

\subsubsection{CTA分析框架}

本演讲介绍了一套系统性的CTA分析方法,用于ViV TAVR术前规划。

\textbf{第一步:确认假体瓣膜类型}

通过CTA和植入卡片确认:
\begin{itemize}
    \item 确认是SAVR还是TAVR
    \item 瓣膜类型(透视标记物;有支架、无支架、缝线式)
    \item 瓣膜尺寸(内径,ID)
\end{itemize}

\textbf{第二步:利用ViV App}

ViV App提供以下信息:
\begin{itemize}
    \item 确认透视标记物
    \item 支架内径(Stent ID)
    \item 可破裂性(Fracturability)
    \item 植入深度
    \item 透视视频
\end{itemize}

\subsubsection{CTA测量方法}

\textbf{1. 瓣环和窦部测量}

\begin{itemize}
    \item 定义精确的瓣环平面
    \item 测量冠状动脉高度、STJ高度、Valsalva窦直径
    \item 评估瓣叶尖端是否超过冠脉口或STJ
\end{itemize}

关键测量包括:
\begin{itemize}
    \item 瓣环尺寸和周长
    \item 窦部高度(分别测量LCC、RCC、NCC)
    \item STJ直径
\end{itemize}

\textbf{2. 虚拟瓣膜技术(Virtual Valve)}

\begin{enumerate}
    \item 选择目标TAVR瓣膜的外径
    \item 使用简单圆柱体模拟TAVR植入物
    \item 将虚拟瓣膜定位于瓣环中心
    \item 根据需要旋转以符合生物假体支架
\end{enumerate}

虚拟瓣膜放置要点:
\begin{itemize}
    \item 基于THV尺寸选择
    \item 与缝合环/支柱中心对齐
    \item 位于支柱顶部
\end{itemize}

\textbf{3. 虚拟瓣膜到冠脉/主动脉的距离测量}

关键参数:
\begin{itemize}
    \item \textbf{VTC(Virtual Valve to Coronary)}:从预期锚定区测量,在收缩期测量
    \begin{itemize}
        \item \textbf{高风险阈值}:<3-4mm
    \end{itemize}
    \item \textbf{VTSTJ/VTA(Virtual Valve to Sinotubular Junction/Aorta)}
    \begin{itemize}
        \item \textbf{高风险阈值}:<2mm
    \end{itemize}
\end{itemize}

\subsubsection{瓣叶改良技术的CTA规划}

对于BASILICA等瓣叶改良技术,需要确定:

\textbf{侧位投照角度}:
\begin{itemize}
    \item 确认正确的瓣叶
    \item 对侧瓣环标记物在侧位投照中重叠
\end{itemize}

\textbf{正位/en face投照角度}:
\begin{itemize}
    \item 确认中部/基底部位置
    \item 瓣环标记物均匀分布
\end{itemize}

\textbf{重要考虑}:
\begin{itemize}
    \item 评估目标瓣叶钙化
    \item 考虑植入特征和TAVR装置
    \item 考虑SAVR类型
    \item 考虑解剖学因素
\end{itemize}

\textbf{瓣叶撕裂位置优化}:
\begin{itemize}
    \item 冠脉口中心对准撕裂线(居中冠脉口)
    \item 偏心冠脉口需要调整撕裂位置
    \item 中线撕裂在以下情况可能不足够:偏心冠脉、VTC不足、THV错位
\end{itemize}

% ============================================
% 主要研究发现
% ============================================
\subsection{主要研究发现}

\subsubsection{ViV TAVR的两大风险}

\textbf{风险\#1:冠状动脉阻塞(Coronary Artery Occlusion, CAO)}

\textbf{风险\#2:患者-假体不匹配(Patient-Prosthesis Mismatch, PPM)}

\subsubsection{冠状动脉阻塞(CAO)的数据}

\begin{table}[h]
\centering
\caption{冠状动脉阻塞发生率和死亡率}
\label{tab:cao_incidence_mortality}
\begin{tabular}{lcc}
\toprule
\textbf{临床情况} & \textbf{CAO发生率} & \textbf{30天死亡率} \\
\midrule
天然瓣膜TAVR & <1\% & -- \\
ViV TAVR总体 & 2.3\% & 40-50\% \\
外置支架/无支架SAVR ViV & 5\% & 40-50\% \\
\bottomrule
\end{tabular}
\end{table}

\textbf{关键数据}:
\begin{itemize}
    \item CAO相对不常见:总体1-5\%
    \item 天然瓣膜TAVR:<1\%
    \item ViV TAVR:2.3\%(外置支架和无支架SAVR为5\%)
    \item \textbf{一旦发生CAO,30天死亡率高达40-50\%}
\end{itemize}

数据来源:Mercanti et al. JACC Cardiovascular Imaging 2020

\subsubsection{CAO风险因素}

\textbf{按因素类别分类}(综合文献总结):

\begin{table}[h]
\centering
\caption{冠状动脉阻塞的危险因素}
\label{tab:cao_risk_factors}
\begin{tabular}{p{4cm}p{10cm}}
\toprule
\textbf{因素类别} & \textbf{具体危险因素} \\
\midrule
\textbf{瓣叶特征} &
\begin{itemize}[leftmargin=*,nosep]
    \item 瓣叶长度超过冠脉口高度和窦管结合部高度
    \item 可能被移位到冠脉口的瓣叶钙化团块
    \item 瓣叶厚度增加TAVR支架移位
    \item 瓣叶回缩(如猪瓣生物假体)降低阻塞风险
\end{itemize} \\
\midrule
\textbf{Valsalva窦因素} &
\begin{itemize}[leftmargin=*,nosep]
    \item 低位冠脉口(<12mm)
    \item 狭窄(缺乏)的Valsalva窦
    \item 低窦管结合部高度
    \item 既往主动脉根部修复(如移植物和冠脉植入)
    \item 窦部尺寸的时相性变化(收缩期vs舒张期)
\end{itemize} \\
\midrule
\textbf{生物假体瓣膜因素} &
\begin{itemize}[leftmargin=*,nosep]
    \item 生物假体构型(瓣叶位于外科支架框架外,瓣环上vs瓣环部,相对于主动脉根部的成角)
    \item 生物假体框架高度相对于窦管结合部
    \item 框架、裙边和瓣叶的体积
    \item \textbf{破裂后框架向外移位}
    \item 无支架和同种移植物装置的长瓣叶特征
\end{itemize} \\
\midrule
\textbf{经导管瓣膜因素} &
\begin{itemize}[leftmargin=*,nosep]
    \item 经导管瓣膜织物裙边(周边可能不均匀)
    \item 经导管瓣膜交界及其旋转对齐
    \item 流入膨胀受限时,经导管瓣膜框架流出扩张更明显;在球囊扩张瓣膜中更明显
    \item 高位植入以避免传导缺陷
    \item 长瓣膜(如Evolut)可能被升主动脉倾斜
    \item 小瓣膜选择治疗小瓣环
\end{itemize} \\
\bottomrule
\end{tabular}
\end{table}

\textbf{生物瓣膜特殊风险}:
\begin{itemize}
    \item \textbf{Mitroflow和Trifecta}:瓣叶位于外科支架外,CAO高风险
    \item 无支架瓣膜:长瓣叶特征
    \item 外置支架瓣膜:CAO风险显著增加
\end{itemize}

\subsubsection{患者-假体不匹配(PPM)和瓣膜破裂/重塑}

\textbf{PPM风险评估}:

为避免PPM和优化血流动力学结果,需要:
\begin{itemize}
    \item \textbf{SoV足够大}以容纳增大的THV尺寸
    \item \textbf{VTC应>5mm}以允许扩张
    \item THV尺寸选择基于SAVR真实ID预期增加3-4mm
\end{itemize}

\textbf{实例}:
\begin{itemize}
    \item 21mm Mitroflow(ID 17mm)→ 推荐S3 20mm
    \item 25mm Mitroflow(ID 21mm)→ 推荐S3 23mm
\end{itemize}

\textbf{生物瓣膜破裂(BVF)vs重塑(BVR)}

数据来源:Allen et al. Annals of Thoracic Surgery 2017; 104: 1501-8

\begin{table}[h]
\centering
\caption{有支架SAVR的可破裂性和可重塑性}
\label{tab:bvf_bvr_classification}
\begin{tabular}{lll}
\toprule
\textbf{可破裂} & \textbf{可重塑} & \textbf{不可破裂或重塑} \\
\midrule
Biocor Epic & Carpentier-Edwards Standard & Avalus \\
Magna & Carpentier-Edwards SAV & Hancock II \\
Magna Ease & Inspiris & \\
Mitroflow & Perimount (older) & \\
Mosaic & Trifecta & \\
Perimount (newer) & & \\
\bottomrule
\end{tabular}
\end{table}

\textbf{球囊破裂压力}(选择性数据):

\begin{itemize}
    \item \textbf{St. Jude Trifecta}:19mm和21mm均不可破裂(Bard TRU和Atlas Gold球囊)
    \item \textbf{Biocor Epic 21mm}:可破裂(8 ATM,两种球囊均可)
    \item \textbf{Mosaic}:19mm和21mm可破裂(10 ATM)
    \item \textbf{Mitroflow}:19mm和21mm可破裂(12 ATM)
    \item \textbf{Edwards MagnaEase}:19mm和21mm可破裂(18 ATM)
    \item \textbf{Edwards Magna}:19mm和21mm可破裂(24 ATM)
\end{itemize}

\textbf{Epic/Inspiris瓣膜}:
\begin{itemize}
    \item 专为ViV设计
    \item 可承受约8个大气压的球囊瓣膜成形术压力
    \item 内置瓣叶和低支柱高度减少冠脉阻塞风险
    \item 在荧光透视下瓣环和支柱的可见性增强(用于经导管主动脉瓣置换术)
\end{itemize}

\subsubsection{TAV-in-TAV的冠脉阻塞风险}

随着TAVR患者的增加,未来将面临更多TAV-in-TAV病例。

\textbf{风险评估方法}(Medranda et al. EuroIntervention 2022):

\begin{itemize}
    \item \textbf{风险平面(Risk Plane)}= THV瓣叶高度
    \item 计算VTC/VTA距离
\end{itemize}

\textbf{风险分层}:

\begin{table}[h]
\centering
\caption{TAV-in-TAV冠脉阻塞风险分层}
\label{tab:tav_in_tav_risk}
\begin{tabular}{p{4cm}p{5cm}p{4cm}}
\toprule
\textbf{风险等级} & \textbf{VTC标准} & \textbf{VTA标准} \\
\midrule
\textbf{高风险} & 两支冠脉均<4mm & 两支冠脉均<2mm \\
\textbf{中等风险} & 一支冠脉<4mm & 至少一支冠脉>2mm \\
\textbf{低风险} & 两支冠脉均>4mm & 两支冠脉均>2mm \\
\midrule
\textbf{处理策略} & & \\
高风险 & \multicolumn{2}{l}{可能需要瓣叶改良,窦部嵌顿高风险} \\
中等风险 & \multicolumn{2}{l}{窦部嵌顿中等风险} \\
低风险 & \multicolumn{2}{l}{无需瓣叶改良即可行*} \\
\bottomrule
\end{tabular}
\end{table}

\subsubsection{当代保护策略}

\textbf{ShortCut或BASILICA(超适应症)}

两种瓣叶改良技术用于预防冠脉阻塞:
\begin{itemize}
    \item \textbf{BASILICA}:Bioprosthetic or native Aortic Scallop Intentional Laceration to prevent Iatrogenic Coronary Artery obstruction
    \item \textbf{ShortCut}:类似技术
\end{itemize}

技术要点:
\begin{itemize}
    \item 瓣叶电切撕裂
    \item 允许瓣叶向两侧分离,为冠脉口留出空间
    \item 可用于ViV TAVR和天然瓣膜TAVR
\end{itemize}

\subsubsection{终生管理(Lifetime Management)模拟}

利用计算机模拟技术(如DASI Simulations)评估:

\textbf{第一个瓣膜的考虑}:
\begin{itemize}
    \item 类型、尺寸、植入深度
    \item PPM风险
    \item 瓣周漏(PVL)风险
    \item 破裂风险
\end{itemize}

\textbf{第二个瓣膜的考虑}:
\begin{itemize}
    \item 类型、尺寸、植入深度
    \item 冠脉风险
    \item 破裂风险
\end{itemize}

目标:优化首次瓣膜选择,为未来的再次干预留出空间。

\subsubsection{CT-荧光融合成像}

\textbf{应用场景}(Basman/Kliger et al. JACC Cardiovascular Interventions 2021):

\begin{enumerate}
    \item \textbf{栓塞保护装置放置}:精确定位
    \item \textbf{共面/瓣尖重叠角度}:优化植入角度
    \item \textbf{冠状动脉定位}:用于三尖瓣同种移植物/天然瓣膜的BASILICA
\end{enumerate}

技术优势:
\begin{itemize}
    \item 实时整合CTA解剖信息
    \item 提供精确的荧光投照角度
    \item 指导复杂操作(如瓣叶改良)
\end{itemize}

% ============================================
% 结论
% ============================================
\subsection{结论}

\subsubsection{核心要点总结}

\begin{enumerate}
    \item \textbf{CTA提供安全手术所需的全部信息}
    \begin{itemize}
        \item 假体瓣膜识别
        \item 冠脉阻塞风险评估
        \item PPM风险评估
        \item 血管通路评估
    \end{itemize}

    \item \textbf{识别和治疗CAO和PPM高危患者是关键}
    \begin{itemize}
        \item ShortCut技术
        \item 超适应症BASILICA
        \item 生物瓣膜破裂(BVF)
    \end{itemize}

    \item \textbf{冠脉再通和终生管理是重要考虑}
    \begin{itemize}
        \item 计划首次干预时考虑未来需求
        \item 计算机模拟辅助决策
    \end{itemize}

    \item \textbf{CT-荧光融合成像可指导}
    \begin{itemize}
        \item 瓣叶改良
        \item 投照角度选择
        \item 脑栓塞保护装置(CEPD)放置
    \end{itemize}

    \item \textbf{总体而言,利用CTA将确保成功}
\end{enumerate}

\subsubsection{关键临床信息}

\textbf{高风险阈值(必须记住)}:
\begin{itemize}
    \item VTC < 3-4mm:冠脉阻塞高风险
    \item VTSTJ/VTA < 2mm:窦部嵌顿高风险
    \item CAO发生后死亡率:40-50\%
\end{itemize}

\textbf{ViV特异性考虑}:
\begin{itemize}
    \item 分析焦点从天然瓣环转向缝合环/TAVR支架
    \item 分析焦点从天然瓣叶转向假体瓣叶
    \item 血管通路评估方法不变
\end{itemize}

% ============================================
% 临床启示
% ============================================
\subsection{临床启示}

\subsubsection{对临床实践的指导}

\textbf{1. 术前评估标准化流程}

所有ViV TAVR患者必须进行:
\begin{enumerate}
    \item 高质量心脏CTA(收缩期和舒张期)
    \item 确认假体瓣膜类型和尺寸(查阅植入卡片)
    \item 使用ViV App进行初步规划
    \item 系统性CTA测量(瓣环、窦部、冠脉高度、VTC、VTSTJ)
    \item 虚拟瓣膜模拟
\end{enumerate}

\textbf{2. 风险分层和管理策略}

\begin{table}[h]
\centering
\caption{ViV TAVR风险分层和管理策略}
\label{tab:viv_risk_management}
\begin{tabular}{p{3cm}p{5cm}p{5.5cm}}
\toprule
\textbf{风险类别} & \textbf{识别标准} & \textbf{管理策略} \\
\midrule
\textbf{CAO高风险} &
\begin{itemize}[leftmargin=*,nosep]
    \item VTC <3-4mm
    \item VTSTJ <2mm
    \item 外置瓣叶瓣膜
    \item 低位冠脉
\end{itemize} &
\begin{itemize}[leftmargin=*,nosep]
    \item 考虑BASILICA/ShortCut
    \item 备用冠脉保护方案
    \item 准备紧急冠脉介入
    \item 考虑改行SAVR
\end{itemize} \\
\midrule
\textbf{PPM高风险} &
\begin{itemize}[leftmargin=*,nosep]
    \item 小尺寸SAVR
    \item VTC <5mm
    \item 窦部空间不足
\end{itemize} &
\begin{itemize}[leftmargin=*,nosep]
    \item 考虑BVF(如可破裂)
    \item 选择更大THV
    \item 评估SAVR可行性
    \item 优化植入深度
\end{itemize} \\
\bottomrule
\end{tabular}
\end{table}

\textbf{3. 瓣膜选择优化}

\begin{itemize}
    \item \textbf{优先选择可破裂瓣膜}:为未来ViV留出空间
    \item \textbf{年轻患者考虑终生管理}:计算机模拟2-3次干预的可行性
    \item \textbf{避免高危瓣膜}:外置瓣叶瓣膜(Mitroflow、Trifecta)在小根部患者中需谨慎
\end{itemize}

\textbf{4. 技术应用建议}

\begin{itemize}
    \item \textbf{CT-荧光融合}:复杂病例常规使用
    \item \textbf{虚拟瓣膜技术}:所有病例术前模拟
    \item \textbf{多学科团队}:影像专家、介入医师、心外科医师共同评估
\end{itemize}

\subsubsection{对患者教育的启示}

\begin{enumerate}
    \item 首次SAVR时应告知患者未来可能需要再次干预
    \item 解释ViV TAVR作为微创选择的优势
    \item 讨论终生管理计划(特别是年轻患者)
    \item 说明CTA在术前规划中的重要性
\end{enumerate}

\subsubsection{对研究方向的启示}

\begin{enumerate}
    \item 开发标准化CTA测量和报告模板
    \item 人工智能辅助风险预测模型
    \item 长期随访数据收集(TAV-in-TAV结果)
    \item 新型瓣膜设计(更适合ViV)
    \item BASILICA/ShortCut技术的优化和标准化
\end{enumerate}

% ============================================
% 研究局限性
% ============================================
\subsection{研究局限性}

\begin{enumerate}
    \item \textbf{文献类型局限}
    \begin{itemize}
        \item 本文献为会议演讲,非原始研究
        \item 数据来源于多项已发表研究的综合
        \item 缺乏系统性文献综述方法学
    \end{itemize}

    \item \textbf{证据等级}
    \begin{itemize}
        \item 主要基于观察性研究和注册数据
        \item CAO发生率基于回顾性分析
        \item 缺乏ViV TAVR的随机对照试验
    \end{itemize}

    \item \textbf{技术局限}
    \begin{itemize}
        \item CTA测量存在观察者间和观察者内变异
        \item 虚拟瓣膜模拟假设刚性模型,未完全模拟真实变形
        \item 收缩期和舒张期窦部尺寸变化的影响未完全阐明
    \end{itemize}

    \item \textbf{瓣膜特异性数据}
    \begin{itemize}
        \item 某些瓣膜型号的破裂数据有限
        \item 新型瓣膜(如Inspiris)的长期ViV数据不足
        \item 不同THV在ViV场景中的比较数据缺乏
    \end{itemize}

    \item \textbf{BASILICA/ShortCut}
    \begin{itemize}
        \item 为超适应症应用
        \item 长期安全性和有效性数据有限
        \item 技术标准化程度不足
        \item 学习曲线未充分研究
    \end{itemize}

    \item \textbf{终生管理}
    \begin{itemize}
        \item 计算机模拟基于理论假设
        \item 缺乏多次ViV的临床结果数据
        \item 患者依从性和实际临床路径可能与规划不符
    \end{itemize}

    \item \textbf{推广性}
    \begin{itemize}
        \item 技术和专业知识在不同中心间存在差异
        \item 资源可用性(CTA、融合成像、ViV App等)不均衡
        \item 主要经验来自高容量中心
    \end{itemize}
\end{enumerate}

% ============================================
% 个人笔记
% ============================================
\subsection{个人笔记}

\subsubsection{关键数字记忆}

\begin{table}[h]
\centering
\caption{ViV TAVR关键数字速查}
\label{tab:key_numbers}
\begin{tabular}{ll}
\toprule
\textbf{参数} & \textbf{数值/阈值} \\
\midrule
\textbf{CAO发生率} & \\
\quad 天然瓣膜TAVR & <1\% \\
\quad ViV TAVR总体 & 2.3\% \\
\quad 外置/无支架SAVR ViV & 5\% \\
\midrule
\textbf{CAO后死亡率} & \\
\quad 30天死亡率 & 40-50\% \\
\midrule
\textbf{高风险阈值} & \\
\quad VTC(虚拟瓣膜到冠脉) & <3-4mm \\
\quad VTSTJ/VTA(虚拟瓣膜到主动脉) & <2mm \\
\quad VTC(允许扩张) & >5mm \\
\midrule
\textbf{THV尺寸选择} & \\
\quad SAVR ID预期增加 & 3-4mm \\
\midrule
\textbf{球囊破裂压力(示例)} & \\
\quad Mitroflow & 12 ATM \\
\quad MagnaEase & 18 ATM \\
\quad Magna & 24 ATM \\
\quad Epic & 8 ATM \\
\bottomrule
\end{tabular}
\end{table}

\subsubsection{重要概念解析}

\begin{description}
    \item[ViV TAVR] Valve-in-Valve TAVR,瓣中瓣TAVR,指在已植入的外科生物瓣膜或TAVR内再次植入TAVR。

    \item[CAO] Coronary Artery Occlusion,冠状动脉阻塞,ViV TAVR最严重的并发症之一,死亡率高达40-50\%。

    \item[PPM] Patient-Prosthesis Mismatch,患者-假体不匹配,指瓣膜有效开口面积相对于患者体表面积过小,导致血流动力学受损。

    \item[VTC] Virtual Valve to Coronary,虚拟瓣膜到冠脉的距离,<3-4mm为CAO高风险。

    \item[VTSTJ/VTA] Virtual Valve to Sinotubular Junction/Aorta,虚拟瓣膜到窦管结合部/主动脉的距离,<2mm为窦部嵌顿高风险。

    \item[BASILICA] Bioprosthetic or native Aortic Scallop Intentional Laceration to prevent Iatrogenic Coronary Artery obstruction,瓣叶撕裂技术预防冠脉阻塞。

    \item[BVF] Bioprosthetic Valve Fracture,生物瓣膜破裂,使用高压球囊故意破裂SAVR支架以增大有效开口面积。

    \item[BVR] Bioprosthetic Valve Remodeling,生物瓣膜重塑,某些瓣膜可被重塑但不能破裂。

    \item[虚拟瓣膜技术] 在CTA上模拟放置TAVR装置,预测植入后的几何关系和潜在并发症。

    \item[CT-荧光融合] 将CTA解剖信息实时叠加到荧光透视图像上,指导精确操作。
\end{description}

\subsubsection{临床思维流程图}

\textbf{ViV TAVR术前评估流程}:

\begin{enumerate}
    \item \textbf{确认假体瓣膜}
    \begin{itemize}
        \item 查阅植入卡片
        \item CTA识别透视标记物
        \item 确定类型(有支架/无支架/缝线式)和尺寸
    \end{itemize}

    \item \textbf{查询ViV App}
    \begin{itemize}
        \item 确认支架ID
        \item 查看可破裂性
        \item 获得推荐THV尺寸
    \end{itemize}

    \item \textbf{CTA系统测量}
    \begin{itemize}
        \item 瓣环尺寸
        \item 窦部高度(LCC、RCC、NCC)
        \item 冠脉口高度
        \item STJ高度和直径
    \end{itemize}

    \item \textbf{虚拟瓣膜模拟}
    \begin{itemize}
        \item 放置虚拟圆柱体
        \item 对齐缝合环/支柱
        \item 测量VTC(三个窦部)
        \item 测量VTSTJ
    \end{itemize}

    \item \textbf{风险分层}
    \begin{itemize}
        \item CAO风险:VTC <3-4mm → 高风险
        \item PPM风险:小瓣膜 + VTC <5mm → 高风险
        \item 窦部嵌顿:VTSTJ <2mm → 高风险
    \end{itemize}

    \item \textbf{制定策略}
    \begin{itemize}
        \item 低风险:常规ViV TAVR
        \item CAO高风险:考虑BASILICA/ShortCut或SAVR
        \item PPM高风险:考虑BVF或更大THV或SAVR
        \item 多重高风险:倾向SAVR
    \end{itemize}

    \item \textbf{终生管理考虑}
    \begin{itemize}
        \item 年龄<65岁:模拟2-3次干预
        \item 评估未来冠脉再通可行性
        \item 优化首次干预以保留未来选择
    \end{itemize}
\end{enumerate}

\subsubsection{特殊瓣膜记忆要点}

\textbf{高CAO风险瓣膜}:
\begin{itemize}
    \item \textbf{Mitroflow}:外置瓣叶,需特别注意VTC
    \item \textbf{Trifecta}:外置瓣叶,且不可破裂
    \item \textbf{无支架瓣膜}:长瓣叶特征
\end{itemize}

\textbf{推荐瓣膜(ViV友好)}:
\begin{itemize}
    \item \textbf{Epic}:可破裂(8 ATM),专为ViV设计
    \item \textbf{Inspiris}:可重塑,低支柱高度
    \item \textbf{Magna系列}:可破裂(18-24 ATM)
\end{itemize}

\subsubsection{与其他主题的联系}

\begin{enumerate}
    \item \textbf{与主题1(指南和基础)的关系}
    \begin{itemize}
        \item ViV TAVR适应症基于AS/AR严重程度
        \item 风险评估遵循通用TAVR评估原则
        \item 团队协作(Heart Team)的重要性
    \end{itemize}

    \item \textbf{与其他主题4内容的关系}
    \begin{itemize}
        \item 本文聚焦术前CTA规划
        \item 需结合ViV TAVR临床结果数据
        \item 需结合瓣膜衰败机制理解
    \end{itemize}

    \item \textbf{与影像评估的关系}
    \begin{itemize}
        \item CTA是ViV规划的核心工具
        \item 超声心动图诊断瓣膜衰败
        \item 多模态影像整合
    \end{itemize}
\end{enumerate}

\subsubsection{实用工具和资源}

\begin{itemize}
    \item \textbf{ViV App}(移动应用)
    \begin{itemize}
        \item 包含所有主要SAVR和TAVR型号
        \item 提供推荐THV尺寸
        \item 显示透视视频和标记物
        \item 免费下载
    \end{itemize}

    \item \textbf{DASI Simulations}
    \begin{itemize}
        \item 计算机终生管理模拟
        \item 预测多次干预的可行性
    \end{itemize}

    \item \textbf{CT-荧光融合系统}
    \begin{itemize}
        \item 各导管室厂商提供
        \item 需要专门培训
    \end{itemize}
\end{itemize}

\subsubsection{值得深入思考的问题}

\begin{enumerate}
    \item \textbf{为什么CAO死亡率如此高(40-50\%)?}
    \begin{itemize}
        \item 突发性完全阻塞
        \item 患者常为高危(外科高风险)
        \item 紧急PCI技术难度大(导丝难以通过变形的瓣叶)
        \item 心肌大面积缺血导致心源性休克
    \end{itemize}

    \item \textbf{BASILICA为何是超适应症?}
    \begin{itemize}
        \item FDA未批准用于ViV TAVR
        \item 缺乏大规模RCT数据
        \item 技术相对较新(约5-6年历史)
        \item 但在高CAO风险患者中越来越多使用
    \end{itemize}

    \item \textbf{为什么虚拟瓣膜要在收缩期测量VTC?}
    \begin{itemize}
        \item 收缩期窦部最扩张
        \item 最能代表瓣膜张开时的空间
        \item 收缩期冠脉充盈最少,最易受阻
    \end{itemize}

    \item \textbf{年轻患者首次SAVR如何选择瓣膜?}
    \begin{itemize}
        \item 优先考虑可破裂瓣膜
        \item 选择较大尺寸(如解剖允许)
        \item 避免外置瓣叶瓣膜
        \item 考虑Epic/Inspiris等"ViV友好"瓣膜
        \item 进行终生管理模拟
    \end{itemize}

    \item \textbf{TAV-in-TAV何时会成为常态?}
    \begin{itemize}
        \item 目前TAVR适应症已扩展到低危患者
        \item 年轻患者(50-60岁)接受TAVR增多
        \item 预计10-15年后TAV-in-TAV将显著增加
        \item 需要现在就开始规划(瓣膜选择、植入技术)
    \end{itemize}
\end{enumerate}

\subsubsection{临床案例思考}

\textbf{案例1:21mm Mitroflow衰败}
\begin{itemize}
    \item 瓣膜ID:17mm
    \item 问题:外置瓣叶,CAO风险高;小瓣膜,PPM风险高
    \item CTA测量:VTC LCC = 2.8mm,RCC = 3.1mm
    \item 策略选择:
    \begin{enumerate}
        \item 评估SAVR可行性(首选)
        \item 如选择ViV:必须行BASILICA + 选择S3 20mm
        \item 备用冠脉介入方案
    \end{enumerate}
\end{itemize}

\textbf{案例2:29mm Magna衰败}
\begin{itemize}
    \item 瓣膜ID:约25mm
    \item CTA测量:VTC均>5mm,VTSTJ >3mm
    \item 可破裂瓣膜(24 ATM)
    \item 策略选择:
    \begin{enumerate}
        \item 考虑BVF + 29mm Evolut PRO/PRO+
        \item 或直接26mm Evolut(如不做BVF)
        \item 低CAO和PPM风险
        \item 为未来TAV-in-TAV留出空间
    \end{enumerate}
\end{itemize}

\subsubsection{对中国临床实践的思考}

\begin{enumerate}
    \item \textbf{CTA可及性}
    \begin{itemize}
        \item 中国主要TAVR中心均具备心脏CTA能力
        \item 需要培训专业影像医师进行ViV特异性测量
        \item 建立标准化测量和报告流程
    \end{itemize}

    \item \textbf{ViV App应用}
    \begin{itemize}
        \item 免费工具,应推广使用
        \item 可能需要翻译为中文界面
    \end{itemize}

    \item \textbf{BASILICA技术}
    \begin{itemize}
        \item 技术难度高,需要专门培训
        \item 可在高容量中心率先开展
        \item 建立国内经验分享平台
    \end{itemize}

    \item \textbf{瓣膜选择}
    \begin{itemize}
        \item 中国SAVR主要使用的瓣膜型号与欧美可能不同
        \item 需要了解国产瓣膜的ViV特性
        \item 推广"ViV友好"瓣膜概念
    \end{itemize}

    \item \textbf{终生管理}
    \begin{itemize}
        \item 中国TAVR患者年龄可能更年轻
        \item 终生管理规划尤为重要
        \item 需要建立长期随访体系
    \end{itemize}
\end{enumerate}

\subsubsection{未来研究方向展望}

\begin{enumerate}
    \item \textbf{人工智能应用}
    \begin{itemize}
        \item 自动化CTA测量和风险评估
        \item AI辅助虚拟瓣膜放置优化
        \item 深度学习预测CAO和PPM风险
    \end{itemize}

    \item \textbf{新型瓣膜设计}
    \begin{itemize}
        \item 专为ViV设计的SAVR(如Epic/Inspiris的改进)
        \item 更薄瓣膜支架以保留更多空间
        \item 可调节支架高度
    \end{itemize}

    \item \textbf{BASILICA技术优化}
    \begin{itemize}
        \item 标准化操作流程
        \item 减少学习曲线
        \item 长期安全性和有效性数据
    \end{itemize}

    \item \textbf{TAV-in-TAV专用装置}
    \begin{itemize}
        \item 可能需要不同设计的TAVR用于TAV-in-TAV
        \item 更好的径向力和定位精度
    \end{itemize}
\end{enumerate}

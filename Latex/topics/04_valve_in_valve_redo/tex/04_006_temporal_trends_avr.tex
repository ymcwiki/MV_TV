\section{美国<65岁患者孤立性和联合主动脉瓣置换术的时间趋势}
\label{sec:04_006_temporal_trends_avr}

% ============================================
% 文献信息
% ============================================
\subsection{文献信息}

\begin{itemize}
    \item \textbf{标题}: Temporal Trends in Isolated and Concomitant Aortic Valve Replacement in Patients Aged <65 Years in the United States
    \item \textbf{作者}: Tanush Gupta, MD; Hannah R. Murphy PhD; Dhaval Kolte MD MPH PhD; Alyssa H. Harris MPH; Patrick O'Gara MD; Harold L. Dauerman MD
    \item \textbf{第一作者机构}: University of Vermont Medical Center
    \item \textbf{会议}: TCT (Transcatheter Cardiovascular Therapeutics)
    \item \textbf{PDF文件名}: tct-1109-temporal-trends-in-isolated-and-concomitant-aortic-valve-replacemen.pdf
    \item \textbf{文献类型}: 会议演讲/临床研究
    \item \textbf{利益冲突声明}: Edwards Lifesciences(机构研究支持)、Medtronic(顾问费)、Anteris Technologies(股票)
\end{itemize}

\subsection{研究背景}

\subsubsection{TAVR与SAVR在年轻患者中的证据缺口}

\textbf{临床试验证据不足}:
\begin{itemize}
    \item 经导管主动脉瓣置换术(TAVR)和外科主动脉瓣置换术(SAVR)在<65岁患者中尚未进行系统性头对头研究
    \item PARTNER-3试验:<65岁患者占比<10\%
    \item Evolut低危试验:<65岁患者占比<10\%
    \item 年轻患者的长期数据严重缺乏
\end{itemize}

\subsubsection{当前指南推荐}

\textbf{2021 ACC/AHA指南}(Otto CM, et al. JACC 2021):
\begin{itemize}
    \item \textbf{推荐SAVR}作为<65岁非高危患者严重AS的首选治疗
    \item 基于年轻患者预期寿命较长
    \item 考虑瓣膜耐久性问题
    \item TAVR瓣中瓣(ViV)的可行性问题
\end{itemize}

\subsubsection{注册数据显示的现实世界趋势}

文献报道(Sharma T, et al. JACC 2022; Gupta T, et al. JSCAI 2024; Alabaddi S, et al.):
\begin{itemize}
    \item <65岁患者中TAVR使用率持续增加
    \item TAVR与孤立性SAVR使用率接近均等化
    \item 但这些研究存在\textbf{重要局限性}:排除了接受机械瓣或联合手术的SAVR患者
\end{itemize}

\subsubsection{既往研究的局限性}

\textbf{关键问题}:
\begin{itemize}
    \item 既往TAVR vs SAVR时间趋势研究排除了:
    \begin{itemize}
        \item 接受机械瓣的SAVR患者
        \item 联合手术患者(CABG、升主动脉手术、二尖瓣/三尖瓣手术)
    \end{itemize}
    \item 这导致排除了\textbf{约四分之三的SAVR患者}
    \item 给出了TAVR与SAVR使用率均等化的\textbf{虚假印象}
    \item \textbf{本研究的必要性}:需要研究TAVR相对于\textbf{全谱AVR手术}的使用情况
\end{itemize}

\subsection{研究方法}

\subsubsection{数据来源}

\textbf{Vizient Clinical Database (CDB)}:
\begin{itemize}
    \item 全国代表性临床数据库
    \item 研究时间段:2016年1月至2024年12月
    \item 纳入医院:190家医院
    \item 覆盖美国多种医疗机构类型
\end{itemize}

\subsubsection{研究人群}

\textbf{纳入标准}:
\begin{itemize}
    \item 年龄<65岁
    \item 因主动脉瓣狭窄(AS)接受以下手术之一:
    \begin{itemize}
        \item TAVR(经导管主动脉瓣置换术)
        \item SAVR(外科主动脉瓣置换术)
        \item Ross手术(肺动脉瓣自体移植术)
    \end{itemize}
\end{itemize}

\textbf{排除标准}:
\begin{itemize}
    \item 既往接受过AVR
    \item 单纯主动脉瓣反流(AR)
    \item 感染性心内膜炎
\end{itemize}

\subsubsection{SAVR分类方法}

\textbf{按瓣膜类型分类}:
\begin{itemize}
    \item 机械瓣膜
    \item 生物瓣膜
\end{itemize}

\textbf{按手术类型分类}:
\begin{enumerate}
    \item \textbf{孤立性SAVR}:仅主动脉瓣置换
    \item \textbf{联合SAVR}:SAVR联合以下任一手术:
    \begin{itemize}
        \item 冠状动脉旁路移植术(CABG)
        \item 升主动脉手术
        \item 二尖瓣手术
        \item 三尖瓣手术
        \item 外科迷宫手术(治疗房颤)
    \end{itemize}
\end{enumerate}

\subsubsection{研究终点}

\textbf{主要研究目标}:
\begin{itemize}
    \item 研究<65岁严重AS患者中以下治疗方式的时间趋势:
    \begin{itemize}
        \item TAVR
        \item 孤立性SAVR
        \item 联合SAVR
        \item Ross手术
    \end{itemize}
\end{itemize}

\textbf{次要研究目标}:
\begin{itemize}
    \item 比较接受TAVR与外科手术患者的人口学特征
    \item 比较基线合并症差异
\end{itemize}

\subsection{主要研究发现}

\subsubsection{研究人群总体特征}

\textbf{总样本量}:N = 34,504例

\begin{table}[h]
\centering
\caption{各治疗方式病例数分布}
\label{tab:case_distribution}
\begin{tabular}{lcc}
\toprule
\textbf{治疗方式} & \textbf{病例数} & \textbf{占比} \\
\midrule
TAVR & 9,834 & 28.5\% \\
孤立性SAVR & 10,982 & 31.8\% \\
联合SAVR & 13,053 & 37.8\% \\
Ross手术 & 635 & 1.8\% \\
\midrule
总计 & 34,504 & 100\% \\
\bottomrule
\end{tabular}
\end{table}

\subsubsection{基线特征对比}

\begin{table}[h]
\centering
\caption{不同治疗方式患者基线特征对比}
\label{tab:baseline_characteristics}
\begin{tabular}{lccccc}
\toprule
\textbf{特征} & \textbf{TAVR} & \textbf{孤立SAVR} & \textbf{联合SAVR} & \textbf{Ross手术} & \textbf{p值} \\
 & \textbf{(n=9,834)} & \textbf{(n=10,982)} & \textbf{(n=13,053)} & \textbf{(n=635)} & \\
\midrule
\multicolumn{6}{l}{\textit{人口学特征}} \\
年龄中位数 (IQR) & 61.3 & 58.8 & 59.1 & 40.3 & <0.001 \\
 & (57.6-63.4) & (53.1-62.2) & (53.5-62.4) & (28.9-50.7) & \\
女性, n (\%) & 3,596 (36.6\%) & 3,576 (32.6\%) & 3,730 (28.6\%) & 244 (38.4\%) & <0.001 \\
\midrule
\multicolumn{6}{l}{\textit{合并症, n (\%)}} \\
既往PCI & 1,943 (19.8\%) & 685 (6.2\%) & 1,250 (9.6\%) & 8 (1.3\%) & <0.001 \\
既往CABG & 1,156 (11.8\%) & 283 (2.6\%) & 344 (2.6\%) & <5 (<1.0\%) & <0.001 \\
既往心肌梗死 & 1,355 (13.8\%) & 642 (5.9\%) & 1,231 (9.4\%) & 11 (1.7\%) & <0.001 \\
心力衰竭 & 6,924 (70.4\%) & 3,787 (34.5\%) & 5,293 (40.6\%) & 169 (26.6\%) & <0.001 \\
肝硬化 & 904 (9.2\%) & 176 (1.6\%) & 303 (2.3\%) & 6 (0.9\%) & <0.001 \\
COPD & 2,209 (22.5\%) & 1,221 (11.1\%) & 1,607 (12.3\%) & <5 (<1.0\%) & <0.001 \\
慢性透析 & 1,297 (13.2\%) & 284 (2.6\%) & 531 (4.1\%) & <5 (<1.0\%) & <0.001 \\
家用氧疗 & 669 (6.8\%) & 119 (1.1\%) & 167 (1.3\%) & <5 (<1.0\%) & <0.001 \\
既往卒中 & 1,250 (12.7\%) & 789 (7.2\%) & 1,169 (9.0\%) & 21 (3.3\%) & <0.001 \\
\midrule
Elixhauser合并症 & 6 (4-7) & 4 (3-6) & 5 (4-7) & 3 (2-5) & <0.001 \\
指数中位数 (IQR) & & & & & \\
\bottomrule
\end{tabular}
\end{table}

\textbf{关键观察}:
\begin{itemize}
    \item \textbf{TAVR患者年龄最大}:中位年龄61.3岁,显著高于孤立SAVR(58.8岁)和联合SAVR(59.1岁)
    \item \textbf{Ross手术患者最年轻}:中位年龄仅40.3岁,年龄范围28.9-50.7岁
    \item \textbf{TAVR患者合并症负担最重}:
    \begin{itemize}
        \item 心力衰竭:70.4\%(显著高于其他组)
        \item 慢性透析:13.2\%(为孤立SAVR的5倍)
        \item 既往PCI:19.8\%(为孤立SAVR的3倍)
        \item COPD:22.5\%(为孤立SAVR的2倍)
        \item Elixhauser合并症指数:6,显著高于孤立SAVR的4
    \end{itemize}
    \item \textbf{重要结论}:<65岁接受TAVR的患者\textbf{并非低危患者},而是高合并症负担患者
\end{itemize}

\subsubsection{AVR治疗方式相对使用率的时间趋势}

\begin{table}[h]
\centering
\caption{2016-2024年各AVR治疗方式相对使用率(占所有AVR的百分比)}
\label{tab:relative_utilization_trends}
\begin{tabular}{lccccccccc}
\toprule
\textbf{治疗方式} & \textbf{2016} & \textbf{2017} & \textbf{2018} & \textbf{2019} & \textbf{2020} & \textbf{2021} & \textbf{2022} & \textbf{2023} & \textbf{2024} \\
\midrule
TAVR & 15.7\% & 21.2\% & 24.2\% & 31.6\% & 36.8\% & 35.4\% & 32.8\% & 30.4\% & 29.0\% \\
孤立SAVR & 43.8\% & 38.0\% & 36.3\% & 31.8\% & 27.4\% & 26.3\% & 28.3\% & 28.5\% & 27.0\% \\
联合SAVR & 40.1\% & 40.4\% & 38.9\% & 35.7\% & 34.3\% & 36.8\% & 36.7\% & 37.8\% & 39.2\% \\
Ross手术 & 0.4\% & 0.5\% & 0.6\% & 0.9\% & 1.5\% & 1.6\% & 2.3\% & 3.3\% & 4.8\% \\
\bottomrule
\end{tabular}
\end{table}

\textbf{时间趋势分析}(p趋势<0.001):

\begin{enumerate}
    \item \textbf{TAVR使用率增长}:
    \begin{itemize}
        \item 2016年:15.7\% → 2024年:29.0\%
        \item 增长幅度:+13.3个百分点(增长85\%)
        \item 2020年达到峰值36.8\%,随后略有下降
        \item \textbf{重要发现}:TAVR仅占所有AVR的约1/3,而非过半
    \end{itemize}

    \item \textbf{孤立SAVR使用率下降}:
    \begin{itemize}
        \item 2016年:43.8\% → 2024年:27.0\%
        \item 下降幅度:-16.8个百分点(下降38\%)
        \item 2020年开始被TAVR超越
    \end{itemize}

    \item \textbf{联合SAVR保持稳定}:
    \begin{itemize}
        \item 2016年:40.1\% → 2024年:39.2\%
        \item \textbf{无显著变化}(p趋势=0.42)
        \item 始终是\textbf{最常用的AVR方式}(约40\%)
        \item 说明大量<65岁患者有复杂多瓣膜/冠脉疾病
    \end{itemize}

    \item \textbf{Ross手术显著增长}:
    \begin{itemize}
        \item 2016年:0.4\% → 2024年:4.8\%
        \item 增长12倍
        \item 反映年轻患者对避免抗凝和长期瓣膜耐久性的需求
    \end{itemize}
\end{enumerate}

\subsubsection{AVR治疗方式绝对病例数的时间趋势}

\begin{table}[h]
\centering
\caption{2016-2024年各AVR治疗方式绝对病例数}
\label{tab:absolute_volume_trends}
\begin{tabular}{lccccccccc}
\toprule
\textbf{治疗方式} & \textbf{2016} & \textbf{2017} & \textbf{2018} & \textbf{2019} & \textbf{2020} & \textbf{2021} & \textbf{2022} & \textbf{2023} & \textbf{2024} \\
\midrule
TAVR & 543 & 774 & 926 & 1,253 & 1,293 & 1,355 & 1,243 & 1,223 & 1,303 \\
孤立SAVR & 1,630 & 1,391 & 1,390 & 1,261 & 963 & 1,008 & 1,073 & 1,148 & 1,216 \\
联合SAVR & 1,491 & 1,478 & 1,488 & 1,416 & 1,204 & 1,407 & 1,392 & 1,522 & 1,762 \\
Ross手术 & 14 & 18 & 23 & 36 & 52 & 61 & 87 & 133 & 221 \\
\midrule
总计 & 3,678 & 3,661 & 3,827 & 3,966 & 3,512 & 3,831 & 3,795 & 4,026 & 4,502 \\
\bottomrule
\end{tabular}
\end{table}

\textbf{绝对病例数趋势分析}:

\begin{itemize}
    \item \textbf{TAVR}:从543例增至1,303例(p趋势=0.005)
    \begin{itemize}
        \item 增长2.4倍
        \item 2019-2021年增长最快
        \item 2020年后达到平台期(约1,200-1,300例/年)
    \end{itemize}

    \item \textbf{孤立SAVR}:从1,630例降至1,216例(p趋势=0.04)
    \begin{itemize}
        \item 下降25\%
        \item 2020年降至最低点(963例)
        \item 2021年后略有回升
    \end{itemize}

    \item \textbf{联合SAVR}:保持相对稳定(p趋势=0.42)
    \begin{itemize}
        \item 2016年:1,491例 → 2024年:1,762例
        \item 轻度增长18\%
        \item 始终维持在1,200-1,800例/年
    \end{itemize}

    \item \textbf{Ross手术}:从14例激增至221例(p趋势=0.001)
    \begin{itemize}
        \item 增长15.8倍
        \item 2022-2024年增长尤其迅速
        \item 反映Ross手术中心的增加和技术推广
    \end{itemize}
\end{itemize}

\subsubsection{TAVR vs 孤立SAVR的直接对比}

\textbf{关键里程碑}:
\begin{itemize}
    \item \textbf{2019年}:TAVR病例数首次接近孤立SAVR(1,253 vs 1,261)
    \item \textbf{2020年}:TAVR病例数首次超过孤立SAVR(1,293 vs 963)
    \begin{itemize}
        \item 注:2020年COVID-19疫情导致择期SAVR显著减少
    \end{itemize}
    \item \textbf{2021-2024年}:TAVR与孤立SAVR病例数基本持平(约1,200例/年)
\end{itemize}

\subsubsection{机械瓣使用趋势}

\begin{table}[h]
\centering
\caption{2016-2024年SAVR患者机械瓣使用率}
\label{tab:mechanical_valve_trends}
\begin{tabular}{lccccccccc}
\toprule
\textbf{SAVR类型} & \textbf{2016} & \textbf{2017} & \textbf{2018} & \textbf{2019} & \textbf{2020} & \textbf{2021} & \textbf{2022} & \textbf{2023} & \textbf{2024} \\
\midrule
所有SAVR & 26.0\% & 22.6\% & 24.0\% & 22.8\% & 28.0\% & 29.2\% & 28.8\% & 28.7\% & 25.4\% \\
孤立SAVR & 26.4\% & 23.0\% & 24.0\% & 23.0\% & 30.1\% & 31.5\% & 31.2\% & 31.7\% & 31.9\% \\
联合SAVR & 25.6\% & 22.5\% & 24.0\% & 22.7\% & 28.5\% & 28.4\% & 28.7\% & 27.2\% & 25.3\% \\
\bottomrule
\end{tabular}
\end{table}

\textbf{机械瓣使用分析}:

\begin{itemize}
    \item \textbf{总体使用率}:约25-30\%
    \item \textbf{时间趋势}:
    \begin{itemize}
        \item 所有SAVR:p趋势=0.035(轻度增长)
        \item 孤立SAVR:p趋势=0.007(显著增长)
        \item 联合SAVR:p趋势=0.138(无显著变化)
    \end{itemize}
    \item \textbf{孤立SAVR机械瓣使用率增长}:26.4\% → 31.9\%
    \begin{itemize}
        \item 可能反映医生为年轻低危患者更倾向选择机械瓣
        \item 避免未来再次干预
    \end{itemize}
    \item \textbf{联合SAVR机械瓣使用率较低}:约25\%
    \begin{itemize}
        \item 可能因联合CABG患者已需长期抗血小板治疗
        \item 增加抗凝负担的顾虑
    \end{itemize}
\end{itemize}

\subsubsection{总体AVR手术量趋势}

\begin{itemize}
    \item 2016年:3,678例 → 2024年:4,502例
    \item 增长22\%
    \item 2020年因COVID-19降至最低点(3,512例)
    \item 2021年后持续回升
\end{itemize}

\subsection{结论}

\subsubsection{主要结论}

\begin{enumerate}
    \item \textbf{TAVR在年轻患者中使用增长,但未成为主流}:
    \begin{itemize}
        \item TAVR使用率从2016年的15.7\%增至2024年的29.0\%
        \item 2020年开始,TAVR病例数超过孤立SAVR
        \item 但\textbf{TAVR仅占所有AVR手术的约1/3}
    \end{itemize}

    \item \textbf{联合SAVR是<65岁患者最常见的AVR方式}:
    \begin{itemize}
        \item 占所有AVR的约40\%
        \item 时间趋势无显著变化
        \item 说明大量年轻患者存在复杂多瓣膜/冠脉疾病
    \end{itemize}

    \item \textbf{TAVR患者具有显著更高的合并症负担}:
    \begin{itemize}
        \item 心力衰竭患病率:70.4\% vs 34.5\%(孤立SAVR)
        \item 慢性透析:13.2\% vs 2.6\%(孤立SAVR)
        \item Elixhauser合并症指数:6 vs 4(孤立SAVR)
        \item \textbf{明确表明:<65岁接受TAVR的患者并非低危患者}
    \end{itemize}

    \item \textbf{Ross手术使用显著增加}:
    \begin{itemize}
        \item 从2016年的0.4\%增至2024年的4.8\%
        \item 增长12倍
        \item 反映年轻患者对避免抗凝和长期耐久性的需求
    \end{itemize}

    \item \textbf{机械瓣使用率保持稳定}:
    \begin{itemize}
        \item 约25\%的SAVR患者接受机械瓣
        \item 孤立SAVR中机械瓣使用率轻度增长至32\%
        \item 研究期间无显著变化
    \end{itemize}

    \item \textbf{在所有AVR手术背景下,TAVR仅用于少数<65岁患者}:
    \begin{itemize}
        \item 主要用于高合并症负担患者
        \item 符合指南推荐(SAVR作为年轻患者首选)
        \item 避免了TAVR在低危年轻患者中的过度使用
    \end{itemize}
\end{enumerate}

\subsubsection{研究意义}

\textbf{纠正既往研究的偏倚}:
\begin{itemize}
    \item 既往研究排除联合手术和机械瓣患者,导致\textbf{虚假的均等化印象}
    \item 本研究纳入全谱AVR手术,提供更准确的现实世界数据
    \item 证明TAVR在<65岁患者中的使用\textbf{并非主流},而是针对特定高危人群
\end{itemize}

\textbf{支持当前指南推荐}:
\begin{itemize}
    \item 数据支持SAVR作为<65岁患者的首选治疗
    \item TAVR主要用于高合并症负担患者
    \item 符合精准医学和个体化治疗原则
\end{itemize}

\subsection{临床启示}

\subsubsection{对临床决策的指导}

\begin{enumerate}
    \item \textbf{遵循指南推荐}:
    \begin{itemize}
        \item <65岁非高危AS患者应首选SAVR
        \item 考虑患者预期寿命和瓣膜耐久性
        \item 现实世界数据支持指南推荐的合理性
    \end{itemize}

    \item \textbf{TAVR的合理使用}:
    \begin{itemize}
        \item 保留给高合并症负担的年轻患者
        \item 如:严重心力衰竭、慢性透析、既往心脏手术史
        \item 避免在低危年轻患者中过度使用TAVR
    \end{itemize}

    \item \textbf{机械瓣的考虑}:
    \begin{itemize}
        \item 约25-30\%的年轻患者选择机械瓣
        \item 适合能耐受长期抗凝的患者
        \item 避免未来生物瓣退化需要再次干预
    \end{itemize}

    \item \textbf{Ross手术的新选择}:
    \begin{itemize}
        \item 对于年轻、低危、希望避免抗凝的患者
        \item 特别是<50岁患者(Ross组中位年龄40.3岁)
        \item 需要在有经验的中心进行
    \end{itemize}

    \item \textbf{复杂病变的处理}:
    \begin{itemize}
        \item 40\%的年轻患者需要联合手术
        \item 需要多学科团队(MDT)评估
        \item 考虑冠脉、其他瓣膜、主动脉的联合病变
    \end{itemize}
\end{enumerate}

\subsubsection{对医疗政策的启示}

\begin{enumerate}
    \item \textbf{资源配置}:
    \begin{itemize}
        \item 继续支持高质量的心脏外科项目
        \item 发展Ross手术等专业化技术
        \item 保持TAVR和SAVR的平衡发展
    \end{itemize}

    \item \textbf{医保政策}:
    \begin{itemize}
        \item 对<65岁患者的TAVR使用应有明确适应证
        \item 避免基于便利性而非医学必要性的TAVR使用
        \item 支持Ross手术等创新术式的合理报销
    \end{itemize}

    \item \textbf{质量监控}:
    \begin{itemize}
        \item 监测年轻患者TAVR使用的适当性
        \item 确保高危患者能及时获得TAVR治疗
        \item 评估不同术式的长期结果
    \end{itemize}
\end{enumerate}

\subsubsection{对未来研究的启示}

\begin{enumerate}
    \item \textbf{长期随访研究}:
    \begin{itemize}
        \item 比较<65岁患者TAVR vs SAVR的10-20年结果
        \item 评估瓣膜耐久性
        \item 研究再次干预率
    \end{itemize}

    \item \textbf{随机对照试验}:
    \begin{itemize}
        \item 在<65岁患者中进行TAVR vs SAVR的RCT
        \item 针对不同风险层级(低危、中危)
        \item 关注长期临床终点
    \end{itemize}

    \item \textbf{比较效果研究}:
    \begin{itemize}
        \item 机械瓣 vs 生物瓣 vs TAVR vs Ross手术
        \item 生活质量评估
        \item 成本效果分析
    \end{itemize}

    \item \textbf{预测模型开发}:
    \begin{itemize}
        \item 开发决策辅助工具
        \item 帮助个体化选择最佳治疗方式
        \item 整合患者偏好和价值观
    \end{itemize}
\end{enumerate}

\subsection{研究局限性}

\begin{enumerate}
    \item \textbf{数据库固有局限性}:
    \begin{itemize}
        \item Vizient数据库为行政数据,可能存在编码错误
        \item 缺乏详细的临床和超声心动图数据
        \item 无法获知AS严重程度、症状状态等关键信息
        \item 无法评估手术风险评分(如STS评分)
    \end{itemize}

    \item \textbf{缺乏长期随访数据}:
    \begin{itemize}
        \item 本研究仅分析使用率趋势,未评估临床结果
        \item 无法比较不同治疗方式的死亡率、再住院率
        \item 缺乏瓣膜耐久性数据
        \item 不知道再次干预率
    \end{itemize}

    \item \textbf{混杂因素}:
    \begin{itemize}
        \item 虽然比较了基线特征,但可能存在未测量的混杂因素
        \item 治疗选择受多种因素影响(患者偏好、医生经验、机构资源等)
        \item 无法完全调整选择偏倚
    \end{itemize}

    \item \textbf{代表性问题}:
    \begin{itemize}
        \item 仅包括190家医院
        \item 可能不完全代表全美所有医疗机构
        \item 可能偏向学术中心和大型医院
    \end{itemize}

    \item \textbf{COVID-19疫情影响}:
    \begin{itemize}
        \item 2020年数据受疫情显著影响
        \item 择期手术显著减少
        \item 可能影响趋势判断
    \end{itemize}

    \item \textbf{缺乏具体手术指征}:
    \begin{itemize}
        \item 不知道为何选择TAVR vs SAVR
        \item 无法区分"适当使用"和"可能过度使用"
        \item 缺乏心脏团队决策过程的信息
    \end{itemize}

    \item \textbf{瓣膜类型信息不完整}:
    \begin{itemize}
        \item 不知道具体的TAVR瓣膜型号
        \item 不知道生物瓣的具体类型
        \item 无法评估不同瓣膜产品的影响
    \end{itemize}

    \item \textbf{研究时间段内指南和技术变化}:
    \begin{itemize}
        \item 2016-2024年间TAVR适应证扩大
        \item TAVR技术持续改进
        \item 可能影响使用模式
    \end{itemize}
\end{enumerate}

\subsection{个人笔记}

\subsubsection{关键数字记忆}

\textbf{总体数据}:
\begin{itemize}
    \item 总样本量:34,504例
    \item TAVR:9,834例(28.5\%)
    \item 孤立SAVR:10,982例(31.8\%)
    \item 联合SAVR:13,053例(37.8\%)
    \item Ross手术:635例(1.8\%)
\end{itemize}

\textbf{年龄中位数}:
\begin{itemize}
    \item TAVR:61.3岁(最大)
    \item 孤立SAVR:58.8岁
    \item 联合SAVR:59.1岁
    \item Ross手术:40.3岁(最小)
\end{itemize}

\textbf{TAVR vs 孤立SAVR关键合并症对比}:
\begin{itemize}
    \item 心力衰竭:70.4\% vs 34.5\%
    \item 慢性透析:13.2\% vs 2.6\%
    \item 既往PCI:19.8\% vs 6.2\%
    \item Elixhauser指数:6 vs 4
\end{itemize}

\textbf{使用率趋势}:
\begin{itemize}
    \item TAVR:15.7\%(2016)→ 29.0\%(2024)
    \item 孤立SAVR:43.8\%(2016)→ 27.0\%(2024)
    \item 联合SAVR:约40\%(稳定)
    \item Ross手术:0.4\%(2016)→ 4.8\%(2024)
\end{itemize}

\textbf{机械瓣使用率}:
\begin{itemize}
    \item 所有SAVR:约25-30\%
    \item 孤立SAVR:26.4\% → 31.9\%
    \item 联合SAVR:约25\%(稳定)
\end{itemize}

\subsubsection{重要概念}

\begin{description}
    \item[孤立SAVR] 仅主动脉瓣置换,无其他心脏手术
    \item[联合SAVR] SAVR联合CABG、升主动脉手术、二尖瓣/三尖瓣手术或迷宫手术
    \item[Ross手术] 肺动脉瓣自体移植术,将患者自己的肺动脉瓣移植到主动脉位置
    \item[Elixhauser合并症指数] 综合评估30种合并症的指数,数值越高表示合并症负担越重
    \item[虚假均等化] 既往研究因排除联合手术患者而夸大了TAVR与SAVR使用率的均等化程度
\end{description}

\subsubsection{研究设计的优势}

\begin{enumerate}
    \item \textbf{全谱AVR纳入}:
    \begin{itemize}
        \item 首次纳入机械瓣和联合手术患者
        \item 提供更准确的现实世界全貌
        \item 纠正既往研究的选择偏倚
    \end{itemize}

    \item \textbf{大样本量}:
    \begin{itemize}
        \item 34,504例患者
        \item 190家医院
        \item 9年研究时间跨度
    \end{itemize}

    \item \textbf{全国代表性数据库}:
    \begin{itemize}
        \item Vizient CDB覆盖多种医疗机构
        \item 反映真实世界临床实践
    \end{itemize}
\end{enumerate}

\subsubsection{对中国的启示}

\begin{enumerate}
    \item \textbf{年轻AS患者的治疗选择}:
    \begin{itemize}
        \item 中国<65岁AS患者比例可能更高(风湿性心脏病、二叶瓣等)
        \item SAVR仍应是主流选择
        \item TAVR应保留给高危患者
    \end{itemize}

    \item \textbf{Ross手术的发展}:
    \begin{itemize}
        \item 中国Ross手术开展较少
        \item 可考虑在有经验的中心推广
        \item 为年轻患者提供更多选择
    \end{itemize}

    \item \textbf{机械瓣的使用}:
    \begin{itemize}
        \item 中国年轻患者机械瓣使用率可能更高
        \item 需要平衡抗凝风险和再次干预风险
        \item 充分的患者教育和抗凝管理至关重要
    \end{itemize}

    \item \textbf{多学科团队决策}:
    \begin{itemize}
        \item 40\%患者需要联合手术
        \item 强调MDT在复杂病例中的重要性
        \item 需要心脏外科、介入、影像等多学科协作
    \end{itemize}

    \item \textbf{医保政策考虑}:
    \begin{itemize}
        \item 避免基于经济利益的不当选择
        \item 确保治疗选择基于医学必要性
        \item 支持长期结果更好的术式
    \end{itemize}
\end{enumerate}

\subsubsection{值得思考的问题}

\begin{enumerate}
    \item \textbf{为何联合SAVR占比如此高(40\%)?}
    \begin{itemize}
        \item 反映<65岁AS患者疾病复杂性
        \item 可能包括冠心病、升主动脉扩张、二叶瓣相关主动脉病变
        \item 提示不能简单地将年轻患者等同于低危患者
    \end{itemize}

    \item \textbf{TAVR患者为何合并症如此重?}
    \begin{itemize}
        \item 严格遵循指南:仅高危年轻患者接受TAVR
        \item 可能包括既往心脏手术、透析依赖、严重心衰等
        \item 这些患者外科风险极高或禁忌
    \end{itemize}

    \item \textbf{Ross手术快速增长的原因?}
    \begin{itemize}
        \item 技术改进和经验积累
        \item Ross手术中心增加
        \item 年轻患者对避免抗凝和长期耐久性的需求
        \item 可能部分替代机械瓣
    \end{itemize}

    \item \textbf{孤立SAVR机械瓣使用率为何增加?}
    \begin{itemize}
        \item TAVR分流了部分高危患者后
        \item 接受孤立SAVR的患者更年轻、更低危
        \item 这些患者更适合机械瓣
        \item 避免未来瓣中瓣的不确定性
    \end{itemize}

    \item \textbf{2020年后TAVR使用率为何下降?}
    \begin{itemize}
        \item 可能是COVID-19后手术恢复的影响
        \item 可能反映对年轻患者TAVR使用的更审慎态度
        \item Ross手术的竞争(部分替代TAVR)
        \item 需要更长时间观察趋势
    \end{itemize}
\end{enumerate}

\subsubsection{与既往文献的对比}

\textbf{Sharma T, et al. JACC 2022}(仅纳入孤立AVR):
\begin{itemize}
    \item 报告<65岁患者TAVR与SAVR接近均等化
    \item 本研究纠正:TAVR仅占全部AVR的29\%
    \item 差异原因:Sharma研究排除了联合手术(占40\%)
\end{itemize}

\textbf{PARTNER-3和Evolut低危试验}:
\begin{itemize}
    \item <65岁患者占比<10\%
    \item 缺乏年轻患者的证据
    \item 本研究提供真实世界补充数据
\end{itemize}

\subsubsection{临床实践建议}

\textbf{对于<65岁AS患者的决策流程}:

\begin{enumerate}
    \item \textbf{全面评估}:
    \begin{itemize}
        \item AS严重程度和症状
        \item 合并症和手术风险
        \item 冠脉、其他瓣膜、主动脉情况
        \item 肾功能、肝功能等
    \end{itemize}

    \item \textbf{低危患者}:
    \begin{itemize}
        \item 首选SAVR
        \item 如需长期耐久性:考虑机械瓣(需耐受抗凝)
        \item 如希望避免抗凝:考虑Ross手术(年龄<50岁)或生物瓣
    \end{itemize}

    \item \textbf{高危患者}:
    \begin{itemize}
        \item 考虑TAVR
        \item 如:严重心衰、透析、既往心脏手术、frail等
    \end{itemize}

    \item \textbf{需要联合手术患者}:
    \begin{itemize}
        \item 优先考虑外科联合手术
        \item TAVR + PCI可能增加风险
        \item 需要MDT讨论
    \end{itemize}

    \item \textbf{患者偏好}:
    \begin{itemize}
        \item 充分告知不同术式的利弊
        \item 长期抗凝的意愿
        \item 再次干预的接受度
        \item 生活方式考虑
    \end{itemize}
\end{enumerate}

\subsubsection{未来展望}

\begin{itemize}
    \item 需要<65岁患者TAVR vs SAVR的RCT
    \item 关注10-20年长期结果
    \item 新一代TAVR瓣膜的耐久性数据
    \item Ross手术的长期结果和推广
    \item 个体化决策工具的开发
    \item 不同术式成本效果比较
\end{itemize}

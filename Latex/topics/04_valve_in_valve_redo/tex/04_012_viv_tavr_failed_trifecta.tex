\section{Trifecta生物瓣失败后的瓣中瓣TAVR:斯洛文尼亚注册研究}
\label{sec:04_012_viv_tavr_failed_trifecta}

% ============================================
% 文献信息
% ============================================
\subsection{文献信息}

\begin{itemize}
    \item \textbf{标题}: ViV TAV in degenerated Trifecta valve - Slovenia registry: Optimal treatment strategy
    \item \textbf{作者}: Prof. Matjaž Bunc, MD PhD, FESC; Gregor Vercek, MD; Klemen Steblovnik MD PhD
    \item \textbf{机构}: UKC Ljubljana, Slovenia
    \item \textbf{会议}: TCT (Transcatheter Cardiovascular Therapeutics)
    \item \textbf{数据来源}: 斯洛文尼亚单中心注册研究
    \item \textbf{文献类型}: 回顾性队列研究
\end{itemize}

% ============================================
% 研究背景
% ============================================
\subsection{研究背景}

\subsubsection{Trifecta瓣膜的特殊性}

\textbf{Trifecta瓣膜结构特点}(Abbott,已停产):
\begin{itemize}
    \item \textbf{外挂式瓣叶}(externally mounted leaflets)
    \item 钛金属支架
    \item 牛心包生物材料
\end{itemize}

\textbf{ViV TAVR的特殊风险}:
\begin{enumerate}
    \item \textbf{冠状动脉阻塞风险增加}:
    \begin{itemize}
        \item 外挂式瓣叶设计
        \item 主动脉根部狭窄患者风险更高
        \item 冠状动脉开口低位患者风险更高
    \end{itemize}

    \item \textbf{术后跨瓣压差升高风险}:
    \begin{itemize}
        \item 小尺寸Trifecta瓣膜(19-21mm)
        \item 钛金属支架无法用球囊破裂(fracture)
        \item 可能导致患者-假体不匹配(PPM)
    \end{itemize}
\end{enumerate}

\subsubsection{Trifecta瓣膜的早期失败问题}

\textbf{耐久性研究}(Anselmi et al, Ann Thorac Cardiovasc Surg 2017):

\begin{figure}[h]
\centering
\caption{Trifecta瓣膜中期耐久性研究}
\label{fig:trifecta_durability}
\end{figure}

\begin{itemize}
    \item 研究人群:824例患者(2008-2014年)
    \item 术后30天存活:793例(96.2\%)
    \item 完整随访:793例(100\%),平均2.2年
    \item 研究期间死亡:54例
    \item 研究结束时存活:739例
\end{itemize}

\textbf{中心信息}:
\begin{quote}
``The Trifecta bioprosthesis is a reliable device for aortic valve replacement. Continued surveillance for SVD events is required.''(Trifecta生物瓣是可靠的主动脉瓣置换装置。需要继续监测SVD事件。)
\end{quote}

\textbf{观点}:
\begin{quote}
``Durability is a pivotal characteristic for modern bioprostheses. In the present mid-term follow-up of 824 implants, the Trifecta valve showed excellent hemodynamic properties and consistent durability. Few SVD events were observed, characterized by peculiar timing, pathophysiology, and clinical presentation. Continued follow-up is required.''
\end{quote}

\textbf{性能比较研究}(Yongue et al, Ann Thorac Surg 2021):

对比Trifecta与Perimount瓣膜:
\begin{itemize}
    \item Trifecta早期血流动力学性能\textbf{优越}
    \item 但5年时:
    \begin{itemize}
        \item 跨瓣压差\textbf{快速增加}
        \item 主动脉瓣反流\textbf{更多}
        \item 对植入5年后的长期耐久性\textbf{存在担忧}
    \end{itemize}
\end{itemize}

\textbf{病理机制研究}(J Thorac Cardiovasc Surg 2017):

Trifecta瓣膜早期失败的机制:
\begin{enumerate}
    \item \textbf{瓣叶撕裂}:最常见机制
    \item \textbf{环形血管翳形成}:
    \begin{itemize}
        \item 流入部分的纤维脂肪组织
    \end{itemize}
    \item \textbf{瓣叶钙化}:
    \begin{itemize}
        \item 集中在流出部分的支柱周围
    \end{itemize}
\end{enumerate}

\textbf{FDA警告和停产}(2023年7月):

\begin{itemize}
    \item 2023年2月:FDA发布\textbf{早期结构性退化}警告
    \item 2023年7月31日:Abbott\textbf{宣布停产}Trifecta系列瓣膜
    \item 临床关注:大量已植入患者的再干预需求
\end{itemize}

\subsubsection{真实世界结局数据}

\textbf{Medicare受益人研究}(Guffinger et al, Cardiovasc Revasc Med 2025):

\begin{itemize}
    \item 研究人群:接受Trifecta瓣膜的Medicare受益人
    \item \textbf{10年免于再干预率}:
    \begin{itemize}
        \item 总体:82.4\% (95\%CI: 81.1-83.5\%)
        \item \textbf{>80\%的患者10年内无需再干预}
    \end{itemize}
\end{itemize}

\textbf{再干预选择比较}:

\begin{table}[h]
\centering
\caption{再干预方式的手术死亡率比较}
\label{tab:trifecta_reintervention_mortality}
\begin{tabular}{lcc}
\toprule
\textbf{再干预方式} & \textbf{手术死亡率} & \textbf{p值} \\
\midrule
再次外科主动脉瓣置换(Re-SAVR) & 12.5\% & <0.001 \\
ViV-TAVI & 3.8\% & \\
\bottomrule
\end{tabular}
\end{table}

\textbf{6年免于重复再干预率}:两种方法均>90\%

% ============================================
% 研究方法
% ============================================
\subsection{研究方法}

\subsubsection{UKC Ljubljana ViV注册研究}

\textbf{研究设计}:
\begin{itemize}
    \item 单中心回顾性队列研究
    \item UKC Ljubljana,斯洛文尼亚
    \item ViV TAVR注册数据库
\end{itemize}

\textbf{研究人群}:
\begin{itemize}
    \item 总ViV TAVR:103例
    \item Trifecta瓣膜衰败:19例(18.4\%)
    \item 其他瓣膜类型:
    \begin{itemize}
        \item Freedom Solo:36例(34.6\%)
        \item MitroFlow:13例(12.5\%)
        \item Perceval:21例(20.2\%)
        \item 其他:14例(13.6\%)
    \end{itemize}
\end{itemize}

\subsubsection{Trifecta亚组特征}(N=19)

\textbf{基线特征}:

\begin{table}[h]
\centering
\caption{Trifecta ViV TAVR患者基线特征}
\label{tab:trifecta_baseline}
\begin{tabular}{lccccc}
\toprule
& \textbf{总体} & \textbf{环上瓣} & \textbf{环内瓣} & \textbf{BVR组} & \textbf{无BVR组} \\
& (N=19) & (N=14) & (N=5) & (N=10) & (N=9) \\
\midrule
年龄(岁) & 76.3±6.8 & 75.2±7.1 & 79.4±5.4 & 72.3±3.8 & 80.8±6.7 \\
男性 & 47.4\% & 50.0\% & 40.0\% & 50.0\% & 44.4\% \\
身高(cm) & 166±10 & 168±10 & 160±8 & 168±9 & 164±11 \\
体重(kg) & 73.9±14.2 & 75.1±14.4 & 70.6±14.8 & 79.0 & 65.0 \\
BMI (kg/m²) & 26.8±4.4 & 26.6±4.3 & 27.4±4.9 & 27.5±4.4 & 26.0±4.5 \\
EuroScore II (\%) & 8.0 & 7.2 & 9.6 & 6.9 & 11.5 \\
STS评分 (\%) & 5.6 & 5.0 & 11.0 & 5.5±3.8 & 8.3±5.4 \\
\bottomrule
\end{tabular}
\end{table}

\textbf{Trifecta尺寸分布}:
\begin{itemize}
    \item 19mm:6例(31.6\%)
    \item 21mm:10例(52.6\%)
    \item 23-27mm:3例(15.8\%)
\end{itemize}

\textbf{衰败Trifecta的术前超声参数}:
\begin{itemize}
    \item Vmax:3.8±1.0 m/s
    \item 平均压差:38.1±17.3 mmHg
    \item 有效瓣口面积(EOA):0.7 (0.6-1.2) cm²
    \item 严重主动脉瓣反流:42.1\%
\end{itemize}

\subsubsection{干预方法}

\textbf{ViV TAVR瓣膜选择}:
\begin{itemize}
    \item Evolut R/Pro+/FX/FX+:15例(78.9\%)
    \item Sapien 3 Ultra/Ultra Resilia:3例(15.8\%)
    \item Portico/Navitor:1例(5.3\%)
\end{itemize}

\textbf{生物假体重塑(BVR)}:
\begin{itemize}
    \item 使用BVR:10例(52.6\%)
    \item 未使用BVR:9例(47.4\%)
\end{itemize}

\textbf{BVR技术}(Saxon et al, Structural Heart 2020):
\begin{itemize}
    \item 使用高压球囊预扩张Trifecta钛金属支架
    \item 目的:改善THV扩张和手术血流动力学
    \item 使Trifecta支架变形,增加内径
\end{itemize}

% ============================================
% 主要发现
% ============================================
\subsection{主要发现}

\subsubsection{手术成功率和安全性}

\textbf{手术结局}(总体N=19):

\begin{table}[h]
\centering
\caption{Trifecta ViV TAVR手术结局}
\label{tab:trifecta_procedural_outcomes}
\begin{tabular}{lccccc}
\toprule
& \textbf{总体} & \textbf{环上瓣} & \textbf{环内瓣} & \textbf{BVR组} & \textbf{无BVR组} \\
\midrule
围手术期死亡 & 0.0\% & 0.0\% & 0.0\% & 0.0\% & 0.0\% \\
院内死亡 & 0.0\% & 0.0\% & 0.0\% & 0.0\% & 0.0\% \\
30天生存率 & 100.0\% & 100.0\% & 100.0\% & 100.0\% & 100.0\% \\
\bottomrule
\end{tabular}
\end{table}

\textbf{并发症}:
\begin{itemize}
    \item \textbf{无心脏并发症}
    \item \textbf{无冠状动脉阻塞}
    \item \textbf{无大出血或危及生命的出血}
    \item \textbf{无急性肾损伤}
    \item \textbf{无需新植入永久起搏器}
    \item 血管入路并发症:1例(5.3\%)
    \item 缺血性卒中:2例(10.5\%)
\end{itemize}

\subsubsection{血流动力学改善}

\textbf{术后即刻血流动力学}:

\begin{table}[h]
\centering
\caption{ViV TAVR术后血流动力学参数}
\label{tab:trifecta_hemodynamics}
\begin{tabular}{lccccc}
\toprule
& \textbf{总体} & \textbf{环上瓣} & \textbf{环内瓣} & \textbf{BVR组} & \textbf{无BVR组} \\
\midrule
Vmax (m/s) & 2.2±0.4 & 2.1±0.3 & 2.6±0.4 & 2.3±0.4 & 2.0±0.4 \\
平均压差 (mmHg) & 11.4±4.0 & 10.2±3.4 & 15.5±3.7 & 12.0±4.2 & 10.6±4.0 \\
DVI & 0.44±0.11 & 0.46±0.12 & 0.37±0.08 & 0.45±0.06 & 0.43±0.16 \\
>轻度PVR & 0.0\% & 0.0\% & 0.0\% & 0.0\% & 0.0\% \\
\bottomrule
\end{tabular}
\end{table}

\textbf{关键对比}:
\begin{itemize}
    \item 环内瓣Vmax显著高于环上瓣(2.6 vs 2.1 m/s,p=0.016)
    \item 环内瓣平均压差显著高于环上瓣(15.5 vs 10.2 mmHg,p=0.015)
    \item \textbf{无中度以上瓣周漏}
\end{itemize}

\subsubsection{患者-假体不匹配(PPM)分析}

\textbf{整体ViV注册数据}(N=103):
\begin{itemize}
    \item 无/轻度PPM:33例(32.0\%)
    \item 中度PPM:25例(24.3\%)
    \item 重度PPM:22例(21.4\%)
    \item 数据缺失:23例(22.3\%)
\end{itemize}

\textbf{ViV TAVR前后压差变化}:
\begin{itemize}
    \item 术前Vmax:4.18 m/s
    \item 术后Vmax:2.54 m/s
    \item 术前平均压差:45.6 mmHg
    \item 术后平均压差:15.3 mmHg
\end{itemize}

\subsubsection{生存分析}

\textbf{Kaplan-Meier生存曲线}:

\begin{figure}[h]
\centering
\caption{Trifecta ViV TAVR总体生存率}
\label{fig:trifecta_survival_overall}
\end{figure}

\begin{itemize}
    \item \textbf{12个月生存率}:84.2\% (95\%CI: 69.3-100\%)
    \item 早期30天生存率:100\%
    \item 随访期间有死亡事件发生
\end{itemize}

\textbf{按瓣叶位置分层的生存分析}:

\begin{figure}[h]
\centering
\caption{环内瓣vs环上瓣生存率比较}
\label{fig:trifecta_survival_position}
\end{figure}

\begin{itemize}
    \item 环内瓣(Intra-annular):5例
    \item 环上瓣(Supra-annular):14例
    \item Log-rank检验:p=0.279(\textbf{无统计学差异})
\end{itemize}

\subsubsection{BVR的作用}

\textbf{BVR对压差的影响}:

虽然数据显示BVR组和非BVR组在术后压差上无显著统计学差异(p=0.489),但:
\begin{itemize}
    \item BVR组平均压差:12.0±4.2 mmHg
    \item 非BVR组平均压差:10.6±4.0 mmHg
    \item 两组差异不大
\end{itemize}

\textbf{文献中BVR的证据}(Saxon et al 2020):

\begin{itemize}
    \item BVR使Trifecta瓣膜支架扭曲变形
    \item 改善THV扩张
    \item 改善手术血流动力学
    \item 可能降低患者-假体不匹配
\end{itemize}

\subsubsection{特殊病例}

\textbf{极年轻患者ViV TAVR}(31岁患者):

\begin{table}[h]
\centering
\caption{31岁患者ViV TAVR前后对比}
\label{tab:trifecta_young_case}
\begin{tabular}{lcc}
\toprule
\textbf{参数} & \textbf{ViV前} & \textbf{ViV后(6/10/25)} \\
\midrule
Vmax (m/s) & 4.5 & 2.2 \\
LVEF (\%) & 58 & 62 \\
平均压差 (mmHg) & 58 & 11 \\
CV压力 (mmHg) & 29+CVP & 21+CVP \\
\midrule
\multicolumn{3}{l}{\textit{原瓣膜}:Trifecta 27mm} \\
\multicolumn{3}{l}{\textit{新瓣膜}:Edwards Resilia 26mm} \\
\bottomrule
\end{tabular}
\end{table}

\textbf{临床意义}:
\begin{itemize}
    \item 血流动力学显著改善
    \item LVEF增加(58\%→62\%)
    \item 压差大幅降低(58→11 mmHg)
    \item 证明ViV TAVR在年轻患者中的可行性
\end{itemize}

\subsubsection{Cleveland Clinic经验}

\textbf{研究}(Zmaili et al, JACC 2023):

\begin{itemize}
    \item 研究时间:2013年1月至2023年7月
    \item 78例Trifecta瓣膜失败的ViV TAVR
    \item 平均年龄:73岁(IQR: 69-78岁)
    \item 女性:38.4\%
    \item 平均STS评分:4.77±0.64
\end{itemize}

\textbf{Trifecta尺寸分布}:
\begin{itemize}
    \item 21mm:27例(34.6\%)
    \item 23mm:25例(32.1\%)
    \item 其他尺寸:26例(33.3\%)
\end{itemize}

\textbf{衰败类型}:
\begin{itemize}
    \item 假体性主动脉狭窄:46例(59\%)
    \item 中位ViV时间:75.71个月(IQR: 67.11-92.44个月)
\end{itemize}

\textbf{使用的THV}:
\begin{itemize}
    \item Sapien 3或Sapien 3 Ultra:80.8\%
    \item 23mm THV最常用:56.4\%
\end{itemize}

\textbf{球囊后扩张}:74.7\%的病例

\textbf{结局}:
\begin{itemize}
    \item 出院时平均跨瓣压差:14.81±0.9 mmHg
    \item 内径<21mm的SAVR患者术后压差更高:16.07±1.10 vs 12.01±6.89 mmHg(p=0.035)
    \item \textbf{30天无死亡}
    \item 30天心衰再入院:3.8\%
\end{itemize}

\textbf{结论}:
\begin{quote}
``This study, the largest to date on ViV TAVR interventions in patients with failing Trifecta SAVRs, demonstrated satisfactory short-term outcomes with no 30-day mortality and a low rate of heart failure admissions.''
\end{quote}

% ============================================
% 结论
% ============================================
\subsection{结论}

\subsubsection{主要结论}

\begin{enumerate}
    \item \textbf{ViV TAVR是Trifecta瓣膜失败的有效治疗选择}:
    \begin{itemize}
        \item \textbf{优异的短期安全性}:无围手术期和院内死亡
        \item \textbf{可接受的12个月生存率}:84.2\%
        \item \textbf{显著的血流动力学改善}
        \item \textbf{低并发症率}:除卒中外
    \end{itemize}

    \item \textbf{ViV TAVR相比再次外科手术的优势}:
    \begin{itemize}
        \item 手术死亡率更低(3.8\% vs 12.5\%,p<0.001)
        \item 创伤更小
        \item 恢复更快
        \item 适合高风险患者
    \end{itemize}

    \item \textbf{需要注意的特殊风险}:
    \begin{itemize}
        \item 冠状动脉阻塞(本研究中未发生)
        \item 患者-假体不匹配(特别是小尺寸瓣膜)
        \item 卒中风险(10.5\%)
    \end{itemize}

    \item \textbf{BVR的潜在价值}:
    \begin{itemize}
        \item 可能改善血流动力学
        \item 减少PPM
        \item 需要更多研究证据
    \end{itemize}
\end{enumerate}

\subsubsection{Trifecta瓣膜的临床考虑}

\begin{itemize}
    \item Trifecta瓣膜已停产,但仍有大量患者在随访中
    \item 早期结构性退化风险需要密切监测
    \item ViV TAVR提供了安全有效的再干预选择
    \item 长期耐久性数据仍需进一步研究
\end{itemize}

% ============================================
% 临床启示
% ============================================
\subsection{临床启示}

\subsubsection{对临床实践的启示}

\begin{enumerate}
    \item \textbf{Trifecta瓣膜患者的监测}:
    \begin{itemize}
        \item 定期超声心动图随访
        \item 关注血流动力学变化
        \item 早期识别结构性退化
        \item 及时评估再干预时机
    \end{itemize}

    \item \textbf{ViV TAVR手术计划}:
    \begin{itemize}
        \item 详细的CT评估冠状动脉高度
        \item 评估主动脉根部解剖
        \item 选择合适的THV尺寸
        \item 考虑BVR策略
        \item 准备应对冠状动脉阻塞
    \end{itemize}

    \item \textbf{瓣膜选择建议}:
    \begin{itemize}
        \item Evolut系列:可重新定位,适合复杂解剖
        \item Sapien系列:良好的密封性
        \item 根据瓣膜位置选择:环上vs环内
        \item 考虑患者预期寿命
    \end{itemize}

    \item \textbf{并发症预防}:
    \begin{itemize}
        \item 卒中预防:围手术期抗栓治疗
        \item 冠状动脉保护:必要时准备CHIMNEY技术
        \item 血管并发症:选择合适的入路
    \end{itemize}
\end{enumerate}

\subsubsection{对患者咨询的启示}

\begin{itemize}
    \item ViV TAVR相比再次外科手术风险更低
    \item 手术成功率高,恢复快
    \item 需要讨论卒中风险
    \item 长期耐久性仍需随访
    \item 可能需要未来的再次干预
\end{itemize}

% ============================================
% 研究局限性
% ============================================
\subsection{研究局限性}

\begin{enumerate}
    \item \textbf{样本量小}:
    \begin{itemize}
        \item 仅19例Trifecta ViV TAVR
        \item 单中心经验
        \item 统计效能有限
    \end{itemize}

    \item \textbf{随访时间有限}:
    \begin{itemize}
        \item 中位随访约1年
        \item 缺乏长期耐久性数据
        \item 长期并发症未知
    \end{itemize}

    \item \textbf{回顾性设计}:
    \begin{itemize}
        \item 选择偏倚
        \item 缺乏对照组
        \item 数据完整性受限
    \end{itemize}

    \item \textbf{缺乏随机化比较}:
    \begin{itemize}
        \item BVR使用由操作者决定
        \item 瓣膜选择非随机
        \item 难以确定最佳策略
    \end{itemize}

    \item \textbf{缺乏影像学随访}:
    \begin{itemize}
        \item 无常规CT随访
        \item 无法评估亚临床瓣叶血栓
        \item 瓣膜形态学演变未知
    \end{itemize}
\end{enumerate}

% ============================================
% 个人笔记
% ============================================
\subsection{个人笔记}

\subsubsection{关键数字记忆}

\begin{itemize}
    \item Trifecta ViV TAVR:19例
    \item 30天生存率:100\%
    \item 12个月生存率:84.2\%
    \item 卒中率:10.5\%
    \item 术后平均压差:11.4 mmHg
    \item 无中度以上瓣周漏
    \item 无需新植入起搏器
    \item ViV vs Re-SAVR死亡率:3.8\% vs 12.5\%
\end{itemize}

\subsubsection{重要概念}

\begin{description}
    \item[Trifecta瓣膜] Abbott公司的外挂式瓣叶生物瓣,钛金属支架,已于2023年停产
    \item[BVR] 生物假体重塑(Bioprosthetic Valve Remodeling),使用高压球囊预扩张Trifecta支架
    \item[外挂式瓣叶] 瓣叶安装在支架外侧,增加冠状动脉阻塞风险
    \item[钛金属支架] 无法用球囊破裂(fracture),可能限制THV扩张
    \item[环上瓣vs环内瓣] 指原生Trifecta瓣膜的瓣叶位置,影响ViV血流动力学
\end{description}

\subsubsection{Trifecta瓣膜的独特挑战}

\begin{enumerate}
    \item \textbf{早期结构性退化}:
    \begin{itemize}
        \item FDA 2023年2月警告
        \item 5年后压差快速增加
        \item 反流增加
        \item 多种失败机制:撕裂、血管翳、钙化
    \end{itemize}

    \item \textbf{ViV TAVR技术挑战}:
    \begin{itemize}
        \item 外挂式瓣叶增加冠脉阻塞风险
        \item 钛支架无法破裂
        \item 小尺寸瓣膜PPM风险高
        \item 需要精确的术前规划
    \end{itemize}

    \item \textbf{解决策略}:
    \begin{itemize}
        \item BVR技术
        \item 详细CT评估
        \item 选择可重新定位的THV
        \item 必要时CHIMNEY技术
    \end{itemize}
\end{enumerate}

\subsubsection{临床实践要点}

\begin{enumerate}
    \item \textbf{术前评估}:
    \begin{itemize}
        \item CT测量冠状动脉高度
        \item 主动脉根部尺寸
        \item Trifecta瓣膜尺寸和类型
        \item 衰败机制(狭窄vs反流)
        \item 预测PPM风险
    \end{itemize}

    \item \textbf{手术技术}:
    \begin{itemize}
        \item 考虑BVR(特别是小尺寸瓣膜)
        \item 选择合适的THV尺寸
        \item 准备冠状动脉保护措施
        \item 术中TEE监测
        \item 必要时球囊后扩张
    \end{itemize}

    \item \textbf{术后管理}:
    \begin{itemize}
        \item 密切监测神经系统状况
        \item 评估血流动力学
        \item 抗栓治疗策略
        \item 定期随访
    \end{itemize}
\end{enumerate}

\subsubsection{值得思考的问题}

\begin{enumerate}
    \item \textbf{为什么Trifecta瓣膜早期失败率高?}
    \begin{itemize}
        \item 外挂式设计可能增加应力
        \item 材料和处理问题
        \item 血流动力学因素
        \item 需要更多病理学研究
    \end{itemize}

    \item \textbf{BVR是否应该常规使用?}
    \begin{itemize}
        \item 理论上可改善血流动力学
        \item 本研究未显示显著差异
        \item 可能增加操作复杂性
        \item 需要RCT验证
    \end{itemize}

    \item \textbf{如何优化THV选择?}
    \begin{itemize}
        \item Evolut:可重新定位,适合复杂解剖
        \item Sapien:更好的环形密封
        \item 根据个体解剖选择
        \item 考虑未来再干预可能
    \end{itemize}

    \item \textbf{ViV TAVR的长期耐久性?}
    \begin{itemize}
        \item 本研究随访时间短
        \item 需要5-10年数据
        \item 年轻患者特别重要
        \item 可能需要三次干预
    \end{itemize}

    \item \textbf{卒中率10.5\%是否可接受?}
    \begin{itemize}
        \item 高于一般TAVR
        \item 可能与Trifecta退化机制有关
        \item 需要加强围手术期脑保护
        \item 考虑术中神经监测
    \end{itemize}
\end{enumerate}

\subsubsection{未来研究方向}

\begin{itemize}
    \item 多中心注册研究
    \item Trifecta ViV TAVR的长期随访
    \item BVR的前瞻性评估
    \item 冠状动脉阻塞预测模型
    \item 卒中风险分层和预防策略
    \item 年轻患者的长期管理
    \item 新一代THV在ViV中的应用
\end{itemize}

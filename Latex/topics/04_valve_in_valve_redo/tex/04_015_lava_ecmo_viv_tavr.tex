\section{LAVA-ECMO支持下的ViV TAVR与瓣周漏闭合治疗主动脉生物瓣膜脱垂}
\label{sec:04_015_lava_ecmo_viv_tavr}

% ============================================
% 文献信息
% ============================================
\subsection{文献信息}

\begin{itemize}
    \item \textbf{标题}: LAVA-ECMO supported ViV TAVR \& PVL closure in aortic prosthetic valve dehiscence
    \item \textbf{作者}: Dr. Alvin KO
    \item \textbf{机构}: Queen Elizabeth Hospital, Hong Kong
    \item \textbf{会议}: TCT (Transcatheter Cardiovascular Therapeutics)
    \item \textbf{PDF文件名}: tct-1302-lava-ecmo-supported-viv-tavr-and-pvl-closure-in-aortic-bioprostheti.pdf
    \item \textbf{文献类型}: 病例报告/会议演讲
\end{itemize}

\subsection{研究背景}

\subsubsection{患者病史}

\textbf{60岁男性患者},经历了极为复杂的主动脉瓣疾病治疗过程:

\textbf{时间轴}:
\begin{itemize}
    \item \textbf{2023年9月}:因重度主动脉瓣反流(AR)接受外科组织瓣AVR
    \begin{itemize}
        \item 组织学显示赘生物伴急性炎症
        \item 给予长期抗生素治疗
    \end{itemize}

    \item \textbf{2023年12月}(术后3个月):
    \begin{itemize}
        \item 新发严重瓣周漏(PVL)
        \item 组织瓣部分脱垂
        \item 接受第一次重做AVR
    \end{itemize}

    \item \textbf{2024年5月}(6个月后):
    \begin{itemize}
        \item 急性肺水肿(APO),ICU收治
        \item AVR再次脱垂
        \item 接受第二次重做AVR
        \item 术中心源性休克,需要ECMO支持
        \item 完全性心脏传导阻滞,植入无导线起搏器
    \end{itemize}

    \item \textbf{2024年后期}(术后9个月):
    \begin{itemize}
        \item 再次出现呼吸困难,运动耐量下降
        \item NYHA III-IV级
        \item 超声显示:瓣膜再次脱垂伴PVL导致重度AR
        \item \textbf{心脏团队评估:外科手术不可行}
    \end{itemize}
\end{itemize}

\subsubsection{术前评估}

\textbf{超声心动图}:
\begin{itemize}
    \item Edwards Perimount Tissue \#27生物瓣膜脱垂
    \item 左室射血分数降低
    \item 严重瓣周漏导致重度主动脉瓣反流
\end{itemize}

\textbf{CT评估}:
\begin{itemize}
    \item 瓣环平均直径:32.5 mm(范围31.8-33.2 mm)
    \item 冠状动脉口高度:左冠30°,右冠-20°
    \item LVOT最小直径:约74.8 mm
\end{itemize}

\subsection{主要发现}

\subsubsection{手术方案}

\textbf{心脏团队决策}:
\begin{itemize}
    \item 患者外科风险极高(多次胸骨切开史,上次术中心源性休克)
    \item 决定经导管介入治疗
    \item 预计手术高风险,计划预防性LAVA-ECMO支持
\end{itemize}

\textbf{手术计划}:
\begin{enumerate}
    \item 全麻 + 经食管超声(TEE)
    \item LAVA-ECMO循环支持
    \item 生物瓣膜破裂(Bioprosthetic valve fracture)
    \item ViV TAVR → Evolut FX+ 34mm
    \item 视情况PVL闭合
\end{enumerate}

\subsubsection{手术过程}

\textbf{1. LAVA-ECMO建立}:
\begin{itemize}
    \item 左心室辅助-体外膜肺氧合(LAVA-ECMO)
    \item 提供血流动力学稳定支持
\end{itemize}

\textbf{2. 生物瓣膜破裂}:
\begin{itemize}
    \item 使用28mm球囊预扩张
    \item 意图性生物瓣膜破裂以扩大开口
\end{itemize}

\textbf{3. 第一个THV植入(Evolut FX+ 34mm)}:
\begin{itemize}
    \item 瓣膜部分迁移至左心室
    \item 显著反流
    \item 瓣膜仍在摇动
    \item 发生室颤,需要除颤
    \item \textbf{LAVA-ECMO提供关键血流动力学支持}
\end{itemize}

\textbf{4. 第二个THV植入}:
\begin{itemize}
    \item 另一个Evolut FX+ 34mm瓣膜植入
    \item 瓣膜位置改善
\end{itemize}

\textbf{5. 残余PVL闭合}:
\begin{itemize}
    \item TEE显示残余瓣周漏
    \item 使用16mm血管塞(AVP II)闭合PVL
    \item 成功闭合漏口
\end{itemize}

\textbf{6. 最终结果}:
\begin{itemize}
    \item 主动脉瓣最大流速(AV Vmax):1.89 m/s
    \item 平均压差(Mean PG):7.61 mmHg
    \item 无残余漏口
    \item 侵入性平均梯度:0 mmHg
\end{itemize}

\subsubsection{术后结果}

\textbf{即刻结果}:
\begin{itemize}
    \item LAVA-ECMO术后成功撤除
    \item 术后超声:THV和血管塞位置满意,无残余漏口
    \item 术后第2天从CCU出院
\end{itemize}

\textbf{门诊随访}:
\begin{itemize}
    \item NYHA I-II级
    \item 临床状况显著改善
\end{itemize}

\subsection{结论}

\subsubsection{主要结论}

\begin{enumerate}
    \item \textbf{细致的术前规划至关重要}
    \begin{itemize}
        \item 评估患者复杂病史
        \item 预见潜在并发症
        \item 制定应急预案
    \end{itemize}

    \item \textbf{预防性LAVA-ECMO在高风险TAVR中提供血流动力学稳定}
    \begin{itemize}
        \item 允许在血流动力学不稳定时继续操作
        \item 为处理并发症(如室颤、瓣膜迁移)争取时间
        \item 减少紧急转换为外科手术的需要
    \end{itemize}

    \item \textbf{TAVR和PVL闭合在极高外科风险患者中是可行的}
    \begin{itemize}
        \item 即使在复杂解剖和反复手术史的情况下
        \item 需要多学科团队协作
        \item 技术上具有挑战性但可以成功完成
    \end{itemize}
\end{enumerate}

\subsection{临床启示}

\subsubsection{对临床实践的指导}

\textbf{1. 适应症选择}:
\begin{itemize}
    \item 反复AVR手术失败的患者
    \item 外科手术禁忌或极高风险
    \item 生物瓣膜脱垂伴PVL
    \item 血流动力学不稳定或预期不稳定
\end{itemize}

\textbf{2. LAVA-ECMO的应用}:
\begin{itemize}
    \item \textbf{预防性应用}优于抢救性应用
    \item 适用于:
    \begin{itemize}
        \item 左室功能严重受损
        \item 严重血流动力学障碍
        \item 预期操作时间长或复杂
        \item 高并发症风险(瓣膜迁移、冠脉阻塞等)
    \end{itemize}
    \item 允许在稳定条件下处理技术挑战
\end{itemize}

\textbf{3. 生物瓣膜破裂技术}:
\begin{itemize}
    \item 在小尺寸生物瓣膜中扩大开口
    \item 减少瓣膜-瓣膜不匹配
    \item 降低患者-瓣膜不匹配(PPM)风险
    \item 改善血流动力学结果
\end{itemize}

\textbf{4. PVL管理}:
\begin{itemize}
    \item 术中即刻评估残余PVL
    \item 根据严重程度决定是否闭合
    \item 血管塞是有效的闭合装置
    \item TEE引导下精确定位和释放
\end{itemize}

\textbf{5. 多学科团队协作}:
\begin{itemize}
    \item 介入心脏病学
    \item 心脏外科
    \item 超声心动图
    \item 麻醉和体外循环
    \item 术前充分讨论和准备
\end{itemize}

\subsubsection{技术要点}

\textbf{LAVA-ECMO管理}:
\begin{itemize}
    \item 术前建立,术中维持
    \item 流量根据血流动力学需求调整
    \item 抗凝管理
    \item 撤除时机的判断
\end{itemize}

\textbf{瓣膜选择}:
\begin{itemize}
    \item Evolut FX+自膨胀瓣膜
    \item 34mm尺寸适合27mm生物瓣膜
    \item 自膨胀特性允许重新定位
    \item 考虑瓣膜高度和冠脉距离
\end{itemize}

\textbf{影像引导}:
\begin{itemize}
    \item TEE实时监测
    \item 透视定位
    \item 评估瓣膜位置和功能
    \item 检测残余PVL
\end{itemize}

\subsection{研究局限性}

\begin{enumerate}
    \item \textbf{单中心病例报告}
    \begin{itemize}
        \item 缺乏对照组
        \item 不能评估不同策略的相对效益
        \item 需要多中心研究验证
    \end{itemize}

    \item \textbf{随访时间相对较短}
    \begin{itemize}
        \item 长期耐久性未知
        \item 血栓形成风险
        \item 瓣膜功能退化
        \item 再次干预需求
    \end{itemize}

    \item \textbf{LAVA-ECMO的成本和资源}
    \begin{itemize}
        \item 需要专业团队和设备
        \item 增加医疗成本
        \item 不是所有中心都具备条件
        \item 成本效益分析缺乏
    </itemize}

    \item \textbf{学习曲线}
    \begin{itemize}
        \item 技术复杂,需要经验
        \item 并发症风险
        \item 操作者依赖性强
    \end{itemize}

    \item \textbf{并发症数据有限}
    \begin{itemize}
        \item 血管并发症
        \item 出血风险
        \item 神经系统并发症
        \item 肾功能影响
    \end{itemize}
\end{enumerate}

\subsection{个人笔记}

\subsubsection{关键数据记忆}

\begin{itemize}
    \item \textbf{患者}:60岁男性,多次AVR手术史
    \item \textbf{手术时间轴}:
    \begin{itemize}
        \item 2023/9:首次AVR(组织瓣)
        \item 2023/12:第一次重做(3个月后)
        \item 2024/5:第二次重做(6个月后,术中ECMO)
        \item 2024年后期:TAVR治疗(9个月后)
    \end{itemize}
    \item \textbf{生物瓣膜}:Edwards Perimount Tissue \#27
    \item \textbf{THV}:Evolut FX+ 34mm × 2
    \item \textbf{PVL闭合装置}:16mm AVP II血管塞
    \item \textbf{最终梯度}:平均7.61 mmHg
    \item \textbf{住院时间}:术后第2天出院
\end{itemize}

\subsubsection{重要概念}

\begin{description}
    \item[LAVA-ECMO] 左心室辅助-体外膜肺氧合。结合了左心室辅助装置和ECMO的功能,提供循环和呼吸支持。

    \item[生物瓣膜破裂] 使用高压球囊故意破裂生物瓣膜环,以扩大开口,便于经导管瓣膜植入,减少PPM。

    \item[ViV TAVR] Valve-in-Valve TAVR,在已植入的生物瓣膜内再植入经导管瓣膜。

    \item[瓣周漏(PVL)] 瓣膜缝合环周围的血流漏口,可导致血流动力学显著影响和心衰症状。

    \item[预防性ECMO] 在预期高风险操作前预先建立ECMO支持,而非等到发生血流动力学崩溃后再建立。
\end{description}

\subsubsection{临床思考}

\textbf{1. 何时考虑预防性机械循环支持?}
\begin{itemize}
    \item 严重左室功能不全(EF <30\%)
    \item 复杂的冠脉或瓣膜解剖
    \item 高风险操作(多瓣膜、再次干预)
    \item 既往血流动力学不稳定史
    \item 预期长时间或复杂的操作
\end{itemize}

\textbf{2. LAVA-ECMO vs 标准ECMO vs Impella}
\begin{itemize}
    \item \textbf{LAVA-ECMO}:
    \begin{itemize}
        \item 优点:同时提供循环和呼吸支持,流量大
        \item 缺点:需要体外循环团队,抗凝需求高
    \end{itemize}
    \item \textbf{Impella}:
    \begin{itemize}
        \item 优点:左室减负荷,操作简便
        \item 缺点:流量有限,不提供氧合
    \end{itemize}
    \item 选择取决于患者具体情况和中心经验
\end{itemize}

\textbf{3. 反复瓣膜脱垂的原因?}
\begin{itemize}
    \item 感染(本例有赘生物)
    \item 缝合技术问题
    \item 组织质量差
    \item 瓣环钙化严重
    \item 血流动力学应力过大
    \item 本例可能与初次感染性心内膜炎有关
\end{itemize}

\textbf{4. 为何不在首次脱垂时就选择TAVR?}
\begin{itemize}
    \item 2023年患者相对年轻(59-60岁)
    \item TAVR耐久性长期数据不足
    \item 外科手术仍是金标准
    \item 保留经导管选项作为后备
    \item 直到第三次脱垂且外科风险极高时才选择TAVR
\end{itemize}

\subsubsection{病例特殊之处}

\begin{enumerate}
    \item \textbf{极其复杂的病史}:一年内4次主动脉瓣手术
    \item \textbf{预防性机械支持}:预见性使用LAVA-ECMO
    \item \textbf{双瓣膜ViV}:一次操作中植入两个经导管瓣膜
    \item \textbf{联合PVL闭合}:在复杂ViV基础上加做PVL闭合
    \item \textbf{优异结果}:尽管极高风险,患者快速恢复
\end{enumerate}

\subsubsection{对未来实践的启示}

\begin{itemize}
    \item \textbf{预防性策略}:高风险患者应考虑预防性支持
    \item \textbf{技术整合}:结合多种技术(瓣膜破裂、ViV、PVL闭合)
    \item \textbf{团队准备}:充分的术前准备和团队协调
    \item \textbf{设备待命}:准备备用装置和应急方案
    \item \textbf{持续创新}:探索新技术应用于复杂病例
\end{itemize}

\subsubsection{值得进一步研究的问题}

\begin{enumerate}
    \item LAVA-ECMO支持的TAVR的多中心注册研究
    \item 预防性vs抢救性机械支持的比较研究
    \item 双瓣膜ViV的长期耐久性
    \item 联合PVL闭合的技术标准化
    \item 成本效益分析
    \item 最佳抗血栓治疗策略
\end{enumerate}

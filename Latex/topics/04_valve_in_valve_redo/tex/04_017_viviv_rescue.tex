\section{瓣中瓣中瓣抢救:术中瓣膜栓塞的序贯经导管瓣膜植入}
\label{sec:04_017_viviv_rescue}

% ============================================
% 文献信息
% ============================================
\subsection{文献信息}

\begin{itemize}
    \item \textbf{标题}: Valve-in-Valve-in-Valve Rescue: Sequential Transcatheter Valve Deployment for Intraprocedural Valve Embolization - Bailout of valve embolization in a native annulus using a three-valve ViViV configuration
    \item \textbf{作者}: Antigone Kostea, MD; Nicolas van Mieghem, MD-PhD; Rutger-Jan Nuis, MD-PhD
    \item \textbf{机构}: Erasmus MC, Rotterdam
    \item \textbf{会议}: TCT (Transcatheter Cardiovascular Therapeutics)
    \item \textbf{PDF文件名}: tct-1304-valve-in-valve-in-valve-rescue-sequential-transcatheter-valve-depl.pdf
    \item \textbf{文献类型}: 病例报告/会议演讲
\end{itemize}

\subsection{研究背景}

\subsubsection{患者病史}

\textbf{91岁男性患者}:

\textbf{基本信息}:
\begin{itemize}
    \item 身高:190 cm
    \item 体重:84 kg
    \item BMI:23.27 kg/m²
\end{itemize}

\textbf{主诉}:
\begin{itemize}
    \item 运动耐量下降,劳力性呼吸困难
    \item NYHA II级
    \item CCS 0级(无心绞痛)
\end{itemize}

\textbf{既往心脏病史}:
\begin{itemize}
    \item 高血压
    \item \textbf{2006年}:稳定型心绞痛,LAD PCI
    \item \textbf{2014年}:LAD和RCA PCI
    \item \textbf{2022年}:
    \begin{itemize}
        \item Mid-LAD、proximal Cx和OM PCI
        \item 中度主动脉瓣狭窄伴左室功能减退
    \end{itemize}
\end{itemize}

\textbf{心电图}:
\begin{itemize}
    \item 窦性心律,61 bpm
    \item 右束支传导阻滞(RBBB)
    \item 一度房室传导阻滞
    \item PR间期:274 ms
    \item QRS时限:194 ms
    \item QTc:515 ms
\end{itemize}

\subsubsection{术前检查}

\textbf{冠状动脉造影(CAG)}:
\begin{itemize}
    \item 右优势系统
    \item Mid-LAD和proximal Cx支架通畅
    \item LAD和Cx残余中等病变
    \item Mid-RCA中等病变
    \item 决定保守治疗
\end{itemize}

\textbf{经胸超声心动图(TTE)}:
\begin{itemize}
    \item 左室功能减退
    \item 左房扩大(LAVi 48.5 ml/m²)
    \item \textbf{重度经典低流量低梯度主动脉瓣狭窄(cLFLG AS)}:
    \begin{itemize}
        \item 平均压差(MPG):34 mmHg
        \item 最大流速(Vmax):4.0 m/s
        \item 每搏量指数(SVi):32.2 ml/m²
        \item 主动脉瓣口面积指数(AVAi):0.36 cm²/m²
    \end{itemize}
    \item 中度主动脉瓣反流
\end{itemize}

\textbf{计算机断层扫描(CT)}:
\begin{itemize}
    \item \textbf{三叶式主动脉瓣}
    \item 瓣叶和融合嵴重度钙化
    \item Agatston钙化评分:3110
    \item \textbf{非常大的主动脉瓣环}:
    \begin{itemize}
        \item 瓣环面积:735.3 mm²
        \item 瓣环周长:97.3 mm
    \end{itemize}
    \item \textbf{冠状动脉高度}:
    \begin{itemize}
        \item 左冠状动脉口高度:16.6 mm
        \item 右冠状动脉口高度:22.5 mm
    \end{itemize}
    \item 膜部间隔长度:3.6 mm(较短)
    \item LVOT直径:最小16 mm,最大20 mm
    \item 冠状窦直径:右20 mm,左19 mm
    \item 冠状窦高度:右10 mm,左10.2 mm
    \item 主动脉窦管交界:19.6 mm
    \item 股动脉入路适宜植入
    \item 两侧髂股动脉最小直径:9.0 mm
\end{itemize}

\subsection{主要发现}

\subsubsection{心脏团队决策}

\textbf{风险评估}:
\begin{itemize}
    \item 高龄(91岁)
    \item 衰弱、功能能力降低、谵妄风险高
    \item 外科风险评分:
    \begin{itemize}
        \item STS评分:4.96\%
        \item EuroSCORE II:4.26\%
    \end{itemize}
    \item 分类为\textbf{高风险}患者
\end{itemize}

\textbf{传导风险}:
\begin{itemize}
    \item 基线RBBB + QRS >160 ms + PR >240 ms
    \item 术后高度房室传导阻滞高风险
    \item 决定预防性植入起搏器
\end{itemize}

\textbf{瓣膜选择}:
\begin{itemize}
    \item 经股动脉TAVR
    \item 32 mm Myval Octapro瓣膜
    \item 球囊扩张瓣膜
\end{itemize}

\subsubsection{手术过程及并发症}

\textbf{初始瓣膜植入(第一个MyVal Octapro 32 mm)}:

\textbf{并发症发生}:
\begin{enumerate}
    \item \textbf{瓣膜误装}:
    \begin{itemize}
        \item MyVal Octapro 32 mm瓣膜装载错误
        \item 球囊充盈时瓣膜仅部分扩张
    \end{itemize}

    \item \textbf{球囊放气时向后反冲}:
    \begin{itemize}
        \item 瓣膜向下移位进入LVOT
        \item 瓣膜栓塞至左心室流出道
        \item 严重的即刻并发症
    \end{itemize}
\end{enumerate}

\textbf{抢救策略 - 第二个瓣膜(Navitor Vision 35mm)}:

\begin{enumerate}
    \item \textbf{瓣膜选择}:
    \begin{itemize}
        \item Navitor Vision 35mm自膨胀瓣膜
        \item 用于固定栓塞的瓣膜
        \item 具有NaviSeal裙边,改善密封和固定
        \item 特别适合大瓣环解剖
    \end{itemize}

    \item \textbf{植入过程}:
    \begin{itemize}
        \item 成功植入Navitor Vision 35mm
        \item 固定了栓塞的MyVal瓣膜
        \item 瓣膜位置改善
    \end{itemize}

    \item \textbf{后扩张}:
    \begin{itemize}
        \item 球囊后扩张以实现完全贴靠
        \item 优化瓣膜位置和密封
    \end{itemize}
\end{enumerate}

\textbf{持续瓣周漏 - 第三个瓣膜(MyVal Octapro 32 mm)}:

\begin{enumerate}
    \item \textbf{残余问题}:
    \begin{itemize}
        \item 尽管植入了两个瓣膜
        \item 仍存在显著瓣周漏
        \item 血流动力学不理想
    \end{itemize}

    \item \textbf{第三个瓣膜植入}:
    \begin{itemize}
        \item 决定植入另一个MyVal Octapro 32 mm
        \item 进一步改善密封
        \item 减少瓣周漏
    \end{itemize}

    \item \textbf{最终结果}:
    \begin{itemize}
        \item 成功植入第三个瓣膜
        \item \textbf{形成三瓣膜ViViV结构}
        \item 无冠状动脉阻塞
        \item 侵入性平均梯度:0 mmHg
    \end{itemize}
\end{enumerate}

\subsubsection{术后评估}

\textbf{术后即刻超声心动图}:
\begin{itemize}
    \item THV功能良好(无PVL)
    \item 平均压差:7 mmHg
    \item 最大流速:1.8 m/s
    \item 深位于LVOT,视觉上阻挡前二尖瓣叶
    \item 但二尖瓣口未测得压差
\end{itemize}

\textbf{术后CT}:
\begin{itemize}
    \item \textbf{经导管心脏瓣膜部署满意}
    \item 支架深位,延伸至LVOT
    \item ViViV结构上缘高于冠状动脉口
    \item 无冠状动脉阻塞证据
    \item \textbf{瓣下壁龛低密度影(2 cm)}:
    \begin{itemize}
        \item 可能为血栓或增生组织
        \item 未造成阻塞
    \end{itemize}
\end{itemize}

\textbf{临床结果}:
\begin{itemize}
    \item 术后过程顺利无事件
    \item 开始治疗性抗凝(阿哌沙班)
    \item 术后第4天出院,状况稳定
\end{itemize}

\subsection{结论}

\subsubsection{主要结论}

\begin{enumerate}
    \item \textbf{THV栓塞至LVOT是罕见但严重的并发症}
    \begin{itemize}
        \item 发生率:原生瓣环TAVR中<1\%
        \item 可由多种原因引起
        \item 需要立即识别和处理
        \item 可能危及生命
    \end{itemize}

    \item \textbf{本例瓣膜栓塞由球囊扩张瓣膜误装引起}
    \begin{itemize}
        \item 技术性错误
        \item 导致瓣膜部分扩张
        \item 球囊放气时缺乏足够锚定
        \item 瓣膜向后反冲栓塞
    \end{itemize}

    \item \textbf{分步救援策略确保结构稳定性和保持血流动力学}
    \begin{itemize}
        \item 35mm Navitor SEV用于固定
        \item NaviSeal裙边改善大瓣环中的密封和固定
        \item 第三个瓣膜进一步优化结果
    \end{itemize}

    \item \textbf{三瓣膜(ViViV)结构的长期预后和耐久性未知}
    \begin{itemize}
        \item 缺乏文献报道
        \item 需要密切随访
        \item 潜在问题:血栓、结构退化、血流动力学恶化
    \end{itemize}
\end{enumerate}

\subsubsection{瓣膜栓塞的原因和预防}

根据TRAVEL注册研究(Kim WK et al, EHJ 2019)和其他文献(Frumkin D et al, Front Cardiovasc Med 2022):

\textbf{主动脉TVEM原因(n=217)}:
\begin{itemize}
    \item 定位错误:47.9\%
    \item 操作失误:22.1%
    \item 后扩张:8.3%
    \item 起搏失败:6.0%
    \item 尺寸错误:2.8%
    \item 其他/未知:14.7%
\end{itemize}

\textbf{救援策略}:
\begin{itemize}
    \item 再定位:46.1%
    \item ViV:88.9%
    \item 转换:13.4%
\end{itemize}

\textbf{主动脉TVEM死亡率}:
\begin{itemize}
    \item \textbf{30天死亡率}:
    \begin{itemize}
        \item 全部TVEM:18.6%
        \item 术中发生:17.5%
        \item 早期发生(<60分钟):30.1%
        \item 延迟发生(>60分钟):3.2%
    \end{itemize}
    \item \textbf{1年死亡率}:
    \begin{itemize}
        \item 全部TVEM:30.5%
        \item 术中发生:26.8%
        \item 早期发生:38.5%
        \item 延迟发生:8.9%
    \end{itemize}
\end{itemize}

\subsection{临床启示}

\subsubsection{对临床实践的指导}

\textbf{1. 预防瓣膜栓塞}:

\begin{itemize}
    \item \textbf{术前评估}:
    \begin{itemize}
        \item 精确的瓣环测量
        \item 评估钙化分布
        \item 识别不利解剖(大瓣环、钙化少)
        \item 选择合适尺寸和类型的瓣膜
    \end{itemize}

    \item \textbf{器械准备}:
    \begin{itemize}
        \item 仔细检查瓣膜装载
        \item 遵循制造商说明
        \item 双人核查关键步骤
        \item 准备备用装置
    \end{itemize}

    \item \textbf{操作技术}:
    \begin{itemize}
        \item 精确定位
        \item 适当的植入深度
        \item 球囊扩张瓣膜需充分扩张
        \item 自膨胀瓣膜避免过早释放
        \item 在不稳定情况下使用快速起搏
    \end{itemize}
\end{itemize}

\textbf{2. 瓣膜栓塞的识别}:

\begin{itemize}
    \item \textbf{即刻征象}:
    \begin{itemize}
        \item 瓣膜位置异常(透视)
        \item 血流动力学不稳定
        \item 严重反流(超声)
        \item 支架移位或变形
    \end{itemize}

    \item \textbf{延迟征象}:
    \begin{itemize}
        \item 术后血流动力学恶化
        \item 心衰症状加重
        \item 超声显示瓣膜移位
        \item 梯度异常升高或AR
    \end{itemize}
\end{itemize}

\textbf{3. 瓣膜栓塞的处理}:

\begin{itemize}
    \item \textbf{即刻稳定}:
    \begin{itemize}
        \item 血流动力学支持
        \item 血管活性药物
        \item 考虑机械循环支持(Impella、ECMO)
        \item 维持冠脉灌注
    \end{itemize}

    \item \textbf{再定位尝试}:
    \begin{itemize}
        \item 如果瓣膜未完全释放
        \item 使用球囊或套索技术
        \item 自膨胀瓣膜可能收回重新释放
        \item 成功率取决于栓塞时机和程度
    \end{itemize}

    \item \textbf{ViV救援}:
    \begin{itemize}
        \item 如果无法再定位
        \item 植入第二个瓣膜固定
        \item 选择较大尺寸瓣膜
        \item 自膨胀瓣膜可能更适合
    \end{itemize}

    \item \textbf{外科转换}:
    \begin{itemize}
        \item 如果经导管救援失败
        \item 严重冠脉阻塞
        \item 瓣膜完全脱入LV或主动脉
        \item 持续血流动力学不稳定
    \end{itemize}
\end{itemize}

\textbf{4. ViViV结构的管理}:

\begin{itemize}
    \item \textbf{抗血栓治疗}:
    \begin{itemize}
        \item 本例使用治疗性抗凝(阿哌沙班)
        \item 考虑瓣下血栓风险
        \item 多层金属结构增加血栓风险
        \item 个体化治疗策略
    \end{itemize}

    \item \textbf{密切随访}:
    \begin{itemize}
        \item 术后即刻和出院前超声
        \item 1个月、3个月、6个月超声
        \item 之后每6-12个月
        \item CT评估结构完整性
    \end{itemize}

    \item \textbf{监测重点}:
    \begin{itemize}
        \item 瓣膜功能(梯度、EOA)
        \item 瓣周漏和瓣内漏
        \item 血栓形成征象
        \item 结构性瓣膜退化
        \item 传导异常进展
        \item 临床症状和功能状态
    \end{itemize}
\end{itemize}

\subsubsection{Navitor瓣膜的特点}

根据PORTICO NG研究(Sondergaard L et al, EuroIntervention 2023):

\textbf{Navitor经导管心脏瓣膜特点}:
\begin{itemize}
    \item 自膨胀镍钛合金支架
    \item \textbf{NaviSeal裙边}:
    \begin{itemize}
        \item 额外的内外层密封裙边
        \item 改善密封,减少PVL
        \item 在大瓣环解剖中特别有用
    \end{itemize}
    \item 可回收和再定位
    \item 低位植入设计
\end{itemize}

\textbf{30天和1年结果(PORTICO NG研究)}:
\begin{itemize}
    \item 全因死亡率:30天2.8\%,1年13.6\%
    \item 中-重度PVL:30天0.9\%,1年0.9\%
    \item 新起搏器植入率:30天14.5\%,1年16.1\%
    \item 平均梯度:30天7.9±3.6 mmHg,1年8.1±3.8 mmHg
\end{itemize}

\subsection{研究局限性}

\begin{enumerate}
    \item \textbf{罕见病例报告}
    \begin{itemize}
        \item 三瓣膜ViViV结构极其罕见
        \item 缺乏类似病例对照
        \item 结果可能不可推广
        \item 需要更多病例积累
    \end{itemize}

    \item \textbf{随访时间有限}
    \begin{itemize}
        \item 仅报告早期结果(出院时)
        \item 长期耐久性完全未知
        \item 血栓、退化风险未明
        \item 需要延长随访至少5-10年
    \end{itemize}

    \item \textbf{技术错误的细节不完整}
    \begin{itemize}
        \item 瓣膜误装的具体原因未详述
        \item 难以从中吸取教训
        \item 预防措施不明确
        \item 制造商应调查和改进
    \end{itemize}

    \item \textbf{血流动力学评估不完整}:
    \begin{itemize}
        \item 缺乏详细的血流动力学数据
        \item EOA未报告
        \item 二尖瓣功能评估有限
        \item LVOT阻塞风险评估不足
    \end{itemize}

    \item \textbf{成本和资源考虑}:
    \begin{itemize}
        \item 三个昂贵的瓣膜
        \item 延长手术时间
        \item 增加辐射暴露
        \item 成本效益未评估
    \end{itemize}
\end{enumerate}

\subsection{个人笔记}

\subsubsection{关键数据记忆}

\begin{itemize}
    \item \textbf{患者}:91岁男性
    \item \textbf{基线心律}:RBBB + 一度AVB,QRS 194 ms,PR 274 ms
    \item \textbf{瓣环}:非常大,面积735.3 mm²,周长97.3 mm
    \item \textbf{钙化评分}:Agatston 3110(重度)
    \item \textbf{术前AS}:cLFLG,MPG 34 mmHg,Vmax 4.0 m/s,AVAi 0.36 cm²/m²
    \item \textbf{植入瓣膜}:
    \begin{itemize}
        \item 第1个:MyVal Octapro 32mm(栓塞)
        \item 第2个:Navitor Vision 35mm(固定)
        \item 第3个:MyVal Octapro 32mm(优化密封)
    \end{itemize}
    \item \textbf{最终梯度}:MPG 7 mmHg,Vmax 1.8 m/s
    \item \textbf{出院}:术后第4天
    \item \textbf{抗凝}:阿哌沙班
\end{itemize}

\subsubsection{重要概念}

\begin{description}
    \item[瓣膜栓塞(TVEM)] 经导管心脏瓣膜在植入过程中或术后移位到预期位置之外,包括主动脉栓塞、左心室栓塞等。是TAVR的严重并发症。

    \item[ViViV] Valve-in-Valve-in-Valve,三个瓣膜嵌套结构。极其罕见,通常是并发症救援的结果而非计划性的。

    \item[NaviSeal裙边] Navitor瓣膜的特殊设计,额外的内外层密封裙边,旨在改善密封、减少PVL,特别适合大瓣环解剖。

    \item[经典低流量低梯度AS(cLFLG AS)] 左室功能减退(EF <50\%)伴低梯度(MPG <40 mmHg)和低流量(SVi <35 ml/m²)的重度AS(AVAi <0.6 cm²/m²)。

    \item[瓣膜误装] 球囊扩张瓣膜在压接和装载到输送系统时的技术错误,可导致瓣膜部分扩张、移位或栓塞。
\end{description}

\subsubsection{临床思考}

\textbf{1. 为何选择三瓣膜结构而非外科转换?}
\begin{itemize}
    \item \textbf{患者因素}:
    \begin{itemize}
        \item 91岁高龄
        \item 外科风险极高(STS 4.96\%,但实际可能更高)
        \item 衰弱状态
        \item 多次PCI史,可能有广泛冠脉疾病
    \end{itemize}
    \item \textbf{技术因素}:
    \begin{itemize}
        \item 瓣膜栓塞但未完全脱入LV
        \item 血流动力学相对稳定
        \item 有经导管救援可能
        \item 外科团队准备时间长
    \end{itemize}
    \item \textbf{结果导向}:
    \begin{itemize}
        \item 最终血流动力学良好
        \item 避免了开胸手术
        \item 快速恢复和出院
        \item 证明了决策的正确性
    \end{itemize}
\end{itemize}

\textbf{2. 三瓣膜结构的潜在问题}
\begin{itemize}
    \item \textbf{血栓风险}:
    \begin{itemize}
        \item 多层金属框架
        \item 血流紊乱增加
        \item 瓣下龛腔
        \item 需要强化抗凝
    \end{itemize}
    \item \textbf{结构耐久性}:
    \begin{itemize}
        \item 复杂应力分布
        \item 瓣叶疲劳可能加速
        \item 框架相互作用
        \item 长期未知
    \end{itemize}
    \item \textbf{血流动力学}:
    \begin{itemize}
        \item 有效开口可能受限
        \item 湍流增加
        \item 溶血风险
        \item 需密切监测
    \end{itemize}
    \item \textbf{再次干预}:
    \begin{itemize}
        \item 如果需要再次瓣膜干预
        \item 选项极其有限
        \item 可能只能外科手术
        \item "burning bridges"
    \end{itemize}
\end{itemize>

\textbf{3. 球囊扩张vs自膨胀瓣膜的栓塞风险}
\begin{itemize}
    \item \textbf{球囊扩张瓣膜}:
    \begin{itemize}
        \item 需要球囊充盈扩张
        \item 扩张前无径向力
        \item 误装或不完全扩张时易栓塞
        \item 本例即为典型
    \end{itemize}
    \item \textbf{自膨胀瓣膜}:
    \begin{itemize}
        \item 逐渐释放和扩张
        \item 持续径向力
        \item 可回收和再定位
        \item 栓塞风险较低但仍存在
    \end{itemize}
    \item 两类瓣膜各有优劣,选择取决于解剖和操作者经验
\end{itemize}

\textbf{4. 预防性起搏器的决策}
\begin{itemize}
    \item \textbf{本例高风险因素}:
    \begin{itemize}
        \item 基线RBBB
        \item QRS >160 ms(194 ms)
        \item PR >240 ms(274 ms)
        \item 膜部间隔短(3.6 mm)
    \end{itemize}
    \item \textbf{预防性策略}:
    \begin{itemize}
        \item 术前植入临时起搏器
        \item 术后监测传导
        \item 必要时植入永久起搏器
    \end{itemize}
    \item \textbf{三瓣膜结构的影响}:
    \begin{itemize}
        \item 深位于LVOT
        \item 更大的传导系统压迫
        \item 起搏器需求可能进一步增加
    \end{itemize>
\end{itemize}

\subsubsection{病例特殊之处}

\begin{enumerate}
    \item \textbf{罕见的三瓣膜ViViV结构}:文献中极少报道
    \item \textbf{术中瓣膜栓塞的成功救援}:避免了外科转换
    \item \textbf{序贯使用不同类型瓣膜}:球囊扩张和自膨胀结合
    \item \textbf{高龄患者(91岁)}:优异结果令人鼓舞
    \item \textbf{快速恢复}:尽管复杂操作,术后第4天出院
    \item \textbf{技术性失误的教训}:瓣膜装载的重要性
\end{enumerate}

\subsubsection{对未来实践的启示}

\begin{itemize}
    \item \textbf{并发症管理}:即使严重并发症也可能经导管救援
    \item \textbf{器械准备}:仔细核查每个步骤,特别是瓣膜装载
    \item \textbf{备用方案}:随时准备B计划、C计划
    \item \textbf{团队协作}:复杂救援需要团队高效配合
    \item \textbf{技术创新}:不同瓣膜类型的互补优势
    \item \textbf{持续学习}:从失误中学习,分享经验
\end{itemize}

\subsubsection{值得进一步研究的问题}

\begin{enumerate}
    \item 三瓣膜ViViV结构的长期预后(耐久性、血栓、血流动力学)
    \item 瓣膜栓塞的预防策略和风险预测模型
    \item 不同瓣膜类型在救援场景中的性能比较
    \item NaviSeal技术在大瓣环和栓塞救援中的价值
    \item 球囊扩张瓣膜装载的标准化和质量控制
    \item ViViV结构的抗血栓治疗最佳策略
    \item 计算流体力学模拟三瓣膜结构的血流动力学
\end{enumerate}

\section{紧急ViV TAVR伴高冠状动脉阻塞风险:ShortCut瓣叶修饰技术应用}
\label{sec:04_021_emergent_viv_coronary_risk}

% ============================================
% 文献信息
% ============================================
\subsection{文献信息}

\begin{itemize}
    \item \textbf{标题}: Emergent Valve-in-Valve TAVR at High Risk of Coronary Occlusion Treated with ShortCut™ Leaflet Modification Device
    \item \textbf{作者}: Curtiss T. Stinis, MD, FACC, FSCAI
    \item \textbf{机构}: Scripps Clinic \& Research Foundation, La Jolla, California, United States
    \item \textbf{会议}: TCT (Transcatheter Cardiovascular Therapeutics)
    \item \textbf{PDF文件名}: tct-1438-emergent-valve-in-valve-tavr-at-high-risk-of-coronary-occlusion-tre.pdf
    \item \textbf{文献类型}: 病例报告
    \item \textbf{利益冲突披露}: 作者为Edwards Lifesciences, Medtronic, Shockwave Medical, Boston Scientific的顾问
\end{itemize}

\subsection{研究背景}

\subsubsection{ViV TAVR中冠状动脉阻塞的风险}

Valve-in-Valve (ViV) TAVR中,冠状动脉阻塞是一个严重但相对罕见的并发症:
\begin{itemize}
    \item \textbf{发生率}:ViV TAVR中约2.5-3.5\%(高于原生瓣膜TAVR的0.5-1\%)
    \item \textbf{高危因素}:
    \begin{itemize}
        \item 冠状动脉开口低位(<12 mm)
        \item 瓣叶至冠状动脉距离(VTC)<4 mm
        \item 外科瓣膜瓣叶长度较长
        \item 主动脉根部较小
        \item 外翻瓣叶
    \end{itemize}
    \item \textbf{后果}:冠状动脉阻塞可导致心肌梗死、心源性休克、死亡
\end{itemize}

\subsubsection{预防策略}

传统预防冠状动脉阻塞的方法:
\begin{enumerate}
    \item \textbf{BASILICA(Bioprosthetic Aortic Scallop Intentional Laceration to prevent Coronary Artery obstruction)}:
    \begin{itemize}
        \item 使用电切导丝劈裂瓣叶
        \item 需要复杂的技术和设备
        \item 手术时间较长
        \item 学习曲线陡峭
    \end{itemize}

    \item \textbf{预防性PCI/CABG}:高风险或复杂

    \item \textbf{外科再次手术}:对高危患者风险过高
\end{enumerate}

\subsubsection{ShortCut装置的创新}

\textbf{ShortCut™}(Pi-Cardia公司)是首个专用瓣叶劈裂装置:
\begin{itemize}
    \item \textbf{FDA批准}:用于劈裂生物瓣膜瓣叶,治疗冠状动脉阻塞高危患者
    \item \textbf{创新设计}:
    \begin{itemize}
        \item 使用同一装置可安全、简单地劈裂单个或双瓣叶
        \item 机械劈裂元件,可控激活
    \end{itemize}
    \item \textbf{直观控制}:响应式系统允许精确定位和瓣叶劈裂
    \item \textbf{高效流程}:无缝整合到常规TAVR工作流程中
    \item \textbf{优势}:相比BASILICA,更简单、更快、复杂性更低
\end{itemize}

\subsection{病例详情}

\subsubsection{患者基本信息}

\begin{itemize}
    \item \textbf{年龄/性别}:69岁男性
    \item \textbf{基础疾病}:
    \begin{itemize}
        \item 高血压(HTN)
        \item 高脂血症(HLD)
        \item 冠心病(CAD):STEMI病史,LAD支架植入(PCI)
        \item 室性心动过速,已行消融
        \item 射血分数保留的心力衰竭(HFpEF)
        \item 脑卒中(CVA)病史
        \item 严重主动脉瓣狭窄(AS)
    \end{itemize}
\end{itemize}

\subsubsection{主动脉瓣手术史}

\begin{table}[h]
\centering
\caption{患者主动脉瓣手术时间线}
\label{tab:av_surgery_timeline}
\begin{tabular}{lp{11cm}}
\toprule
\textbf{时间} & \textbf{事件} \\
\midrule
1999年 & \textbf{首次SAVR},治疗严重AS \\
2010年 & \textbf{再次SAVR}(Redo SAVR),原因:感染性心内膜炎 \\
 & • 植入27mm Magna Ease生物瓣膜 \\
2025年 & 急性失代偿心力衰竭和心源性休克 \\
 & • 需要多巴酚丁胺(dobutamine)支持 \\
 & • 超声:AVA = 2.4 cm²,平均梯度 = 16 mmHg \\
 & • \textbf{严重主动脉瓣反流(Severe AR)} \\
 & • 转诊\textbf{急诊ViV TAVR} \\
\bottomrule
\end{tabular}
\end{table}

\subsubsection{术前风险评估}

\textbf{外科瓣膜参数(27mm Magna Ease)}:
\begin{itemize}
    \item \textbf{True ID(真实内径)}:25 mm
    \item \textbf{瓣膜高度}:17 mm
\end{itemize}

\textbf{冠状动脉高度测量(CT评估)}:
\begin{itemize}
    \item \textbf{左冠状动脉(LCA)}:
    \begin{itemize}
        \item 冠状动脉高度:12.5 mm
        \item 瓣叶至冠状动脉距离(LC VTC):7.3 mm(相对安全)
    \end{itemize}
    \item \textbf{右冠状动脉(RCA)}:
    \begin{itemize}
        \item 冠状动脉高度:14.5 mm
        \item 瓣叶至冠状动脉距离(RC VTC):\textbf{1.2 mm}(极高危!)
    \end{itemize}
\end{itemize}

\textbf{风险分层}:
\begin{itemize}
    \item \textbf{RC VTC仅1.2 mm}:远低于4 mm的安全阈值
    \item 提示ViV TAVR后\textbf{RCA阻塞风险极高}
    \item 患者处于心源性休克,需要紧急干预
    \item 外科再次手术(第三次)风险极高
    \item 决定:\textbf{急诊ViV TAVR + ShortCut瓣叶修饰}
\end{itemize}

\subsubsection{手术计划}

\begin{itemize}
    \item \textbf{经导管瓣膜}:26mm SAPIEN 3 Ultra RESILIA THV
    \item \textbf{瓣叶修饰}:ShortCut装置劈裂RC(右冠)瓣叶
    \item \textbf{理由}:预防RCA阻塞,确保冠状动脉血流
\end{itemize}

\subsection{手术过程}

\subsubsection{基线评估(TEE)}

\textbf{术前超声心动图}:
\begin{itemize}
    \item \textbf{严重主动脉瓣反流}
    \item 可能存在瓣周漏(PVL)
    \item 两个切面均显示显著的反流信号
\end{itemize}

\subsubsection{ShortCut瓣叶修饰步骤}

\textbf{1. ShortCut定位}:

\begin{itemize}
    \item \textbf{目标}:右冠(RC)瓣叶
    \item \textbf{定位策略}:
    \begin{itemize}
        \item 定位臂(positioning arm)\textbf{偏心放置}
        \item 朝向偏心位置的RCA
        \item 确保最大限度地劈裂瓣叶,为RCA留出空间
    \end{itemize}
    \item \textbf{影像引导}:
    \begin{itemize}
        \item 透视下可见ShortCut装置和定位臂
        \item TEE确认位置
        \item 可见ShortCut定位臂和瓣膜支柱(valve post)
    \end{itemize}
\end{itemize}

\textbf{2. ShortCut激活和瓣叶劈裂}:

\begin{itemize}
    \item 激活劈裂元件(splitting element)
    \item 机械劈裂RC瓣叶
    \item 透视下可见:
    \begin{itemize}
        \item 劈裂元件激活(左图)
        \item RC瓣叶成功劈裂(右图)
    \end{itemize}
    \item 劈裂后瓣叶向外移位,为冠状动脉留出空间
\end{itemize}

\textbf{3. 劈裂后评估(TEE)}:

\begin{itemize}
    \item 评估瓣叶劈裂效果
    \item 仍有主动脉瓣反流(预期)
    \item 准备进行ViV TAVR
\end{itemize}

\subsubsection{ViV TAVR植入}

\textbf{瓣膜植入}:
\begin{itemize}
    \item 26mm SAPIEN 3 Ultra RESILIA THV
    \item 在劈裂的外科瓣膜内植入
    \item 透视下成功释放
\end{itemize}

\textbf{即刻术后TEE评估}:
\begin{itemize}
    \item 发现\textbf{瓣周漏(PVL)}
    \item 瓣膜可能扩张不足
\end{itemize}

\subsubsection{球囊后扩张优化}

\textbf{高压球囊扩张}:
\begin{itemize}
    \item 使用\textbf{28mm True球囊}
    \item 高压扩张(具体压力值未提及)
    \item 目的:优化瓣膜贴壁,消除PVL
\end{itemize}

\textbf{后扩张后TEE}:
\begin{itemize}
    \item \textbf{PVL消除}
    \item 瓣膜扩张良好
    \item 反流显著改善
\end{itemize}

\subsubsection{冠状动脉评估}

\textbf{终末造影}:
\begin{itemize}
    \item \textbf{RCA血流良好维持}
    \item 未发生冠状动脉阻塞
    \item ShortCut瓣叶修饰成功预防了RCA阻塞
\end{itemize}

\subsection{术后结果}

\subsubsection{超声心动图随访}

\begin{table}[h]
\centering
\caption{术后超声心动图结果}
\label{tab:post_procedure_echo}
\begin{tabular}{lcc}
\toprule
\textbf{参数} & \textbf{次日} & \textbf{1个月} \\
\midrule
瓣膜面积(Valve Area) & 2.3 cm² & 2.3 cm² \\
平均梯度(Mean Gradient) & 15.2 mmHg & 14.7 mmHg \\
瓣周漏(PVL) & 无 & 无 \\
中央反流(Central AR) & 微量(Trace) & 无 \\
射血分数(EF) & 42.8\% & \textbf{56.7\%} \\
\bottomrule
\end{tabular}
\end{table}

\textbf{关键观察}:
\begin{itemize}
    \item 瓣膜面积和梯度稳定,血流动力学良好
    \item PVL完全消除(球囊后扩张成功)
    \item 中央反流从微量改善至无
    \item \textbf{射血分数显著改善}:42.8\% → 56.7\%(改善13.9\%)
    \item 提示左室功能恢复
\end{itemize}

\subsubsection{临床结果}

\textbf{急性期}:
\begin{itemize}
    \item 心源性休克迅速缓解
    \item 停用多巴酚丁胺支持
    \item 血流动力学稳定
\end{itemize}

\textbf{30天随访}:
\begin{itemize}
    \item \textbf{症状完全缓解}
    \item 患者报告生活质量显著改善
    \item \textbf{有动力恢复锻炼,特别是举重(powerlifting)}
    \item 功能状态优异
\end{itemize}

\subsection{主要发现与结论}

\subsubsection{核心结论}

\begin{enumerate}
    \item \textbf{ShortCut装置的有效性}:
    \begin{itemize}
        \item 在RCA阻塞高危患者中实现了\textbf{安全、可控、快速}的RC瓣叶劈裂
        \item 患者因严重AI处于休克状态,紧急情况下成功应用
        \item 成功预防了RCA阻塞并发症
    \end{itemize}

    \item \textbf{ShortCut相比BASILICA的优势}:
    \begin{itemize}
        \item \textbf{简单性}:使心脏团队能够治疗原本不符合条件的患者
        \item \textbf{复杂性更低}:技术要求降低,学习曲线平缓
        \item \textbf{手术时间更快}:快速整合到TAVR流程
        \item \textbf{可及性}:扩大了可治疗患者群体
    \end{itemize}

    \item \textbf{综合治疗策略的成功}:
    \begin{itemize}
        \item 有效的瓣叶修饰(ShortCut)
        \item S3UR瓣膜植入
        \item 高压球囊优化
        \item 三者结合导致:
        \begin{itemize}
            \item 中央反流快速消除
            \item 瓣周反流快速消除
            \item 心源性休克缓解
        \end{itemize}
    \end{itemize}

    \item \textbf{优异的临床结果}:
    \begin{itemize}
        \item 症状完全缓解
        \item 射血分数显著改善(42.8\% → 56.7\%)
        \item 患者功能状态优秀,能够恢复剧烈运动
    \end{itemize}
\end{enumerate}

\subsection{临床启示}

\subsubsection{对冠状动脉阻塞风险评估的启示}

\begin{enumerate}
    \item \textbf{关键测量指标}:
    \begin{itemize}
        \item \textbf{VTC(瓣叶至冠状动脉距离)}是最重要的风险指标
        \item VTC <4 mm:高危
        \item VTC <2 mm:极高危(本例RCA VTC = 1.2 mm)
        \item 冠状动脉高度 <12 mm:增加风险
        \item 外科瓣膜瓣叶高度:影响VTC
    \end{itemize}

    \item \textbf{术前CT评估必不可少}:
    \begin{itemize}
        \item 精确测量冠状动脉高度
        \item 计算VTC距离
        \item 评估主动脉根部解剖
        \item 识别偏心冠状动脉开口
        \item 规划瓣叶修饰策略
    \end{itemize}

    \item \textbf{个体化风险分层}:
    \begin{itemize}
        \item 左右冠状动脉风险可能不同
        \item 本例:LCA相对安全(VTC 7.3 mm),RCA极高危(VTC 1.2 mm)
        \item 仅需劈裂RC瓣叶,无需双瓣叶修饰
    \end{itemize}
\end{enumerate}

\subsubsection{对ShortCut技术的启示}

\begin{enumerate}
    \item \textbf{适应证}:
    \begin{itemize}
        \item ViV TAVR中VTC <4 mm的患者
        \item 特别是VTC <2 mm的极高危患者
        \item 可用于单瓣叶或双瓣叶修饰
        \item 适用于紧急/急诊情况
    \end{itemize}

    \item \textbf{技术要点}:
    \begin{itemize}
        \item \textbf{精确定位}:使用TEE和透视双重引导
        \item \textbf{偏心放置}:针对偏心冠状动脉,可调整定位臂位置
        \item \textbf{可控劈裂}:机械激活,可预测结果
        \item \textbf{快速整合}:无缝融入TAVR流程,不显著延长手术时间
    \end{itemize}

    \item \textbf{安全性考虑}:
    \begin{itemize}
        \item 本例无ShortCut相关并发症
        \item 劈裂后仍有反流(预期),ViV TAVR后消除
        \item 成功预防了冠状动脉阻塞
        \item 未发生瓣叶撕裂延伸、主动脉损伤等并发症
    \end{itemize}

    \item \textbf{与BASILICA的比较}:
    \begin{itemize}
        \item ShortCut:机械劈裂,专用装置
        \item BASILICA:电切导丝,需要更复杂的技术
        \item ShortCut优势:
        \begin{itemize}
            \item 更简单、更直观
            \item 学习曲线更平缓
            \item 手术时间更短
            \item 设备可及性可能更好
        \end{itemize}
        \item BASILICA优势:
        \begin{itemize}
            \item 更多临床经验和数据
            \item 已在多中心验证
        \end{itemize}
    \end{itemize}
\end{enumerate}

\subsubsection{对紧急/急诊ViV TAVR的启示}

\begin{enumerate}
    \item \textbf{紧急情况下的决策}:
    \begin{itemize}
        \item 本例:心源性休克,严重AR,需要急诊干预
        \item 外科再次手术(第三次)风险极高
        \item ViV TAVR是合理选择,但有冠状动脉阻塞风险
        \item ShortCut使紧急ViV TAVR成为可能
    \end{itemize}

    \item \textbf{快速评估和执行}:
    \begin{itemize}
        \item 即使在紧急情况下,也要完成关键评估(CT, TEE)
        \item ShortCut的简单性使其适用于紧急情况
        \item 不需要延长准备时间或复杂设备
    \end{itemize}

    \item \textbf{综合优化策略}:
    \begin{itemize}
        \item 瓣叶修饰(ShortCut)
        \item 适当瓣膜选择(26mm SAPIEN 3)
        \item 球囊后扩张优化(28mm True球囊)
        \item 三者结合确保最佳结果
    \end{itemize}
\end{enumerate}

\subsubsection{对球囊后扩张的启示}

\begin{itemize}
    \item 本例中,初次植入后有PVL
    \item \textbf{28mm True球囊}高压扩张(超过瓣膜标称直径26mm)
    \item 成功消除PVL,优化瓣膜性能
    \item 提示:ViV TAVR中,积极的球囊后扩张可能有益
    \item 需要平衡:优化扩张 vs 瓣膜损伤/瓣周漏加重
\end{itemize}

\subsection{与现有证据的关联}

\subsubsection{ViV TAVR中冠状动脉阻塞}

\begin{itemize}
    \item \textbf{发生率}:ViV TAVR中2.5-3.5\%,高于原生瓣膜TAVR
    \item \textbf{VIVID注册研究}:VTC <4 mm时冠状动脉阻塞风险显著增加
    \item \textbf{预测模型}:已开发多种风险评分和计算器
    \item 本例VTC 1.2 mm,处于极高危范围
\end{itemize}

\subsubsection{BASILICA经验}

\begin{itemize}
    \item BASILICA首次报道于2017年(Khan等)
    \item 多中心经验显示技术成功率约90\%
    \item 可有效预防冠状动脉阻塞
    \item 但需要专门培训和设备
    \item 手术时间相对较长
\end{itemize}

\subsubsection{ShortCut的新证据}

\begin{itemize}
    \item ShortCut是较新的FDA批准装置
    \item 本病例展示了其在紧急情况下的可行性
    \item 需要更多多中心数据验证其安全性和有效性
    \item 与BASILICA的头对头比较研究尚缺乏
\end{itemize}

\subsection{研究局限性}

\begin{enumerate}
    \item 单一病例报告,缺乏对照组和大样本数据
    \item 未提供ShortCut的详细技术参数(如劈裂深度、宽度)
    \item 随访时间相对较短(1个月)
    \item 未提供ShortCut装置的成本信息
    \item 与BASILICA缺乏直接比较
    \item 作者有利益冲突(多家公司顾问),可能存在偏倚
    \item 未讨论潜在的ShortCut相关并发症或失败模式
\end{enumerate}

\subsection{个人笔记}

\subsubsection{关键数字记忆}

\begin{itemize}
    \item 患者年龄:69岁
    \item 手术史:1999年首次SAVR,2010年再次SAVR(心内膜炎),2025年ViV TAVR
    \item 外科瓣膜:27mm Magna Ease,True ID 25mm,高度17mm
    \item \textbf{冠状动脉参数}:
    \begin{itemize}
        \item LCA高度:12.5 mm,LC VTC:7.3 mm
        \item RCA高度:14.5 mm,\textbf{RC VTC:1.2 mm(极危险!)}
    \end{itemize}
    \item 术前:AVA 2.4 cm²,平均梯度16 mmHg,严重AR,心源性休克
    \item 经导管瓣膜:26mm SAPIEN 3 Ultra RESILIA
    \item 球囊后扩张:28mm True球囊
    \item 术后次日:AVA 2.3 cm²,平均梯度15.2 mmHg,EF 42.8\%
    \item 术后1月:AVA 2.3 cm²,平均梯度14.7 mmHg,\textbf{EF 56.7\%}(改善13.9\%)
\end{itemize}

\subsubsection{重要概念}

\begin{description}
    \item[VTC (Valve-to-Coronary distance)] 瓣叶至冠状动脉距离 - ViV TAVR中预测冠状动脉阻塞的关键参数
    \item[ShortCut] 首个FDA批准的专用瓣叶劈裂装置,用于预防TAVR中冠状动脉阻塞
    \item[BASILICA] Bioprosthetic Aortic Scallop Intentional Laceration to prevent Coronary Artery obstruction - 使用电切导丝的瓣叶修饰技术
    \item[瓣叶劈裂 (Leaflet Splitting)] 预防性劈裂外科瓣膜瓣叶,使其向外移位,为冠状动脉留出空间
    \item[心源性休克 (Cardiogenic Shock)] 严重心脏泵血功能障碍导致的休克状态,需要正性肌力药物支持
\end{description}

\subsubsection{技术亮点}

\begin{enumerate}
    \item \textbf{ShortCut的创新设计}:
    \begin{itemize}
        \item 机械劈裂元件,可控激活
        \item 定位臂可偏心放置,针对偏心冠状动脉
        \item 同一装置可劈裂单或双瓣叶
        \item 与常规TAVR流程无缝整合
    \end{itemize}

    \item \textbf{多模态影像引导}:
    \begin{itemize}
        \item 术前CT:精确测量VTC和冠状动脉高度
        \item 透视:实时监测ShortCut定位和激活
        \item TEE:确认ShortCut位置,评估劈裂效果和术后结果
    \end{itemize}

    \item \textbf{球囊后扩张优化}:
    \begin{itemize}
        \item 使用超过标称尺寸的球囊(28mm vs 26mm瓣膜)
        \item 成功消除PVL
        \item 提示ViV TAVR中球囊后扩张的重要性
    \end{itemize}

    \item \textbf{紧急情况下的快速执行}:
    \begin{itemize}
        \item 患者处于心源性休克
        \item 仍完成必要的术前评估(CT)
        \item ShortCut简单性使其适用于紧急情况
        \item 从评估到手术完成迅速
    \end{itemize}
\end{enumerate}

\subsubsection{临床思考}

\begin{enumerate}
    \item \textbf{VTC 1.2 mm是否还有其他选择?}
    \begin{itemize}
        \item 如此极端的VTC,不进行瓣叶修饰几乎肯定会发生冠状动脉阻塞
        \item 外科手术:第三次手术,风险极高,且患者处于休克
        \item 预防性冠状动脉保护(chimney stenting等):技术复杂,长期效果不确定
        \item ShortCut/BASILICA:唯一合理的微创选择
        \item 本例选择ShortCut是正确的
    \end{itemize}

    \item \textbf{为何仅劈裂RC瓣叶?}
    \begin{itemize}
        \item LC VTC 7.3 mm相对安全
        \item RC VTC 1.2 mm极危险
        \item 个体化策略:仅劈裂高危侧
        \item 减少干预范围,降低复杂性
        \item 单瓣叶劈裂可能保留更好的瓣膜功能(虽然是外科瓣膜)
    \end{itemize}

    \item \textbf{ShortCut vs BASILICA如何选择?}
    \begin{itemize}
        \item 如果两者都可用,如何决策?
        \item ShortCut优势:简单、快速、学习曲线平缓
        \item BASILICA优势:更多经验、已在多中心验证
        \item 可能取决于:
        \begin{itemize}
            \item 中心经验和设备可及性
            \item 紧急程度(紧急情况可能倾向ShortCut)
            \item 解剖特点(某些情况可能更适合一种技术)
            \item 成本考虑
        \end{itemize}
    \end{itemize}

    \item \textbf{球囊后扩张的策略?}
    \begin{itemize}
        \item 本例使用28mm球囊扩张26mm瓣膜
        \item Oversizing约8\%
        \item 成功消除PVL
        \item 但oversizing过度可能导致:
        \begin{itemize}
            \item 瓣膜支架断裂或变形
            \item 主动脉根部损伤
            \item 传导阻滞
        \end{itemize}
        \item 需要平衡:充分扩张 vs 并发症风险
        \item 经验性策略:逐步增加球囊尺寸,直到PVL消除
    \end{itemize}

    \item \textbf{心源性休克患者EF改善的机制?}
    \begin{itemize}
        \item EF从42.8\%改善至56.7\%(1个月)
        \item 可能机制:
        \begin{itemize}
            \item 严重AR消除,减少容量负荷
            \item 左室后负荷优化
            \item 多巴酚丁胺影响消除(术前使用)
            \item 心肌功能恢复(stunned myocardium recovery)
        \end{itemize}
        \item 提示:即使基线EF降低,ViV TAVR后仍有恢复潜力
    \end{itemize}
\end{enumerate}

\subsubsection{对未来实践的启示}

\begin{enumerate}
    \item \textbf{扩大ViV TAVR适应证}:
    \begin{itemize}
        \item ShortCut使原本因冠状动脉阻塞风险而禁忌的患者可接受治疗
        \item VTC <4 mm甚至<2 mm的患者现在有微创选择
        \item 特别是高危、禁忌再次外科手术的患者
    \end{itemize}

    \item \textbf{术前评估标准化}:
    \begin{itemize}
        \item 所有ViV TAVR前应行CT评估
        \item 测量VTC作为常规
        \item VTC <4 mm时考虑瓣叶修饰
        \item 建立本中心的风险分层和决策流程
    \end{itemize}

    \item \textbf{技术培训和推广}:
    \begin{itemize}
        \item ShortCut的简单性使其易于学习和推广
        \item 可能成为ViV TAVR的标准技术之一
        \item 需要更多培训项目和教育资源
    \end{itemize}

    \item \textbf{长期随访的重要性}:
    \begin{itemize}
        \item 本例仅随访1个月
        \item 需要长期随访评估:
        \begin{itemize}
            \item 瓣膜耐久性
            \item 劈裂瓣叶的长期效果
            \item 晚期冠状动脉并发症
            \item 再次干预需求
        \end{itemize}
    \end{itemize}
\end{enumerate}

\subsubsection{未来研究方向}

\begin{itemize}
    \item ShortCut多中心注册研究
    \item ShortCut vs BASILICA的随机对照试验
    \item 长期随访数据(>5年)
    \item 不同VTC阈值下ShortCut的必要性
    \item ShortCut在原生瓣膜TAVR中的应用(小主动脉根部)
    \item 成本-效益分析
    \item 学习曲线研究
    \item 失败模式和并发症分析
\end{itemize}

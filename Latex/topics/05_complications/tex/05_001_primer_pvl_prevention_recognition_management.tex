\section{主动脉瓣周围漏(PVL)的预防、识别与管理入门}
\label{sec:05_001_primer_pvl}

% ============================================
% 文献信息
% ============================================
\subsection{文献信息}

\begin{itemize}
    \item \textbf{标题}: A Primer on Prevention, Recognition and Management of Aortic PVL
    \item \textbf{作者}: Hasan Jilaihawi, MD
    \item \textbf{机构}: Interventional Cardiologist, Director of Multimodality Imaging Corelab, Professor of Cardiology, Smidt Heart Institute, Cedars-Sinai Medical Center, Los Angeles, CA
    \item \textbf{会议}: TCT (Transcatheter Cardiovascular Therapeutics) 2025
    \item \textbf{演讲时间}: October 27, 2025
    \item \textbf{PDF文件名}: a-primer-on-prevention-recognition-and-management-of-aortic-pvl.pdf
    \item \textbf{文献类型}: 会议演讲 / 教学讲座
\end{itemize}

\subsection{研究背景}

\subsubsection{历史回顾:2010年的PVL问题}

在TAVR发展早期(约2010年),主动脉瓣周围漏(Paravalvular Leak, PVL)是术后死亡率的\textbf{主要且相对常见}的预测因子。

\textbf{多项研究证据}:

\begin{table}[h]
\centering
\caption{早期研究中PVL与死亡率的关系}
\label{tab:pvl_mortality_2010}
\begin{tabular}{lccc}
\toprule
\textbf{研究} & \textbf{患者数} & \textbf{显著PV AR比例} & \textbf{HR (95\% CI)} \\
\midrule
Sinning et al & 146 & 22 (15.0\%) & 2.4 (1.0–5.4) \\
Tamburino et al & 663 & 139 (21.0\%) & 3.79 (1.57-9.10) \\
Moat et al & 877 & 115/849 (13.6\%) & 1.66 (1.10–2.51) \\
Gilard et al & 3195 & 316/1915 (16.5\%) & 2.49 (1.91-3.25) \\
Abdel-Wahab et al & 690 & 119 (17.2\%) & 2.43 (1.22-4.85) \\
Vasa-Nicotera et al & 122 & 20 (16.3\%) & 4.19 (2.05-8.59) \\
\bottomrule
\end{tabular}
\end{table}

\textbf{关键发现}:
\begin{itemize}
    \item 中到重度PVL发生率:13.6-21.0\%
    \item 死亡风险增加:1.66-4.19倍
    \item 所有研究均显示PVL与不良预后显著相关
\end{itemize}

\subsubsection{PARTNER试验时代的局限性}

在PARTNER试验中,\textbf{3D测量未被用于TAVR术前评估},这导致了一些严重的后果:

\textbf{经典案例}(见第7-8页):
\begin{itemize}
    \item 使用2D TEE最小径向测量:21.6 mm
    \item 回顾性CT测量瓣环面积:797 mm\textsuperscript{2}
    \item 植入26 mm Edwards SAPIEN瓣膜
    \item 结果:严重的、动态的、致命性PVL,伴瓣膜摇摆
    \item \textbf{结论}:严重尺寸不足导致严重AR和瓣膜不稳定
\end{itemize}

\subsection{预防策略}

\subsubsection{1. 成像优化:3D CT vs 2D TEE}

\textbf{Cedars-Sinai数据}(Jilaihawi et al, JACC 2012):

横断面CT测量比2D TEE最大径向测量对PVL有\textbf{更好的区分价值}:

\begin{itemize}
    \item CT横断面测量的ROC曲线下面积(AUC)显著优于2D TEE
    \item CT能更准确反映瓣环的真实大小和形态
    \item 2D TEE容易低估瓣环尺寸,特别是在椭圆形瓣环中
\end{itemize}

\subsubsection{2. 3D TEE的证据}

\textbf{3D TEE (Qlab) vs 2D TEE}(Jilaihawi et al, JACC 2013):

研究显示横断面3D TEE测量比2D TEE最大径向测量对PVL有更好的区分价值:

\begin{itemize}
    \item $\Delta D_{mean\ Qlab} = (D_{mean\ Qlab} - TAVR\ size)$
    \item AUC 0.68 (95\% CI 0.54-0.81), p=0.031
    \item $\Delta D_{2d\ TEE}$ AUC仅为0.52 (95\% CI 0.35-0.68)
    \item 结论:3D TEE测量对PVL有显著更好的预测价值
\end{itemize}

\subsubsection{3. 设备迭代改进}

\textbf{密封裙技术的重要性}:

不仅3D尺寸选择重要,\textbf{设备迭代}也能减少PVL:

案例:重度钙化AS伴LVOT钙化
\begin{itemize}
    \item 瓣环面积:643 mm\textsuperscript{3}
    \item 使用29 mm Sapien 3(S3)瓣膜
    \item 钙化结节阻止瓣膜完全膨胀
    \item 但密封裙技术实现\textbf{零PVL},无贴壁不良
    \item 现在S3 Ultra Resilia (S3UR)进一步改进
\end{itemize}

\textbf{其他平台}:多个平台也采用了密封裙技术的优势

\subsubsection{4. 计算机模拟辅助}

\textbf{DASI模拟:降阶建模(ROM)}

案例:二叶瓣巨大瓣环
\begin{itemize}
    \item 75岁男性
    \item 非缺血性心肌病;ICD;低流量低梯度AS
    \item EF 27\%; DI 0.2; AVA 0.5 cm\textsuperscript{2}; 平均梯度14 mmHg
    \item 1型LR融合,无钙化嵴
    \item 瓣环879 mm\textsuperscript{2},ICD 40.3 mm
    \item LVOT 1000 mm\textsuperscript{2}
    \item 升主动脉最大径44 mm
\end{itemize}

\textbf{主要关注点}:
\begin{enumerate}
    \item 瓣膜栓塞/移位风险
    \item 瓣周漏风险
\end{enumerate}

\textbf{DASI模拟结果}(BE 29 球囊扩张瓣膜):

\begin{table}[h]
\centering
\caption{不同充盈体积下的瓣膜尺寸}
\label{tab:dasi_simulation}
\begin{tabular}{lccc}
\toprule
\textbf{位置 (mm)} & \textbf{标准体积} & \textbf{+3cc} & \textbf{+5cc} \\
\midrule
流入道 & 26.8/27.4 & 27.7/28.1 & 28.8/28.9 \\
腰部 & 24.8/25.2 & 27.0/27.1 & 28.5/28.9 \\
流出道 & 27.4/27.8 & 28.3/28.5 & 29.2/29.4 \\
\bottomrule
\end{tabular}
\end{table}

\textbf{关键发现}:
\begin{itemize}
    \item 标准充盈时腰部形成 → 提示瓣膜稳定性良好
    \item 主动脉根部破裂风险评估:
    \begin{itemize}
        \item 标准:最大拉伸1.2
        \item +3cc:最大拉伸1.4
        \item +5cc:最大拉伸1.5
        \item 拉伸≥1.6可能增加主动脉根部损伤风险
    \end{itemize}
    \item PVL风险评估(S3U瓣膜):
    \begin{itemize}
        \item 标准:间隙1.0mm; 1.9mm
        \item +3cc:间隙0.9mm; 1.7mm
        \item +5cc:间隙0.7mm; 1.4mm
        \item 间隙≥2mm可能增加PVL风险
        \item 间隙<2mm提示无显著PVL
    \end{itemize}
\end{itemize}

\textbf{实际手术结果}:
\begin{itemize}
    \item 使用29 S3U,标准充盈部署
    \item 后扩张×1次
    \item 无PVL
    \item 平均主动脉瓣梯度4 mmHg
    \item EF从27\%改善至34\%
\end{itemize}

\subsection{识别策略}

\subsubsection{1. 识别高风险解剖结构}

\textbf{案例}:73岁男性,1型L-R融合伴钙化嵴

\textbf{CT风险表型}:高风险
\begin{itemize}
    \item 瓣环面积:615.2 mm\textsuperscript{2}
    \item 平均直径:27.7 mm
    \item ICD:40.8 mm
    \item 钙化嵴
    \item 升主动脉最大径:44.1 mm
\end{itemize}

\textbf{治疗计划}:29 mm S3 - 双扩张策略("Double tap")

\textbf{术中策略}("MAXIMALIST"策略):
\begin{enumerate}
    \item 脑保护装置(Sentinel)
    \item 22mm Z MED II预扩张
    \item -4 cc初始部署(低于标准体积)
    \item -2 cc后扩张
\end{enumerate}

\textbf{最终结果}:
\begin{itemize}
    \item 术中TEE:圆形部署,良好血流动力学
    \item 出院TTE:无显著PVL
    \item TEE明确显示无PVL
\end{itemize}

\subsubsection{2. 经导管血流动力学评估}

\textbf{CHAI评分系统}(Jilaihawi et al, EuroIntervention 2015):

经导管血流动力学可帮助评估有预后意义的主动脉PVL:

\textbf{评估流程}:
\begin{enumerate}
    \item TEE评估术后确认PV漏
    \item 如为中度或重度 → 经导管血流动力学评估
    \item 心率校正的舒张压差 = (AoDBP-LVEDP)/HR*80
\end{enumerate}

\textbf{CHAI评分分级}:
\begin{itemize}
    \item CHAI score 0:心率校正舒张压差≥25 → 无需进一步干预
    \item CHAI score 1:心率校正舒张压差<25
    \item CHAI score 2:需要侵入性手段纠正漏
    \begin{itemize}
        \item (1) 后扩张
        \item (2) TV-in-TV(弥散性漏)
        \item (3) PV漏封堵(局灶性PV漏)
    \end{itemize}
    \item CHAI score 3:需要紧急手术(如经皮方法失败)
\end{itemize}

\textbf{生存分析}(按TOE分级):
\begin{itemize}
    \item 无/轻度:1年生存率91.0\%
    \item 轻度:1年生存率72.5\%
    \item 中到重度:1年生存率68.7\%(p<0.001)
\end{itemize}

\textbf{生存分析}(按LA CHAI评分):
\begin{itemize}
    \item Score 0:1年生存率91.0\%
    \item Score 1:1年生存率83.5\%
    \item Score 2或3:1年生存率47.3\%(p<0.001)
\end{itemize}

\textbf{血流动力学示例}:

\begin{itemize}
    \item \textbf{首次TAVR}(A组):
    \begin{itemize}
        \item HR 60:ARi=10, HRA-DD=15, 舒张压差32, DD 21
        \item HR 80:ARi=17, HRA-DD=20, 舒张压差40, DD 20
        \item HR 100:ARi=28, HRA-DD=23, 舒张压差42, DD 13
    \end{itemize}
    \item \textbf{TV-in-TV}(B组):
    \begin{itemize}
        \item HR 60:ARi=24, HRA-DD=35, 舒张压差41, DD 15
        \item HR 80:ARi=33, HRA-DD=34(红色标注:高风险)
        \item HR 100:ARi=42, HRA-DD=32, 舒张压差48, DD 8
    \end{itemize}
\end{itemize}

HR 80时B组CHAI评分显著升高,提示预后较差。

\subsection{管理策略}

\subsubsection{决策流程}

\textbf{1. 决定是否治疗}(基于TTE/TEE/血流动力学)

\textbf{2. 决定如何治疗}:
\begin{itemize}
    \item 基线成像(CT)
    \item 当前成像(CT + TEE)
    \item 了解\textbf{机制}
    \item 治疗选项:
    \begin{enumerate}
        \item 后扩张(Post-dilatation)
        \item 封堵器(Plug)
        \item 再次TAVR(Redo TAV)
    \end{enumerate}
\end{itemize}

\subsubsection{案例1:封堵器治疗}

\textbf{患者}:88岁男性,1型L-R融合

\textbf{CT风险表型}:高风险
\begin{itemize}
    \item 瓣环面积:722.4 mm\textsuperscript{2}
    \item 平均直径:30.9 mm
    \item ICD:43.5 mm
    \item SOV:面积1568 mm\textsuperscript{2},周长142 mm
    \item LVOT:面积694.3 mm\textsuperscript{2},周长95.4 mm
    \item 冠脉高度:RCA 14.8 mm, LCA 16.4 mm
    \item 重要发现:LVOT钙化
\end{itemize}

\textbf{手术过程}:
\begin{enumerate}
    \item 脑保护(Sentinel装置)
    \item 29 mm Sapien 3植入
    \item 20mm Z MED II预扩张
    \item 标准体积初始部署
    \item 标准体积后扩张
\end{enumerate}

\textbf{术后发现}:
\begin{itemize}
    \item 持续性PVL,归因于钙化
    \item TEE明确显示PVL位置和大小
    \item 测量:0.658 cm × 0.486 cm
\end{itemize}

\textbf{治疗}:
\begin{itemize}
    \item 使用\textbf{12mm AVP II封堵器}
    \item 成功封堵PVL
    \item 最终无残余PVL
\end{itemize}

\subsubsection{案例2:Redo TAVR}

\textbf{病史}:
\begin{itemize}
    \item 2022年外院植入26 S3
    \item 2024年来Cedars-Sinai就诊
    \item 诊断:生物瓣膜功能障碍(BVD)= PVL + AS
\end{itemize}

\textbf{2024年TEE评估}:
\begin{itemize}
    \item 显著瓣周漏
    \item 生物瓣膜叶片运动受限
\end{itemize}

\textbf{当前CT评估}(26 S3植入后):
\begin{itemize}
    \item 流出道:26.5 mm
    \item 中段:25.6 mm
    \item 流入道:26.5 mm
    \item 升主动脉最大径:42.7 mm
    \item LVOT面积:619 mm\textsuperscript{2}(较大)
    \item 发现:对联对齐;钙化瓣叶;最小腰部
    \item \textbf{结论}:CT显示体外尺寸适合26 S3,但LVOT较大
\end{itemize}

\textbf{回顾初次TAVR前CT}(2022年):

\textbf{解剖特点}:1型LR二叶瓣,无钙化嵴
\begin{itemize}
    \item 瓣环:576 mm\textsuperscript{2}
    \item SOV:38.8 × 39.9 × 38.4 mm
    \item STJ:38.6 × 39.1 mm
    \item ICD:32.5 mm
    \item LVOT:566 mm\textsuperscript{2}
\end{itemize}

\textbf{关键发现}:\textbf{瓣环和瓣环上尺寸支持选择29 S3}

\subsubsection{为什么初次选择了26 S3?}

这是一个\textbf{尺寸选择不当}的案例:
\begin{itemize}
    \item 瓣环576 mm\textsuperscript{2},周长85.7 mm
    \item ICD 32.5 mm支持更大尺寸
    \item LVOT 566 mm\textsuperscript{2}较大
    \item 理论上29 S3更合适
    \item 可能是外院仅基于某一个径向测量选择了26 S3
\end{itemize}

\textbf{Redo TAVR策略}:

\textbf{手术过程}:
\begin{enumerate}
    \item 脑栓子保护(Sentinel)
    \item 植入\textbf{29 mm Sapien 3 Ultra Resilia}
    \item 标准体积部署
    \item Commander球囊标准体积后扩张
\end{enumerate}

\textbf{最终结果}:
\begin{itemize}
    \item \textbf{术中}:平均梯度4 mmHg,无PVL
    \item \textbf{出院}:平均梯度11 mmHg,无PVL
    \item 成功纠正PVL和AS
\end{itemize}

\subsection{总结与临床意义}

\subsubsection{核心要点}

PVL的\textbf{预防、识别和管理}三个方面都严重依赖于成像:

\textbf{1. 预防}:
\begin{itemize}
    \item 使用3D成像(CT或3D TEE)优化瓣膜尺寸选择
    \item 考虑使用计算机模拟(如DASI)辅助决策
    \item 术中优化:适当的预扩张和后扩张
    \item 选择具有密封裙技术的新一代瓣膜
\end{itemize}

\textbf{2. 识别}:
\begin{itemize}
    \item TTE容易低估PVL严重程度
    \item 根据临床线索(如心衰症状、溶血迹象)低阈值使用TEE
    \item 血流动力学评估可以预后分层(CHAI评分)
    \item 识别高风险解剖:
    \begin{itemize}
        \item 二叶瓣
        \item 重度钙化
        \item LVOT钙化
        \item 巨大瓣环
        \item 椭圆形瓣环
    \end{itemize}
\end{itemize}

\textbf{3. 管理}:
\begin{itemize}
    \item 理解PVL的\textbf{机制}至关重要
    \item 基于基线和当前成像制定治疗策略
    \item 治疗选择:
    \begin{itemize}
        \item 后扩张:适用于尺寸不足导致的PVL
        \item 封堵器:适用于局灶性PVL(如钙化导致的间隙)
        \item Redo TAVR:适用于瓣膜尺寸选择错误或合并其他功能障碍
    \end{itemize}
    \item 目标:安全有效的管理
\end{itemize}

\subsubsection{从2010到2025的进步}

\begin{itemize}
    \item \textbf{2010年}:PVL是主要并发症,发生率13-21\%,死亡风险增加2-4倍
    \item \textbf{2025年}:
    \begin{itemize}
        \item 3D成像成为标准
        \item 新一代瓣膜(密封裙技术)显著减少PVL
        \item 计算机模拟辅助复杂病例决策
        \item 标准化的血流动力学评估方法(CHAI评分)
        \item 多样化的治疗选择(后扩张、封堵器、Redo TAVR)
    \end{itemize}
\end{itemize}

\subsubsection{临床实践建议}

\textbf{术前}:
\begin{enumerate}
    \item 常规使用CT进行3D瓣环测量
    \item 评估钙化分布,特别是LVOT钙化
    \item 识别高风险解剖结构
    \item 复杂病例考虑计算机模拟
\end{enumerate}

\textbf{术中}:
\begin{enumerate}
    \item 适当的瓣膜尺寸选择(宁大勿小)
    \item 考虑预扩张,特别是重度钙化病例
    \item 术中TEE评估PVL
    \item 血流动力学评估(CHAI评分)
    \item 必要时积极后扩张或即刻封堵
\end{enumerate}

\textbf{术后}:
\begin{enumerate}
    \item 出院前TTE评估
    \item 临床线索提示时低阈值使用TEE
    \item 显著PVL考虑血流动力学评估
    \item 及时干预中到重度PVL
    \item 长期随访监测PVL变化
\end{enumerate}

\subsection{研究局限性}

\begin{itemize}
    \item 本文为会议演讲,主要基于单中心经验
    \item CHAI评分需要更多中心验证
    \item 计算机模拟尚在FDA评估中,未广泛应用
    \item 不同瓣膜平台的PVL处理策略可能不同
\end{itemize}

\subsection{未来方向}

\begin{itemize}
    \item 继续改进瓣膜设计以进一步减少PVL
    \item 人工智能辅助瓣膜尺寸选择
    \item 术中实时3D成像引导
    \item 标准化的PVL评估和处理流程
    \item 新型封堵器技术发展
    \item Redo TAVR的长期结果研究
\end{itemize}

\newpage

\section{急性主动脉综合征:谁来处理?多学科协作}
\label{sec:05_002_acute_aortic_syndromes}

% ============================================
% 文献信息
% ============================================
\subsection{文献信息}

\begin{itemize}
    \item \textbf{标题}: Acute Aortic Syndromes: Who's Going to do Them?
    \item \textbf{副标题}: Collaboration amongst specialties
    \item \textbf{作者}: Jason T. Lee, MD
    \item \textbf{职务}: Chief, Vascular Surgery; Professor of Surgery
    \item \textbf{机构}: Stanford Health Care
    \item \textbf{会议}: TCT (Transcatheter Cardiovascular Therapeutics) 2025
    \item \textbf{PDF文件名}: acute-aortic-syndromes-whos-going-to-do-them.pdf
    \item \textbf{文献类型}: 会议演讲 / 综述
\end{itemize}

\subsection{研究背景}

\subsubsection{主动脉夹层的流行病学}

\textbf{发病率}:
\begin{itemize}
    \item 10-15例/100,000成人/年
    \item 每年新发病例:10,000-12,000例
    \item 类型分布:2/3为A型,1/3为B型
\end{itemize}

\subsubsection{Stanford分型的历史}

演讲首先回顾了\textbf{Stanford分型}的历史:

引用自Daily PO等人,Ann Thorac Surg 1970年的经典文献:

\begin{quote}
``急性主动脉夹层动脉瘤持续伴有极高的死亡率。因此,外科和非外科治疗方法都有所发展。一种方法优于另一种方法尚未得到明确证明。在大多数机构中,倾向于使用一种方法而排除另一种方法。''
\end{quote}

\textbf{Stanford经验}表明:
\begin{itemize}
    \item 主动脉升主动脉受累的夹层患者与病变未延伸至左锁骨下动脉近端的患者在临床过程和预后上存在显著且明显的差异
    \item 主动脉造影几乎可以在所有病例中确定初始内膜撕裂的位置
    \item 治疗方法根据夹层起源采用外科和内科治疗手段
\end{itemize}

\subsubsection{当代主动脉夹层管理}

\textbf{A型夹层}:
\begin{itemize}
    \item 通常紧急/急诊修复
    \item 常残留夹层,需要长期随访处理慢性问题
\end{itemize}

\textbf{急性灌注不良综合征}:
\begin{itemize}
    \item 涉及主动脉弓和降主动脉
    \item 构成大多数\textbf{急性}问题
    \item 真腔(TL)和假腔(FL)的动态变化
\end{itemize}

\subsection{Stanford主动脉中心模式}

\subsubsection{主动脉中心的使命声明}

\textbf{Stanford Health Care Aortic Center}(2016年5月建立):

\textbf{目的/使命}:

Stanford健康护理主动脉中心提供美国最现代、富有同情心和专家级的主动脉疾病护理。我们采用\textbf{综合性和多学科方法}处理这些问题,目标是:
\begin{itemize}
    \item 推进临床护理
    \item 提供尖端研究/技术
    \item 培养未来主动脉疾病管理的领导者
\end{itemize}

\subsubsection{共享实践模式的原则}

\textbf{CT和血管外科共享的教师领导}:

\textbf{联合主任}负责:
\begin{itemize}
    \item 手术(包括护理会议和"肿瘤委员会"活动)
    \item 方案制定和实施(例如破裂AAA方案、创伤性主动脉横断方案等)
    \item 人事决策
    \item 行业关系
    \item 登记和质量管理
    \item 值班表
    \item 教育框架
\end{itemize}

\textbf{资金支持}:
\begin{itemize}
    \item 通过CVH领导层(执行委员会和运营主任)提供临床运营的财务支持和监督
\end{itemize}

\textbf{主动脉特定资金流和服务线归属}:

\begin{itemize}
    \item 所有由两个专科的外科医生作为"共同外科医生"进行的病例,允许-62修饰符并\textbf{最大化和平等分配RVU/报销}
    \item 非手术治疗的复杂主动脉疾病患者(例如B型主动脉夹层伴远端灌注不良)指定为主动脉中心患者,用于出院医师专科归属和病例追踪
    \item 单一专科治疗的复杂主动脉疾病患者(例如A型夹层或EVAR/开放AAA修复)也计入主动脉中心的出院医师专科归属 - 专业收入归各自科室/部门
\end{itemize}

\textbf{主动脉中心优先事项}:

\begin{itemize}
    \item \textbf{共享决策制定和协作},除了收入优化(-62修饰符)外,\textbf{在多个专科之间}
\end{itemize}

\subsection{急性主动脉问题协作的要求}

\subsubsection{关键要素}

\textbf{1. 患者转诊}

\textbf{2. 多学科专业知识}:
\begin{itemize}
    \item 心脏病学
    \item 心胸外科
    \item 血管外科
    \item 血管内科
    \item 心血管麻醉
    \item 心血管放射学
    \item 护理
    \item 康复
    \item 病例管理
\end{itemize}

\textbf{3. 临床协作意愿}

\textbf{4. 学术协作愿望}

\textbf{5. 医院和服务线领导的支持}

\textbf{6. 最佳培训范例}

\subsection{内血管技术的发展}

\subsubsection{B型夹层的内血管支架移植物}

\textbf{技术原理}:
\begin{itemize}
    \item 内膜撕裂口(Entry Tear)
    \item 真腔(True Lumen)
    \item 假腔(False Lumen)
    \item 支架移植物(Stent-Graft)覆盖近端撕裂口
    \item 血栓形成假腔
    \item 远端再入口撕裂(Re-entry Tear)和远端撕裂(Distal Tear)
\end{itemize}

\textbf{目标}:
\begin{itemize}
    \item 封闭近端内膜撕裂
    \item 促进假腔血栓形成
    \item 恢复真腔血流
    \item 改善器官灌注
\end{itemize}

\subsection{临床案例展示}

\subsubsection{案例1:A型夹层伴灌注不良}

\textbf{患者信息}:
\begin{itemize}
    \item 50岁女性
    \item 主诉:胸痛、腹痛、左下肢冰冷
\end{itemize}

\textbf{影像学发现}:
\begin{itemize}
    \item 急性A型主动脉夹层
    \item 累及腹主动脉
    \item 左下肢灌注不良
\end{itemize}

\textbf{治疗策略(多学科协作)}:

\textbf{第一步}:心胸外科开放A型修复
\begin{itemize}
    \item 升主动脉和主动脉弓置换
    \item 修复近端夹层
\end{itemize}

\textbf{术后问题}:
\begin{itemize}
    \item CTA显示开放修复后灌注不良加重
    \item 腹腔和下肢缺血持续
\end{itemize}

\textbf{第二步}:血管外科内血管干预(IVUS引导)

\textbf{IVUS的重要性}:
\begin{itemize}
    \item 明确真假腔
    \item 识别内膜瓣
    \item 测量血管直径
    \item 引导导丝和器械
    \item 24.0 mm直径测量
\end{itemize}

\textbf{内血管治疗}:
\begin{enumerate}
    \item \textbf{近端TEVAR}(胸主动脉腔内修复)
    \item \textbf{远端夹层支架}
    \item \textbf{SMA裸支架}(肠系膜上动脉)
\end{enumerate}

\textbf{最终结果}:
\begin{itemize}
    \item 成功恢复腹腔脏器灌注
    \item 下肢血流恢复
    \item 患者康复良好
\end{itemize}

\subsubsection{案例2:A型夹层伴灌注不良(创新技术)}

\textbf{患者信息}:
\begin{itemize}
    \item 54岁女性
    \item 主诉:胸痛、腹痛、右下肢冰冷
\end{itemize}

\textbf{影像学发现}:
\begin{itemize}
    \item 急性A型主动脉夹层(Zone 2)
    \item 累及主动脉弓
    \item 内脏器官灌注不良
\end{itemize}

\textbf{文献参考}:

引用Ohira等人发表的研究:"Zone 2 arch repair for acute type A dissection: Evolution from arch-first to proximal-first repair"

\textbf{治疗策略}:

\textbf{创新技术}:"漂浮内脏征"(Floating Viscera Sign)
\begin{itemize}
    \item 造影显示肠道呈"漂浮"状
    \item 提示严重的内脏灌注不良
\end{itemize}

\textbf{使用Gore Thoracic Branch Endoprosthesis}:
\begin{itemize}
    \item 分支型主动脉腔内支架系统
    \item 保留弓上血管
    \item 同时修复夹层
\end{itemize}

\textbf{术中技术}:
\begin{itemize}
    \item 主动脉弓分支重建
    \item 近端夹层封闭
    \item 内脏动脉灌注恢复
\end{itemize}

\textbf{结果}:
\begin{itemize}
    \item 内脏灌注不良改善
    \item CT显示良好的器官灌注
    \item 患者成功康复
\end{itemize}

\subsection{学术协作成果}

\subsubsection{1. 球囊假腔闭塞技术}

\textbf{发表}:JTCVS Tech 2023

\textbf{研究内容}:
\begin{quote}
``慢性主动脉夹层降胸主动脉的可控球囊假腔闭塞的内血管管理''
\end{quote}

\textbf{作者}:A. Claire Watkins, MD, MS等

\textbf{目的}:
\begin{itemize}
    \item 逆行假腔灌注限制了主动脉支架移植治疗慢性主动脉夹层的效用
    \item 研究球囊隔膜破裂是否能改善慢性主动脉夹层的内血管管理结果
\end{itemize}

\textbf{方法}:
\begin{itemize}
    \item 纳入患者接受假腔闭塞
    \item 在胸主动脉内血管修复期间使用顺应性球囊在支架移植物内创建单腔主动脉着陆区
    \item 远端胸主动脉支架移植物大小适应总主动脉腔直径
    \item 在距远端织物边缘近端5 cm的区域内使用顺应性球囊进行隔膜破裂
\end{itemize}

\textbf{结果}(40例患者):
\begin{itemize}
    \item 74\%完全假腔血栓形成
    \item 1.6年随访时生存率97.5\%
    \item 技术成功率高
    \item 并发症少
\end{itemize}

\subsubsection{2. 经皮隔膜切开术技术}

\textbf{发表}:多篇文献

\textbf{"芝士线"技术}:
\begin{quote}
``使用'芝士线'穿刺慢性主旁肾主动脉夹层瓣以促进EVAR期间近端颈部固定''
\end{quote}

\textbf{作者}:Mark Colin Gissler等

\textbf{技术原理}:
\begin{itemize}
    \item 在EVAR前经皮切开夹层内膜瓣
    \item 创建统一的腔隙用于近端固定
    \item "芝士线"穿刺技术
\end{itemize}

\textbf{另一项研究}:
\begin{quote}
``经皮隔膜切开术在慢性夹层伴腹主动脉瘤中创建EVAR的单腔颈''
\end{quote}

\textbf{应用}:
\begin{itemize}
    \item 慢性主旁肾夹层合并AAA
    \item 为EVAR创造近端着陆区
    \item 成功完成III型内漏修复,需要再次手术和肾动脉闭塞
\end{itemize}

\subsubsection{3. A型夹层混合手术室研究}

\textbf{发表}:多项研究

\textbf{创新实践}:
\begin{itemize}
    \item 建立A型夹层转诊床位(始终开放)
    \item 混合手术室能力,可在开放修复后进行内血管手术
    \item CT外科和血管外科进一步合作
\end{itemize}

\textbf{成果}:
\begin{itemize}
    \item 2017-2021年完成301例A型修复
    \item 增加服务线和多个科室的收入
    \item 改善所有科室的教育工作
\end{itemize}

\textbf{研究标题}:
\begin{quote}
``混合手术室中有无灌注不良的A型主动脉夹层修复的中期结果''
\end{quote}

\textbf{作者}:Alex R. Dalal, MD等

\textbf{研究人群}:
\begin{itemize}
    \item 301例急性A型夹层
    \item 传统手术室组:144例
    \item 混合手术室组:157例
\end{itemize}

\textbf{倾向评分匹配}:
\begin{itemize}
    \item 每组各125例患者
\end{itemize}

\textbf{混合手术室结果}:
\begin{itemize}
    \item 提供术中即时评估远端灌注不良的能力
    \item 必要时进行内血管干预
    \item 与传统手术室方法相比,显著提高30天和2年生存率
\end{itemize}

\textbf{关键发现}:
\begin{itemize}
    \item 混合手术室组41\% (64/157)接受术中血管造影
    \item 其中58\% (37/64)至少接受一次内血管干预
    \item 30天生存率:混合组96.7\% vs 传统组92.8\% (p=0.0074)
    \item 中期生存率显著改善
\end{itemize}

\subsubsection{4. TAVR相关协作}

\textbf{发表}:J Invasive Cardiol. 2023

\textbf{标题}:
\begin{quote}
``经皮救援技术用于捕获ViV TAVR期间的栓塞瓣膜''
\end{quote}

\textbf{作者}:Stathogiannis KE, MacArthur JW, Lee JT, Sharma RP

\textbf{病例特点}:
\begin{itemize}
    \item 75岁男性,NYHA III级症状
    \item 15年复杂治疗史
    \item 二叶主动脉瓣(BAV)合并室间隔缺损(VSD)
    \item 2005年行AV置换和VSD修补
    \item 2015年行再次AV置换和根部重建
\end{itemize}

\textbf{当前问题}:
\begin{itemize}
    \item 超声心动图显示严重的生物瓣膜AS和中度AR
    \item 推荐ViV TAVR,使用Sentinel脑保护装置
    \item 术前CT显示扩张的主动脉根和降主动脉,伴假性主动脉缩窄证据
\end{itemize}

\textbf{强调}:
\begin{quote}
\textcolor{red}{``本病例强调了多学科团队方法的必要性,以及对各种装置和技术的深入了解。''}
\end{quote}

\subsubsection{5. TAMBE后批准研究}

\textbf{最新进展}:2025年1月16日

\textbf{研究}:TAMBE Post Approval Study (AAA)

\textbf{标题}:
\begin{quote}
``四分支现成解决方案治疗复杂腹主动脉瘤和IV型胸腹主动脉瘤的关键试验早期结果''
\end{quote}

\textbf{主要研究者}:Mark A. Farber, MD等

\textbf{里程碑}:
\begin{center}
\large\textbf{入组已开始!!!}
\end{center}

\textbf{祝贺}:
\begin{quote}
``祝贺DR. JASON LEE及其在STANFORD UNIVERSITY的研究团队成功入组TAMBE后批准研究的前两名患者,标志着我们第一个研究里程碑的完成!!''
\end{quote}

\subsubsection{6. 国际合作}

\textbf{Osaka Police Hospital}

\textbf{第3例TAMBE}在日本完成 - 2025年10月

\begin{itemize}
    \item Stanford团队参与日本的TAMBE手术
    \item 国际学术交流和技术转移
    \item 促进全球主动脉疾病治疗进步
\end{itemize}

\textbf{TCT 2025演讲者}:

\textbf{Osamu Iida, MD, PhD}
\begin{itemize}
    \item Cardiovascular Division, Osaka Keisatsu Hospital
    \item 演讲主题:Endovascular Spotlight 3. CLTI State of the Art with Live from Germany
    \item 日期:Sunday, October 26, 2025
\end{itemize}

\subsection{Stanford模式总结}

\subsubsection{成功要素}

\textbf{1. 医院和服务线领导的支持}:
\begin{itemize}
    \item 心血管服务线与血管外科、心胸外科、心脏病学、心血管麻醉、心血管影像学合作
\end{itemize}

\textbf{2. 与医院其他学科合作的意愿}:
\begin{itemize}
    \item 归根结底是\textbf{人和钱}
    \item 多个科室和部门
    \item 培训项目应允许有兴趣的介入医师发展技能组合
\end{itemize}

\textbf{3. 尚未完美}:
\begin{itemize}
    \item 培养年轻教师
    \item 激励部门协同工作
    \item 谁支付人员费用/组织随访
\end{itemize}

\textbf{4. 最终}:
\begin{center}
\large\textbf{想要合作的团队最终会实现}
\end{center}

\subsection{临床意义与启示}

\subsubsection{对TAVR并发症管理的启示}

虽然这篇演讲主要关注主动脉夹层,但其跨学科协作模式对TAVR并发症管理具有重要借鉴意义:

\textbf{1. 多学科团队的必要性}:
\begin{itemize}
    \item TAVR并发症(如主动脉破裂、夹层)需要多学科协作
    \item 心脏病学、心胸外科、血管外科的紧密合作
    \item 快速决策和行动能力
\end{itemize}

\textbf{2. 混合手术室的价值}:
\begin{itemize}
    \item 可在TAVR术中即时处理并发症
    \item 如主动脉根部破裂可立即转开放手术
    \item 减少转运时间,提高救治成功率
\end{itemize}

\textbf{3. 内血管技术的应用}:
\begin{itemize}
    \item TAVR后主动脉并发症可能需要内血管修复
    \item IVUS引导在复杂解剖中的重要性
    \item 多种技术的掌握(支架、球囊、封堵器等)
\end{itemize}

\textbf{4. 培训和教育}:
\begin{itemize}
    \item 培养具备多种技能的介入医师
    \item 跨学科培训的重要性
    \item 模拟培训和病例讨论
\end{itemize}

\subsubsection{主动脉中心模式的可复制性}

\textbf{核心要素}:
\begin{enumerate}
    \item \textbf{共同领导}:打破科室壁垒
    \item \textbf{收入共享}:-62修饰符,公平分配
    \item \textbf{病例共享}:共同管理复杂病例
    \item \textbf{资源共享}:设备、人员、经验
    \item \textbf{学术共享}:共同发表,共同进步
\end{enumerate}

\textbf{对其他中心的启示}:
\begin{itemize}
    \item 建立结构性心脏病中心可借鉴此模式
    \item TAVR项目应包含多学科协作机制
    \item 处理并发症需要预先建立的协作关系
    \item 不能等到并发症发生时才开始沟通
\end{itemize}

\subsection{研究局限性}

\begin{itemize}
    \item 单中心经验,可能不适用于所有医疗环境
    \item 需要大量资源投入(混合手术室、多学科团队)
    \item 收入分配模式在不同医疗体系中可能难以实施
    \item 文化和组织结构的差异可能影响可复制性
\end{itemize}

\subsection{未来方向}

\subsubsection{技术发展}

\begin{itemize}
    \item 新一代分支型主动脉支架
    \item 更好的成像技术(实时3D、融合成像)
    \item 微创技术的进一步发展
    \item 人工智能辅助决策
\end{itemize}

\subsubsection{协作模式}

\begin{itemize}
    \item 更多中心采用主动脉中心模式
    \item 远程协作和会诊
    \item 国际合作网络
    \item 标准化的培训项目
\end{itemize}

\subsubsection{研究方向}

\begin{itemize}
    \item 长期结果的多中心研究
    \item 成本效益分析
    \item 最佳团队配置研究
    \item 培训效果评估
\end{itemize}

\subsection{关键要点}

\begin{enumerate}
    \item \textbf{急性主动脉综合征}需要多学科协作处理
    \item \textbf{Stanford主动脉中心模式}提供了成功的协作框架
    \item \textbf{混合手术室}对处理复杂病例至关重要
    \item \textbf{内血管技术发展}扩展了治疗选择
    \item \textbf{学术协作}推动技术创新和临床进步
    \item \textbf{跨学科培训}培养下一代专家
    \item \textbf{成功的关键}:人员、资金、意愿、领导支持
    \item 对TAVR并发症管理有重要借鉴意义
\end{enumerate}

\newpage

\section{TAVR瓣膜脱位的预防与管理技巧}

\subsection{文献信息}

\begin{itemize}
    \item \textbf{标题}: My Tips and Tricks for Prevention and Management of Aortic Valve Embolizations
    \item \textbf{作者}: Chad Kliger MD MS
    \item \textbf{职务}: Director of Structural Heart, Associate Professor of Cardiology and Cardiothoracic Surgery
    \item \textbf{机构}: Northwell Health – Lenox Hill Hospital, New York, NY
    \item \textbf{会议}: CRF TCT (Transcatheter Cardiovascular Therapeutics)
    \item \textbf{主题}: TAVR瓣膜脱位的预防、识别与管理
    \item \textbf{利益冲突}: 无相关财务关系
\end{itemize}

\subsection{TAVR脱位流行病学}

\subsubsection{1. 发生率数据}

\textbf{历史数据}:
\begin{itemize}
    \item TAVR脱位总体发生率:0.1-3.7\%
    \item TRAVEL注册研究(2010-17年,26个中心):1\%
\end{itemize}

\textbf{新一代瓣膜的脱位率}:
\begin{itemize}
    \item \textbf{Evolut Fx}:0.4\%
    \item \textbf{Edwards Sapien S3 Ultra Resilia}:0.1\%
    \item \textbf{Abbott Navitor}:0.9\%
\end{itemize}

\textbf{脱位位置分布}:
\begin{table}[h]
\centering
\caption{TAVR脱位位置分类}
\begin{tabular}{p{6cm}p{2cm}p{5cm}}
\toprule
\textbf{脱位位置} & \textbf{比例} & \textbf{处理方式} \\
\midrule
主动脉脱位(Aortic Embolization) & 38\% & 内血管取回、开放手术 \\
流出道和瓣环脱位(Outflow Tract and Annular Embolization) & 31\% & 瓣中瓣部署 \\
心室脱位(Ventricular Embolization) & 23\% & 开放手术、内血管取回 \\
\bottomrule
\end{tabular}
\end{table}

\subsubsection{2. 脱位原因}

\textbf{患者相关因素}:
\begin{itemize}
    \item \textbf{缺乏钙化}:瓣膜缺乏固定锚定点
    \item \textbf{不对称钙化}:钙化分布不均匀影响瓣膜固定
    \item \textbf{水平主动脉}:解剖角度不利于瓣膜稳定
    \item \textbf{二叶主动脉瓣(BAV)}:解剖异常增加脱位风险
    \item \textbf{大主动脉瓣环}:瓣膜尺寸匹配困难
    \item \textbf{升主动脉扩张}:影响瓣膜在升主动脉的稳定性
\end{itemize}

\textbf{技术因素}:
\begin{itemize}
    \item \textbf{不正确的瓣膜尺寸}:
    \begin{itemize}
        \item 瓣膜过小:无法获得足够径向力
        \item 尺寸选择边界情况(borderline sizing)
    \end{itemize}
    \item \textbf{不适当的定位}:
    \begin{itemize}
        \item 释放位置过高或过低
        \item 释放角度不当
    \end{itemize}
    \item \textbf{起搏故障}:
    \begin{itemize}
        \item 快速起搏失败
        \item 起搏捕获丢失
        \item 起搏不足导致心输出量过高
    \end{itemize}
    \item \textbf{术后扩张}:
    \begin{itemize}
        \item 球囊扩张力度过大
        \item 扩张导致瓣膜移位
    \end{itemize}
\end{itemize}

\subsection{临床案例分析}

\subsubsection{案例1:Edwards Sapien瓣膜脱位(成功救治)}

\textbf{患者特点}:
\begin{itemize}
    \item 瓣环/STJ边界:460 mm²
    \item 使用23mm S3 Ultra Resilia + 2cc
    \item 边界尺寸选择(Borderline sizing)
\end{itemize}

\textbf{并发症发生}:
\begin{itemize}
    \item 释放过程中起搏故障
    \item 瓣膜可能尺寸过小
    \item 瓣膜通过输送系统被拉入降主动脉
\end{itemize}

\textbf{处理策略}:
\begin{enumerate}
    \item \textbf{使用25mm ZMed球囊术后扩张}
    \begin{itemize}
        \item 扩张脱位的瓣膜
        \item 尝试改善瓣膜形态
    \end{itemize}

    \item \textbf{植入26mm S3瓣膜}
    \begin{itemize}
        \item 在原瓣环位置植入新瓣膜
        \item 处理主动脉瓣狭窄
    \end{itemize}

    \item \textbf{使用Palmaz XL支架固定脱位瓣膜}
    \begin{itemize}
        \item 10×40mm Palmaz XL支架
        \item 预先压接在25mm ZMed球囊上
        \item 在降主动脉固定脱位瓣膜
        \item 防止瓣膜进一步移位
    \end{itemize}
\end{enumerate}

\textbf{结果}:
\begin{itemize}
    \item 成功避免外科手术
    \item 患者血流动力学稳定
    \item 新植入瓣膜功能良好
\end{itemize}

\subsubsection{案例2:Edwards Sapien瓣膜脱位(转外科)}

\textbf{患者特点}:
\begin{itemize}
    \item 使用26mm S3 Ultra(S3U)
\end{itemize}

\textbf{并发症发生}:
\begin{itemize}
    \item 释放过程中起搏故障
    \item 瓣膜释放位置不理想
    \item 瓣膜部分脱位到主动脉
\end{itemize}

\textbf{处理尝试}:
\begin{itemize}
    \item 尝试将瓣膜拉回
    \item 无法将瓣膜拉回超过某一点
    \item 瓣膜卡在不利位置
\end{itemize}

\textbf{最终处理}:
\begin{itemize}
    \item \textbf{转外科主动脉瓣置换术(SAVR)}
    \begin{itemize}
        \item 内血管取回失败
        \item 瓣膜位置无法纠正
        \item 需要紧急外科干预
    \end{itemize}
\end{itemize}

\textbf{教训}:
\begin{itemize}
    \item 起搏功能的关键重要性
    \item 术前起搏导线测试必不可少
    \item 边界情况应保守选择瓣膜尺寸
    \item 准备好外科后备方案
\end{itemize}

\subsubsection{案例3:Evolut Fx瓣膜脱位(瓣中瓣救治)}

\textbf{患者特点}:
\begin{itemize}
    \item 既往19mm Magna Ease外科生物瓣
    \item 23mm Evolut Fx瓣中瓣(ViV)
    \item 升主动脉扩张
\end{itemize}

\textbf{技术细节}:
\begin{itemize}
    \item 导丝上起搏(Pacing over wire)
    \item 防止起搏导线丢失
\end{itemize}

\textbf{并发症发生}:
\begin{itemize}
    \item 瓣膜释放后脱位到升主动脉
    \item 升主动脉扩张使瓣膜无法固定
\end{itemize}

\textbf{处理策略}:
\begin{enumerate}
    \item \textbf{双套索技术捕获脱位瓣膜}
    \begin{itemize}
        \item 使用JR4引导导管
        \item 使用Ensnare套索
        \item 同时套索两个Evolut标签
        \item 尝试将瓣膜固定在升主动脉
    \end{itemize}

    \item \textbf{升主动脉扩张导致无法固定}
    \begin{itemize}
        \item 升主动脉直径过大
        \item 瓣膜无法获得足够径向力
        \item 无法在innominate动脉之前固定
    \end{itemize}

    \item \textbf{植入第二个Evolut瓣膜固定}
    \begin{itemize}
        \item 在原瓣环位置
        \item 第二个Evolut瓣膜作为"锚定"
        \item 固定第一个脱位的瓣膜
    \end{itemize}
\end{enumerate}

\textbf{结果}:
\begin{itemize}
    \item 成功使用瓣中瓣技术
    \item 避免外科手术
    \item 两个瓣膜联合功能
\end{itemize}

\subsubsection{案例4:Evolut Pro+瓣膜脱位(复杂救治)}

\textbf{患者特点}:
\begin{itemize}
    \item 使用29mm Evolut Pro+
    \item 主动脉解剖复杂
\end{itemize}

\textbf{并发症发生}:
\begin{itemize}
    \item 瓣膜脱位到升主动脉
    \item 使用套索捕获
    \item 无法将瓣膜移动到窦管连接处(STJ)之上
\end{itemize}

\textbf{关键问题}:
\begin{itemize}
    \item \textbf{瓣膜位置在STJ处}
    \begin{itemize}
        \item 阻塞冠状动脉开口
        \item 妨碍舒张期冠脉充盈
        \item 主动脉造影显示瓣叶上方无造影剂流入
    \end{itemize}
\end{itemize}

\textbf{处理策略}:
\begin{enumerate}
    \item \textbf{主动脉造影评估}
    \begin{itemize}
        \item 猪尾导管置于瓣叶上方
        \item 评估冠脉充盈情况
        \item 确认瓣膜位置对冠脉的影响
    \end{itemize}

    \item \textbf{植入23mm Sapien 3瓣膜}
    \begin{itemize}
        \item 在原瓣环位置
        \item 处理主动脉瓣狭窄
        \item 恢复正常血流动力学
    \end{itemize}

    \item \textbf{使用Palmaz XL支架固定脱位瓣膜}
    \begin{itemize}
        \item 10×40mm Palmaz XL支架
        \item 固定Evolut Pro+在升主动脉
        \item 防止瓣膜进一步移位或造成并发症
    \end{itemize}
\end{enumerate}

\textbf{关键学习点}:
\begin{itemize}
    \item 评估脱位瓣膜对冠脉的影响至关重要
    \item 主动脉造影是评估冠脉充盈的关键工具
    \item 灵活运用不同瓣膜类型组合
    \item 支架技术在固定脱位瓣膜中的价值
\end{itemize}

\subsubsection{案例5:Evolut Pro避免强行牵拉}

\textbf{关键教训}:
\begin{center}
\large\textbf{不要强行牵拉!尽量避免这种情况!}
\end{center}

\textbf{危险情况}:
\begin{itemize}
    \item 瓣膜脱位后卡在不利位置
    \item 尝试通过牵拉输送系统取回瓣膜
\end{itemize}

\textbf{强行牵拉的风险}:
\begin{enumerate}
    \item \textbf{瓣膜结构损坏}
    \begin{itemize}
        \item 瓣膜支架变形
        \item 瓣叶撕裂
        \item 输送系统损坏
    \end{itemize}

    \item \textbf{血管并发症}
    \begin{itemize}
        \item 主动脉夹层
        \item 主动脉破裂
        \item 血管撕裂
    \end{itemize}

    \item \textbf{瓣环和LVOT损伤}
    \begin{itemize}
        \item 瓣环撕裂
        \item LVOT破裂
        \item 心包填塞
    \end{itemize}

    \item \textbf{瓣膜进一步脱位}
    \begin{itemize}
        \item 进入心室
        \item 进入冠状动脉
        \item 其他更不利位置
    \end{itemize}
\end{enumerate}

\textbf{正确处理原则}:
\begin{itemize}
    \item \textbf{停止强行操作}
    \begin{itemize}
        \item 评估瓣膜位置和状态
        \item 不要持续用力牵拉
    \end{itemize}

    \item \textbf{使用套索或其他捕获装置}
    \begin{itemize}
        \item Ensnare套索
        \item Gooseneck套索
        \item 专用捕获工具
    \end{itemize}

    \item \textbf{考虑替代救治方案}
    \begin{itemize}
        \item 瓣中瓣技术
        \item 支架固定
        \item 转外科手术
    \end{itemize}
\end{itemize}

\textbf{影像学证据}:
\begin{itemize}
    \item CT显示瓣膜在心腔内的复杂位置
    \item 强行牵拉可能导致灾难性后果
    \item 需要慎重评估和谨慎操作
\end{itemize}

\subsection{TAVR脱位管理策略}

\subsubsection{自膨胀瓣膜(Self Expanding THVs)脱位管理}

\textbf{1. 主动脉脱位(Aortic Embolization)}

\textit{捕获技术}:
\begin{itemize}
    \item \textbf{套索装置}:
    \begin{itemize}
        \item Ensnare套索
        \item Gooseneck套索
        \item 捕获瓣膜标签或支架
    \end{itemize}

    \item \textbf{Lasso技术}:
    \begin{itemize}
        \item 使用导丝形成套索
        \item 捕获瓣膜结构
    \end{itemize}
\end{itemize}

\textit{固定策略}:
\begin{itemize}
    \item \textbf{在升主动脉固定}:
    \begin{itemize}
        \item 理想位置:innominate动脉之前
        \item 需要足够的径向力
        \item 避免影响分支血管
    \end{itemize}

    \item \textbf{如无法在升主动脉固定}:
    \begin{itemize}
        \item 使用第二个自膨胀瓣膜固定
        \item 或使用支架(如Palmaz XL)
        \item 在降主动脉固定
    \end{itemize}
\end{itemize}

\textbf{2. LVOT脱位(LVOT Embolization)}

\textit{处理方案}:
\begin{itemize}
    \item \textbf{套索拉回主动脉}:
    \begin{itemize}
        \item 使用套索捕获瓣膜
        \item 尝试拉回主动脉位置
        \item 重新定位
    \end{itemize}

    \item \textbf{瓣中瓣(Valve in Valve)}:
    \begin{itemize}
        \item 在脱位瓣膜内植入新瓣膜
        \item 最常用的救治方法
        \item 成功率较高
    \end{itemize}
\end{itemize}

\subsubsection{球囊扩张瓣膜(Balloon Expandable THVs)脱位管理}

\textbf{1. 主动脉脱位(Aortic Embolization)}

\textit{牵拉技术}:
\begin{itemize}
    \item \textbf{通过输送系统球囊拉回}:
    \begin{itemize}
        \item 如果瓣膜仍在输送系统上
        \item 谨慎回撤到降主动脉
        \item 避免强行牵拉
    \end{itemize}

    \item \textbf{使用充气球囊拉回}:
    \begin{itemize}
        \item Coda球囊
        \item ZMed球囊
        \item 其他非顺应性球囊
        \item 在瓣膜内充气后回撤
    \end{itemize}
\end{itemize}

\textit{固定策略}:
\begin{itemize}
    \item \textbf{支架固定技术}(存疑):
    \begin{itemize}
        \item 理论上可使用支架
        \item 固定在瓣叶后方
        \item 临床经验有限
        \item 需要进一步研究
    \end{itemize}
\end{itemize}

\textbf{2. LVOT脱位(LVOT Embolization)}

\textit{处理方案}:
\begin{itemize}
    \item \textbf{瓣中瓣(Valve in Valve)}:
    \begin{itemize}
        \item 首选救治方法
        \item 在脱位瓣膜内植入新瓣膜
        \item 球囊扩张瓣膜适合作为救援瓣膜
        \item 可精确定位和快速部署
    \end{itemize}
\end{itemize}

\textbf{3. 心室脱位(Ventricular Embolization)}

\textit{处理方案}:
\begin{itemize}
    \item \textbf{外科手术}:
    \begin{itemize}
        \item 唯一可靠的选择
        \item 紧急转外科
        \item 取出脱位瓣膜
        \item 外科主动脉瓣置换
    \end{itemize}
\end{itemize}

\subsection{预防策略}

\subsubsection{术前评估与规划}

\textbf{1. 详细的影像学评估}

\begin{itemize}
    \item \textbf{CT测量}:
    \begin{itemize}
        \item 精确的瓣环尺寸测量
        \item 评估钙化分布和程度
        \item 测量升主动脉直径
        \item 评估主动脉角度
    \end{itemize}

    \item \textbf{识别高危特征}:
    \begin{itemize}
        \item 缺乏或轻度钙化
        \item 不对称钙化
        \item 二叶主动脉瓣
        \item 升主动脉扩张
        \item 水平主动脉
        \item 大瓣环
    \end{itemize}
\end{itemize}

\textbf{2. 瓣膜选择策略}

\begin{itemize}
    \item \textbf{边界情况处理}:
    \begin{itemize}
        \item 倾向于选择大一号瓣膜
        \item 考虑使用自膨胀瓣膜(更好的固定)
        \item 评估过度尺寸风险
    \end{itemize}

    \item \textbf{特殊解剖考虑}:
    \begin{itemize}
        \item 升主动脉扩张:考虑球囊扩张瓣膜
        \item 钙化不足:考虑更大过度尺寸
        \item 二叶瓣:详细规划釈放策略
    \end{itemize}
\end{itemize}

\subsubsection{术中技术要点}

\textbf{1. 起搏管理}

\begin{itemize}
    \item \textbf{术前起搏测试}:
    \begin{itemize}
        \item 确认起搏捕获
        \item 测试起搏阈值
        \item 确保起搏导线稳定
    \end{itemize}

    \item \textbf{术中起搏监测}:
    \begin{itemize}
        \item 持续监测起搏功能
        \item 备用起搏导线准备
        \item 快速起搏参数优化
    \end{itemize}

    \item \textbf{导丝上起搏}:
    \begin{itemize}
        \item 复杂病例考虑
        \item 防止起搏导线丢失
        \item 确保整个过程有起搏保护
    \end{itemize}
\end{itemize}

\textbf{2. 释放技术}

\begin{itemize}
    \item \textbf{精确定位}:
    \begin{itemize}
        \item 多角度确认瓣膜位置
        \item 与瓣环的关系
        \item 冠状动脉开口的位置
    \end{itemize}

    \item \textbf{慢速释放}:
    \begin{itemize}
        \item 允许实时评估
        \item 必要时中止释放
        \item 对于自膨胀瓣膜尤其重要
    \end{itemize}

    \item \textbf{血流动力学稳定}:
    \begin{itemize}
        \item 确保有效快速起搏
        \item 维持适当血压
        \item 减少心输出量影响
    \end{itemize}
\end{itemize}

\textbf{3. 谨慎术后处理}

\begin{itemize}
    \item \textbf{评估术后扩张必要性}:
    \begin{itemize}
        \item 权衡获益和风险
        \item 球囊扩张可能导致瓣膜移位
        \item 仅在必要时进行
    \end{itemize}

    \item \textbf{术后即刻评估}:
    \begin{itemize}
        \item 多角度透视确认瓣膜位置
        \item 超声评估瓣膜功能
        \item 早期识别瓣膜不稳定
    \end{itemize}
\end{itemize}

\subsection{救援装备准备}

\subsubsection{必备器械清单}

\textbf{捕获装置}:
\begin{itemize}
    \item Ensnare套索(多种尺寸)
    \item Gooseneck套索
    \item 其他专用捕获装置
\end{itemize}

\textbf{球囊}:
\begin{itemize}
    \item ZMed球囊(多种尺寸)
    \item Coda球囊
    \item 其他非顺应性球囊
\end{itemize}

\textbf{支架}:
\begin{itemize}
    \item Palmaz XL支架(10×40mm等)
    \item 其他支架选择
\end{itemize}

\textbf{备用瓣膜}:
\begin{itemize}
    \item 第二个TAVR瓣膜
    \item 不同类型瓣膜(SEV和BEV)
    \item 不同尺寸选择
\end{itemize}

\subsection{总结与临床意义}

\subsubsection{关键要点}

\begin{enumerate}
    \item \textbf{预防永远优于治疗}
    \begin{itemize}
        \item 详细的术前评估和规划
        \item 识别高危患者特征
        \item 适当的瓣膜选择
        \item 精确的技术执行
    \end{itemize}

    \item \textbf{起搏是关键}
    \begin{itemize}
        \item 起搏故障是脱位的主要原因
        \item 术前测试和术中监测至关重要
        \item 准备备用起搏方案
    \end{itemize}

    \item \textbf{不要强行牵拉}
    \begin{itemize}
        \item 可能导致灾难性后果
        \item 使用专用捕获装置
        \item 考虑替代救治方案
    \end{itemize}

    \item \textbf{灵活的救治策略}
    \begin{itemize}
        \item 根据脱位位置选择方法
        \item 根据瓣膜类型调整策略
        \item 准备多种救援选择
    \end{itemize}

    \item \textbf{团队准备和沟通}
    \begin{itemize}
        \item 术前团队讨论
        \item 准备好所有救援器械
        \item 外科后备支持
    \end{itemize}
\end{enumerate}

\subsubsection{不同瓣膜类型的管理差异}

\begin{table}[h]
\centering
\caption{自膨胀vs球囊扩张瓣膜脱位管理}
\begin{tabular}{p{3cm}p{5.5cm}p{5.5cm}}
\toprule
\textbf{脱位位置} & \textbf{自膨胀瓣膜(SEV)} & \textbf{球囊扩张瓣膜(BEV)} \\
\midrule
主动脉脱位 & • 套索/Lasso捕获 \newline • 升主动脉固定 \newline • 第二个SEV/支架固定 & • 通过输送系统拉回 \newline • 充气球囊拉回 \newline • 支架固定(存疑) \\
\midrule
LVOT脱位 & • 套索拉回主动脉 \newline • 瓣中瓣 & • 瓣中瓣 \\
\midrule
心室脱位 & • 外科手术 & • 外科手术 \\
\bottomrule
\end{tabular}
\end{table}

\subsubsection{临床实践建议}

\textbf{术前}:
\begin{itemize}
    \item 详细的CT评估,识别高危特征
    \item 多学科团队讨论瓣膜选择
    \item 准备好所有可能需要的救援器械
    \item 外科团队待命
\end{itemize}

\textbf{术中}:
\begin{itemize}
    \item 确保起搏功能可靠
    \item 精确的瓣膜定位和慢速释放
    \item 早期识别脱位迹象
    \item 冷静、系统的救治方法
    \item 避免强行牵拉
\end{itemize}

\textbf{术后}:
\begin{itemize}
    \item 详细记录脱位原因和处理过程
    \item 团队debriefing学习经验
    \item 持续随访评估结果
    \item 优化未来病例的策略
\end{itemize}

\subsubsection{新一代瓣膜的进步}

\textbf{脱位率显著下降}:
\begin{itemize}
    \item Edwards Sapien S3 Ultra Resilia: 0.1\%
    \item Evolut Fx: 0.4\%
    \item Abbott Navitor: 0.9\%
\end{itemize}

\textbf{改进因素}:
\begin{itemize}
    \item 更好的瓣膜设计
    \item 改进的输送系统
    \item 优化的尺寸选择指导
    \item 更准确的成像技术
    \item 操作者经验积累
\end{itemize}

\subsubsection{未来展望}

\begin{itemize}
    \item 继续优化瓣膜设计以减少脱位风险
    \item 开发更好的预测工具识别高危患者
    \item 改进的捕获和固定装置
    \item 标准化的救治流程和培训
    \item 新技术如实时3D成像指导释放
\end{itemize}

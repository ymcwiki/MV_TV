\section{自膨胀瓣膜严重内折的延迟治疗}

\subsection{文献信息}

\begin{itemize}
    \item \textbf{标题}:Delayed Treatment of Severe Infolding of Self-Expanding Transcatheter Heart Valve
    \item \textbf{作者}:Charlene L. Rohm, MD; Eric Quintana, MD; Jared O'Leary, MD
    \item \textbf{单位}:Vanderbilt University Medical Center
    \item \textbf{会议}:TCT 2024
    \item \textbf{编号}:TCT-1372
    \item \textbf{研究类型}:病例报告
\end{itemize}

\subsection{病例背景}

\textbf{患者基本信息}:
\begin{itemize}
    \item 75岁男性
    \item 症状性重度主动脉瓣狭窄
    \item 外科手术高危患者
    \item 决定行TAVR治疗
\end{itemize}

\subsection{术前评估}

\subsubsection{心脏CT测量}

\textbf{主动脉瓣环测量结果}:
\begin{table}[h]
\centering
\caption{术前CT瓣环测量参数}
\begin{tabular}{p{5cm}p{4cm}}
\toprule
\textbf{测量参数} & \textbf{数值} \\
\midrule
瓣环最小径 & 25.6 mm \\
瓣环最大径 & 30.1 mm \\
瓣环平均径 & 27.8 mm \\
面积推导直径 & 27.8 mm \\
\textbf{周长推导直径} & \textbf{28.0 mm} \\
瓣环面积 & 605.2 mm² \\
\textbf{瓣环周长} & \textbf{88.0 mm} \\
\midrule
\multicolumn{2}{l}{\textit{LVOT测量}} \\
LVOT最小径 & 24.0 mm \\
LVOT最大径 & 30.4 mm \\
LVOT平均径 & 27.2 mm \\
LVOT周长 & 85.4 mm \\
\midrule
\multicolumn{2}{l}{\textit{窦管交界}} \\
STJ平均径 & 24.6 mm \\
\bottomrule
\end{tabular}
\end{table}

\textbf{瓣膜选择}:
\begin{itemize}
    \item 基于瓣环周长88mm,选择\textbf{34mm自膨胀瓣膜(SE-THV)}
    \item 这是较大尺寸的瓣膜
\end{itemize}

\subsection{首次手术过程}

\subsubsection{TAVR操作步骤}

\textbf{瓣膜释放过程中发现内折}:

\begin{enumerate}
    \item \textbf{瓣膜定位}:
    \begin{itemize}
        \item 34mm SE-THV成功送入主动脉瓣位置
        \item 透视下可见瓣膜框架的4个节点(Node 1-4)
    \end{itemize}

    \item \textbf{发现内折}:
    \begin{itemize}
        \item 释放过程中透视显示瓣膜叶片\textcolor{red}{内折}
        \item 瓣膜框架形态异常
        \item 植入深度:5mm(较浅)
    \end{itemize}

    \item \textbf{即刻处理尝试}:
    \begin{itemize}
        \item 使用24mm Atlas Gold球囊进行球囊扩张
        \item 试图纠正内折畸形
    \end{itemize}
\end{enumerate}

\subsubsection{术后即刻超声评估}

\textbf{经食道超声(TEE)检查}:
\begin{itemize}
    \item 瓣膜框架扩张不充分
    \item 仍存在明显内折
    \item \textcolor{red}{跨瓣压差增高}
\end{itemize}

\subsection{术后随访}

\subsubsection{随访影像学检查}

\textbf{超声心动图}:
\begin{itemize}
    \item \textbf{主动脉瓣平均压差}:\textcolor{red}{24 mmHg}(异常升高)
    \item 多普勒显示跨瓣血流速度增快
    \item 提示瓣膜功能受损
\end{itemize}

\textbf{心脏CT}:
\begin{itemize}
    \item 瓣膜框架形态评估
    \item 确认持续性内折
    \item 瓣环周长:83.8mm
    \item 瓣膜未完全扩张
\end{itemize}

\subsubsection{临床决策}

基于以下因素决定再次干预:
\begin{itemize}
    \item 跨瓣压差显著升高(24 mmHg)
    \item 影像学证实持续性严重内折
    \item 患者可能出现症状
    \item 长期预后不佳风险
\end{itemize}

\subsection{第二次手术}

\subsubsection{延迟球囊扩张术}

\textbf{手术时机}:首次TAVR术后\textbf{24天}

\textbf{手术过程}:

\begin{enumerate}
    \item \textbf{术前评估}:
    \begin{itemize}
        \item 透视下确认瓣膜位置
        \item TEE评估瓣膜功能和内折程度
        \item 确认内折区域
    \end{itemize}

    \item \textbf{球囊扩张}:
    \begin{itemize}
        \item 使用\textbf{24mm Atlas Gold球囊}
        \item 精确定位于瓣膜框架内
        \item 缓慢充盈球囊
        \item 高压扩张(具体压力未报告)
    \end{itemize}

    \item \textbf{术中监测}:
    \begin{itemize}
        \item 实时透视监测框架扩张
        \item TEE评估瓣叶活动度改善
        \item 多普勒测量压差变化
    \end{itemize}
\end{enumerate}

\subsubsection{即刻结果}

\textbf{透视显示}:
\begin{itemize}
    \item 瓣膜框架充分扩张
    \item 内折完全纠正
    \item 框架形态对称、规整
\end{itemize}

\textbf{超声心动图}:
\begin{itemize}
    \item \textbf{主动脉瓣平均压差}:\textcolor{blue}{2-3 mmHg}(\textcolor{green}{优秀结果})
    \item 从24 mmHg降至2-3 mmHg(\textcolor{green}{降低约90\%})
    \item 瓣叶活动良好
    \item 无明显瓣周漏
\end{itemize}

\subsection{病例分析与讨论}

\subsubsection{瓣膜内折的发生机制}

\textbf{内折的定义}:
\begin{itemize}
    \item 瓣膜叶片在释放过程中向内卷曲
    \item 导致有效开口面积减少
    \item 影响瓣膜血流动力学性能
\end{itemize}

\textbf{可能的发生原因}:
\begin{enumerate}
    \item \textbf{大尺寸瓣膜因素}
    \begin{itemize}
        \item 本病例使用34mm SE-THV(较大尺寸)
        \item 大尺寸瓣膜叶片面积更大
        \item 释放过程中更容易发生内折
        \item 文献报道主要发生于≥29mm瓣膜
    \end{itemize}

    \item \textbf{解剖因素}
    \begin{itemize}
        \item 瓣环周长88mm,相对较大
        \item 可能存在瓣环椭圆形态(最小径25.6mm vs 最大径30.1mm)
        \item 瓣环-窦管交界空间关系
    \end{itemize}

    \item \textbf{技术因素}
    \begin{itemize}
        \item 植入深度较浅(5mm)
        \item 释放角度和速度
        \item 瓣膜与瓣环的相对位置
    \end{itemize}

    \item \textbf{瓣膜设计因素}
    \begin{itemize}
        \item 自膨胀瓣膜特性
        \item 镍钛合金框架的径向力
        \item 叶片材料和厚度
    \end{itemize}
\end{enumerate}

\subsubsection{内折的血流动力学影响}

\begin{table}[h]
\centering
\caption{内折前后血流动力学参数对比}
\begin{tabular}{p{5cm}p{3cm}p{3cm}}
\toprule
\textbf{参数} & \textbf{内折时} & \textbf{纠正后} \\
\midrule
主动脉瓣平均压差 & 24 mmHg & 2-3 mmHg \\
压差降低幅度 & -- & \textcolor{green}{约90\%} \\
有效瓣口面积 & 减少 & 正常 \\
瓣叶活动度 & 受限 & 良好 \\
血流速度 & 增快 & 正常 \\
\bottomrule
\end{tabular}
\end{table}

\textbf{临床意义}:
\begin{itemize}
    \item 平均压差24 mmHg提示中度狭窄
    \item 长期可能导致左室负荷增加
    \item 影响患者症状改善
    \item 可能影响瓣膜耐久性
\end{itemize}

\subsubsection{延迟治疗的可行性}

\textbf{本病例的独特之处}:
\begin{itemize}
    \item 首次报告\textcolor{red}{延迟治疗}严重瓣膜内折
    \item 既往报告多为\textcolor{blue}{即刻处理}
    \item 24天后再次干预仍获得成功
\end{itemize}

\textbf{延迟治疗的理论基础}:

\begin{enumerate}
    \item \textbf{瓣膜框架特性}:
    \begin{itemize}
        \item 镍钛合金具有超弹性
        \item 即使术后数周,框架仍可能被重新塑形
        \item 组织内生长可能尚未完全固定框架
    \end{itemize}

    \item \textbf{组织反应时程}:
    \begin{itemize}
        \item 24天时瓣膜-组织界面可能仍在重塑
        \item 纤维化和钙化尚未发生
        \item 仍有机械干预的空间
    \end{itemize}

    \item \textbf{风险-获益评估}:
    \begin{itemize}
        \item 球囊扩张风险相对较低
        \item 持续高压差的长期危害更大
        \item 延迟治疗给予了观察和计划的时间
    \end{itemize}
\end{enumerate}

\subsubsection{治疗策略选择}

\textbf{可选的治疗方案}:

\begin{table}[h]
\centering
\caption{瓣膜内折的治疗选项}
\begin{tabular}{p{4cm}p{5cm}p{4.5cm}}
\toprule
\textbf{治疗方法} & \textbf{优点} & \textbf{缺点} \\
\midrule
\textbf{球囊扩张} &
\begin{itemize}[leftmargin=*]
    \item 微创
    \item 技术相对简单
    \item 可重复操作
    \item 成功率高
\end{itemize} &
\begin{itemize}[leftmargin=*]
    \item 可能损伤瓣叶
    \item 可能加重瓣周漏
    \item 效果可能不持久
\end{itemize} \\
\midrule
瓣中瓣TAVR &
\begin{itemize}[leftmargin=*]
    \item 确定性解决方案
    \item 提供新的功能瓣膜
\end{itemize} &
\begin{itemize}[leftmargin=*]
    \item 费用高
    \item 冠脉阻塞风险
    \item 有效瓣口面积减少
    \item 未来再干预困难
\end{itemize} \\
\midrule
外科瓣膜置换 &
\begin{itemize}[leftmargin=*]
    \item 彻底解决问题
    \item 可处理其他病变
\end{itemize} &
\begin{itemize}[leftmargin=*]
    \item 创伤大
    \item 本例患者高危
    \item 恢复时间长
    \item 手术风险高
\end{itemize} \\
\midrule
保守观察 &
\begin{itemize}[leftmargin=*]
    \item 无操作风险
    \item 给予自然恢复机会
\end{itemize} &
\begin{itemize}[leftmargin=*]
    \item 症状可能持续
    \item 左室重构风险
    \item 长期预后差
\end{itemize} \\
\bottomrule
\end{tabular}
\end{table}

\textbf{本病例选择球囊扩张的理由}:
\begin{itemize}
    \item 风险最小
    \item 技术可行
    \item 保留未来治疗选择
    \item 费用相对较低
    \item 既往即刻球囊扩张部分有效,提示延迟扩张可能成功
\end{itemize}

\subsubsection{球囊扩张技术要点}

\textbf{球囊选择}:
\begin{itemize}
    \item 本病例选择24mm Atlas Gold球囊
    \item 球囊直径约为瓣膜标称直径的70\%(24/34)
    \item 目的是扩张框架而非过度扩张
\end{itemize}

\textbf{操作技术}:
\begin{enumerate}
    \item \textbf{精确定位}:
    \begin{itemize}
        \item 透视下将球囊置于瓣膜框架中心
        \item TEE监测球囊与瓣叶的关系
        \item 避免球囊位置过高或过低
    \end{itemize}

    \item \textbf{扩张策略}:
    \begin{itemize}
        \item 缓慢充盈球囊
        \item 允许框架逐渐扩张
        \item 可能需要多次扩张
        \item 实时监测血流动力学变化
    \end{itemize}

    \item \textbf{安全监测}:
    \begin{itemize}
        \item 持续TEE监测瓣周漏
        \item 观察瓣叶活动度
        \item 评估冠脉口位置关系
        \item 监测心律失常
    \end{itemize}
\end{enumerate}

\subsection{文献回顾与对比}

\subsubsection{瓣膜内折的发生率}

\textbf{文献报道}:
\begin{itemize}
    \item 自膨胀瓣膜内折是\textcolor{blue}{罕见并发症}
    \item 主要发生于\textbf{大尺寸瓣膜(≥29mm)}
    \item 确切发生率缺乏大规模统计
    \item 可能因轻微内折未被识别而低估
\end{itemize}

\textbf{与瓣膜尺寸的关系}:
\begin{table}[h]
\centering
\caption{不同尺寸自膨胀瓣膜内折风险}
\begin{tabular}{p{3cm}p{4cm}p{6cm}}
\toprule
\textbf{瓣膜尺寸} & \textbf{内折风险} & \textbf{可能原因} \\
\midrule
23-26mm & 低 & 叶片面积较小,易于展开 \\
29mm & 中等 & 叶片面积增大,开始出现风险 \\
31-34mm & \textcolor{red}{较高} & 叶片面积大,框架扩张力需求高 \\
\bottomrule
\end{tabular}
\end{table}

\subsubsection{即刻处理 vs 延迟治疗}

\textbf{既往文献报道的处理时机}:
\begin{itemize}
    \item 绝大多数病例报告为\textcolor{blue}{术中即刻发现和处理}
    \item 即刻球囊扩张成功率较高
    \item 少数即刻处理失败的病例采用瓣中瓣或外科手术
    \item \textcolor{red}{本病例首次报告延迟24天成功治疗}
\end{itemize}

\textbf{延迟治疗的优势}:
\begin{enumerate}
    \item \textbf{时间优势}:
    \begin{itemize}
        \item 充分评估患者整体状况
        \item 详细制定治疗计划
        \item 多学科团队讨论
        \item 优化患者术前准备
    \end{itemize}

    \item \textbf{诊断优势}:
    \begin{itemize}
        \item 多模态影像评估(TTE、TEE、CT)
        \item 精确测量压差和瓣口面积
        \item 明确内折的严重程度和影响
        \item 评估瓣膜其他功能参数
    \end{itemize}

    \item \textbf{安全优势}:
    \begin{itemize}
        \item 避免术中过度操作
        \item 患者血流动力学稳定
        \item 术者准备更充分
        \item 团队配合更流畅
    \end{itemize}
\end{enumerate}

\textbf{延迟治疗的潜在风险}:
\begin{itemize}
    \item 组织内生长可能固定内折
    \item 患者症状持续
    \item 左室重构风险
    \item 需要额外一次操作
    \item 可能增加总体费用
\end{itemize}

\subsection{临床意义与启示}

\subsubsection{对临床实践的启示}

\textbf{术中管理}:
\begin{enumerate}
    \item \textbf{预防策略}:
    \begin{itemize}
        \item 大瓣环患者谨慎选择瓣膜尺寸
        \item 考虑瓣环形态(圆形vs椭圆形)
        \item 优化植入深度(避免过浅)
        \item 释放过程密切监测
    \end{itemize}

    \item \textbf{早期识别}:
    \begin{itemize}
        \item 透视下观察框架展开形态
        \item TEE实时评估瓣叶活动
        \item 多普勒即刻测量压差
        \item 可疑时立即详细评估
    \end{itemize}

    \item \textbf{即刻处理}:
    \begin{itemize}
        \item 术中发现应尝试球囊扩张
        \item 选择合适球囊尺寸
        \item 可能需要多次尝试
        \item 评估处理效果
    \end{itemize}
\end{enumerate}

\textbf{术后随访}:
\begin{enumerate}
    \item \textbf{早期监测}:
    \begin{itemize}
        \item 出院前超声评估
        \item 测量跨瓣压差
        \item 评估瓣叶活动度
        \item 可疑时进行CT检查
    \end{itemize}

    \item \textbf{压差异常的处理}:
    \begin{itemize}
        \item 术后压差>10-15 mmHg应引起重视
        \item 详细影像学评估原因
        \item 考虑球囊扩张可能性
        \item 权衡干预时机和风险
    \end{itemize}
\end{enumerate}

\subsubsection{延迟球囊扩张的适应证和时机}

\textbf{可能适应证}:
\begin{itemize}
    \item 影像学证实的严重内折
    \item 跨瓣压差显著升高(>15-20 mmHg)
    \item 有效瓣口面积明显减少
    \item 患者症状未改善或恶化
    \item 术中即刻处理效果不佳
\end{itemize}

\textbf{时间窗口}:
\begin{itemize}
    \item 本病例在术后24天成功治疗
    \item 提示至少在\textcolor{blue}{术后数周内}可能有效
    \item 建议在\textcolor{red}{组织内生长完全发生前}(通常3个月内)
    \item 具体时间窗口需更多病例验证
\end{itemize}

\textbf{禁忌证或相对禁忌}:
\begin{itemize}
    \item 患者血流动力学不稳定
    \item 严重瓣周漏(球囊扩张可能加重)
    \item 已发生显著组织内生长
    \item 框架明显扭曲或变形
    \item 合并其他需要外科处理的病变
\end{itemize}

\subsubsection{对未来研究的提示}

\begin{enumerate}
    \item \textbf{内折机制研究}:
    \begin{itemize}
        \item 体外模型研究内折发生条件
        \item 计算机模拟不同瓣环形态的影响
        \item 分析瓣膜设计与内折风险的关系
        \item 探索预测内折的影像学指标
    \end{itemize}

    \item \textbf{预防策略研究}:
    \begin{itemize}
        \item 优化瓣膜尺寸选择算法
        \item 改进瓣膜设计减少内折风险
        \item 研究最佳植入技术和深度
        \item 开发术中预警系统
    \end{itemize}

    \item \textbf{治疗时机研究}:
    \begin{itemize}
        \item 建立延迟治疗病例库
        \item 比较即刻vs延迟治疗结果
        \item 确定最佳治疗时间窗口
        \item 制定标准化治疗流程
    \end{itemize}

    \item \textbf{长期随访研究}:
    \begin{itemize}
        \item 评估球囊扩张后瓣膜耐久性
        \item 监测晚期再狭窄风险
        \item 分析对生活质量的长期影响
        \item 比较不同治疗策略的远期结果
    \end{itemize}
\end{enumerate}

\subsection{要点总结}

\subsubsection{病例核心信息}

\begin{enumerate}
    \item \textbf{病例特点}:
    \begin{itemize}
        \item 75岁男性,高危AS患者
        \item 34mm SE-THV植入后发生严重内折
        \item 术中即刻球囊扩张效果不佳
        \item 术后跨瓣压差24 mmHg
    \end{itemize}

    \item \textbf{创新治疗}:
    \begin{itemize}
        \item 术后24天行延迟球囊扩张
        \item 使用24mm球囊成功纠正内折
        \item 压差从24 mmHg降至2-3 mmHg
        \item \textcolor{green}{首次报告延迟治疗成功}
    \end{itemize}
\end{enumerate}

\subsubsection{关键学习要点}

\begin{enumerate}
    \item \textbf{内折特征}:
    \begin{itemize}
        \item 自膨胀瓣膜的罕见并发症
        \item \textcolor{red}{主要发生于大尺寸瓣膜(≥29mm)}
        \item 可能与瓣环解剖、技术因素相关
        \item 影响血流动力学性能
    \end{itemize}

    \item \textbf{诊断要点}:
    \begin{itemize}
        \item 透视下观察框架形态
        \item TEE评估瓣叶活动和压差
        \item CT精确评估框架扩张程度
        \item 多模态影像综合判断
    \end{itemize}

    \item \textbf{治疗策略}:
    \begin{itemize}
        \item 首选球囊扩张
        \item 即刻处理 vs 延迟治疗均可
        \item \textcolor{blue}{延迟治疗仍可获得优秀结果}
        \item 选择合适球囊尺寸和技术
    \end{itemize}

    \item \textbf{临床意义}:
    \begin{itemize}
        \item 扩展了内折治疗的时间窗口
        \item 为术中即刻处理失败提供了选择
        \item 允许更充分的评估和计划
        \item 避免了更激进的干预(ViV或外科)
    \end{itemize}
\end{enumerate}

\subsubsection{实践建议}

\begin{table}[h]
\centering
\caption{自膨胀瓣膜内折的预防、识别和管理建议}
\begin{tabular}{p{3cm}p{11cm}}
\toprule
\textbf{阶段} & \textbf{建议} \\
\midrule
\textbf{术前预防} &
\begin{itemize}[leftmargin=*]
    \item 大瓣环患者(周长>85mm)谨慎选择瓣膜尺寸
    \item 评估瓣环椭圆度,椭圆形瓣环风险可能更高
    \item 使用≥29mm瓣膜时特别警惕
    \item 制定详细操作计划
\end{itemize} \\
\midrule
\textbf{术中识别} &
\begin{itemize}[leftmargin=*]
    \item 透视多角度观察框架展开
    \item TEE实时监测瓣叶活动
    \item 即刻测量跨瓣压差
    \item 可疑时立即详细评估
\end{itemize} \\
\midrule
\textbf{即刻处理} &
\begin{itemize}[leftmargin=*]
    \item 术中发现立即球囊扩张
    \item 球囊直径约为瓣膜直径的70\%
    \item 可能需要多次尝试
    \item 评估处理效果
\end{itemize} \\
\midrule
\textbf{术后随访} &
\begin{itemize}[leftmargin=*]
    \item 出院前超声评估
    \item 压差>10-15 mmHg需详细评估
    \item 考虑CT检查明确诊断
    \item 制定个体化随访计划
\end{itemize} \\
\midrule
\textbf{延迟治疗} &
\begin{itemize}[leftmargin=*]
    \item 压差显著升高考虑再次干预
    \item 建议在术后3个月内进行
    \item 球囊扩张为首选方法
    \item 多学科团队讨论决策
\end{itemize} \\
\bottomrule
\end{tabular}
\end{table}

\subsection{结论}

本病例报告了首例延迟成功治疗自膨胀瓣膜严重内折的经验,主要结论包括:

\begin{enumerate}
    \item \textbf{内折可延迟治疗}:球囊主动脉瓣成形术可以在TAVR术后数周安全有效地纠正严重瓣膜内折,不局限于术中即刻处理。

    \item \textbf{疗效确切}:延迟球囊扩张可获得与即刻处理相似的优秀血流动力学结果,本病例压差从24 mmHg降至2-3 mmHg。

    \item \textbf{扩展治疗选择}:为术中即刻处理效果不佳或未识别的内折病例提供了二次干预机会,避免了更激进的治疗方案。

    \item \textbf{临床实践指导}:提示临床医生在遇到TAVR术后压差异常升高时,应考虑瓣膜内折可能,并积极评估球囊扩张的可行性。

    \item \textbf{未来研究方向}:需要更多病例和长期随访数据来确定延迟治疗的最佳时间窗口、适应证和长期疗效。
\end{enumerate}

这一病例为TAVR并发症管理提供了重要的临床经验,强调了密切术后随访和积极干预的重要性。

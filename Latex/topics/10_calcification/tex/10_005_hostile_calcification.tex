\section{敌对性主动脉瓣钙化:TAVR还是不TAVR?}
\label{sec:10_005_hostile_calcification}

% ============================================
% 文献信息
% ============================================
\subsection{文献信息}

\begin{itemize}
    \item \textbf{标题}: Hostile Aortic Valve Calcification: To TAVR or not to TAVR?
    \item \textbf{作者}: Konstantinos Stathogiannis, MD, FACC, PhD
    \item \textbf{机构}: Transcatheter Heart Valves Department, Hygeia Hospital, Athens, Greece
    \item \textbf{会议}: TCT (Transcatheter Cardiovascular Therapeutics)
    \item \textbf{PDF文件名}: tct-1395-hostile-aortic-valve-calcification-to-tavr-or-not-to-tavr.pdf
    \item \textbf{文献类型}: 会议演讲/病例报告
\end{itemize}

\subsection{研究背景}

\subsubsection{敌对性钙化的定义与挑战}

敌对性主动脉瓣钙化(Hostile Aortic Valve Calcification)是指主动脉瓣及瓣环区域存在极度严重、不对称分布的钙化,这种解剖特征对经导管主动脉瓣置换术(TAVR)构成重大技术挑战。

\textbf{常见于以下情况}:
\begin{itemize}
    \item 二叶主动脉瓣畸形(Bicuspid Aortic Valve, BAV)
    \item 巨大瓣环(Bulky annulus)病例
    \item 钙化延伸至左心室流出道(LVOT)
    \item 不对称钙化分布
\end{itemize}

\textbf{技术挑战}:
\begin{itemize}
    \item 瓣膜定位困难
    \item 瓣膜扩张受阻
    \item 增加瓣周漏(PVL)风险
    \item 冠脉开口阻塞风险
    \item 瓣环破裂风险
    \item 传导阻滞风险增加
\end{itemize}

\subsection{病例展示}

\subsubsection{患者基本信息}

\begin{table}[h]
\centering
\caption{患者基本临床资料}
\label{tab:patient_baseline}
\begin{tabular}{ll}
\toprule
\textbf{参数} & \textbf{数值} \\
\midrule
年龄 & 84岁 \\
性别 & 男性 \\
体重指数(BMI) & 29.7 kg/m² \\
主要症状 & 意识丧失(LOC)事件 \\
LOC发生时间 & 1年前1次,1周前再次发生 \\
伴随症状 & 疲劳 \\
\bottomrule
\end{tabular}
\end{table}

\subsubsection{既往病史}

\begin{itemize}
    \item \textbf{创伤性脑血肿}:10年前
    \item \textbf{高血压}(HTN)
    \item \textbf{慢性肾脏病}(CKD):肾小球滤过率(GFR)67 mL/min
    \item \textbf{冠脉造影}:无冠心病(外院检查)
\end{itemize}

\subsubsection{超声心动图检查结果}

\begin{table}[h]
\centering
\caption{术前超声心动图参数}
\label{tab:pre_tavr_echo}
\begin{tabular}{lcc}
\toprule
\textbf{参数} & \textbf{术前数值} & \textbf{严重程度} \\
\midrule
AV Vmax & 4.43 m/s & 严重 \\
AV Vmean & 3.36 m/s & 严重 \\
AV max PG & 78.55 mmHg & 严重 \\
AV mean PG & 51.33 mmHg & 严重 \\
AV VTI & 117.5 cm & — \\
AV Env.Ti & 349 ms & — \\
心率(HR) & 168 BPM & 快速房颤 \\
\bottomrule
\end{tabular}
\end{table}

\textbf{关键发现}:
\begin{itemize}
    \item 严重主动脉瓣狭窄
    \item 二叶主动脉瓣(Type I, R-L型):右冠瓣与左冠瓣融合
    \item 极度钙化累及瓣叶和缝合线(raphe)
\end{itemize}

\subsubsection{CT检查结果}

\textbf{钙化评估}:
\begin{itemize}
    \item \textbf{Agatston评分}:\textbf{9850 HU}(极度严重)
    \item 钙化分布:主动脉瓣、缝合线、延伸至LVOT
    \item 钙化特点:重度、不对称、bulky
\end{itemize}

\textbf{解剖测量}(瓣环上方不同水平):
\begin{table}[h]
\centering
\caption{CT解剖测量数据}
\label{tab:ct_measurements}
\begin{tabular}{lcc}
\toprule
\textbf{测量位置} & \textbf{直径1} & \textbf{直径2} \\
\midrule
瓣环水平 & 26 mm & 32 mm \\
瓣环上3mm & 25.23 mm & 38.55 mm \\
瓣环上4mm & 29.55 mm & — \\
瓣环上5mm & 29.33 mm & — \\
左冠脉开口高度 & \multicolumn{2}{c}{12.81 mm} \\
\bottomrule
\end{tabular}
\end{table}

\subsubsection{风险评分}

\begin{itemize}
    \item \textbf{EuroSCORE II}:3.6\%
    \item \textbf{STS评分}:2.4\%
    \item \textbf{STS发病率/死亡率}:7\%
\end{itemize}

\subsection{Heart Team决策}

\subsubsection{患者特点总结}

\begin{itemize}
    \item 84岁男性二叶主动脉瓣狭窄(Type I, R-L型)
    \item 主要症状:反复意识丧失事件
    \item 主动脉瓣和缝合线极度钙化,延伸至LVOT
    \item Agatston评分:9850 HU
    \item 外科风险:EuroSCORE II 3.6\%, STS 2.4\%, STS m/m 7\%
\end{itemize}

\subsubsection{治疗决策}

经Heart Team讨论,决定实施\textbf{TAVR(经导管主动脉瓣置换术)}。

\textbf{关键考虑因素}:
\begin{itemize}
    \item 患者高龄(84岁)
    \item 有症状(晕厥事件)
    \item 外科风险中等
    \item 极度钙化但新一代TAVR瓣膜可能适用
    \item 需要仔细的术前规划和影像引导
\end{itemize}

\subsection{TAVR手术过程}

\subsubsection{瓣膜选择}

基于CT测量和钙化分布特点,选择适合的经导管瓣膜系统(演讲中未明确说明具体型号,但从影像看使用了新一代可扩展瓣膜)。

\subsubsection{手术步骤}

演讲展示了详细的手术透视图像序列,显示:

\begin{enumerate}
    \item \textbf{血管通路建立}:经股动脉入路
    \item \textbf{导丝通过}:克服重度钙化,成功通过主动脉瓣
    \item \textbf{球囊预扩张}:必要的预扩张步骤
    \item \textbf{瓣膜输送}:瓣膜输送系统通过钙化区域
    \item \textbf{瓣膜定位}:精确定位于瓣环水平
    \item \textbf{瓣膜释放}:逐步释放瓣膜
    \item \textbf{瓣膜扩张}:在极度钙化环境中充分扩张
\end{enumerate}

\textbf{术中挑战}:
\begin{itemize}
    \item 导丝穿过严重钙化的瓣叶
    \item 输送系统通过僵硬的瓣环
    \item 在不对称钙化中实现瓣膜对称扩张
    \item 避免冠脉阻塞(左冠开口高度仅12.81mm)
\end{itemize}

\subsection{主要研究发现}

\subsubsection{术后即刻结果}

\textbf{血流动力学改善}:

\begin{table}[h]
\centering
\caption{TAVR术前术后血流动力学对比}
\label{tab:pre_post_tavr}
\begin{tabular}{lccc}
\toprule
\textbf{参数} & \textbf{术前} & \textbf{术后} & \textbf{改善率} \\
\midrule
AV Vmax & 4.43 m/s & 2.43 m/s & 45\% ↓ \\
AV mean PG & 51.33 mmHg & 13 mmHg & 75\% ↓ \\
AV max PG & 78.55 mmHg & 24 mmHg & 69\% ↓ \\
AV VTI & 117.5 cm & 46.2 cm & 61\% ↓ \\
\bottomrule
\end{tabular}
\end{table}

\textbf{超声心动图评估}:
\begin{itemize}
    \item 术后主动脉瓣反流(AV VR):0.34(轻度)
    \item 无明显瓣周漏
    \item 瓣膜功能良好
\end{itemize}

\subsubsection{术后CT评估}

术后CT扫描显示:

\textbf{瓣膜位置与扩张}:
\begin{itemize}
    \item 瓣膜位置良好
    \item 在极度钙化环境中实现充分扩张
    \item 瓣架与瓣环良好贴合
    \item 无瓣环破裂
\end{itemize}

\textbf{钙化与瓣膜的关系}:
\begin{itemize}
    \item 重度钙化被瓣膜支架向外推移
    \item 瓣膜支架在钙化区域实现机械扩张
    \item 无明显瓣周间隙
    \item 冠脉开口未受影响
\end{itemize}

\textbf{LVOT评估}:
\begin{itemize}
    \item 无LVOT梗阻
    \item 延伸至LVOT的钙化未影响血流
    \item 瓣膜深度适当
\end{itemize}

\subsubsection{并发症}

演讲未提及明显并发症,提示:
\begin{itemize}
    \item 无冠脉阻塞
    \item 无瓣环破裂
    \item 无需植入永久起搏器(演讲中未提及)
    \item 无血管并发症
    \item 无卒中事件
\end{itemize}

\subsection{结论}

\subsubsection{主要结论}

\begin{enumerate}
    \item \textbf{重度、不对称钙化是TAVR的手术挑战}
    \begin{itemize}
        \item 需要仔细的术前评估
        \item 需要精确的手术规划
        \item 需要高级的影像引导技术
    \end{itemize}

    \item \textbf{常见于二叶瓣和巨大瓣环病例}
    \begin{itemize}
        \item 二叶瓣患者钙化模式不同于三叶瓣
        \item 缝合线(raphe)钙化是关键挑战
        \item 钙化常延伸至LVOT
    \end{itemize}

    \item \textbf{CT形态学指导瓣膜选择和手术规划}
    \begin{itemize}
        \item Agatston评分量化钙化严重程度
        \item 钙化分布模式影响瓣膜选择
        \item 瓣环测量指导瓣膜尺寸选择
        \item 冠脉高度评估预测冠脉阻塞风险
    \end{itemize}

    \item \textbf{新一代装置使TAVR可行}
    \begin{itemize}
        \item 更强的径向力
        \item 更好的瓣膜贴合性
        \item 可控释放和重新定位
        \item 优化的瓣膜设计减少PVL
    \end{itemize}

    \item \textbf{影像引导策略将"禁忌"解剖转化为成功}
    \begin{itemize}
        \item 多模态影像整合(CT + Echo + Fluoro)
        \item 3D重建辅助手术规划
        \item 术中实时影像引导
        \item 将曾经的"no-go"解剖变为可行
    \end{itemize}
\end{enumerate}

\subsection{临床启示}

\subsubsection{术前评估要点}

\textbf{1. 钙化评估}:
\begin{itemize}
    \item \textbf{定量评估}:Agatston评分
    \begin{itemize}
        \item <1000 HU:轻度
        \item 1000-2000 HU:中度
        \item 2000-4000 HU:重度
        \item >4000 HU:极度(本例9850 HU)
    \end{itemize}
    \item \textbf{定性评估}:
    \begin{itemize}
        \item 钙化分布模式(对称vs不对称)
        \item 钙化位置(瓣叶、瓣环、LVOT)
        \item 钙化体积和厚度
        \item 钙化对瓣叶运动的影响
    \end{itemize}
\end{itemize}

\textbf{2. 解剖评估}:
\begin{itemize}
    \item 瓣环大小和形态
    \item 主动脉根部几何形态
    \item Sinus of Valsalva直径和高度
    \item 冠脉开口高度和距离
    \item LVOT直径和钙化情况
    \item 升主动脉直径和成角
\end{itemize}

\textbf{3. 二叶瓣特殊考虑}:
\begin{itemize}
    \item Sievers分型(本例Type I, R-L)
    \item 缝合线位置和钙化程度
    \item 瓣环椭圆度
    \item 升主动脉扩张程度
\end{itemize}

\subsubsection{瓣膜选择策略}

\textbf{针对极度钙化的瓣膜选择考虑}:

\begin{enumerate}
    \item \textbf{径向力}:选择径向力强的瓣膜
    \begin{itemize}
        \item 球扩瓣膜通常有更强径向力
        \item 自膨瓣膜可能在极度钙化中扩张不足
    \end{itemize}

    \item \textbf{瓣膜高度}:
    \begin{itemize}
        \item 考虑钙化延伸至LVOT的情况
        \item 避免瓣膜过深植入
        \item 评估冠脉阻塞风险
    \end{itemize}

    \item \textbf{封闭性能}:
    \begin{itemize}
        \item 不对称钙化增加PVL风险
        \item 选择有良好封闭裙的瓣膜
    \end{itemize}
\end{enumerate}

\subsubsection{手术技巧}

\textbf{1. 通路选择}:
\begin{itemize}
    \item 经股动脉入路通常首选
    \item 评估髂股血管条件
    \item 必要时考虑替代通路
\end{itemize}

\textbf{2. 预扩张策略}:
\begin{itemize}
    \item 极度钙化常需要预扩张
    \item 逐步递增球囊尺寸
    \item 注意瓣环破裂风险
\end{itemize}

\textbf{3. 瓣膜植入}:
\begin{itemize}
    \item 影像引导精确定位
    \item 考虑钙化分布的不对称性
    \item 缓慢释放,必要时调整
    \item 评估是否需要后扩张
\end{itemize}

\textbf{4. 并发症预防}:
\begin{itemize}
    \item 冠脉阻塞:术前评估冠脉高度,准备冠脉保护策略
    \item 瓣环破裂:避免过度预扩张
    \item 传导阻滞:监测心律,准备临时起搏
    \item PVL:选择合适瓣膜,必要时后扩张
\end{itemize}

\subsubsection{患者选择}

\textbf{适合TAVR的敌对性钙化患者}:
\begin{itemize}
    \item 有症状的严重主动脉瓣狭窄
    \item 外科风险中等或更高
    \item 解剖条件允许TAVR入路
    \item 冠脉开口高度足够(通常>10-12mm)
    \item 无其他禁忌证
\end{itemize}

\textbf{需要慎重考虑的情况}:
\begin{itemize}
    \item 冠脉开口极低(<10mm)
    \item LVOT钙化极度严重可能导致梗阻
    \item 瓣环极小或极大超出可用瓣膜范围
    \item 升主动脉严重成角影响通路
\end{itemize}

\subsubsection{随访要点}

\begin{enumerate}
    \item \textbf{即刻评估}:
    \begin{itemize}
        \item 术后超声评估瓣膜功能
        \item 评估PVL程度
        \item 监测传导系统
        \item 评估血管并发症
    \end{itemize}

    \item \textbf{术后CT}:
    \begin{itemize}
        \item 评估瓣膜位置和扩张
        \item 评估钙化与瓣膜的关系
        \item 排除并发症
    \end{itemize}

    \item \textbf{长期随访}:
    \begin{itemize}
        \item 定期超声评估
        \item 监测瓣膜功能衰退
        \item 评估结构性瓣膜退化(SVD)
    \end{itemize}
\end{enumerate}

\subsection{研究局限性}

\begin{enumerate}
    \item \textbf{病例报告性质}:
    \begin{itemize}
        \item 这是单一病例展示,而非系统性研究
        \item 缺乏对照组和长期随访数据
        \item 无法评估该策略的普遍适用性
    \end{itemize}

    \item \textbf{缺乏详细数据}:
    \begin{itemize}
        \item 未明确说明使用的具体瓣膜型号
        \item 缺乏手术时间、造影剂用量等详细数据
        \item 未提供住院时间和恢复情况
        \item 缺乏长期随访结果
    \end{itemize}

    \item \textbf{选择偏倚}:
    \begin{itemize}
        \item 展示的是成功病例
        \item 可能存在未报告的失败或并发症病例
        \item 难以评估真实的成功率
    \end{itemize}

    \item \textbf{普遍性问题}:
    \begin{itemize}
        \item 极度钙化的定义标准不统一
        \item 不同中心的技术水平和经验差异大
        \item 不同瓣膜系统的表现可能不同
    \end{itemize}

    \item \textbf{缺乏对比}:
    \begin{itemize}
        \item 未与外科手术结果对比
        \item 未讨论保守治疗的预后
        \item 未提供费用效益分析
    \end{itemize}
\end{enumerate}

\subsection{个人笔记}

\subsubsection{关键数字记忆}

\textbf{病例基本数据}:
\begin{itemize}
    \item 年龄:\textbf{84岁}
    \item Agatston评分:\textbf{9850 HU}(极度钙化,正常<100)
    \item 冠脉高度:\textbf{12.81 mm}(相对较低,需警惕)
\end{itemize}

\textbf{术前血流动力学}:
\begin{itemize}
    \item AV Vmax:4.43 m/s(正常<2.0 m/s)
    \item Mean PG:51.33 mmHg(正常<10 mmHg)
    \item Max PG:78.55 mmHg(严重狭窄)
\end{itemize}

\textbf{术后改善}:
\begin{itemize}
    \item AV Vmax:4.43 → 2.43 m/s(\textbf{↓45\%})
    \item Mean PG:51.33 → 13 mmHg(\textbf{↓75\%})
    \item Max PG:78.55 → 24 mmHg(\textbf{↓69\%})
\end{itemize}

\textbf{风险评分}:
\begin{itemize}
    \item EuroSCORE II:3.6\%(低-中危)
    \item STS:2.4\%(低-中危)
    \item STS m/m:7\%(发病率风险)
\end{itemize}

\subsubsection{重要概念}

\begin{description}
    \item[Hostile Calcification] 敌对性钙化 — 指极度严重、分布不对称的主动脉瓣钙化,传统上被认为是TAVR的相对禁忌证,但随着新一代瓣膜和技术的发展,已逐渐成为可治疗的情况。

    \item[Agatston Score] Agatston评分 — 冠脉和瓣膜钙化的定量评分方法,基于CT扫描。正常<100 HU,重度>1000 HU,极度>4000 HU。本例9850 HU代表极度钙化。

    \item[Bicuspid Aortic Valve - Type I, R-L] 二叶主动脉瓣I型(R-L)— Sievers分型中的Type I指有一条缝合线(raphe),R-L表示右冠瓣与左冠瓣融合。这种类型常伴随缝合线严重钙化。

    \item[LVOT Extension] LVOT延伸 — 钙化从主动脉瓣延伸至左心室流出道,增加TAVR后LVOT梗阻和二尖瓣损伤的风险。

    \item[Radial Force] 径向力 — 瓣膜支架向外扩张的力量,对抗钙化的压迫。球扩瓣膜通常比自膨瓣膜有更强的径向力,更适合极度钙化病例。

    \item[Coronary Obstruction Risk] 冠脉阻塞风险 — TAVR后瓣叶或钙化可能阻塞冠脉开口。冠脉高度<12mm、Sinus of Valsalva窄小、瓣叶钙化重是高危因素。

    \item[PVL (Paravalvular Leak)] 瓣周漏 — 瓣膜支架与瓣环之间的间隙导致的反流。极度不对称钙化增加PVL风险。
\end{description}

\subsubsection{临床思考要点}

\textbf{1. 为什么极度钙化曾被认为是TAVR禁忌?}

\begin{itemize}
    \item \textbf{瓣膜扩张不全}:钙化阻碍瓣膜充分扩张,导致残余狭窄
    \item \textbf{瓣周漏}:不对称钙化导致瓣膜贴合不良
    \item \textbf{瓣环破裂}:极度钙化使瓣环僵硬脆弱,扩张时可能破裂
    \item \textbf{冠脉阻塞}:钙化组织被推向冠脉开口
    \item \textbf{瓣膜移位}:不均匀钙化可能导致瓣膜偏移
    \item \textbf{传导阻滞}:钙化压迫传导系统
\end{itemize}

\textbf{2. 新一代TAVR技术如何克服这些挑战?}

\begin{itemize}
    \item \textbf{更强的径向力}:能够更好地对抗钙化压迫
    \item \textbf{改进的封闭裙}:减少PVL
    \item \textbf{可控释放系统}:允许重新定位和回收
    \item \textbf{优化的瓣膜几何}:更好适应不规则瓣环
    \item \textbf{多尺寸选择}:更精确匹配瓣环大小
\end{itemize}

\textbf{3. 影像引导的关键作用}

\begin{itemize}
    \item \textbf{术前CT}:
    \begin{itemize}
        \item 精确测量瓣环尺寸(多平面测量)
        \item 评估钙化分布和严重程度
        \item 预测手术难度和风险
        \item 制定个体化手术策略
    \end{itemize}
    \item \textbf{术中透视}:
    \begin{itemize}
        \item 实时引导瓣膜定位
        \item 监测瓣膜释放过程
        \item 评估即刻结果
    \end{itemize}
    \item \textbf{术中超声}:
    \begin{itemize}
        \item 评估瓣膜功能
        \item 检测PVL
        \item 评估血流动力学改善
    \end{itemize}
    \item \textbf{术后CT}:
    \begin{itemize}
        \item 确认瓣膜位置
        \item 评估瓣膜扩张程度
        \item 检测并发症
    \end{itemize}
\end{itemize}

\textbf{4. 二叶瓣TAVR的特殊考虑}

\begin{itemize}
    \item \textbf{解剖异质性}:
    \begin{itemize}
        \item Type 0:无缝合线(两个瓣叶)
        \item Type I:一条缝合线(本例)
        \item Type II:两条缝合线(接近三叶瓣)
    \end{itemize}
    \item \textbf{钙化模式}:
    \begin{itemize}
        \item 缝合线常严重钙化
        \item 钙化分布更不对称
        \item 更易延伸至LVOT和升主动脉
    \end{itemize}
    \item \textbf{瓣环特点}:
    \begin{itemize}
        \item 通常更椭圆
        \item 尺寸可能更大
        \item 主动脉常伴扩张
    \end{itemize}
    \item \textbf{TAVR结果}:
    \begin{itemize}
        \item 早期研究显示结果稍差于三叶瓣
        \item 新一代瓣膜改善了结果
        \item PVL率可能略高
    \end{itemize}
\end{itemize}

\subsubsection{文献相关发现}

演讲最后一页展示了相关文献:

\textbf{Cardiovascular Revascularization Medicine (CRM) 期刊文章}:

主题涉及:
\begin{itemize}
    \item 主动脉瓣二叶瓣伴hostile钙化的TAVR
    \item 轻-中度瓣环钙化对TAVR的影响
    \item 经食道超声引导在TAVR中的应用
    \item 瓣膜选择和优化策略
\end{itemize}

关键发现:
\begin{itemize}
    \item TAVR在二叶瓣中可行
    \item 影像引导改善结果
    \item 新一代瓣膜扩大了适应证
    \item 多学科团队评估至关重要
\end{itemize}

\subsubsection{实践要点总结}

\textbf{术前Must-Do}:
\begin{enumerate}
    \item 详细CT分析(Agatston评分、钙化分布、瓣环测量、冠脉高度)
    \item 超声全面评估(血流动力学、LVEF、瓣膜形态)
    \item Heart Team讨论(与外科、影像科、麻醉科)
    \item 患者知情同意(充分告知风险)
\end{enumerate}

\textbf{术中Must-Check}:
\begin{enumerate}
    \item 瓣膜定位是否精确(覆盖瓣环、不过深)
    \item 瓣膜是否充分扩张(考虑后扩张)
    \item 冠脉血流是否通畅(随时准备冠脉介入)
    \item PVL程度(>mild需要处理)
    \item 心律传导(准备临时起搏)
\end{enumerate}

\textbf{术后Must-Follow}:
\begin{enumerate}
    \item 即刻超声(24小时内)
    \item 术后CT(1周内,评估瓣膜扩张和位置)
    \item 出院前超声(评估稳定状态)
    \item 定期随访(1月、3月、6月、12月,然后每年)
\end{enumerate}

\subsubsection{个人思考}

\textbf{1. 这个病例的成功意义何在?}

这个病例打破了"极度钙化是TAVR禁忌证"的传统观念,展示了:
\begin{itemize}
    \item 9850 HU的Agatston评分曾被认为不可能进行TAVR
    \item 通过精心规划和影像引导,成功完成手术
    \item 术后血流动力学显著改善(mean PG从51降至13 mmHg)
    \item 扩展了TAVR的适应证范围
    \item 为类似高危患者提供了治疗选择
\end{itemize}

\textbf{2. 如何判断"敌对性钙化"是否可行TAVR?}

建议综合评估以下因素:

\textbf{有利因素(Go)}:
\begin{itemize}
    \item 钙化虽重但分布相对均匀
    \item 冠脉高度足够(>12-14mm)
    \item 瓣环尺寸在可用瓣膜范围内
    \item LVOT直径足够,钙化不延伸太深
    \item 有经验的术者和团队
    \item 新一代高径向力瓣膜可用
\end{itemize}

\textbf{不利因素(No-Go或慎重)}:
\begin{itemize}
    \item 冠脉极低(<10mm)且无法保护
    \item LVOT严重钙化可能导致梗阻
    \item 瓣环过小(<18mm)或过大(>30mm)
    \item 升主动脉严重成角或扩张(>45mm)
    \item 瓣环有脆性钙化,破裂风险极高
    \item 患者预期寿命极短(<6月)
\end{itemize}

\textbf{3. 中国实践的特殊考虑}

\begin{itemize}
    \item 中国患者主动脉瓣钙化可能与西方不同(风湿性病因更多)
    \item 瓣环尺寸可能偏小,需要小号瓣膜
    \item 医保覆盖和费用考虑
    \item 外科手术可及性和风险
    \item 患者和家属对TAVR的接受度
\end{itemize}

\subsubsection{启发性问题}

\begin{enumerate}
    \item 如果冠脉高度<10mm,是否仍可尝试TAVR?
    \begin{itemize}
        \item 可考虑BASILICA/LAMPOON等冠脉保护技术
        \item 准备紧急冠脉支架植入
        \item 充分告知患者风险
    \end{itemize}

    \item 如何预测极度钙化患者的PVL风险?
    \begin{itemize}
        \item CT评估钙化分布的不对称性
        \item 计算钙化体积
        \item 评估瓣环椭圆度
        \item 选择封闭性能好的瓣膜
    \end{itemize}

    \item 术后follow-up应特别关注什么?
    \begin{itemize}
        \item 瓣膜是否充分扩张(可能需要后扩张)
        \item PVL是否进展
        \item 传导阻滞是否出现
        \item 结构性瓣膜退化(SVD)
        \item 抗凝/抗血小板策略
    \end{itemize}
\end{enumerate}

\subsubsection{Take-Home Messages}

\begin{enumerate}
    \item \textbf{极度钙化不再是TAVR的绝对禁忌证}
    \item \textbf{详细的术前CT评估是成功的关键}
    \item \textbf{新一代瓣膜显著改善了结果}
    \item \textbf{多学科团队评估和规划必不可少}
    \item \textbf{影像引导策略将"no-go"变为"go"}
    \item \textbf{选择合适的患者、合适的瓣膜、合适的技术}
\end{enumerate}

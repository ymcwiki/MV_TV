\section{TAVR患者冠状动脉血运重建的最新数据:适应症与时机}
\label{sec:11_001_coronary_revascularization}

% ============================================
% 文献信息
% ============================================
\subsection{文献信息}

\begin{itemize}
    \item \textbf{标题}: Coronary Revascularization in TAVR Patients: Latest Data on Indications and Timing
    \item \textbf{作者}: Philippe Généreux, MD
    \item \textbf{机构}: Gagnon Cardiovascular Institute at Morristown Medical Center
    \item \textbf{会议}: TCT (Transcatheter Cardiovascular Therapeutics)
    \item \textbf{PDF文件名}: coronary-revascularization-in-tavr-patients-latest-data-on-indications-and-t.pdf
    \item \textbf{文献类型}: 会议演讲/综合评述
\end{itemize}

% ============================================
% 研究背景
% ============================================
\subsection{研究背景}

\subsubsection{TAVR患者中CAD的普遍性}

冠状动脉疾病(CAD)在主动脉瓣狭窄(AS)患者中十分常见,但其患病率因\textbf{定义标准}和\textbf{研究人群}的不同而存在显著差异。

\textbf{CAD患病率的变异性(Stefanini, Eurointervention 2013)}:

\begin{itemize}
    \item \textbf{高患病率研究}(定义为任何CAD):
    \begin{itemize}
        \item PARTNER 1B:68\%
        \item PARTNER 1A:75\%
        \item CoreValve US Extreme Risk:82\%
        \item CoreValve US High Risk:75\%
        \item PARTNER 2:69\%
    \end{itemize}

    \item \textbf{低患病率研究}(定义更严格):
    \begin{itemize}
        \item SURVIV:62\%
        \item PARTNER 3:27\%
        \item Evolut Low Risk:16\%
    \end{itemize}

    \item \textbf{中等患病率研究}(登记研究):
    \begin{itemize}
        \item STACCATO TAV Registry:63\%
        \item German TAV Registry:61\%
        \item ADVANCE:57\%
        \item SOURCE:51\%
        \item FRANCE 2:48\%
        \item UK TAV Registry:44\%
    \end{itemize}
\end{itemize}

\textbf{关键观察}:
\begin{itemize}
    \item 患病率从\textbf{16\%到82\%}不等
    \item 高危/极高危患者CAD患病率更高
    \item 低危患者(如PARTNER 3、Evolut Low Risk)CAD患病率显著降低
    \item 定义标准的差异是造成报道率不同的重要原因
\end{itemize}

\subsubsection{CAD管理的核心问题}

对于合并CAD的主动脉瓣置换患者,需要回答以下关键问题:

\begin{enumerate}
    \item \textbf{是否需要治疗?}
    \begin{itemize}
        \item 患者年龄
        \item CAD的严重程度、复杂性和范围
        \item 是否引起症状或影响生存?
    \end{itemize}

    \item \textbf{如何治疗?}
    \begin{itemize}
        \item PCI vs. CABG
    \end{itemize}

    \item \textbf{何时治疗?}
    \begin{itemize}
        \item TAVR之前 vs. TAVR之后 vs. 同时进行?
    \end{itemize}
\end{enumerate}

% ============================================
% 主要研究发现
% ============================================
\subsection{主要研究发现}

\subsubsection{Meta分析:TAVR+PCI vs TAVR单独}

\textbf{Lateef等人,Am J Cardiol 2019}

\textbf{研究设计}:
\begin{itemize}
    \item Meta分析,纳入多项观察性研究
    \item 比较TAVR+PCI组(N=1194)vs TAVR单独组(N=3386)
\end{itemize}

\textbf{主要结果}(无显著差异):

\begin{table}[h]
\centering
\caption{TAVR+PCI vs TAVR单独的Meta分析结果}
\label{tab:tavr_pci_meta}
\begin{tabular}{lccc}
\toprule
\textbf{终点} & \textbf{OR (95\% CI)} & \textbf{P值} & \textbf{I²} \\
\midrule
30天全因死亡 & 1.30 (0.85-1.98) & 0.22 & 37.5\% \\
30天卒中 & 0.70 (0.36-1.45) & 0.36 & 32.8\% \\
30天心梗 & 2.71 (0.55-12.23) & 0.22 & 41.3\% \\
30天急性肾损伤 & 0.70 (0.46-1.06) & 0.08 & 14.4\% \\
\textbf{1年全因死亡} & \textbf{1.19 (0.92-1.52)} & \textbf{0.18} & \textbf{0.0\%} \\
\bottomrule
\end{tabular}
\end{table}

\textbf{结论}:在观察性研究中,TAVR前行PCI未显示出明显的临床获益。

\subsubsection{ACTIVATION试验}

\textbf{Patterson等人,J Am Coll Cardiol Intv 2021}

\textbf{研究设计}:
\begin{itemize}
    \item 前瞻性随机对照试验
    \item 入组时间:2012年12月4日至2019年1月11日
    \item 样本量:N=235(PCI组119例,无PCI组116例)
\end{itemize}

\textbf{入组标准}:
\begin{itemize}
    \item 计划行TAVR的患者
    \item 存在冠状动脉疾病
\end{itemize}

\textbf{冠脉病变分布(PCI组,N=119)}:

\begin{table}[h]
\centering
\caption{ACTIVATION试验PCI组冠脉病变分布}
\label{tab:activation_cad}
\begin{tabular}{lcc}
\toprule
\textbf{冠脉病变} & \textbf{PCI组 (n=119)} & \textbf{无PCI组 (n=116)} \\
\midrule
左前降支 >70\% & 73 (61.3\%) & 69 (60.5\%) \\
回旋支 >70\% & 42 (35.3\%) & 38 (33.3\%) \\
右冠状动脉 >70\% & 47 (39.5\%) & 59 (51.8\%) \\
左主干 >70\% & 3 (2.5\%) & 6 (5.3\%) \\
\midrule
裸金属支架植入患者 & 21 (17.6\%) & - \\
支架/病变数 & 39/194 (20\%) & - \\
\bottomrule
\end{tabular}
\end{table}

\textbf{主要终点}:1年死亡或再住院的复合终点

\textbf{主要结果}:

\begin{itemize}
    \item \textbf{无统计学差异}:绝对差异-2.5\%(上限CI 8.5\%,P=0.067)
    \item PCI组未显示死亡或再住院获益
\end{itemize}

\textbf{30天出血结果}:

\begin{table}[h]
\centering
\caption{ACTIVATION试验30天出血并发症}
\label{tab:activation_bleeding}
\begin{tabular}{lccc}
\toprule
\textbf{出血事件} & \textbf{PCI组 (n=119)} & \textbf{无PCI组 (n=116)} & \textbf{HR/P值} \\
\midrule
任何出血 & 49 (41.2\%) & 31 (26.7\%) & 1.46 (0.93-2.29); 0.098 \\
TAVR术中出血 & 8 (6.7\%) & 4 (3.4\%) & - \\
\midrule
大出血 & 31 (26.1\%) & 21 (18.1\%) & 1.23 (0.68-2.22); 0.49 \\
TAVR术中大出血 & 7 (5.8\%) & 2 (1.7\%) & - \\
\bottomrule
\end{tabular}
\end{table}

\textbf{关键发现}:
\begin{itemize}
    \item PCI组出血风险增加趋势(41.2\% vs 26.7\%)
    \item 无临床获益但增加出血风险
\end{itemize}

\subsubsection{NOTION 3试验}

\textbf{Lønborg等人,N Engl J Med 2024;391:2189-2200}

\textbf{研究设计}:
\begin{itemize}
    \item 前瞻性随机对照试验
    \item PCI组:N=227
    \item 保守治疗组:N=228
\end{itemize}

\textbf{PCI指征}:
\begin{enumerate}
    \item 所有直径狭窄(DS)\textbf{≥90\%}的病变
    \item DS <90\%但\textbf{FFR ≤0.80}的病变
\end{enumerate}

\textbf{主要终点}:死亡-心梗-紧急血运重建的复合终点

\textbf{主要结果}:

\begin{table}[h]
\centering
\caption{NOTION 3试验主要和次要终点}
\label{tab:notion3_endpoints}
\begin{tabular}{lccc}
\toprule
\textbf{终点} & \textbf{PCI组 (N=227)} & \textbf{保守治疗组 (N=228)} & \textbf{HR (95\% CI); P值} \\
\midrule
\textbf{主要终点:MACE†} & \textbf{60 (26\%)} & \textbf{81 (36\%)} & \textbf{0.71 (0.51-0.99); 0.04} \\
\midrule
\multicolumn{4}{l}{\textit{次要终点:}} \\
全因死亡 & 53 (23\%) & 62 (27\%) & 0.85 (0.59-1.23) \\
心肌梗死‡ & 17 (7\%) & 31 (14\%) & 0.54 (0.30-0.97) \\
紧急血运重建¶ & 5 (2\%) & 22 (11\%) & 0.20 (0.08-0.51) \\
心血管死亡¶ & 20 (9\%) & 30 (13\%) & 0.67 (0.38-1.19) \\
\midrule
任何血运重建 & 6 (3\%) & 44 (21\%) & 0.12 (0.05-0.27) \\
卒中¶ & 21 (10\%) & 35 (15\%) & 0.67 (0.39-1.14) \\
\midrule
\multicolumn{4}{l}{\textit{安全性终点:}} \\
任何出血事件§ & 64 (28\%) & 45 (20\%) & 1.51 (1.03-2.22) \\
致命/致残性出血 & 23 (10\%) & 16 (7\%) & - \\
大出血 & 26 (11\%) & 22 (10\%) & - \\
轻微出血 & 53 (23\%) & 36 (16\%) & - \\
\midrule
支架血栓形成 & 1 (<1\%) & 2 (1\%) & 0.45 (0.23-0.89) \\
急性肾损伤 & 12 (5\%) & 26 (11\%) & 0.45 (0.23-0.89) \\
\bottomrule
\end{tabular}
\end{table}

\textit{† MACE定义为全因死亡、心肌梗死或紧急血运重建的复合终点}

\textit{‡ 心肌梗死包括TAVR后<72小时、PCI后<48小时发生的围手术期心梗}

\textit{¶ 紧急血运重建定义为因急性冠脉综合征(心梗或不稳定型心绞痛)导致的非计划住院血运重建}

\textit{§ 出血按照VARC-2标准记录}

\textbf{分组分析结果}:

\begin{table}[h]
\centering
\caption{NOTION 3试验按狭窄程度分组分析}
\label{tab:notion3_subgroup}
\begin{tabular}{lcccc}
\toprule
\textbf{亚组} & \textbf{PCI组} & \textbf{保守治疗组} & \textbf{HR (95\% CI)} \\
 & \textbf{事件数/总数 (\%)} & \textbf{事件数/总数 (\%)} & \\
\midrule
\multicolumn{4}{l}{\textbf{按狭窄直径分层:}} \\
<90\% & 27/88 (31\%) & 32/96 (33\%) & 1.04 (0.62-1.73) \\
≥90\% & 33/139 (24\%) & 49/132 (37\%) & 0.53 (0.34-0.82) \\
\midrule
\multicolumn{4}{l}{\textbf{按年龄分层:}} \\
<82岁 & 21/106 (20\%) & 40/117 (34\%) & 0.56 (0.33-0.95) \\
≥82岁 & 39/121 (32\%) & 41/111 (37\%) & 0.81 (0.52-1.26) \\
\midrule
\multicolumn{4}{l}{\textbf{按糖尿病分层:}} \\
无 & 40/168 (24\%) & 55/167 (33\%) & 0.67 (0.45-1.01) \\
有 & 20/59 (34\%) & 26/61 (43\%) & 0.78 (0.44-1.42) \\
\midrule
\multicolumn{4}{l}{\textbf{按SYNTAX评分分层:}} \\
≤9 & 33/109 (30\%) & 39/106 (37\%) & 0.74 (0.47-1.18) \\
>9 & 27/118 (23\%) & 42/122 (34\%) & 0.66 (0.41-1.07) \\
\bottomrule
\end{tabular}
\end{table}

\textbf{重要局限性}:

\begin{center}
\fbox{\parbox{0.9\textwidth}{
\textbf{关键限制}:FFR在一般情况下仅在\textbf{<35\%}的测试病变中≤0.80。因此,该试验策略实际上仅在\textbf{少量DS<90\%的病变}中得到测试。基于如此小的测试病变样本量得出"DS<90\%病变无差异"的结论可能具有\textbf{误导性}。
}}
\end{center}

\subsubsection{TCW试验(重磅研究)}

\textbf{Kedhi等人,Lancet 2025;404(10471):2593-2602}

\textbf{研究设计}:
\begin{itemize}
    \item 国际多中心前瞻性随机对照试验
    \item 入组患者:≥70岁,重度AS,≥2支血管病变或复杂LAD病变
    \item 经心脏团队讨论后随机分组
    \item 试验组(N=164):FFR指导PCI + TAVI(所有FFR≤0.80病变均行PCI)
    \item 对照组(N=164):CABG + SAVR
    \item 随访评估心绞痛症状:若已知FFR≤0.85患者仍有持续心绞痛,且复查FFR≤0.80,可行PCI
\end{itemize}

\textbf{实际入组和完成情况}:

\begin{table}[h]
\centering
\caption{TCW试验患者流程图}
\label{tab:tcw_flowchart}
\begin{tabular}{lcc}
\toprule
\textbf{分组} & \textbf{TAVI+FFR指导PCI} & \textbf{SAVR+CABG} \\
\midrule
入组患者 & 91 & 81 \\
\midrule
\multicolumn{3}{l}{\textit{术前退出:}} \\
术前死亡 & - & 4 \\
交叉至对侧组 & 1 (接受SAVR+CABG) & 7 (交叉至PCI+TAVI) \\
撤回同意 & - & 3 \\
医师决定终止 & - & 1 \\
\midrule
实际接受指定治疗 & 89 (TAVI+FFR指导PCI) & 64 (SAVR+CABG) \\
仅接受部分治疗 & 2 (1例仅PCI,1例仅TAVI) & - \\
\midrule
\textbf{ITT分析} & \textbf{91} & \textbf{77} \\
\bottomrule
\end{tabular}
\end{table}

\textbf{主要终点}:1年复合终点
\begin{itemize}
    \item 全因死亡
    \item 心肌梗死
    \item 致残性卒中
    \item 非计划的临床驱动的靶血管血运重建
    \item 瓣膜再干预
    \item 致命性或致残性出血
\end{itemize}

\textbf{30天结果}:

\begin{table}[h]
\centering
\caption{TCW试验30天临床结果}
\label{tab:tcw_30day}
\begin{tabular}{lccccc}
\toprule
\textbf{终点} & \textbf{TAVI+PCI} & \textbf{SAVR+CABG} & \textbf{HR (95\% CI)} & \textbf{P值} \\
 & \textbf{(n=91)} & \textbf{(n=77)} & & \\
\midrule
\textbf{主要终点} & \textbf{1 (1.10\%)} & \textbf{11 (14.42\%)} & \textbf{0.07 (0.01-0.55)} & \textbf{0.001} \\
\midrule
\multicolumn{5}{l}{\textit{次要终点:}} \\
MACE & 1 (1.10\%) & 4 (5.23\%) & 0.20 (0.02-1.82) & 0.16 \\
全因死亡+卒中 & 1 (1.10\%) & 3 (3.93\%) & 0.27 (0.03-2.62) & 0.23 \\
\midrule
全因死亡 & 0 (0\%) & 1 (1.30\%) & - & 0.28 \\
心血管死亡 & 0 (0\%) & 1 (1.30\%) & - & 0.28 \\
卒中或TIA & 0 (0\%) & 2 (2.70\%) & - & 0.12 \\
\quad 致残性卒中 & 0 (0\%) & 1 (1.35\%) & - & 0.28 \\
\quad 非致残性卒中 & 0 (0\%) & 0 (0\%) & - & - \\
\quad TIA & 0 (0\%) & 1 (1.35\%) & - & 0.28 \\
心肌梗死(任何) & 1 (1.10\%) & 1 (1.30\%) & 0.82 (0.05-13.18) & 0.89 \\
\quad 围手术期心梗 & 1 (1.10\%) & 1 (1.30\%) & 0.82 (0.05-13.18) & 0.89 \\
\quad 自发性心梗 & 0 (0\%) & 0 (0\%) & - & - \\
任何血运重建 & 0 (0\%) & 1 (1.30\%) & - & 0.28 \\
CD-TVR & 0 (0\%) & 1 (1.30\%) & - & 0.28 \\
瓣膜再干预 & 0 (0\%) & 1 (1.30\%) & - & 0.28 \\
\midrule
\multicolumn{5}{l}{\textit{安全性终点:}} \\
\textbf{致命/致残性出血} & \textbf{1 (1.10\%)} & \textbf{7 (9.24\%)} & \textbf{0.11 (0.01-0.93)} & \textbf{0.01} \\
大出血(VARC-2) & 2 (2.20\%) & 6 (7.79\%) & 0.28 (0.06-1.39) & 0.09 \\
轻微出血(VARC-2) & 8 (8.79\%) & 3 (3.90\%) & 2.24 (0.59-8.44) & 0.22 \\
永久起搏器植入 & 8 (8.79\%) & 1 (1.33\%) & 6.74 (0.84-53.88) & 0.04 \\
大血管并发症 & 4 (4.40\%) & 1 (1.35\%) & 3.36 (0.38-30.09) & 0.25 \\
\textbf{再次开胸} & \textbf{0 (0\%)} & \textbf{4 (5.19\%)} & \textbf{-} & \textbf{0.03} \\
\textbf{房颤} & \textbf{2 (2.20\%)} & \textbf{10 (13.05\%)} & \textbf{0.16 (0.03-0.72)} & \textbf{0.006} \\
\bottomrule
\end{tabular}
\end{table}

\textbf{1年(365天)结果}:

\begin{table}[h]
\centering
\caption{TCW试验1年临床结果(ITT分析)}
\label{tab:tcw_1year}
\begin{tabular}{lccccc}
\toprule
\textbf{终点} & \textbf{TAVI+PCI} & \textbf{SAVR+CABG} & \textbf{HR (95\% CI)} & \textbf{P值} \\
 & \textbf{(n=91)} & \textbf{(n=77)} & & \\
\midrule
\textbf{主要终点} & \textbf{4 (4.4\%)} & \textbf{18 (22.9\%)} & \textbf{0.17 (0.06-0.51)} & \textbf{<0.001} \\
\midrule
\textbf{全因死亡} & \textbf{0 (0\%)} & \textbf{7 (9.74\%)} & \textbf{-} & \textbf{0.002} \\
\textbf{心血管死亡} & \textbf{0 (0\%)} & \textbf{6 (8.35\%)} & \textbf{-} & \textbf{0.005} \\
全部卒中和TIA & 1 (1.11\%) & 3 (4.20\%) & 0.25 (0.03-2.45) & 0.20 \\
\quad 致残性卒中 & 1 (1.11\%) & 2 (2.85\%) & 0.38 (0.03-4.19) & 0.41 \\
\quad TIA & 0 (0\%) & 1 (1.35\%) & - & 0.27 \\
心肌梗死(任何) & 2 (2.21\%) & 1 (1.30\%) & 1.58 (0.14-17.48) & 0.71 \\
\quad 围手术期心梗 & 1 (1.10\%) & 1 (1.30\%) & 0.82 (0.05-13.18) & 0.89 \\
\quad 自发性心梗 & 1 (1.11\%) & 0 (0\%) & - & 0.40 \\
CD-TVR & 0 (0\%) & 1 (1.30\%) & - & 0.28 \\
瓣膜再干预 & 0 (0\%) & 1 (1.30\%) & - & 0.28 \\
\midrule
\multicolumn{5}{l}{\textit{安全性终点:}} \\
\textbf{致命/致残性出血} & \textbf{2 (2.21\%)} & \textbf{9 (12.10\%)} & \textbf{0.17 (0.04-0.80)} & \textbf{0.01} \\
大出血(VARC-2) & 5 (5.56\%) & 7 (9.21\%) & 0.57 (0.18-1.79) & 0.32 \\
轻微出血(VARC-2) & 12 (13.27\%) & 4 (5.40\%) & 2.52 (0.81-7.81) & 0.10 \\
永久起搏器植入 & 9 (9.89\%) & 2 (2.87\%) & 3.74 (0.81-17.30) & 0.07 \\
大血管并发症 & 4 (4.40\%) & 1 (1.35\%) & 3.36 (0.38-30.09) & 0.25 \\
再次开胸 & 0 (0\%) & 4 (5.19\%) & - & 0.02 \\
房颤 & 2 (2.20\%) & 11 (13.05\%) & 0.28 (0.09-0.88) & 0.03 \\
\bottomrule
\end{tabular}
\end{table}

\textbf{TCW试验的突破性发现}:

\begin{center}
\fbox{\parbox{0.9\textwidth}{
\textbf{卓越结果}:与SAVR+CABG相比,FFR指导的PCI+TAVI策略显著降低:
\begin{itemize}
    \item 1年主要复合终点:\textbf{4.4\% vs 22.9\%}(HR 0.17,P<0.001)
    \item 1年全因死亡:\textbf{0\% vs 9.74\%}(P=0.002)
    \item 1年心血管死亡:\textbf{0\% vs 8.35\%}(P=0.005)
    \item 全因死亡+卒中:\textbf{1.1\% vs 12.5\%}(HR 0.08,P=0.003)
    \item 致命/致残性出血:\textbf{2.21\% vs 12.10\%}(HR 0.17,P=0.01)
    \item 房颤:\textbf{2.20\% vs 13.05\%}(P=0.03)
    \item 再次开胸:\textbf{0\% vs 5.19\%}(P=0.02)
\end{itemize}
}}
\end{center}

\textbf{重要提示}:

\begin{itemize}
    \item 样本量有限(N=172),需要更大规模的随机试验证实
    \item 存在显著的交叉(7名SAVR+CABG组患者交叉至TAVI+PCI组)
    \item 术前死亡4例(均为SAVR+CABG组)
\end{itemize}

% ============================================
% 临床启示
% ============================================
\subsection{临床启示}

\subsubsection{ESC/EACTS 2021指南建议}

\textbf{心肌血运重建的适应症}:

\begin{table}[h]
\centering
\caption{ESC/EACTS 2021瓣膜性心脏病指南:血运重建建议}
\label{tab:esc_guidelines}
\begin{tabular}{lcc}
\toprule
\textbf{临床情景} & \textbf{推荐等级} & \textbf{证据等级} \\
\midrule
\textbf{CABG推荐用于:} & & \\
主动脉瓣/二尖瓣/三尖瓣手术合并 & \textbf{I} & \textbf{C} \\
冠脉直径狭窄≥70\% & & \\
\midrule
\textbf{应考虑CABG用于:} & & \\
主动脉瓣/二尖瓣/三尖瓣手术合并 & \textbf{IIa} & \textbf{C} \\
冠脉直径狭窄≥50-70\% & & \\
\midrule
\textbf{应考虑PCI用于:} & & \\
计划行TAVR且合并 & \textbf{IIa} & \textbf{C} \\
近段冠脉直径狭窄>70\%的患者 & & \\
\bottomrule
\end{tabular}
\end{table}

\textbf{关键要点}:
\begin{itemize}
    \item 对于外科瓣膜置换,≥70\%狭窄建议CABG(I类,C级)
    \item 对于外科瓣膜置换,50-70\%狭窄应考虑CABG(IIa类,C级)
    \item 对于TAVR,近段>70\%狭窄应考虑PCI(IIa类,C级)
\end{itemize}

\subsubsection{PCI时机的考虑}

\textbf{三种时机策略的优缺点}:

\begin{table}[h]
\centering
\caption{PCI时机选择的优缺点比较}
\label{tab:pci_timing}
\begin{tabular}{p{3cm}p{5cm}p{5cm}}
\toprule
\textbf{时机} & \textbf{优势} & \textbf{劣势} \\
\midrule
\textbf{TAVI前PCI} &
• 冠脉通路更容易(特别是自扩张THV瓣上瓣叶位置)
\newline • 降低缺血导致的血流动力学不稳定风险(如快速起搏时)
\newline • 相比同时手术减少对比剂用量 &
• 交界病变FFR/iFR评估不可靠
\newline • 因AS导致的血流动力学不稳定风险更高 \\
\midrule
\textbf{TAVI后PCI} &
• 中度病变FFR/iFR评估更可靠
\newline • 复杂PCI期间血流动力学不稳定风险降低(如旋磨和左室功能受损)
\newline • 相比同时手术减少对比剂用量 &
• 冠脉通路更具挑战性且可能受损
\newline • 冠脉导丝支撑稳定性降低
\newline • 潜在的THV移位风险 \\
\midrule
\textbf{同时手术} &
• 使用相同的动脉通路
\newline • 降低成本 &
• 对比剂用量更大,AKI风险更高
\newline • 手术时间更长
\newline • TAVI时需要DAPT,因此出血风险增加 \\
\bottomrule
\end{tabular}
\end{table}

\textbf{AS:主动脉瓣狭窄;AKI:急性肾损伤;DAPT:双联抗血小板治疗;FFR:血流储备分数;iFR:瞬时无波形比值;LV:左心室;PCI:经皮冠状动脉介入;TAVI:经导管主动脉瓣植入;THV:经导管心脏瓣膜}

\subsubsection{正在进行的临床试验}

\textbf{1. TAVI-PCI试验}

\begin{itemize}
    \item \textbf{入组标准}:≥1处直径狭窄≥70\%的病变,可在TAVI前后45天内行PCI
    \item \textbf{试验组}:PCI+OMT \textbf{先于}TAVI
    \item \textbf{对照组}:\textbf{成功}TAVI后行PCI+OMT
    \item \textbf{主要终点}:复合终点
    \begin{itemize}
        \item 全因死亡
        \item 非致死性心肌梗死
        \item 缺血驱动的血运重建
        \item 再住院(瓣膜或手术相关,包括心衰)
        \item 致命性/致残性或大出血(VARC-2标准)
    \end{itemize}
    \item \textbf{试验设计}:优效性检验
    \item \textbf{主要研究者}:B. Stähli,苏黎世,瑞士
\end{itemize}

\textbf{2. FAITAVI试验}

\begin{itemize}
    \item \textbf{试验组}:\textbf{FFR指导PCI}(仅FFR≤0.80的病变行PCI)
    \item \textbf{对照组}:\textbf{造影指导PCI}(所有>50\%直径狭窄的病变均行PCI)
    \item \textbf{主要终点}:复合终点
    \begin{itemize}
        \item 全因死亡
        \item 心肌梗死
        \item 卒中
        \item 大出血
        \item 靶血管血运重建需求
    \end{itemize}
    \item \textbf{主要研究者}:Flavio Ribichini,维罗纳,意大利
\end{itemize}

% ============================================
% 结论
% ============================================
\subsection{结论}

\subsubsection{CAD和主动脉瓣狭窄的管理结论}

基于目前的证据,可以得出以下结论:

\begin{enumerate}
    \item \textbf{血运重建的必要性}:
    \begin{itemize}
        \item CAD血运重建\textbf{可能需要},特别是对于合并严重(真正缺血性)CAD的年轻患者
        \item 是否治疗取决于年龄、CAD严重程度和范围、症状及对生存的影响
    \end{itemize}

    \item \textbf{治疗策略选择}:
    \begin{itemize}
        \item 对于\textbf{≥70岁}患者,\textbf{TAVI+PCI}可能是治疗此类患者的最佳方式
        \item TCW试验显示FFR指导的PCI+TAVI相比CABG+SAVR有显著优势
        \item 需要大规模试验进一步验证
    \end{itemize}

    \item \textbf{时机选择}:
    \begin{itemize}
        \item TAVI前还是TAVI后行PCI仍存在争议
        \item 取决于多种因素(冠脉解剖、病变复杂程度、血流动力学状态等)
        \item 需要个体化决策
    \end{itemize}

    \item \textbf{缺血评估的作用}:
    \begin{itemize}
        \item FFR在指导血运重建中的作用仍需明确
        \item NOTION 3试验显示FFR指导策略的获益,但存在局限性
        \item FAITAVI试验将比较FFR指导vs造影指导策略
    \end{itemize}

    \item \textbf{未来研究方向}:
    \begin{itemize}
        \item 需要完成"COMPLETE"式的大规模随机试验
        \item TAVI-PCI试验将明确PCI的最佳时机
        \item 需要更多关于FFR在AS患者中应用的数据
    \end{itemize}
\end{enumerate}

\subsubsection{当前实践建议}

\begin{center}
\fbox{\parbox{0.9\textwidth}{
\textbf{基于现有证据的临床实践建议}:

\begin{itemize}
    \item 对于≥70岁、合并重度AS和复杂CAD的患者,应优先考虑\textbf{TAVI+FFR指导的PCI}策略
    \item 对于<70岁患者,需要心脏团队讨论决定TAVI+PCI vs SAVR+CABG
    \item \textbf{常规PCI}(针对所有CAD患者)\textbf{不推荐},应基于缺血评估或严重狭窄(≥90\%或FFR≤0.80)
    \item PCI时机应根据冠脉解剖、病变复杂程度、血流动力学状态个体化决定
    \item 等待正在进行的随机试验结果以获得更明确的指导
\end{itemize}
}}
\end{center}

% ============================================
% 研究局限性
% ============================================
\subsection{研究局限性}

\subsubsection{现有证据的局限性}

\textbf{1. Meta分析和观察性研究}:
\begin{itemize}
    \item 大多数早期证据来自观察性研究和meta分析
    \item 存在选择偏倚和混杂因素
    \item 缺乏标准化的PCI适应症和策略
\end{itemize}

\textbf{2. ACTIVATION试验}:
\begin{itemize}
    \item 样本量相对较小(N=235)
    \item 未显示PCI明确获益
    \item 出血风险增加
    \item 可能纳入了不需要PCI的患者
\end{itemize}

\textbf{3. NOTION 3试验}:
\begin{itemize}
    \item \textbf{关键局限}:FFR在<90\%狭窄病变中的应用有限
    \begin{itemize}
        \item FFR一般仅在35\%的测试病变中≤0.80
        \item 实际测试的DS<90\%病变数量很少
        \item 对中度狭窄病变的结论可能不可靠
    \end{itemize}
    \item 主要获益可能来自≥90\%狭窄病变的治疗
    \item 缺乏完整的缺血评估
\end{itemize}

\textbf{4. TCW试验}:
\begin{itemize}
    \item \textbf{样本量有限}(N=172),这是最重要的局限性
    \item 显著的交叉(7例SAVR+CABG组交叉至TAVI+PCI组)
    \item 术前死亡全部发生在SAVR+CABG组(4例)
    \item 单中心或少中心研究,外推性有限
    \item 需要大规模多中心试验验证
    \item 缺乏长期随访数据
\end{itemize}

\textbf{5. PCI时机研究}:
\begin{itemize}
    \item 缺乏直接比较TAVI前vs TAVI后PCI的随机试验
    \item 大多数建议基于专家意见和理论考虑
    \item TAVI-PCI试验结果尚未公布
\end{itemize}

\textbf{6. FFR在AS患者中的应用}:
\begin{itemize}
    \item AS对冠脉血流动力学的影响可能影响FFR准确性
    \item 缺乏AS患者中FFR cutoff值验证的数据
    \item TAVI前后FFR值的变化及其临床意义尚不明确
\end{itemize}

% ============================================
% 个人笔记
% ============================================
\subsection{个人笔记}

\subsubsection{关键数字记忆}

\textbf{CAD患病率范围}:
\begin{itemize}
    \item 最低:\textbf{16\%}(Evolut Low Risk)
    \item 最高:\textbf{82\%}(CoreValve US Extreme Risk)
    \item 一般范围:\textbf{40-70\%}
\end{itemize}

\textbf{ACTIVATION试验(N=235)}:
\begin{itemize}
    \item PCI组:119例
    \item 无PCI组:116例
    \item 1年主要终点:无显著差异(P=0.067)
    \item 30天出血:41.2\% vs 26.7\%
\end{itemize}

\textbf{NOTION 3试验(N=455)}:
\begin{itemize}
    \item PCI组:227例
    \item 保守治疗组:228例
    \item 主要终点(MACE):26\% vs 36\%(HR 0.71,P=0.04)
    \item 心梗:7\% vs 14\%(HR 0.54)
    \item 紧急血运重建:2\% vs 11\%(HR 0.20)
    \item 出血:28\% vs 20\%(HR 1.51)
\end{itemize}

\textbf{TCW试验(N=172)}:
\begin{itemize}
    \item TAVI+PCI:91例
    \item SAVR+CABG:81例
    \item \textbf{1年主要终点}:\textbf{4.4\% vs 22.9\%}(HR 0.17,P<0.001)
    \item \textbf{1年全因死亡}:\textbf{0\% vs 9.74\%}(P=0.002)
    \item \textbf{30天主要终点}:\textbf{1.10\% vs 14.42\%}(P=0.001)
    \item 致命/致残性出血(30天):1.10\% vs 9.24\%(P=0.01)
    \item 房颤(1年):2.20\% vs 13.05\%(P=0.03)
\end{itemize}

\subsubsection{重要概念与机制}

\begin{description}
    \item[CAD在AS患者中的患病率] 冠状动脉疾病在主动脉瓣狭窄患者中非常常见,但患病率因定义标准和研究人群的不同而有显著差异(16-82\%)。高危患者CAD患病率更高,而低危年轻患者相对较低。

    \item[FFR指导的血运重建] 血流储备分数(FFR)≤0.80被认为是血流动力学显著狭窄的标准。在AS患者中,FFR评估可能受到AS本身对冠脉血流动力学的影响,特别是在TAVI前评估交界病变时可靠性较低。

    \item[TAVI vs SAVR+CABG的选择] 对于≥70岁、合并重度AS和复杂CAD的患者,TCW试验显示TAVI+FFR指导的PCI策略显著优于SAVR+CABG,主要体现在降低死亡率、出血和房颤风险。但该研究样本量有限,需要更大规模试验验证。

    \item[PCI时机的选择] TAVI前、TAVI后或同时进行PCI各有优劣:
    \begin{itemize}
        \item \textbf{TAVI前}:冠脉通路更容易,降低缺血风险,但交界病变FFR评估不可靠
        \item \textbf{TAVI后}:FFR评估更可靠,血流动力学更稳定,但冠脉通路更困难
        \item \textbf{同时}:使用相同通路、降低成本,但对比剂用量大、出血风险高
    \end{itemize}

    \item[MACE (Major Adverse Cardiac Events)] 主要不良心血管事件,通常包括死亡、心肌梗死和血运重建。在不同研究中定义可能略有差异。

    \item[VARC-2出血标准] Valve Academic Research Consortium-2标准,用于TAVR研究中标准化出血事件的定义和分级(致命/致残性、大出血、轻微出血)。

    \item[FFR的局限性] 在NOTION 3试验中,FFR在一般情况下仅在约35\%的测试病变中≤0.80,这意味着大多数中度狭窄(<90\%)病变FFR实际上是正常的。这限制了试验对中度狭窄病变治疗策略的评估能力。

    \item[心脏团队决策] 对于合并AS和CAD的复杂患者,应由包括介入心脏病学家、心脏外科医生、影像专家等组成的多学科团队共同讨论决定最佳治疗策略。
\end{description}

\subsubsection{临床决策要点}

\textbf{何时考虑血运重建}:
\begin{enumerate}
    \item \textbf{严重狭窄}:直径狭窄≥90\%或FFR≤0.80
    \item \textbf{症状性CAD}:心绞痛或缺血证据
    \item \textbf{近段病变}:近段大血管病变(如左主干、LAD近段)
    \item \textbf{年轻患者}:预期寿命长,需要完全血运重建
    \item \textbf{复杂多支病变}:特别是合并左主干或三支病变
\end{enumerate}

\textbf{TAVI+PCI vs SAVR+CABG的选择}:
\begin{itemize}
    \item \textbf{TAVI+PCI优先}:
    \begin{itemize}
        \item ≥70岁
        \item 手术高危
        \item 冠脉病变适合PCI(非弥漫性、非严重钙化、非CTO)
        \item 患者偏好微创治疗
    \end{itemize}

    \item \textbf{SAVR+CABG优先}:
    \begin{itemize}
        \item <70岁且手术风险可接受
        \item 复杂CAD不适合PCI(左主干、弥漫性病变、CTO)
        \item 合并其他需要手术治疗的心脏疾病
        \item 二叶主动脉瓣等TAVR不适合的解剖
    \end{itemize}

    \item \textbf{需要心脏团队讨论}的情况:
    \begin{itemize}
        \item 年龄60-75岁
        \item 复杂CAD但PCI技术上可行
        \item 合并症较多但手术风险中等
    \end{itemize}
\end{itemize}

\textbf{PCI时机选择建议}:
\begin{itemize}
    \item \textbf{TAVI前PCI}:
    \begin{itemize}
        \item 严重狭窄(≥90\%)明确需要治疗
        \item 左主干或LAD近段严重病变
        \item 血流动力学不稳定高风险
        \item 计划使用自扩张瓣膜(瓣上瓣叶可能影响冠脉通路)
    \end{itemize}

    \item \textbf{TAVI后PCI}:
    \begin{itemize}
        \item 中度狭窄(50-90\%)需要FFR评估
        \item 交界病变需要可靠的缺血评估
        \item 复杂PCI(旋磨、分叉等)
        \item 左室功能明显受损
    \end{itemize}

    \item \textbf{同时手术}:
    \begin{itemize}
        \item 简单病变
        \item 减少手术次数和住院时间
        \item 降低总体成本
        \item 需注意对比剂用量和出血风险
    \end{itemize}
\end{itemize}

\subsubsection{与其他研究的比较}

\textbf{本综述的独特贡献}:

\begin{enumerate}
    \item \textbf{整合了最新的RCT证据}:
    \begin{itemize}
        \item ACTIVATION(2021)
        \item NOTION 3(2024)
        \item TCW(2025)
    \end{itemize}

    \item \textbf{提供了不同策略的系统性比较}:
    \begin{itemize}
        \item PCI vs 保守治疗
        \item FFR指导vs造影指导
        \item TAVI+PCI vs SAVR+CABG
        \item 不同PCI时机的优劣
    \end{itemize}

    \item \textbf{纳入了指南建议}:
    \begin{itemize}
        \item ESC/EACTS 2021指南
        \item 基于证据等级的推荐
    \end{itemize}

    \item \textbf{介绍了正在进行的试验}:
    \begin{itemize}
        \item TAVI-PCI(PCI时机)
        \item FAITAVI(FFR指导vs造影指导)
    \end{itemize}
\end{enumerate}

\textbf{与既往观念的变化}:

\begin{itemize}
    \item \textbf{过去}:认为所有TAVR患者合并CAD都应行PCI
    \item \textbf{现在}:
    \begin{itemize}
        \item 常规PCI不推荐(ACTIVATION试验)
        \item 应基于缺血评估或严重狭窄选择性PCI(NOTION 3)
        \item FFR指导可能优于单纯造影指导
    \end{itemize}

    \item \textbf{过去}:年龄>70岁合并复杂CAD倾向SAVR+CABG
    \item \textbf{现在}:
    \begin{itemize}
        \item TCW试验显示TAVI+PCI可能优于SAVR+CABG
        \item 但需要更大规模试验验证
        \item 个体化决策仍然重要
    \end{itemize}
\end{itemize}

\subsubsection{对未来研究的建议}

\begin{enumerate}
    \item \textbf{大规模RCT}:
    \begin{itemize}
        \item 需要类似"COMPLETE"的大规模试验验证TCW试验结果
        \item 明确FFR指导vs造影指导的优劣
        \item 比较不同PCI时机的临床结果
    \end{itemize}

    \item \textbf{FFR在AS患者中的验证}:
    \begin{itemize}
        \item AS对FFR值的影响
        \item TAVI前后FFR值的变化
        \item AS患者中FFR最佳cutoff值
        \item 其他缺血评估方法(iFR、QFR等)的应用
    \end{itemize}

    \item \textbf{长期随访}:
    \begin{itemize}
        \item 目前大多数试验随访时间≤1年
        \item 需要3-5年甚至更长期的随访
        \item 评估瓣膜耐久性和冠脉病变进展
    \end{itemize}

    \item \textbf{亚组分析}:
    \begin{itemize}
        \item 不同年龄组(<70岁、70-80岁、>80岁)
        \item 不同CAD复杂程度(SYNTAX评分)
        \item 不同AS严重程度
        \item 合并症负担(糖尿病、肾功能不全等)
    \end{itemize}

    \item \textbf{成本效益分析}:
    \begin{itemize}
        \item TAVI+PCI vs SAVR+CABG的卫生经济学评估
        \item 不同PCI时机的成本比较
        \item FFR指导策略的成本效益
    \end{itemize}

    \item \textbf{技术改进}:
    \begin{itemize}
        \item 新一代TAVR瓣膜对冠脉通路的影响
        \item 冠脉成像技术(OCT、IVUS)在TAVR前后的应用
        \item 生理学评估新技术
    \end{itemize}
\end{enumerate}

\subsubsection{记忆口诀}

\textbf{CAD患病率"16-82"规律}:
\begin{itemize}
    \item 最低\textbf{16\%}(低危年轻患者)
    \item 最高\textbf{82\%}(极高危患者)
\end{itemize}

\textbf{NOTION 3 "FFR 35\%"局限}:
\begin{itemize}
    \item FFR阳性率(≤0.80)仅约\textbf{35\%}
    \item 中度狭窄病变测试样本量小
\end{itemize}

\textbf{TCW试验"4-23"优势}:
\begin{itemize}
    \item 1年主要终点:\textbf{4.4\%} vs \textbf{22.9\%}
    \item 绝对差异约\textbf{18\%}
\end{itemize}

\textbf{TCW试验"0-10"死亡率}:
\begin{itemize}
    \item TAVI+PCI:\textbf{0\%}死亡
    \item SAVR+CABG:约\textbf{10\%}死亡
\end{itemize}

\textbf{PCI指征"90-80"标准}:
\begin{itemize}
    \item 直径狭窄≥\textbf{90\%}:直接PCI
    \item FFR≤\textbf{0.80}:需要PCI
\end{itemize}

\textbf{ESC指南"70-50"推荐}:
\begin{itemize}
    \item ≥\textbf{70\%}狭窄:I类推荐(外科)
    \item \textbf{50-70\%}狭窄:IIa类推荐(外科)
    \item >\textbf{70\%}近段狭窄:IIa类推荐PCI(TAVR患者)
\end{itemize}

\subsubsection{值得深入思考的问题}

\begin{enumerate}
    \item \textbf{为什么TAVI+PCI的死亡率如此之低?}
    \begin{itemize}
        \item TCW试验中TAVI+PCI组1年\textbf{0死亡}是惊人的结果
        \item 可能原因:
        \begin{itemize}
            \item 微创手术减少手术创伤和并发症
            \item 避免了体外循环相关的炎症和器官损伤
            \item FFR指导确保了精准的血运重建
            \item 患者选择(适合TAVI的解剖和风险)
        \end{itemize}
        \item 但需要注意样本量小(N=91),需大规模试验验证
    \end{itemize}

    \item \textbf{为什么ACTIVATION和NOTION 3结果不一致?}
    \begin{itemize}
        \item ACTIVATION:PCI无获益
        \item NOTION 3:PCI有获益(HR 0.71)
        \item 可能原因:
        \begin{itemize}
            \item ACTIVATION可能纳入了不需要PCI的患者(缺乏严格的缺血或严重狭窄标准)
            \item NOTION 3使用FFR指导,更精准地识别需要治疗的病变
            \item 患者人群差异(CAD严重程度、AS严重程度)
            \item 随访时间和终点定义不同
        \end{itemize}
    \end{itemize}

    \item \textbf{FFR在AS患者中真的可靠吗?}
    \begin{itemize}
        \item 理论上AS会影响冠脉血流动力学:
        \begin{itemize}
            \item 增加左室压力和心肌耗氧
            \item 可能影响冠脉灌注压
            \item 左室肥厚增加微血管阻力
        \end{itemize}
        \item 但目前证据有限:
        \begin{itemize}
            \item 缺乏AS患者中FFR与预后关系的验证
            \item TAVI前后FFR值变化及其意义不清楚
            \item 是否需要AS特异的FFR cutoff值?
        \end{itemize}
        \item 可能替代方案:
        \begin{itemize}
            \item iFR(不需要腺苷,可能更适合AS患者)
            \item 影像学评估(IVUS、OCT)
            \item QFR等非侵入性方法
        \end{itemize}
    \end{itemize}

    \item \textbf{TAVI后PCI是否增加THV移位风险?}
    \begin{itemize}
        \item 理论担忧:
        \begin{itemize}
            \item 导管操作可能推动或拉动THV
            \item 特别是通过THV瓣叶时
            \item 支架释放产生的力可能影响THV位置
        \end{itemize}
        \item 但实际报道少见:
        \begin{itemize}
            \item 大多数新一代THV锚定良好
            \item 操作者经验和技巧很重要
            \item 可能需要特殊技术(如支撑导管)
        \end{itemize}
        \item 需要更多数据
    \end{itemize}

    \item \textbf{如何定义"完全血运重建"在TAVR患者中的意义?}
    \begin{itemize}
        \item 在ACS患者中,COMPLETE试验显示完全血运重建优于罪犯病变PCI
        \item 但TAVR患者不同:
        \begin{itemize}
            \item 年龄更大、预期寿命可能更短
            \item 合并症更多
            \item AS本身对症状和预后的影响更大
        \end{itemize}
        \item 可能需要"AS患者的COMPLETE试验"来回答这个问题
        \item 完全vs罪犯病变(或FFR阳性病变)血运重建的比较
    \end{itemize}

    \item \textbf{为什么SAVR+CABG组房颤发生率如此高?}
    \begin{itemize}
        \item TCW试验:13.05\% vs 2.20\%(P=0.03)
        \item 已知原因:
        \begin{itemize}
            \item 体外循环引起的炎症反应
            \item 心房操作和损伤
            \item 电解质紊乱
            \item 心包积液和炎症
        \end{itemize}
        \item 临床意义:
        \begin{itemize}
            \item 房颤增加卒中风险
            \item 需要抗凝治疗
            \item 可能影响长期预后
        \end{itemize}
        \item 这是TAVI相对于SAVR的重要优势之一
    \end{itemize}
\end{enumerate}

\subsubsection{实用技巧总结}

\textbf{TAVR+CAD患者评估"五步法"}:
\begin{enumerate}
    \item \textbf{第一步}:明确AS严重程度和TAVR适应症
    \item \textbf{第二步}:评估CAD范围和严重程度(造影)
    \item \textbf{第三步}:缺血评估(FFR、负荷试验、核素显像等)
    \item \textbf{第四步}:心脏团队讨论决定治疗策略
    \begin{itemize}
        \item TAVR+PCI vs SAVR+CABG
        \item 哪些病变需要血运重建
        \item PCI时机(TAVR前、后或同时)
    \end{itemize}
    \item \textbf{第五步}:制定详细的治疗计划和随访策略
\end{enumerate}

\textbf{PCI决策"三问法"}:
\begin{enumerate}
    \item \textbf{是否需要PCI}?
    \begin{itemize}
        \item 直径狭窄≥90\%:是
        \item FFR≤0.80:是
        \item 症状性心绞痛+中重度狭窄:是
        \item 其他情况:可能不需要
    \end{itemize}

    \item \textbf{何时PCI}?
    \begin{itemize}
        \item 简单病变、严重狭窄、自扩张瓣膜→TAVI前
        \item 交界病变、需FFR评估、复杂PCI→TAVI后
        \item 简单病变、降低成本→同时
    \end{itemize}

    \item \textbf{如何PCI}?
    \begin{itemize}
        \item FFR指导vs造影指导
        \item 完全vs不完全血运重建
        \item DES选择、DAPT时间
    \end{itemize}
\end{enumerate}

\textbf{围手术期管理要点}:

\begin{itemize}
    \item \textbf{TAVI前PCI}:
    \begin{itemize}
        \item PCI与TAVI间隔:通常1-4周
        \item DAPT管理:至少1个月后TAVI
        \item 对比剂用量:注意总量控制
    \end{itemize}

    \item \textbf{TAVI后PCI}:
    \begin{itemize}
        \item 时机:通常TAVI后1-6个月
        \item 冠脉通路:评估THV对冠脉开口的影响
        \item 导管选择:可能需要特殊形状的导管
        \item 支撑:注意导丝和导管的支撑力
    \end{itemize}

    \item \textbf{同时手术}:
    \begin{itemize}
        \item 对比剂:严格控制总量(<300mL)
        \item DAPT:TAVI时已在DAPT,注意出血
        \item 顺序:通常先PCI后TAVI
        \item 时间:尽量控制总手术时间<2小时
    \end{itemize}
\end{itemize}

\subsubsection{对中国临床实践的思考}

\begin{enumerate}
    \item \textbf{指南推荐的适用性}:
    \begin{itemize}
        \item ESC/EACTS 2021指南推荐基于欧美数据
        \item 中国人群CAD特点可能不同(更年轻、更多糖尿病)
        \item 需要中国人群的本土研究数据
    \end{itemize}

    \item \textbf{FFR在中国的应用}:
    \begin{itemize}
        \item FFR在中国的普及程度相对较低
        \item 成本和医保覆盖是限制因素
        \item QFR等国产技术可能是替代方案
    \end{itemize}

    \item \textbf{TAVR+PCI vs SAVR+CABG的选择}:
    \begin{itemize}
        \item 中国TAVR瓣膜(包括国产)的长期数据有限
        \item 成本差异可能影响治疗选择
        \item 需要考虑医保政策和患者经济负担
    \end{itemize}

    \item \textbf{心脏团队建设}:
    \begin{itemize}
        \item 大型中心基本建立了心脏团队
        \item 基层医院可能缺乏多学科协作
        \item 需要加强心脏团队的培训和推广
    \end{itemize}

    \item \textbf{研究机会}:
    \begin{itemize}
        \item 中国TAVR患者CAD患病率和特点研究
        \item 国产TAVR瓣膜+PCI的真实世界数据
        \item QFR在AS患者中的应用研究
        \item 中国人群的PCI时机和策略研究
    \end{itemize}
\end{enumerate}

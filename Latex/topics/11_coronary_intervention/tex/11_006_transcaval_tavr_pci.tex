\section{严重髂股动脉迂曲情况下经腔静脉入路联合TAVR和PCI治疗}
\label{sec:11_006_transcaval_tavr_pci}

% ============================================
% 文献信息
% ============================================
\subsection{文献信息}

\begin{itemize}
    \item \textbf{标题}: Transcaval approach for combined TAVR and PCI in the setting of prohibitive iliofemoral tortuosity
    \item \textbf{作者}: Andrea Mariani, MD; Nicolas M Van Mieghem, MD, PhD
    \item \textbf{机构}: Erasmus MC University Medical Center Rotterdam, Netherlands
    \item \textbf{会议}: TCT 2025 (Transcatheter Cardiovascular Therapeutics)
    \item \textbf{PDF文件名}: tct-1398-transcaval-approach-for-combined-tavr-and-pci-in-the-setting-of-pro.pdf
    \item \textbf{文献类型}: 会议演讲/挑战性病例报告 (Challenging Cases)
\end{itemize}

% ============================================
% 研究背景
% ============================================
\subsection{研究背景}

\subsubsection{经腔静脉入路在TAVR中的应用}

经腔静脉(Transcaval)入路已被确立为非经股动脉TAVR候选者的\textbf{救援性(bailout)入路},主要适用于:

\begin{itemize}
    \item 严重外周动脉疾病(PAD)患者
    \item 髂股动脉解剖学禁忌患者
    \item 其他常规入路不可行的情况
\end{itemize}

\subsubsection{TAVR联合PCI的挑战}

在TAVR手术中同时进行冠状动脉介入治疗(PCI)面临特殊挑战:

\begin{itemize}
    \item 需要足够的导管支撑力进行PCI操作
    \item 复杂解剖(如髂股动脉严重迂曲)可能限制导管操纵性
    \item 经腔静脉入路用于联合PCI的报道极少
\end{itemize}

\subsubsection{病例背景}

本病例报告了一例\textbf{严重髂股动脉迂曲}患者,常规经股入路无法完成TAVR和PCI,成功采用\textbf{经腔静脉入路完成联合TAVR和PCI}的挑战性病例。

% ============================================
% 病例介绍
% ============================================
\subsection{病例介绍}

\subsubsection{患者基本信息}

\begin{table}[h]
\centering
\caption{患者人口学特征}
\label{tab:patient_demographics}
\begin{tabular}{lc}
\toprule
\textbf{特征} & \textbf{值} \\
\midrule
性别 & 男性 \\
年龄 & 81岁 \\
体重 & 82 kg \\
身高 & 157 cm \\
BMI & 33.32 kg/m² \\
\bottomrule
\end{tabular}
\end{table}

\subsubsection{心血管病史}

\begin{table}[h]
\centering
\caption{心血管疾病史}
\label{tab:cardiac_history}
\begin{tabular}{ll}
\toprule
\textbf{时间} & \textbf{疾病/事件} \\
\midrule
1990年 & 动脉高血压 \\
2015年 & 腹主动脉瘤外科修复 \\
2024年12月 & 脑血管意外(左顶枕叶栓塞性卒中) \\
2024年12月 & 永久性房颤伴快速心室率(RVR) \\
\midrule
\multicolumn{2}{l}{\textit{抗凝治疗:}} \\
\multicolumn{2}{l}{阿哌沙班 2.5 mg BID + 美托洛尔 + 地高辛} \\
\bottomrule
\end{tabular}
\end{table}

\subsubsection{其他重要病史}

\begin{itemize}
    \item \textbf{痛风}
    \item \textbf{阻塞性睡眠呼吸暂停综合征(OSAS)}
    \item \textbf{4期慢性肾脏病}:eGFR 26 ml/min/1.73m²
    \item \textbf{高胆固醇血症}
\end{itemize}

\subsubsection{临床表现}

\begin{itemize}
    \item \textbf{急性肺水肿}:NYHA心功能IV级
    \item \textbf{心绞痛}:CCS分级III级
\end{itemize}

\subsubsection{心电图}

\begin{itemize}
    \item 心率:\textbf{121 bpm}
    \item 节律:\textbf{心房颤动}
    \item 其他:\textbf{左室肥厚伴劳损}
\end{itemize}

% ============================================
% 检查结果
% ============================================
\subsection{检查结果}

\subsubsection{经胸超声心动图(TTE)}

\textbf{左室功能与结构}:

\begin{itemize}
    \item 左室收缩功能正常
    \item \textbf{严重左室肥厚}(LVH)
    \item 双房扩大:左房容积指数(LAVi)= \textbf{40 ml/m²}
\end{itemize}

\textbf{主动脉瓣病变}:

\begin{center}
\fbox{\parbox{0.9\textwidth}{
\textbf{严重矛盾低流量低梯度(pLFLG)主动脉狭窄},瓣叶钙化
}}
\end{center}

\begin{table}[h]
\centering
\caption{主动脉瓣血流动力学参数}
\label{tab:echo_as_parameters}
\begin{tabular}{lc}
\toprule
\textbf{参数} & \textbf{值} \\
\midrule
每搏输出量指数(SVi) & 24 ml/m² \\
主动脉瓣口面积指数(AVAi) & 0.38 cm²/m² \\
\bottomrule
\end{tabular}
\end{table}

\subsubsection{冠状动脉造影(CAG)}

\textbf{冠脉解剖与病变}:

\begin{itemize}
    \item \textbf{右侧优势循环}
    \item \textbf{RCA中段显著钙化性狭窄}
    \item \textbf{LCX中段显著钙化性狭窄}
    \item LAD弥漫性非显著性病变
\end{itemize}

\subsubsection{CT血管造影(CTA)评估}

\textbf{主动脉瓣解剖参数}:

\begin{table}[h]
\centering
\caption{主动脉瓣CT测量参数}
\label{tab:cta_valve_parameters}
\begin{tabular}{lc}
\toprule
\textbf{参数} & \textbf{值} \\
\midrule
瓣膜形态 & 中度钙化三叶主动脉瓣 \\
VBR水平钙化 & 小钙化斑点 \\
LVOT钙化 & 无 \\
Agatston评分 & 1900 \\
膜部间隔(MS)长度 & 6 mm \\
左冠脉开口高度(LCA) & 13.2 mm \\
右冠脉开口高度(RCA) & 16.8 mm \\
\bottomrule
\end{tabular}
\end{table}

\textbf{髂股动脉解剖(关键发现)}:

\begin{center}
\fbox{\parbox{0.9\textwidth}{
\textbf{高度迂曲和动脉瘤样改变的髂股动脉}
}}
\end{center}

\begin{table}[h]
\centering
\caption{髂股动脉测量参数}
\label{tab:iliofemoral_parameters}
\begin{tabular}{lc}
\toprule
\textbf{血管} & \textbf{最大直径(Dmax)} \\
\midrule
右侧髂外动脉(EIA) & 27.2 mm \\
左侧髂外动脉(EIA) & 32.1 mm \\
\midrule
\multicolumn{2}{c}{\textbf{经股动脉TAVR入路不可行}} \\
\bottomrule
\end{tabular}
\end{table}

\textbf{经腔静脉入路规划参数}:

\begin{table}[h]
\centering
\caption{经腔静脉入路CT评估参数}
\label{tab:transcaval_parameters}
\begin{tabular}{lc}
\toprule
\textbf{参数} & \textbf{值} \\
\midrule
穿刺目标位置 & L3椎体上缘 \\
目标区域钙化 & 无钙化 \\
内脏器官遮挡 & 无 \\
与重要动脉分支关系 & 远离 \\
VCI-腹主动脉距离 & 9.4 mm \\
腹主动脉直径 & 22.5 mm \\
\bottomrule
\end{tabular}
\end{table}

% ============================================
% 心脏团队讨论与决策
% ============================================
\subsection{心脏团队讨论与决策}

\subsubsection{多学科评估}

\textbf{老年医学评估}:

\begin{itemize}
    \item 虚弱患者
    \item 功能状态受损
    \item 谵妄和卒中高风险(既往有CVA病史)
\end{itemize}

\textbf{心脏外科评估}:

\begin{itemize}
    \item 患者外科手术风险过高
    \item \textbf{STS-PROM评分:5.69\%}
\end{itemize}

\subsubsection{治疗决策}

\begin{center}
\fbox{\parbox{0.9\textwidth}{
\textbf{共识决定}:\\
1. 首先进行RCA和LCX的PCI\\
2. 随后经腔静脉入路TAVI\\
3. 使用Edwards Sapien 3 Ultra 23 mm瓣膜
}}
\end{center}

\textbf{选择经腔静脉入路的理由}:

\begin{enumerate}
    \item 严重髂股动脉迂曲和动脉瘤样改变,经股入路不可行
    \item CT评估显示理想的穿刺目标(L3上缘无钙化)
    \item 无内脏器官遮挡
    \item 远离重要动脉分支
    \item VCI-主动脉距离适宜(9.4 mm)
\end{enumerate}

% ============================================
% 手术过程
% ============================================
\subsection{手术过程}

\subsubsection{第一步:RCA-PCI(成功)}

\textbf{操作细节}:

\begin{itemize}
    \item 植入\textbf{3.50 × 15 mm依维莫司洗脱支架(EES)}
    \item 使用\textbf{4.00 mm球囊}后扩张
    \item 最终结果良好
\end{itemize}

\subsubsection{第二步:第一次LCX-PCI尝试(失败)}

\textbf{操作过程}:

尽管使用了多种策略,LCX-PCI仍无法完成:

\begin{itemize}
    \item 使用超支撑导丝(extra-support guidewire)
    \item 使用6 Fr导引延长导管(guide extension catheter)
    \item 使用长达65 cm的7 Fr导引鞘管
\end{itemize}

\textbf{失败原因分析}:

\begin{enumerate}
    \item \textbf{严重髂股动脉迂曲}:导致导管支撑力不足
    \item \textbf{LCX迂曲和钙化}:增加了操作难度
    \item 两者叠加导致无法获得足够的导管支撑进行PCI
\end{enumerate}

\begin{center}
\fbox{\parbox{0.9\textwidth}{
\textbf{关键点}:常规经股入路因髂股动脉严重迂曲无法提供足够的导管支撑完成LCX-PCI
}}
\end{center}

\subsubsection{第三步:经腔静脉TAVR(成功)}

\textbf{操作步骤}:

\begin{enumerate}
    \item \textbf{建立经腔静脉通道}:
    \begin{itemize}
        \item 使用\textbf{电导0.014" Astato XS20导丝}
        \item 使用\textbf{25 mm圈套器}(snare)配合
        \item 标准方式获得经腔静脉通道
    \end{itemize}

    \item \textbf{输送系统通过}:
    \begin{itemize}
        \item 经超硬Lunderquist导丝
        \item 推送\textbf{14 Fr eSheath}通过经腔静脉通道
    \end{itemize}

    \item \textbf{瓣膜植入}:
    \begin{itemize}
        \item \textbf{Edwards Sapien 3 Ultra 23 mm}
        \item 成功植入
    \end{itemize}
\end{enumerate}

\subsubsection{第四步:经腔静脉LCX-PCI(成功)}

\textbf{创新性应用}:

利用已建立的经腔静脉通道进行LCX-PCI:

\begin{itemize}
    \item 植入\textbf{3.00 × 8 mm依维莫司洗脱支架(EES)}
    \item 使用\textbf{3.50 mm OPN球囊}后扩张
    \item 手术成功完成
\end{itemize}

\textbf{通道关闭}:

\begin{itemize}
    \item 使用\textbf{8×10 mm ADO-1}(Amplatzer Duct Occluder)
    \item 成功关闭经腔静脉通道
    \item 关闭类型:\textbf{Type 1}(完全闭合)
\end{itemize}

\subsubsection{手术流程总结}

\begin{table}[h]
\centering
\caption{联合手术步骤与结果}
\label{tab:procedure_summary}
\begin{tabular}{llc}
\toprule
\textbf{步骤} & \textbf{操作内容} & \textbf{结果} \\
\midrule
1 & 经股入路RCA-PCI & 成功 \\
  & (3.50×15 mm EES + 4.00 mm球囊后扩张) & \\
\midrule
2 & 第一次经股入路LCX-PCI尝试 & 失败 \\
  & (髂股动脉严重迂曲) & \\
\midrule
3 & 经腔静脉TAVR & 成功 \\
  & (Sapien 3 Ultra 23 mm) & \\
\midrule
4 & 经腔静脉LCX-PCI & 成功 \\
  & (3.00×8 mm EES + 3.50 mm OPN球囊后扩张) & \\
\midrule
5 & 经腔静脉通道关闭 & 成功 \\
  & (8×10 mm ADO-1, Type 1) & \\
\bottomrule
\end{tabular}
\end{table}

% ============================================
% 文献回顾
% ============================================
\subsection{文献回顾}

\subsubsection{经腔静脉入路联合TAVR和PCI的文献报道}

\textbf{既往报道极少}:

文献检索发现仅有\textbf{两个病例报告}:

\begin{enumerate}
    \item \textbf{ACC.20 World Congress of Cardiology报告}(JACC March 24, 2020):
    \begin{itemize}
        \item 标题:Transcaval Impella-Protected Left Main PCI with Rotational Atherectomy Followed by TAVR and Renal Artery Stenting in an Elderly Female with Cardiogenic Shock and Renal Failure Due to Severe Aortic Stenosis and Regurgitation
        \item PCI在THV植入\textbf{后}进行
        \item 使用\textbf{Impella保护}
    \end{itemize}

    \item \textbf{JACC: Cardiovascular Interventions Vol. 17, No. 21, 2024报告}:
    \begin{itemize}
        \item 标题:Transcaval Impella-Assisted CHIP-PCI and Transcaval TAVR With Impella Removal in Freshly Implanted TAVR
        \item PCI在THV植入\textbf{后}进行
        \item 使用\textbf{Impella辅助}
    \end{itemize}
\end{enumerate}

\subsubsection{本病例的独特性}

\textbf{与既往报道的关键区别}:

\begin{table}[h]
\centering
\caption{本病例与既往报道的比较}
\label{tab:case_comparison}
\begin{tabular}{lll}
\toprule
\textbf{特征} & \textbf{既往报道} & \textbf{本病例} \\
\midrule
PCI时机 & THV植入后 & TAVR前后均有 \\
机械循环支持 & 使用Impella & 无 \\
选择经腔静脉入路原因 & PAD严重性 & 髂股动脉严重迂曲 \\
PCI复杂程度 & 左主干、旋磨 & 常规PCI \\
经腔静脉PCI目的 & 血流动力学支持下PCI & 克服导管操纵困难 \\
\bottomrule
\end{tabular}
\end{table}

\textbf{本病例的创新点}:

\begin{enumerate}
    \item \textbf{首次报道}因\textbf{髂股动脉严重迂曲}而非PAD严重性选择经腔静脉入路
    \item \textbf{首次展示}经腔静脉入路可以\textbf{改善导管操纵性},便于完成复杂冠脉病变的PCI
    \item 证明经腔静脉入路不仅是救援性通道,还可作为\textbf{优化导管支撑的策略}
\end{enumerate}

% ============================================
% 主要发现与结论
% ============================================
\subsection{主要发现与结论}

\subsubsection{核心发现}

\begin{enumerate}
    \item \textbf{经腔静脉入路的多功能性}:
    \begin{itemize}
        \item 传统上仅作为非经股TAVR候选者的救援性入路
        \item 本病例证明可同时用于TAVR和PCI
        \item 能够改善导管操纵性和支撑力
    \end{itemize}

    \item \textbf{髂股动脉严重迂曲的解决方案}:
    \begin{itemize}
        \item 髂股动脉严重迂曲导致导管支撑力不足
        \item 即使使用超支撑导丝、导引延长导管和长鞘管仍无法完成PCI
        \item 经腔静脉入路提供了更直接、更短的路径
        \item 显著改善导管操纵性和支撑力
    \end{itemize}

    \item \textbf{技术可行性与安全性}:
    \begin{itemize}
        \item 标准方式建立经腔静脉通道
        \item 14 Fr大鞘管可安全通过
        \item 既可完成TAVR又可完成PCI
        \item 通道可使用ADO装置成功关闭(Type 1)
    \end{itemize}
\end{enumerate}

\subsubsection{临床结论}

\begin{center}
\fbox{\parbox{0.9\textwidth}{
\textbf{经腔静脉入路是髂股动脉解剖学禁忌患者的可行且有效的替代方案,不仅适用于TAVR,也适用于需要良好导管支撑的复杂冠脉介入治疗。}
}}
\end{center}

% ============================================
% 临床启示
% ============================================
\subsection{临床启示}

\subsubsection{对经腔静脉入路适应症的拓展}

\textbf{传统适应症}:

\begin{itemize}
    \item 严重外周动脉疾病(PAD)
    \item 髂股动脉闭塞或严重狭窄
    \item 髂股动脉严重钙化
    \item 腹主动脉瘤伴髂动脉受累
\end{itemize}

\textbf{新增适应症}(基于本病例):

\begin{itemize}
    \item \textbf{髂股动脉严重迂曲和动脉瘤样改变}
    \item 需要良好导管支撑的联合PCI手术
    \item 经股入路导管操纵困难的复杂病例
\end{itemize}

\subsubsection{对术前评估的启示}

\textbf{髂股动脉迂曲的评估}:

\begin{enumerate}
    \item \textbf{CT三维重建评估至关重要}:
    \begin{itemize}
        \item 不仅评估血管直径和钙化
        \item 必须评估血管迂曲程度
        \item 评估动脉瘤样改变
    \end{itemize}

    \item \textbf{迂曲程度的功能性影响}:
    \begin{itemize}
        \item 严重迂曲可导致导管支撑力不足
        \item 影响PCI操作的可行性
        \item 需要考虑替代入路
    \end{itemize}

    \item \textbf{联合手术的额外考虑}:
    \begin{itemize}
        \item TAVR联合PCI时,需评估经股入路是否能满足两种手术的要求
        \item PCI对导管支撑力要求更高
        \item 可能需要分期手术或选择替代入路
    \end{itemize}
\end{enumerate}

\subsubsection{对手术策略的启示}

\textbf{手术顺序的灵活性}:

\begin{enumerate}
    \item \textbf{先尝试经股入路PCI}:
    \begin{itemize}
        \item 评估导管支撑力是否足够
        \item 如失败,可改用经腔静脉入路
    \end{itemize}

    \item \textbf{先完成TAVR建立经腔静脉通道}:
    \begin{itemize}
        \item 随后利用该通道完成PCI
        \item 一次建立通道,完成两种手术
    \end{itemize}

    \item \textbf{分期手术的考虑}:
    \begin{itemize}
        \item 如术前预计经股入路PCI困难
        \item 可考虑先行简单病变PCI,随后经腔静脉TAVR联合复杂病变PCI
    \end{itemize}
\end{enumerate}

\textbf{导管选择与技术}:

\begin{itemize}
    \item 经腔静脉入路PCI可能需要特殊导管配置
    \item 导管长度需适应经腔静脉路径
    \item 支撑力评估在经腔静脉入路中同样重要
\end{itemize}

\subsubsection{对患者选择的启示}

\textbf{理想候选患者}:

\begin{enumerate}
    \item \textbf{髂股动脉解剖禁忌}:
    \begin{itemize}
        \item 严重迂曲和动脉瘤样改变
        \item 髂股动脉直径过大(如本例>27 mm)
        \item 既往腹主动脉瘤修复史
    \end{itemize}

    \item \textbf{需要联合冠脉介入}:
    \begin{itemize}
        \item 显著冠脉病变需要PCI
        \item PCI病变复杂程度需要良好支撑
    \end{itemize}

    \item \textbf{适合的经腔静脉解剖}:
    \begin{itemize}
        \item 合适的穿刺目标(无钙化、无内脏遮挡)
        \item VCI-主动脉距离适宜(通常>5 mm)
        \item 主动脉直径适中
    \end{itemize}
\end{enumerate}

\subsubsection{对并发症预防的启示}

\textbf{经腔静脉通道建立的安全性}:

\begin{itemize}
    \item 详细的CT评估选择最佳穿刺点
    \item 避开钙化、内脏器官和重要动脉分支
    \item 使用电导导丝和圈套器标准技术
\end{itemize}

\textbf{通道关闭的重要性}:

\begin{itemize}
    \item 使用ADO等专用装置关闭
    \item 本例使用8×10 mm ADO-1获得Type 1闭合
    \item 需要超声或造影确认完全闭合
\end{itemize}

\textbf{抗凝管理}:

\begin{itemize}
    \item 本例患者长期服用抗凝药物(阿哌沙班)
    \item 围手术期抗凝管理需要谨慎
    \item 平衡卒中风险(既往CVA)和出血风险
\end{itemize}

% ============================================
% 研究局限性
% ============================================
\subsection{研究局限性}

\subsubsection{病例报告的固有局限性}

\begin{enumerate}
    \item \textbf{单一病例经验}:
    \begin{itemize}
        \item 无法提供系统性证据
        \item 结果可重复性未知
        \item 需要更多病例验证
    \end{itemize}

    \item \textbf{缺乏对照组}:
    \begin{itemize}
        \item 无法与其他替代策略比较(如分期手术、外科手术)
        \item 无法评估相对优劣
    \end{itemize}

    \item \textbf{短期结果报告}:
    \begin{itemize}
        \item 仅报告手术成功
        \item 缺乏术后随访数据
        \item 中长期结果未知
    \end{itemize}
\end{enumerate}

\subsubsection{技术方面的局限性}

\begin{enumerate}
    \item \textbf{操作者经验依赖}:
    \begin{itemize}
        \item 经腔静脉入路需要专门培训
        \item 来自Erasmus MC等高容量中心
        \item 结果可能无法完全推广至所有中心
    \end{itemize}

    \item \textbf{设备可用性}:
    \begin{itemize}
        \item 需要特殊设备(电导导丝、圈套器、ADO装置)
        \item 并非所有中心均可获得
    \end{itemize}

    \item \textbf{影像学支持要求}:
    \begin{itemize}
        \item 需要高质量CT评估
        \item 需要术中透视和超声指导
    \end{itemize}
\end{enumerate}

\subsubsection{患者选择的局限性}

\begin{enumerate}
    \item \textbf{复杂合并症}:
    \begin{itemize}
        \item 本例患者多种合并症(CKD、房颤、既往卒中)
        \item 难以区分哪些因素影响结果
    \end{itemize}

    \item \textbf{特殊解剖}:
    \begin{itemize}
        \item 严重髂股动脉迂曲是相对少见的情况
        \item 经验可能不适用于其他类型的血管病变
    \end{itemize}
\end{enumerate}

% ============================================
% 个人笔记
% ============================================
\subsection{个人笔记}

\subsubsection{关键数字记忆}

\textbf{患者特征}:
\begin{itemize}
    \item 年龄:\textbf{81岁}
    \item BMI:\textbf{33.32 kg/m²}(肥胖)
    \item eGFR:\textbf{26 ml/min/1.73m²}(4期CKD)
    \item STS-PROM评分:\textbf{5.69\%}
\end{itemize}

\textbf{主动脉瓣参数}:
\begin{itemize}
    \item SVi:\textbf{24 ml/m²}(低流量)
    \item AVAi:\textbf{0.38 cm²/m²}(严重狭窄)
    \item Agatston评分:\textbf{1900}(中度钙化)
    \item LCA高度:\textbf{13.2 mm}
    \item RCA高度:\textbf{16.8 mm}
\end{itemize}

\textbf{髂股动脉参数(关键)}:
\begin{itemize}
    \item 右EIA Dmax:\textbf{27.2 mm}
    \item 左EIA Dmax:\textbf{32.1 mm}(严重动脉瘤样改变)
\end{itemize}

\textbf{经腔静脉入路参数}:
\begin{itemize}
    \item 穿刺位置:\textbf{L3椎体上缘}
    \item VCI-主动脉距离:\textbf{9.4 mm}
    \item 腹主动脉直径:\textbf{22.5 mm}
    \item 鞘管尺寸:\textbf{14 Fr eSheath}
\end{itemize}

\textbf{手术参数}:
\begin{itemize}
    \item TAVR瓣膜:\textbf{Sapien 3 Ultra 23 mm}
    \item RCA支架:\textbf{3.50×15 mm EES},\textbf{4.00 mm}球囊后扩张
    \item LCX支架:\textbf{3.00×8 mm EES},\textbf{3.50 mm OPN}球囊后扩张
    \item 通道关闭装置:\textbf{8×10 mm ADO-1}
\end{itemize}

\subsubsection{重要概念与机制}

\begin{description}
    \item[经腔静脉入路(Transcaval Access)] 通过下腔静脉(IVC)穿刺进入腹主动脉的非常规血管通路。使用电导导丝从IVC穿刺主动脉后壁,建立从股静脉到主动脉的直接通道。

    \item[pLFLG主动脉狭窄(Paradoxical Low-Flow Low-Gradient AS)] 矛盾性低流量低梯度主动脉狭窄。特征为:左室收缩功能正常(LVEF≥50\%)、主动脉瓣口面积小、平均梯度低(<40 mmHg)、每搏输出量低(SVi<35 ml/m²)。常见于严重LVH、小腔综合征患者。

    \item[髂股动脉迂曲(Iliofemoral Tortuosity)] 髂动脉和股动脉的过度弯曲和迂回。严重迂曲会导致:(1) 导管推送困难;(2) 导管支撑力不足;(3) 器械输送受限;(4) 血管损伤风险增加。本例中,严重迂曲使得即使使用超支撑系统仍无法完成PCI。

    \item[动脉瘤样改变(Aneurysmal Changes)] 动脉局部直径异常增大,通常定义为正常直径的1.5倍以上。本例双侧髂外动脉直径达27-32 mm(正常约8-10 mm),属于严重动脉瘤样改变。

    \item[导管支撑力(Catheter Support)] 介入导管提供的力量,用于推送球囊、支架等器械通过病变。支撑力取决于:(1) 入路血管的走行和迂曲程度;(2) 导管本身的刚度;(3) 导丝的支撑;(4) 导引延长导管的使用。

    \item[ADO装置(Amplatzer Duct Occluder)] 用于封堵血管或导管的自膨胀镍钛合金装置。本例使用8×10 mm ADO-1关闭经腔静脉通道,获得Type 1(完全)闭合。

    \item[Type 1闭合(Type 1 Closure)] 经腔静脉通道关闭的分类。Type 1为完全闭合,无残余分流;Type 2为小残余分流;Type 3为需要额外干预。

    \item[导引延长导管(Guide Extension Catheter)] 也称为"母-子"导管系统。通过标准导引导管内插入更远端的延长导管,提供额外的支撑和同轴性。常用于复杂PCI,但在严重迂曲时仍可能支撑不足。

    \item[Lunderquist导丝] 超硬交换导丝,直径通常0.035-0.038英寸,具有极高的支撑力。常用于需要推送大鞘管或输送系统的复杂介入手术,如TAVR、EVAR等。

    \item[Astato XS20导丝] 电导(electrified)导丝,导丝头端可通电产生射频能量,用于穿透组织。在经腔静脉入路中用于从IVC穿刺主动脉后壁。
\end{description}

\subsubsection{临床决策要点}

\textbf{何时考虑经腔静脉入路}:

\begin{enumerate}
    \item \textbf{评估标准经股入路}:
    \begin{itemize}
        \item CT三维重建评估髂股动脉
        \item 评估直径、钙化、迂曲、动脉瘤
        \item 预估导管操纵性和支撑力
    \end{itemize}

    \item \textbf{经股入路禁忌指标}:
    \begin{itemize}
        \item 严重迂曲(如本例)
        \item 严重动脉瘤(直径>25 mm)
        \item 严重狭窄或闭塞
        \item 严重钙化
    \end{itemize}

    \item \textbf{联合手术的额外考虑}:
    \begin{itemize}
        \item PCI对支撑力要求高于单纯TAVR
        \item 病变复杂性(钙化、迂曲、慢性闭塞)
        \item 是否需要旋磨等复杂技术
    \end{itemize}
\end{enumerate}

\textbf{经腔静脉入路评估清单}:

\begin{enumerate}
    \item \textbf{穿刺目标选择}:
    \begin{itemize}
        \item 通常选择L3椎体上缘或L4水平
        \item 目标区域必须无钙化
        \item 无内脏器官遮挡(肠管、肝脏等)
        \item 远离重要动脉分支(如肠系膜动脉、肾动脉)
    \end{itemize}

    \item \textbf{距离测量}:
    \begin{itemize}
        \item VCI-主动脉距离:理想>5 mm,可接受范围5-15 mm
        \item 本例9.4 mm,属于理想范围
        \item 距离过短增加穿刺难度
        \item 距离过长可能导致通道不稳定
    \end{itemize}

    \item \textbf{主动脉直径}:
    \begin{itemize}
        \item 理想直径18-25 mm
        \item 本例22.5 mm,理想
        \item 过小可能导致穿刺困难
        \item 过大可能影响通道稳定性
    \end{itemize}
\end{enumerate}

\textbf{手术技巧要点}:

\begin{enumerate}
    \item \textbf{通道建立}:
    \begin{itemize}
        \item 标准技术:电导导丝+圈套器
        \item 透视和超声双重指导
        \item 确认导丝在主动脉真腔
    \end{itemize}

    \item \textbf{鞘管推送}:
    \begin{itemize}
        \item 先交换超硬导丝(如Lunderquist)
        \item 逐步扩张通道
        \item 14 Fr鞘管可满足大多数TAVR需求
    \end{itemize}

    \item \textbf{通道关闭}:
    \begin{itemize}
        \item 使用ADO装置主动闭合
        \item 尺寸选择基于主动脉直径和通道大小
        \item 本例8×10 mm ADO-1
        \item 造影或超声确认完全闭合
    \end{itemize}
\end{enumerate}

\subsubsection{与其他替代入路的比较}

\textbf{非经股TAVR入路选择}:

\begin{table}[h]
\centering
\caption{非经股TAVR入路比较}
\label{tab:alternative_access}
\begin{tabular}{lccc}
\toprule
\textbf{入路} & \textbf{优势} & \textbf{劣势} & \textbf{适用情况} \\
\midrule
经腔静脉 & 完全经皮;可联合PCI & 需要特殊技术;通道闭合 & 髂股迂曲/动脉瘤 \\
经心尖 & 直接路径;短距离 & 需要小切口;心脏穿刺 & LVEF正常;无粘连 \\
经锁骨下 & 经皮;相对简单 & 左侧需要弯曲;卒中风险 & 锁骨下动脉条件好 \\
经主动脉 & 直接;短距离 & 需要胸骨切开或小切口 & 其他入路均不可行 \\
经颈动脉 & 经皮;路径直 & 卒中风险高;需要特殊鞘管 & 颈动脉条件好 \\
\bottomrule
\end{tabular}
\end{table}

\textbf{本例中选择经腔静脉的理由}:

\begin{itemize}
    \item 完全经皮,符合患者虚弱状态
    \item 既往卒中史,避免经颈或经锁骨下入路
    \item 正常左室功能,但严重LVH,经心尖可能增加心律失常风险
    \item 既往腹主动脉瘤修复,经主动脉可能有粘连
    \item CT评估经腔静脉条件理想
    \item 可以同时解决TAVR和PCI的入路问题
\end{itemize}

\subsubsection{对未来研究的建议}

\begin{enumerate}
    \item \textbf{病例系列研究}:
    \begin{itemize}
        \item 收集更多经腔静脉入路联合TAVR和PCI病例
        \item 建立多中心注册研究
        \item 系统评估安全性和有效性
    \end{itemize}

    \item \textbf{技术标准化}:
    \begin{itemize}
        \item 建立经腔静脉入路PCI的技术指南
        \item 明确适应症和禁忌症
        \item 标准化操作流程
    \end{itemize}

    \item \textbf{比较性研究}:
    \begin{itemize}
        \item 经腔静脉 vs 其他非经股入路
        \item 联合手术 vs 分期手术
        \item 成本效益分析
    \end{itemize}

    \item \textbf{长期随访}:
    \begin{itemize}
        \item 通道闭合的持久性
        \item 迟发并发症
        \item TAVR和PCI的长期结果
    \end{itemize}

    \item \textbf{器械改进}:
    \begin{itemize}
        \item 开发专用于经腔静脉PCI的导管系统
        \item 改进通道闭合装置
        \item 优化影像学指导技术
    \end{itemize}
\end{enumerate}

\subsubsection{对中国临床实践的思考}

\begin{enumerate}
    \item \textbf{技术培训需求}:
    \begin{itemize}
        \item 经腔静脉入路在中国尚不普及
        \item 需要专门培训和经验积累
        \item 可能需要国际交流和学习
    \end{itemize}

    \item \textbf{患者人群特点}:
    \begin{itemize}
        \item 中国患者髂股动脉迂曲和动脉瘤相对少见
        \item 但严重PAD和钙化患者增多
        \item 经腔静脉入路可能有不同适应症分布
    \end{itemize}

    \item \textbf{医保和成本考虑}:
    \begin{itemize}
        \item 需要特殊器械(电导导丝、ADO装置)
        \item 成本可能高于标准经股入路
        \item 但可能低于外科手术或分期手术
        \item 需要卫生经济学评估
    \end{itemize}

    \item \textbf{多学科协作}:
    \begin{itemize}
        \item 需要影像科、介入科、心脏外科密切合作
        \item 建立心脏团队决策机制
        \item 复杂病例应有专家会诊
    \end{itemize}
\end{enumerate}

\subsubsection{记忆口诀}

\textbf{经腔静脉入路"3个9"法则}:
\begin{itemize}
    \item VCI-主动脉距离约\textbf{9} mm(理想范围)
    \item Agatston评分\textbf{1900}(约\textbf{19}百)
    \item 穿刺目标无钙化(\textbf{0}分,记忆为"完美9")
\end{itemize}

\textbf{髂股动脉"双3"记忆}:
\begin{itemize}
    \item 双侧EIA直径\textbf{27-32} mm(接近\textbf{30} mm)
    \item BMI \textbf{33}.32(约\textbf{33})
    \item 两者都是"3"字头,提示肥胖患者易有动脉瘤样改变
</itemize>

\textbf{手术"1-2-3"步骤}:
\begin{itemize}
    \item \textbf{1}次经股PCI成功(RCA)
    \item \textbf{2}次尝试(第一次经股LCX失败,第二次经腔静脉成功)
    \item \textbf{3}种器械(Sapien 3 Ultra,2个支架,1个ADO)
\end{itemize}

\subsubsection{值得深入思考的问题}

\begin{enumerate}
    \item \textbf{为什么髂股动脉严重迂曲和动脉瘤会导致PCI失败?}
    \begin{itemize}
        \item 迂曲增加导管路径长度,能量损失增加
        \item 动脉瘤处血管壁缺乏支撑,导管推力无法有效传递
        \item 即使使用超支撑系统,力量仍被迂曲和动脉瘤"吸收"
        \item 经腔静脉入路提供更直接、更短的路径,支撑力显著改善
    \end{itemize}

    \item \textbf{为什么可以先完成TAVR再进行PCI?}
    \begin{itemize}
        \item 通常担心冠脉阻塞,应先PCI
        \item 本例先完成RCA-PCI,确保右侧循环通畅
        \item TAVR后冠脉阻塞风险低(LCA高度13.2 mm,RCA高度16.8 mm)
        \item 利用已建立的经腔静脉通道完成LCX-PCI,避免二次建立通道
        \item 这种策略需要精确的术前规划和风险评估
    \end{itemize}

    \item \textbf{经腔静脉入路的导管支撑力为何优于严重迂曲的经股入路?}
    \begin{itemize}
        \item 路径更短:从IVC穿刺点到主动脉根部距离短
        \item 路径更直:避开了迂曲的髂股动脉
        \item 主动脉段支撑好:主动脉本身直径大、走行直
        \item 无动脉瘤干扰:避开了髂股动脉瘤样改变
    \end{itemize}

    \item \textbf{ADO装置关闭经腔静脉通道的机制是什么?}
    \begin{itemize}
        \item ADO是自膨胀镍钛合金双盘装置
        \item 一个盘置于主动脉腔内,一个盘置于IVC腔内
        \item 中间连接腰部填充穿刺通道
        \item 装置释放后自动扩张,压迫通道壁止血
        \item Type 1闭合表示完全闭合无残余分流
    \end{itemize}

    \item \textbf{为什么本例没有使用Impella等机械循环支持?}
    \begin{itemize}
        \item 左室收缩功能正常(尽管有严重LVH)
        \item 患者血流动力学相对稳定(急性肺水肿已控制)
        \item RCA-PCI先行完成,保证右侧循环
        \item pLFLG主动脉狭窄,但非心源性休克
        \item Impella增加成本、复杂性和血管并发症风险
        \item 既往报道使用Impella可能是因为患者处于心源性休克状态
    \end{itemize}

    \item \textbf{如果经腔静脉LCX-PCI也失败怎么办?}
    \begin{itemize}
        \item 可以考虑术后分期手术
        \item 桡动脉入路PCI(如果RCA-PCI已完成,LCX可能不那么紧迫)
        \item 外科搭桥手术(但本例患者手术风险高)
        \item 药物治疗(如果LCX狭窄不是罪犯病变)
        \item 强调术前规划和多种预案准备的重要性
    \end{itemize}
\end{enumerate}

\subsubsection{实用技巧总结}

\textbf{经腔静脉入路评估"5步法"}:

\begin{enumerate}
    \item \textbf{第一步}:评估经股入路可行性(迂曲、钙化、动脉瘤、直径)
    \item \textbf{第二步}:选择穿刺目标(L3-L4水平,无钙化,无内脏遮挡)
    \item \textbf{第三步}:测量VCI-主动脉距离(理想5-15 mm)
    \item \textbf{第四步}:测量主动脉直径(理想18-25 mm)
    \item \textbf{第五步}:评估是否有其他更优入路(如经锁骨下、经心尖)
\end{enumerate}

\textbf{联合TAVR和PCI手术"3原则"}:

\begin{enumerate}
    \item \textbf{先易后难}:先完成简单病变PCI(如本例RCA),再处理复杂病变
    \item \textbf{保护为先}:确保主要冠脉通畅后再进行TAVR
    \item \textbf{灵活应变}:根据术中情况调整策略(如本例经股失败改经腔静脉)
\end{enumerate}

\textbf{通道关闭"3确认"}:

\begin{enumerate}
    \item 造影确认ADO位置正确
    \item 超声确认无残余分流
    \item 压迫或观察确认无出血
\end{enumerate}


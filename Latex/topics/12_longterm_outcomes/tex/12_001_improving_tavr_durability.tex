\section{我们能否通过影像学程序优化改善TAVR耐久性?}
\label{sec:12_001_improving_tavr_durability}

% ============================================
% 文献信息
% ============================================
\subsection{文献信息}

\begin{itemize}
    \item \textbf{标题}: Can We Improve TAVR Durability Today? Imaging-based Procedural Solutions
    \item \textbf{作者}: João Cavalcante, MD, FASE, FSCCT, FSCMR
    \item \textbf{机构}: Allina Health Minneapolis Heart Institute
    \begin{itemize}
        \item Section Head, Cardiac Imaging
        \item Scientific Director, Cardiovascular Imaging Core Lab and Research Center
        \item Chair, Van Tassel Innovation Center
    \end{itemize}
    \item \textbf{会议}: TCT (Transcatheter Cardiovascular Therapeutics)
    \item \textbf{PDF文件名}: can-we-improve-tavr-durability-today-imaging-based-procedural-solutions.pdf
    \item \textbf{文献类型}: 会议演讲/专家综述
\end{itemize}

% ============================================
% 研究背景
% ============================================
\subsection{研究背景}

\subsubsection{TAVR耐久性的重要性}

随着TAVR适应症扩展至低危和无症状患者,瓣膜耐久性成为关键问题。年轻患者预期寿命更长,对人工瓣膜长期性能提出更高要求。理解和改善TAVR耐久性的影响因素是当前结构性心脏病领域的重点。

\subsubsection{耐久性是多因素、相互关联的问题}

TAVR瓣膜耐久性受多种因素影响,可分为三大类:

\textbf{1. 宿主相关因素}(不可改变或难以改变):
\begin{itemize}
    \item 宿主主动脉瓣瓣叶、解剖结构和周围组织
    \item 宿主合并症:年龄、慢性肾脏病(CKD)、钙磷代谢异常等
\end{itemize}

\textbf{2. 瓣膜相关因素}(部分可选择):
\begin{itemize}
    \item 人工瓣膜类型
    \item 瓣膜尺寸与患者-假体不匹配(PPM)
    \item 支架和瓣叶设计
    \item 瓣叶组织材料
\end{itemize}

\textbf{3. 程序相关因素}(\textbf{可优化}):
\begin{itemize}
    \item THV装置准备和压握
    \item \textbf{THV支架变形/支架扩张}(核心可改善因素)
    \item 球囊预扩张/后扩张策略
    \item 瓣膜尺寸选择算法
    \item 钙化修饰技术
\end{itemize}

\subsubsection{瓣膜退化的定义演变}

演讲强调了\textbf{如何定义}耐久性本身就是一个问题,这也影响我们对耐久性的理解。

% ============================================
% 主要研究发现
% ============================================
\subsection{主要研究发现}

\subsubsection{VARC-3定义:BVD和BVF的标准化}

根据Genereux等人(JACC 2021)提出的VARC-3定义:

\textbf{关键概念:形态学改变(Stage 1)先于血流动力学改变(Stages 2-3)}

\begin{table}[h]
\centering
\caption{VARC-3生物人工瓣膜退化(BVD)分类}
\label{tab:varc3_bvd_classification}
\begin{tabular}{p{4cm}p{10cm}}
\toprule
\textbf{分类} & \textbf{定义} \\
\midrule
\multicolumn{2}{l}{\textbf{问题:BVD是否与人工瓣膜固有的永久性改变相关?}} \\
\midrule
Non-Structural BVD & 任何异常,\textbf{不是}瓣膜固有的,导致BVD \\
(非结构性BVD) & - 假体-患者不匹配 \\
 & - 瓣周反流 \\
 & - 其他:位置不良、栓塞等 \\
 & \textbf{结果}:随访期间无血流动力学瓣膜退化 \\
\midrule
Structural BVD & \textbf{瓣膜固有的永久性结构改变} \\
(结构性BVD) & \textbf{Stage 1 SVD(结构性瓣膜退化)} \\
 & \textbf{结果}:随访期间出现血流动力学瓣膜退化 \\
\midrule
Thrombosis/Endocarditis & 可能可逆 \\
(血栓形成/心内膜炎) & (但可能→不可逆) \\
\midrule
\multicolumn{2}{l}{\textbf{血流动力学瓣膜退化}} \\
\midrule
Non-Structural BVF & 无血流动力学瓣膜退化 \\
\midrule
Structural BVF & \textbf{Stage 2(中度):Stage 3(严重)SVD} \\
\bottomrule
\end{tabular}
\end{table}

\textbf{生物人工瓣膜失败(BVF)定义}:
\begin{enumerate}
    \item 任何有临床表达标准的BVD \textbf{或}不可逆的Stage 3 BVD
    \item 再次干预或有再次干预指征
    \item 瓣膜相关死亡
\end{enumerate}

\subsubsection{THV支架变形:耐久性的关键因素}

\textbf{研究来源}:Fukui M, Cavalcante JL, Bapat VN. J Cardiol 2024 Jun;83(6):351-358

\textbf{核心发现}:

\begin{enumerate}
    \item \textbf{功能部分的变形至关重要}
    \begin{itemize}
        \item Intra-annular THV:功能部分在瓣环内
        \item Supra-annular THV:功能部分在瓣环上方
        \item 瓣叶承受应力主要在功能部分
    \end{itemize}

    \item \textbf{变形是三维现象}
    \begin{itemize}
        \item 体外(In Vitro):圆形、完全扩张
        \item 体内(In Vivo):偏心性、欠扩张、垂直变形
    \end{itemize}

    \item \textbf{变形导致瓣膜退化的机制}
\end{enumerate}

\begin{figure}[h]
\centering
\begin{tikzpicture}[node distance=2cm, auto, thick]
\node[draw, rectangle, fill=gray!30, text width=3cm, align=center] (A) {THV支架变形};
\node[draw, rectangle, fill=orange!40, text width=2.5cm, align=center, right=1.5cm of A] (B) {瓣叶应力\\与应变};
\node[draw, rectangle, fill=orange!40, text width=2.5cm, align=center, below=0.8cm of B] (C) {HALT\\血栓形成};
\node[draw, rectangle, fill=orange!60, text width=3cm, align=center, right=1.5cm of B] (D) {纤维化\\钙化};
\node[draw, rectangle, fill=red!50, text width=3cm, align=center, right=1.5cm of D] (E) {THV退化};

\draw[->, line width=1.5pt] (A) -- (B);
\draw[->, line width=1.5pt] (A) -- (C);
\draw[->, line width=1.5pt] (B) -- (D);
\draw[->, line width=1.5pt] (C) -- (D);
\draw[->, line width=1.5pt] (D) -- (E);
\end{tikzpicture}
\caption{THV支架变形导致瓣膜退化的机制}
\label{fig:thv_deformation_mechanism}
\end{figure}

\subsubsection{HALT的流行病学和自然史}

\textbf{HALT(Hypoattenuated Leaflet Thickening)}:CT上瓣叶低密度增厚,提示亚临床瓣叶血栓形成。

\textbf{1. HALT发生率}

\begin{table}[h]
\centering
\caption{不同研究中TAVR术后HALT发生率}
\label{tab:halt_incidence}
\begin{tabular}{lcccc}
\toprule
\textbf{研究} & \textbf{时间点} & \textbf{HALT发生率} & \textbf{样本量} & \textbf{瓣膜类型} \\
\midrule
\multirow{2}{*}{Blanke et al. JACC 2020} & 30天 & 17.3\% & \multirow{2}{*}{TAVR队列} & \multirow{2}{*}{混合} \\
 & 1年 & 30.9\% & & \\
\midrule
\multirow{2}{*}{Makkar et al. JACC 2020} & 30天 & 13\% & \multirow{2}{*}{TAVR vs SAVR} & \multirow{2}{*}{PARTNER II} \\
 & 1年 & 28\% & & \\
\midrule
Fukui et al. Circulation 2022 & 30天 & 19\% & 565 & S3 62\%, Evolut 38\% \\
\bottomrule
\end{tabular}
\end{table}

\textbf{2. HALT程度分级}(Blanke et al.)

\begin{table}[h]
\centering
\caption{TAVR术后不同程度HALT的比例}
\label{tab:halt_severity}
\begin{tabular}{lcc}
\toprule
\textbf{HALT程度} & \textbf{30天} & \textbf{1年} \\
\midrule
≤25\% & 10.6\% & 17.8\% \\
>25\%-50\% & 2.8\% / 6.5\% & 6.6\% / 9.5\% \\
>50\%-75\% & 2.2\% / 4.3\% & 6.9\% / 6.0\% \\
>75\% & 1.7\% & 1.3\% / 6.6\% \\
\midrule
\textbf{总计} & \textbf{17.3\%} & \textbf{30.9\% (28.4\%)} \\
\bottomrule
\end{tabular}
\end{table}

\textbf{3. TAVR vs 外科瓣膜}(Makkar et al.)

亚临床瓣叶血栓:
\begin{itemize}
    \item TAVR 30天:13\%;1年:28\%
    \item 外科瓣膜 30天:5\%;1年:20\%
\end{itemize}

\textbf{4. HALT的自然史动态变化}

\begin{table}[h]
\centering
\caption{HALT的自然演变(Blanke et al.)}
\label{tab:halt_natural_history}
\begin{tabular}{lccc}
\toprule
\textbf{30天状态} & \textbf{1年状态} & \textbf{患者数} & \textbf{解读} \\
\midrule
HALT (n=26) & HALT (N=15, 3 on OAC) & 15 & 持续存在 \\
 & No HALT (N=11, 2 on OAC) & 11 & 自行消退 \\
\midrule
No HALT (n=126) & HALT (N=32, 0 on OAC) & 32 & 新发HALT \\
 & No HALT (N=94, 5 on OAC) & 94 & 持续无HALT \\
\bottomrule
\end{tabular}
\end{table}

\textbf{关键结论}:
\begin{itemize}
    \item \textbf{HALT不是二元过程,而是分级谱}
    \item 轻度HALT(<25\%)可能是真实发现,但可能自行消退
    \item 严重HALT(>50\%)更难以漏诊,不太可能自发消退
\end{itemize}

\subsubsection{THV形状与HALT的关系}

\textbf{研究}:Fukui M et al. Circulation 2022 Aug 9;146(6):480-493

\textbf{研究设计}:
\begin{itemize}
    \item 565例患者,TAVR术后30天CT筛查
    \item 瓣膜类型:352例Sapien 3(62\%);213例Evolut R/Pro+(38\%)
    \item HALT发生率:19\%
\end{itemize}

\textbf{关键测量指标}:

\begin{enumerate}
    \item \textbf{变形指数(Deformation Index)}
    \begin{itemize}
        \item 计算公式:$\text{DI} = \frac{(\mu - r)}{(2 \times r)}$
        \item $\mu$:最大直径;$r$:最小直径
        \item HALT组:1.13-1.21(BEV和SEV)
        \item No HALT组:1.04
    \end{itemize}

    \item \textbf{新生窦容积指数(Neo-sinus Volume Index)}
    \begin{itemize}
        \item HALT组:0.82-0.89
        \item No HALT组:1.08-1.13
        \item 数值越小,提示容积受压缩
    \end{itemize}

    \item \textbf{偏心度(Eccentricity)}
    \begin{itemize}
        \item HALT组(BEV):0.73
        \item HALT组(SEV):高度变异
        \item No HALT组:0.22
    \end{itemize}

    \item \textbf{不对称瓣叶扩张(Asymmetric Leaflet Expansion)}
    \begin{itemize}
        \item HALT组:12°-34°的瓣叶间差异
        \item No HALT组:2°-6°的瓣叶间差异
    \end{itemize}
\end{enumerate}

\textbf{重要发现}:
\begin{itemize}
    \item \textbf{HALT+组与HALT-组之间,THV跨瓣压差无差异}
    \item 这表明HALT的血流动力学影响在早期可能不明显
    \item 但形态学改变已经存在
\end{itemize}

\subsubsection{HALT的治疗:华法林的作用}

\textbf{研究}:Garcia S et al. Circ Cardiovasc Interv. 2022;15:e011480

\textbf{主要发现}:

\begin{enumerate}
    \item \textbf{华法林治疗与HALT消退}
    \begin{itemize}
        \item 接受华法林治疗的患者中,\textbf{82\%}的HALT在连续影像学检查中消退
        \item 两种瓣膜类型(Sapien 3和Evolut)均有效
    \end{itemize}

    \item \textbf{出血安全性}
    \begin{itemize}
        \item 华法林治疗\textbf{未}增加出血风险(按VARC-2标准)
        \item 累积出血事件曲线:HALT+组 vs HALT-组,p=0.62
    \end{itemize}

    \item \textbf{临床意义}
    \begin{itemize}
        \item 对所有无HALT患者抗凝会增加出血风险,且无获益
        \item 提示\textbf{选择性抗凝}策略的重要性
        \item 需要影像学筛查来识别HALT
    \end{itemize}
\end{enumerate}

\subsubsection{HALT的临床重要性}

\textbf{问题}:HALT是无害的旁观者,还是有临床意义?

\textbf{证据1:病理学数据}(Sellers et al. JACC Imaging 2019)

\begin{itemize}
    \item 对取出的THV进行病理分析
    \item \textbf{发现:所有瓣膜都有血栓}
    \item 与CT报告的7-14\% HALT发生率不符
    \item 提示:\textbf{CT可能低估HALT的真实发生率}
\end{itemize}

\textbf{证据2:THV退化的阶梯式进展}

病理组织学分析显示:
\begin{itemize}
    \item \textbf{血栓}(Thrombus):植入早期(<60天)
    \item \textbf{纤维化}(Fibrosis):60天后
    \item \textbf{钙化}(Calcification):4年后
    \item 整个植入过程中,瓣叶厚度呈增加趋势
\end{itemize}

\textbf{证据3:Atlantis-4D研究}(Montalescot et al. JACC Interv 2022)

\textbf{主要结果}:

\begin{table}[h]
\centering
\caption{Atlantis-4D研究主要结果}
\label{tab:atlantis_4d_results}
\begin{tabular}{lcc}
\toprule
\textbf{终点} & \textbf{事件率/HR} & \textbf{P值} \\
\midrule
\multicolumn{3}{l}{\textbf{主要疗效终点(ITT)}} \\
Standard-of-Care & n=392 & \\
Apixaban & n=370 & \\
\midrule
血栓 & 25例 vs 19例 & \\
RLM 3-4或HALT 3-4 & 13例 vs 9例 & \\
HALT 3-4 & 11例 vs 8例 & \\
RLM 3-4 & 7例 vs 1例 & \\
\midrule
\multicolumn{3}{l}{\textbf{1年缺血/栓塞事件}} \\
RLM/HALT 3-4存在 & HR 1.58 (95\% CI 0.77-3.21) & \\
无RLM/HALT 3-4 & - & \\
\midrule
\multicolumn{3}{l}{\textbf{缺血事件定义}:复合死亡、心肌梗死、卒中或外周栓塞} \\
\bottomrule
\end{tabular}
\end{table}

\textbf{证据4:HALT与1年临床结果}(Fukui et al. Circulation 2022)

\begin{table}[h]
\centering
\caption{HALT与1年临床结果(N=565)}
\label{tab:halt_clinical_outcomes}
\begin{tabular}{lcccccc}
\toprule
\textbf{变量} & \textbf{所有患者} & \textbf{HALT} & \textbf{无HALT} & \textbf{未调整HR} & \textbf{调整后HR} & \textbf{P值} \\
 & \textbf{(n=565)} & \textbf{(n=108)} & \textbf{(n=457)} & \textbf{(95\% CI)} & \textbf{(95\% CI)} & \\
\midrule
全因死亡 & 40 (7\%) & 16 (15\%) & 24 (5\%) & 2.90 (1.54-5.46) & 2.98 (1.57-5.63)* & 0.001 \\
\midrule
心源性死亡 & 18 (3\%) & 9 (8\%) & 9 (2\%) & 4.29 (1.70-10.8) & 4.58 (1.81-11.6)† & 0.001 \\
\midrule
心衰住院 & 35 (6\%) & 10 (9\%) & 25 (6\%) & 1.77 (0.85-3.69) & 1.91 (0.91-4.02)* & 0.09 \\
\midrule
复合终点 & 66 (12\%) & 21 (19\%) & 45 (10\%) & 2.08 (1.24-3.49) & 1.94 (1.14-3.30)‡ & 0.02 \\
(全因死亡+心衰住院) & & & & & & \\
\midrule
心肌梗死 & 9 (2\%) & 6 (6\%) & 3 (1\%) & 4.10 (1.02-16.4) & - & <0.05 \\
\midrule
卒中/TIA & 21 (4\%) & 8 (7\%) & 13 (3\%) & 1.29 (0.54-3.13) & 1.27 (0.50-3.23)† & 0.61 \\
\midrule
出血事件 & 56 (10\%) & 11 (10\%) & 45 (10\%) & 1.07 (0.55-2.07) & 1.03 (0.53-2.00)* & 0.92 \\
\bottomrule
\end{tabular}
\end{table}

\textit{注释}:*调整年龄、性别和log STS-PROM评分;†调整log STS-PROM评分;‡调整年龄、性别、log STS-PROM评分、基线LV射血分数、TAVR后30天LV卒中容积指数

\textbf{关键结论}:
\begin{itemize}
    \item HALT患者1年全因死亡率增加3倍(HR 2.98)
    \item HALT患者1年心源性死亡率增加4.5倍(HR 4.58)
    \item HALT可能对患者\textbf{不是立即有害},但对\textbf{THV耐久性有意义}
\end{itemize}

\subsubsection{早期BVD的预测因素}

\textbf{研究}:Chedid et al. Can J Cardiol 2025 Sep 22:S0828-282X(25)01179-1

\textbf{研究设计}:
\begin{itemize}
    \item 2013-2022年间接受TAVR的患者:N=1,291
    \item 随访≥3年或TAVR后5年内发生BVD的患者:N=306
    \item 早期BVD(5年内):44例(14.3\%)
\end{itemize}

\textbf{与早期BVD相关的因素}:

\begin{table}[h]
\centering
\caption{早期BVD的危险因素}
\label{tab:early_bvd_factors}
\begin{tabular}{lcc}
\toprule
\textbf{因素} & \textbf{早期BVD组} & \textbf{无早期BVD组} \\
\midrule
HALT & 30\% & 5\% \\
小瓣膜尺寸(20-23 mm) & 57\% & 23\% \\
BMI >30 & 45\% & 29\% \\
\midrule
抗凝治疗 & 4.3\% & 20.6\% \\
\bottomrule
\end{tabular}
\end{table}

\textbf{早期BVD的临床结果}(p<0.0001):
\begin{itemize}
    \item 心衰住院率增加
    \item 再次干预(Redo-TAVR)率增加
    \item 早期死亡率增加
\end{itemize}

\textbf{关键结论}:
\begin{itemize}
    \item \textbf{HALT、小瓣膜尺寸、增加的BMI与更高的早期BVD率相关}
    \item \textbf{抗凝治疗对早期BVD有保护作用}
    \item 提示应对高危患者进行HALT筛查和考虑抗凝治疗
\end{itemize}

\subsubsection{支架欠扩张与临床结果}

\textbf{研究1:Acurate neo2瓣膜}(新闻报道,EuroPCR 2025)

\begin{itemize}
    \item 欠扩张的Acurate neo2瓣膜与更差结果相关
    \item Boston Scientific已停止该瓣膜的全球销售
    \item 失败原因:瓣膜扩张不足
\end{itemize}

\textbf{研究2:支架不对称性}(Krishnamoorthy et al. JACC Interv 2025)

\textbf{关键发现}:

\begin{table}[h]
\centering
\caption{TAVR不对称性与人工瓣膜功能障碍}
\label{tab:tavr_asymmetry}
\begin{tabular}{lccc}
\toprule
\textbf{TAVR扩张类型} & \textbf{直径A} & \textbf{直径B} & \textbf{人工瓣膜功能障碍率} \\
\midrule
对称性扩张 & 24.6 mm & 24.6 mm & 1.5\% (n=15/1,007) \\
不对称性扩张 & 30.1 mm & 25.5 mm & 24.4\% (n=51/209) \\
\bottomrule
\end{tabular}
\end{table}

\textbf{定义}:
\begin{itemize}
    \item 低不对称指数:≤5.5\%
    \item 高不对称指数:>5.5\%
\end{itemize}

\textbf{高TAVR不对称性指数的影响}:
\begin{itemize}
    \item 人工瓣膜性能受损(平均残余梯度≥20 mmHg 和/或 ≥中度瓣周漏)
    \item 出院前超声心动图即可显示
    \item 17\%的患者发生不对称性人工心脏瓣膜扩张
\end{itemize}

\textbf{TAVR不对称指数}:
\begin{itemize}
    \item 与受损的血流动力学瓣膜性能相关
    \item 与临床结果\textbf{无}关联(随访期内)
    \item 提示早期血流动力学改变可能在更长期随访中显现临床意义
\end{itemize}

\textbf{研究3:Redo-TAVR支架扩张}(Maznyczka et al. JACC Interv 2024)

\textbf{MDCT和透视评估Redo-TAVR后支架扩张}:

\begin{itemize}
    \item 研究期间:2023年1月-2025年4月
    \item 40例Redo-TAVR患者,30例进行了Redo-TAVR前后MDCT
\end{itemize}

\textbf{关键发现}:
\begin{itemize}
    \item \textbf{Index TAV(第一个瓣膜)功能区100\%欠扩张}
    \item Redo-TAVR后显著扩张
\end{itemize}

\textbf{临床意义}:
\begin{itemize}
    \item 即使透视下看起来扩张良好,功能区仍可能欠扩张
    \item MDCT对评估真实支架扩张至关重要
    \item 欠扩张可能是BVD的早期标志
\end{itemize}

\subsubsection{可改善的因素:基于影像的程序优化}

根据Fukui M, Cavalcante JL, Bapat VN. J Cardiol 2024 Jun;83(6):351-358,以下因素可通过影像学评估和程序优化来改善:

\textbf{TAVR in Native AS(原生主动脉瓣狭窄)}:

\begin{enumerate}
    \item \textbf{过大尺寸(Oversizing)}
    \begin{itemize}
        \item 过度的oversizing可能导致支架变形
        \item 需要平衡瓣周漏风险与支架变形风险
    \end{itemize}

    \item \textbf{钙分布(Calcium Distribution)}
    \begin{itemize}
        \item 不对称的钙化导致不均匀的支架扩张
        \item 钙化修饰技术的潜在作用
    \end{itemize}

    \item \textbf{瓣膜形态(二叶瓣 Bicuspid)}
    \begin{itemize}
        \item 二叶主动脉瓣解剖更复杂
        \item 可能需要不同的尺寸选择策略
    \end{itemize}

    \item \textbf{欠充盈(Underfilling - BEV)}
    \begin{itemize}
        \item 球囊扩张瓣膜充盈不足
        \item 导致功能区欠扩张
    \end{itemize}
\end{enumerate}

\textbf{Valve-in-Valve(瓣中瓣:主动脉、二尖瓣)}:

\begin{enumerate}
    \item \textbf{过大尺寸(Oversizing)}
    \begin{itemize}
        \item 在已有瓣膜环内,oversizing空间有限
        \item True ID(真实内径)测量至关重要
    \end{itemize}

    \item \textbf{植入深度(Implant Depth)}
    \begin{itemize}
        \item 影响功能区位置
        \item 影响血流动力学性能
    \end{itemize}
\end{enumerate}

% ============================================
% 结论
% ============================================
\subsection{结论}

\subsubsection{主要结论}

\begin{enumerate}
    \item \textbf{TAVR耐久性需要长期研究}
    \begin{itemize}
        \item 需要使用标准化的VARC-3定义
        \item 形态学改变先于血流动力学改变
        \item Stage 1 SVD是预测长期耐久性的关键
    \end{itemize}

    \item \textbf{HALT与支架变形的关键关联}
    \begin{itemize}
        \item HALT与支架框架变形相关
        \item 机制:欠扩张 → 风车样变形(pinwheeling) → 瓣叶应力 → 血栓、纤维化、增厚
        \item HALT是THV退化的早期标志
    \end{itemize}

    \item \textbf{预防优于治疗}
    \begin{itemize}
        \item 预防HALT优于治疗
        \item 虽然华法林有效(82\%消退率),但存在出血风险
        \item 理想策略是通过程序优化避免HALT发生
    \end{itemize}
\end{enumerate}

\subsubsection{如何预测或避免HALT?}

演讲提出了以下策略:

\begin{enumerate}
    \item \textbf{球囊预扩张/后扩张}
    \begin{itemize}
        \item 问题:TAVR是否应重新考虑球囊预扩张/后扩张以改善THV支架扩张?
        \item 当前趋势是减少球囊操作,但可能需要重新评估
        \item 个体化策略:对高钙化、不规则瓣环患者考虑
    \end{itemize}

    \item \textbf{更好的尺寸选择算法}
    \begin{itemize}
        \item 传统sizing可能过于简化
        \item 中间尺寸(Intermediate sizing)的瓣膜:MyVal, Braile, X4
        \item 3D影像学指导的sizing
    \end{itemize}

    \item \textbf{主动脉瓣钙化修饰技术}
    \begin{itemize}
        \item 钙化瓣叶的修饰
        \item 可能改善支架扩张的均匀性
        \item Shockwave等技术的潜在应用
    \end{itemize}

    \item \textbf{高危患者的HALT筛查}
    \begin{itemize}
        \item 支架不对称性患者
        \item 高钙化患者
        \item 激进oversizing患者
        \item 术后30天CT筛查
    \end{itemize}
\end{enumerate}

\subsubsection{无症状严重AS的特殊考虑}

\textbf{关键观点}:无症状严重AS患者的stakes更高

\begin{itemize}
    \item 这些患者通常更年轻
    \item 预期寿命更长
    \item 对瓣膜耐久性要求更高
\end{itemize}

\textbf{建议}:
\begin{enumerate}
    \item 程序优化至关重要
    \item 主动进行HALT筛查
    \item THV耐久性需要深入研究
    \item 可能需要更严格的瓣膜选择标准
\end{enumerate}

\subsubsection{未来方向}

\begin{enumerate}
    \item \textbf{新THV支架和瓣叶设计}
    \begin{itemize}
        \item 更好的支架扩张性能
        \item 减少变形的支架设计
        \item 更耐用的瓣叶材料
    \end{itemize}

    \item \textbf{影像学指导策略}
    \begin{itemize}
        \item 实现更好的THV支架扩张
        \item 实现层流(laminar flow)
        \item 最小化瓣叶应力
        \item 最终改善耐久性
    \end{itemize}

    \item \textbf{个体化治疗}
    \begin{itemize}
        \item 基于患者解剖的瓣膜选择
        \item 基于钙化分布的程序规划
        \item 基于风险分层的随访策略
    \end{itemize}
\end{enumerate}

% ============================================
% 临床启示
% ============================================
\subsection{临床启示}

\subsubsection{影像学在TAVR耐久性中的核心作用}

\textbf{1. 术前规划}

\begin{itemize}
    \item \textbf{CT测量}:
    \begin{itemize}
        \item 瓣环尺寸(多平面测量)
        \item 钙化分布和程度
        \item 主动脉根部几何形态
        \item 预测支架变形风险
    \end{itemize}

    \item \textbf{瓣膜选择}:
    \begin{itemize}
        \item 不仅基于瓣环直径
        \item 考虑钙化分布
        \item 考虑瓣膜类型(BEV vs SEV)
        \item 个体化oversizing策略
    \end{itemize}
\end{itemize}

\textbf{2. 术中指导}

\begin{itemize}
    \item 融合影像(Fusion imaging)
    \item 实时评估支架扩张
    \item 决策球囊后扩张
\end{itemize}

\textbf{3. 术后评估}

\begin{itemize}
    \item \textbf{30天CT}(关键时间点):
    \begin{itemize}
        \item 评估支架扩张
        \item 筛查HALT
        \item 测量支架几何参数(变形指数、偏心度等)
    \end{itemize}

    \item \textbf{长期随访CT}:
    \begin{itemize}
        \item 监测HALT演变
        \item 早期发现结构性瓣膜退化
        \item 指导抗凝决策
    \end{itemize}
\end{itemize}

\subsubsection{对TAVR程序的实践建议}

\begin{enumerate}
    \item \textbf{瓣膜选择}
    \begin{itemize}
        \item 避免过度aggressive的oversizing
        \item 考虑中间尺寸瓣膜
        \item 对严重、不对称钙化患者,考虑钙化修饰
    \end{itemize}

    \item \textbf{植入技术}
    \begin{itemize}
        \item 对BEV,确保充分充盈
        \item 考虑球囊后扩张,特别是:
        \begin{itemize}
            \item 支架欠扩张(透视或TEE提示)
            \item 严重钙化
            \item 残余跨瓣压差高
        \end{itemize}
        \item 优化植入深度
    \end{itemize}

    \item \textbf{术后管理}
    \begin{itemize}
        \item 对高危患者(小瓣膜、高钙化、不对称扩张),术后30天CT筛查
        \item HALT阳性患者考虑抗凝治疗
        \item 建立长期影像学随访计划
    \end{itemize}
\end{enumerate}

\subsubsection{对无症状AS患者的特殊建议}

\begin{enumerate}
    \item \textbf{更严格的瓣膜选择}
    \begin{itemize}
        \item 优先选择耐久性数据更好的瓣膜
        \item 避免可能影响耐久性的因素(如过度oversizing)
    \end{itemize}

    \item \textbf{程序优化}
    \begin{itemize}
        \item 追求最优支架扩张
        \item 更积极地使用球囊后扩张
        \item 术后CT评估应作为常规
    \end{itemize}

    \item \textbf{主动监测}
    \begin{itemize}
        \item 定期CT随访筛查HALT
        \item 超声心动图监测血流动力学
        \item 早期识别Stage 1 SVD
    \end{itemize}
\end{enumerate}

\subsubsection{抗凝治疗的个体化策略}

基于现有证据:

\begin{table}[h]
\centering
\caption{TAVR术后抗凝治疗的建议策略}
\label{tab:anticoagulation_strategy}
\begin{tabular}{p{4cm}p{10cm}}
\toprule
\textbf{患者群体} & \textbf{建议} \\
\midrule
所有TAVR患者 & \textbf{不}推荐常规抗凝(增加出血风险,无明确获益) \\
\midrule
高危患者 & 术后30天CT筛查HALT \\
(小瓣膜、高钙化、 & - HALT阴性:常规抗血小板治疗 \\
不对称扩张、BMI>30) & - HALT阳性:考虑华法林抗凝 \\
\midrule
已确诊HALT患者 & 华法林抗凝(82\%消退率,无增加出血风险) \\
 & 3-6个月后复查CT评估HALT是否消退 \\
\midrule
HALT消退后 & 可考虑停止抗凝,继续监测 \\
\midrule
房颤等其他抗凝指征 & 按指南常规抗凝 \\
\bottomrule
\end{tabular}
\end{table}

\subsubsection{对中国临床实践的启示}

\begin{enumerate}
    \item \textbf{建立CT随访体系}
    \begin{itemize}
        \item 目前国内TAVR术后CT随访不普遍
        \item 建议对选定患者群体进行CT筛查
        \item 特别是无症状AS、年轻患者
    \end{itemize}

    \item \textbf{重视程序优化}
    \begin{itemize}
        \item 不仅追求手术成功
        \item 更要追求最优支架扩张
        \item 建立术后CT质控机制
    \end{itemize}

    \item \textbf{多学科协作}
    \begin{itemize}
        \item 心脏影像医师的深度参与
        \item 不仅是术前测量,更要术后评估
        \item 建立影像-介入-随访闭环
    \end{itemize}

    \item \textbf{数据库建设}
    \begin{itemize}
        \item 收集长期随访数据
        \item 建立中国人群的HALT发生率和预后数据
        \item 为临床决策提供本土证据
    \end{itemize}
\end{enumerate}

% ============================================
% 研究局限性
% ============================================
\subsection{研究局限性}

\begin{enumerate}
    \item \textbf{文献类型局限}
    \begin{itemize}
        \item 本文是会议演讲,综合了多项研究
        \item 不是单一原始研究
        \item 数据来源于多个研究,异质性较大
    \end{itemize}

    \item \textbf{HALT检测的局限性}
    \begin{itemize}
        \item CT对HALT的检测可能不够敏感
        \item 病理研究显示所有瓣膜都有血栓,但CT仅检出部分
        \item HALT分级(<25\%, 25-50\%等)的临床意义仍不完全清楚
        \item 轻度HALT可能自行消退,是否需要干预存疑
    \end{itemize}

    \item \textbf{随访时间局限}
    \begin{itemize}
        \item 大多数研究随访时间为1-3年
        \item TAVR耐久性是10-20年的问题
        \item 早期HALT与远期耐久性的关系仍需长期数据
        \item 目前无法确定哪些Stage 1 SVD最终会进展为临床相关的BVF
    \end{itemize}

    \item \textbf{抗凝治疗的不确定性}
    \begin{itemize}
        \item 华法林消退HALT的效果来自观察性研究
        \item 最佳抗凝方案(华法林 vs DOAC)不明确
        \item 最佳抗凝持续时间未知
        \item HALT消退后是否改善长期预后未得到证实
    \end{itemize}

    \item \textbf{程序优化策略缺乏RCT证据}
    \begin{itemize}
        \item 球囊后扩张、钙化修饰等策略主要基于观察和机制推理
        \item 缺乏前瞻性随机对照试验
        \item 不同瓣膜类型可能需要不同策略
    \end{itemize}

    \item \textbf{支架测量的标准化问题}
    \begin{itemize}
        \item 变形指数、偏心度等参数的测量方法不完全统一
        \item 阈值cutoff值(如不对称指数5.5\%)需要更多验证
        \item CT扫描方案、重建方法可能影响测量
    \end{itemize}

    \item \textbf{瓣膜类型差异}
    \begin{itemize}
        \item 不同研究包含不同比例的BEV和SEV
        \item 新一代瓣膜的数据有限
        \item 研究结果的普适性需要验证
    \end{itemize}

    \item \textbf{患者选择偏倚}
    \begin{itemize}
        \item 接受CT随访的患者可能不代表所有TAVR患者
        \item 高危患者、肾功能不全患者可能未接受CT
        \item 可能低估真实世界的HALT发生率
    \end{itemize}
\end{enumerate}

% ============================================
% 个人笔记
% ============================================
\subsection{个人笔记}

\subsubsection{关键数字记忆}

\begin{description}
    \item[HALT发生率] ~
    \begin{itemize}
        \item TAVR 30天:13-19\%
        \item TAVR 1年:28-31\%
        \item 外科瓣膜30天:5\%;1年:20\%
    \end{itemize}

    \item[华法林治疗效果] HALT消退率82\%,无增加出血风险

    \item[HALT的临床影响(1年)] ~
    \begin{itemize}
        \item 全因死亡:HR 2.98(15\% vs 5\%)
        \item 心源性死亡:HR 4.58(8\% vs 2\%)
        \item 复合终点:HR 1.94(19\% vs 10\%)
    \end{itemize}

    \item[早期BVD] ~
    \begin{itemize}
        \item 5年内发生率:14.3\%
        \item HALT存在率:早期BVD组30\% vs 无BVD组5\%
    \end{itemize}

    \item[支架不对称性] ~
    \begin{itemize}
        \item 发生率:17\%
        \item 高不对称(>5.5\%):人工瓣膜功能障碍24.4\%
        \item 低不对称(≤5.5\%):人工瓣膜功能障碍1.5\%
    \end{itemize}

    \item[Atlantis-4D] 有症状HVD 3年事件率:HALT+ 9.4\% vs HALT- 1.5\%(HR 6.10)
\end{description}

\subsubsection{重要概念}

\begin{description}
    \item[VARC-3定义] 生物人工瓣膜退化(BVD)和失败(BVF)的标准化定义,强调形态学改变(Stage 1)先于血流动力学改变(Stages 2-3)

    \item[HALT] Hypoattenuated Leaflet Thickening(低密度瓣叶增厚),CT上表现为瓣叶低密度增厚,提示亚临床瓣叶血栓形成

    \item[Stage 1 SVD] 结构性瓣膜退化的早期阶段,已有瓣膜固有的永久性结构改变,但尚未出现血流动力学恶化

    \item[THV变形三联征] 欠扩张(Underexpansion)→ 风车样变形(Pinwheeling)→ 瓣叶应力增加

    \item[THV退化阶梯] 血栓(早期)→ 纤维化(60天后)→ 钙化(4年后)

    \item[变形指数] Deformation Index = (最大直径 - 最小直径) / (2 × 最小直径),评估支架圆度

    \item[新生窦容积指数] Neo-sinus Volume Index,评估瓣叶运动空间

    \item[不对称性指数] TAVR Asymmetry Index,评估支架不同方向扩张的差异,>5.5\%为高不对称

    \item[预防优于治疗] 虽然华法林可消退HALT,但通过程序优化避免HALT发生是更好的策略
\end{description}

\subsubsection{机制理解}

\textbf{HALT形成的病理生理机制}:

\begin{enumerate}
    \item \textbf{起始因素}:支架变形
    \begin{itemize}
        \item 原因:欠扩张、不对称钙化、过度oversizing
        \item 结果:功能区几何形态异常
    \end{itemize}

    \item \textbf{血流动力学改变}
    \begin{itemize}
        \item 支架变形→非层流(湍流、涡流)
        \item 瓣叶运动受限或不协调
        \item 瓣叶应力和应变增加
    \end{itemize}

    \item \textbf{血栓形成}
    \begin{itemize}
        \item 血流停滞
        \item 瓣叶损伤暴露血栓原性表面
        \item 瓣叶表面血栓沉积(HALT)
    \end{itemize}

    \item \textbf{组织反应}
    \begin{itemize}
        \item 血栓机化→纤维化(60天)
        \item 持续应力→进一步组织损伤
        \item 钙化沉积(4年)
    \end{itemize}

    \item \textbf{临床表现}
    \begin{itemize}
        \item 早期:无血流动力学改变(Stage 1 SVD)
        \item 中期:中度血流动力学恶化(Stage 2 SVD)
        \item 晚期:严重血流动力学恶化(Stage 3 SVD)→ BVF
    \end{itemize}
\end{enumerate}

\subsubsection{临床思考}

\textbf{1. 为什么支架扩张如此重要?}

\begin{itemize}
    \item 这是我们\textbf{可以控制}的因素
    \item 通过影像学可以\textbf{评估和优化}
    \item 直接影响HALT发生和THV耐久性
    \item 对长期预后有深远影响
\end{itemize}

\textbf{2. CT在TAVR中的角色演变}

\begin{itemize}
    \item 过去:主要用于术前测量(瓣环尺寸、血管通路)
    \item 现在:扩展至术后质控(支架扩张、HALT筛查)
    \item 未来:可能成为常规随访工具(耐久性监测)
\end{itemize}

\textbf{3. 如何平衡oversizing的利弊?}

\begin{itemize}
    \item 过度oversizing:减少瓣周漏,但增加支架变形和HALT风险
    \item 不足oversizing:增加瓣周漏风险
    \item 个体化策略:基于钙化分布、瓣环几何、瓣膜类型
\end{itemize}

\textbf{4. 哪些患者最需要HALT筛查?}

基于现有证据,以下患者应考虑术后30天CT:
\begin{itemize}
    \item 年轻患者(预期寿命长)
    \item 无症状AS患者
    \item 小瓣膜尺寸(20-23 mm)
    \item 严重或不对称钙化
    \item BMI>30
    \item 术中支架扩张不理想
    \item 术后残余跨瓣压差高
\end{itemize}

\textbf{5. 球囊后扩张:何时、如何使用?}

\begin{itemize}
    \item \textbf{何时考虑}:
    \begin{itemize}
        \item 透视或TEE提示支架欠扩张
        \item 严重、不对称钙化
        \item 残余跨瓣压差>20 mmHg
        \item ≥中度瓣周漏(非位置相关)
    \end{itemize}

    \item \textbf{如何使用}:
    \begin{itemize}
        \item 选择合适尺寸(通常与支架标称直径相当或略小)
        \item 缓慢充盈,避免瓣环损伤
        \item 即刻评估效果(压差、瓣周漏、支架位置)
    \end{itemize}

    \item \textbf{风险}:
    \begin{itemize}
        \item 瓣环破裂
        \item 瓣膜位移
        \item 冠脉阻塞
        \item 需要权衡利弊
    \end{itemize}
\end{itemize}

\subsubsection{对未来研究的思考}

\begin{enumerate}
    \item \textbf{需要的RCT}:
    \begin{itemize}
        \item 常规球囊后扩张 vs 选择性球囊后扩张
        \item HALT筛查+抗凝 vs 常规治疗
        \item 不同瓣膜类型的耐久性比较(head-to-head)
    \end{itemize}

    \item \textbf{影像学技术发展}:
    \begin{itemize}
        \item 更敏感的HALT检测方法(4D Flow MRI?)
        \item AI辅助的支架扩张评估
        \item 实时术中支架扩张评估
    \end{itemize}

    \item \textbf{瓣膜设计改进}:
    \begin{itemize}
        \item 更好的径向支撑力
        \item 减少变形的支架几何设计
        \item 更耐久的瓣叶材料
        \item 抗血栓涂层
    \end{itemize}

    \item \textbf{风险预测模型}:
    \begin{itemize}
        \item 整合患者因素、解剖因素、程序因素
        \item 预测HALT风险
        \item 预测长期耐久性
        \item 指导个体化治疗决策
    \end{itemize}
\end{enumerate}

\subsubsection{对中国TAVR发展的启示}

\begin{enumerate}
    \item \textbf{不能只追求短期成功率}
    \begin{itemize}
        \item 手术成功只是起点
        \item 长期耐久性才是终点
        \item 需要建立长期随访体系
    \end{itemize}

    \item \textbf{影像学能力建设}
    \begin{itemize}
        \item 培养心脏CT专业人才
        \item 不仅会测量,更要会解读术后CT
        \item 影像医师深度参与TAVR团队
    \end{itemize}

    \item \textbf{数据库的重要性}
    \begin{itemize}
        \item 建立中国TAVR注册研究
        \item 收集术后CT数据
        \item 长期随访(10年+)
        \item 为临床决策提供本土证据
    \end{itemize}

    \item \textbf{适应症扩展需谨慎}
    \begin{itemize}
        \item 无症状AS、年轻患者对耐久性要求更高
        \item 需要更严格的程序优化
        \item 需要更密切的长期监测
        \item 在证据充分前谨慎推进
    \end{itemize}

    \item \textbf{多中心协作}
    \begin{itemize}
        \item 单中心样本量有限
        \item 需要多中心合作研究
        \item 分享经验和数据
        \item 共同提高TAVR质量
    \end{itemize}
\end{enumerate}

\subsubsection{值得记忆的金句}

\begin{enumerate}
    \item \textbf{"Morphological Changes (Stage 1) PRECEDE Hemodynamic Changes (Stages 2-3)"}
    \begin{itemize}
        \item 形态学改变先于血流动力学改变
        \item 提示早期影像学监测的重要性
    \end{itemize}

    \item \textbf{"HALT is not a binary process, but rather a graded spectrum"}
    \begin{itemize}
        \item HALT不是二元过程,而是分级谱
        \item 提示需要细化评估和分层管理
    \end{itemize}

    \item \textbf{"It is better to prevent HALT than to treat it"}
    \begin{itemize}
        \item 预防HALT优于治疗
        \item 提示程序优化的核心地位
    \end{itemize}

    \item \textbf{"Perhaps not immediately harmful to the patient – but meaningful to the THV durability"}
    \begin{itemize}
        \item HALT可能对患者不是立即有害,但对THV耐久性有意义
        \item 提示长期视角的重要性
    \end{itemize}

    \item \textbf{"Stakes should be higher in asymptomatic severe AS"}
    \begin{itemize}
        \item 无症状严重AS中stakes更高
        \item 提示需要更高的程序标准
    \end{itemize}

    \item \textbf{"Deformation at functional portion is crucial"}
    \begin{itemize}
        \item 功能部分的变形至关重要
        \item 提示评估重点
    \end{itemize}
\end{enumerate}

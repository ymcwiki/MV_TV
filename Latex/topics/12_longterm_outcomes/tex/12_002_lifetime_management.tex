\section{重度主动脉瓣狭窄的终生管理考虑}
\label{sec:12_002_lifetime_management}

% ============================================
% 文献信息
% ============================================
\subsection{文献信息}

\begin{itemize}
    \item \textbf{标题}: Considerations for Lifetime Management of Severe Aortic Stenosis
    \item \textbf{作者}: Aakriti Gupta, MD MS
    \item \textbf{机构}: Interventional Cardiology, Cedars-Sinai Medical Center, Los Angeles; Executive Associate Editor, JACC
    \item \textbf{会议}: TCT (Transcatheter Cardiovascular Therapeutics)
    \item \textbf{日期}: October 28, 2025
    \item \textbf{PDF文件名}: considerations-for-lifetime-management-of-severe-aortic-stenosis.pdf
    \item \textbf{文献类型}: 会议演讲/专家讲座
    \item \textbf{利益冲突声明}: Edwards Lifesciences \& Boston Scientific的顾问/讲者
\end{itemize}

\subsection{研究背景}

\subsubsection{TAVR在年轻患者中的应用趋势}

随着TAVR技术的成熟和适应证的扩展,年轻患者接受TAVR的比例显著增加:

\textbf{关键数据}(来源:Gupta Aakriti et al. JACC. 2021; Sharma T et al. JACC 2022):
\begin{itemize}
    \item \textbf{2021年美国约50\%的<65岁患者接受TAVR}
    \item TAVR手术量从2012年到2019年持续增长
    \item SAVR(单纯外科主动脉瓣置换术)手术量逐渐下降
    \item 总体AVR手术量(SAVR + TAVR)持续增加
\end{itemize}

\subsubsection{终生管理的重要性}

对于年轻患者(如图示的65岁Jim、75岁Bill、85岁Sam),需要考虑:
\begin{itemize}
    \item 患者可能需要多次瓣膜干预
    \item 初次瓣膜选择将影响未来治疗选项
    \item 需要制定10-20年的长期治疗计划
    \item 治疗策略排序(Therapy Sequencing)成为关键考虑因素
\end{itemize}

\subsection{主要研究发现}

\subsubsection{治疗策略排序的ABCD框架}

根据Windecker et al, Eur Heart J. 2022的治疗策略排序概念,决策需考虑四个维度:

\textbf{A - Anatomical(解剖学因素)}:
\begin{itemize}
    \item 二叶主动脉瓣形态
    \item 窦管尺寸
    \item 冠状动脉开口高度
    \item 主动脉根部解剖
    \item 主动脉扩张情况
\end{itemize}

\textbf{B - Behavioral(患者偏好)}:
\begin{itemize}
    \item 患者价值观
    \item 对手术的接受程度
    \item 预期寿命期望
    \item 生活方式需求
\end{itemize}

\textbf{C - Clinical(临床因素)}:
\begin{itemize}
    \item 合并症
    \item 衰弱程度
    \item 冠状动脉疾病
    \item 多瓣膜病变
\end{itemize}

\textbf{D - Durability(耐久性)}:
\begin{itemize}
    \item 10-20年的长期计划
    \item 瓣膜耐久性预期
    \item 再次干预的可行性
\end{itemize}

\subsubsection{解剖学考虑:二叶主动脉瓣(BAV)}

\textbf{1. BAV患者的主动脉病变风险}

Yoon, S.-H. et al. J Am Coll Cardiol. 2020;76(9):1018-30研究显示:

\begin{table}[h]
\centering
\caption{二叶瓣患者按形态学特征分类的全因死亡率}
\label{tab:bav_morphology_mortality}
\begin{tabular}{lc}
\toprule
\textbf{形态学特征} & \textbf{2年全因死亡率} \\
\midrule
无钙化瓣叶或过多小叶钙化 & 31.3\% \\
钙化瓣叶或过多小叶钙化 & 42.6\% \\
钙化瓣叶+过多小叶钙化 & 26.0\% \\
\bottomrule
\end{tabular}
\end{table}

关键发现(p<0.001 log-rank):
\begin{itemize}
    \item 无钙化:4.6\%(180天)→ 9.5\%(540天)
    \item 钙化瓣叶或过多小叶钙化:3.8\%(180天)→ 13.6\%(360天)→ 25.7\%(720天)
    \item 钙化瓣叶+过多小叶钙化:最高死亡率
\end{itemize}

\textbf{2. BAV患者TAVR的1年结果令人鼓舞}

Makkar R et al. JAMA. 2021;326(11):1034-1044报告的STS-TVT注册研究(3168对倾向匹配对):

\begin{table}[h]
\centering
\caption{BAV vs TAV患者TAVR后1年结果}
\label{tab:bav_vs_tav_outcomes}
\begin{tabular}{lcc}
\toprule
\textbf{结局指标} & \textbf{二叶瓣} & \textbf{三叶瓣} \\
\midrule
1年死亡率 & 4.6\% & 6.7\% \\
死亡HR (95\% CI) & \multicolumn{2}{c}{0.75 (0.55-1.02), P=0.06} \\
1年卒中率 & 1.8\% & 2.2\% \\
卒中HR (95\% CI) & \multicolumn{2}{c}{1.03 (0.69-1.53), P=0.89} \\
\bottomrule
\end{tabular}
\end{table}

\textbf{结论}:BAV患者TAVR后1年死亡率和卒中率与TAV患者相似,结果令人鼓舞。

\textbf{3. TAVR vs SAVR在BAV中的对比研究}

根据两个注册数据库的比较(Makkar R et al. JAMA. 2021; Hirji et al. Ann Thorac Surg. 2023):

\begin{table}[h]
\centering
\caption{BAV患者TAVR vs SAVR基线特征和结局}
\label{tab:tavr_vs_savr_bav}
\begin{tabular}{lcc}
\toprule
\textbf{指标} & \textbf{TAVR (TVT)} & \textbf{SAVR (STS)} \\
\midrule
\multicolumn{3}{c}{\textit{基线特征}} \\
平均年龄 & 69岁 & 70岁 \\
平均STS评分 & 1.7 & 1.28\% \\
NYHA III/IV & 55.1\% & 18.8\% \\
\midrule
\multicolumn{3}{c}{\textit{死亡率结局}} \\
30天死亡率 & 0.9\% & 1.3\% \\
30天卒中率 & 1.4\% & 1.2\% \\
1年死亡率 & 4.6\% & 3.2\% \\
\bottomrule
\end{tabular}
\end{table}

\begin{table}[h]
\centering
\caption{BAV患者TAVR vs SAVR并发症比较}
\label{tab:tavr_vs_savr_complications}
\begin{tabular}{lcc}
\toprule
\textbf{并发症} & \textbf{TAVR} & \textbf{SAVR} \\
\midrule
永久起搏器 & 6.2\% & 5.8\% \\
新发房颤 & 1.0\% & 36.6\% \\
肾脏并发症 & 0.1\%(新透析) & 1.1\%(急性肾衰) \\
再次手术/第二个瓣膜 & 0.1\% & 3.4\% \\
\bottomrule
\end{tabular}
\end{table}

\textbf{关键观察}:
\begin{itemize}
    \item TAVR组新发房颤率显著低于SAVR组(1.0\% vs 36.6\%)
    \item TAVR组需要再次干预的比例极低(0.1\%)
    \item 两组永久起搏器植入率相似(6.2\% vs 5.8\%)
    \item TAVR组NYHA III/IV比例更高(55.1\% vs 18.8\%),提示症状更重
\end{itemize}

\textbf{4. NOTION试验:BAV vs TAV队列比较}

Ole De Backer et al. EHJ 2024报告的NOTION试验结果:

\textbf{主要终点}:死亡、卒中或心衰住院的复合终点

三叶瓣队列(n=270):
\begin{itemize}
    \item TAVI:8.7\%(12个月)
    \item 外科手术:8.3\%(12个月)
    \item HR 1.0 (95\% CI: 0.5-2.3), P=0.9
\end{itemize}

二叶瓣队列(n=100):
\begin{itemize}
    \item TAVI:14.3\%(12个月)
    \item 外科手术:3.9\%(12个月)
    \item HR 3.8 (95\% CI: 0.8-18.5), P=0.07
\end{itemize}

\textbf{解读}:
\begin{itemize}
    \item 三叶瓣患者TAVI与外科手术结果相当
    \item 二叶瓣患者TAVI结果趋向较差(虽未达统计学显著性)
    \item 二叶瓣样本量较小(n=100),需要更大规模研究
\end{itemize}

\subsubsection{解剖学考虑:TAV-in-TAV可行性}

Tarantini G et al. EuroIntervention 2023;19:37-52研究了TAVI失败后TAV-in-TAV的可行性:

\textbf{影响TAV-in-TAV可行性的解剖学因素}:

\begin{enumerate}
    \item \textbf{冠状动脉开口位置}:
    \begin{itemize}
        \item 冠状动脉开口\textbf{高于}新瓣膜支架:TAV-in-TAV可行性好
        \item 冠状动脉开口\textbf{低于}新瓣膜支架:
        \begin{itemize}
            \item 短支架THV:较好
            \item 高支架THV:风险增加
        \end{itemize}
    \end{itemize}

    \item \textbf{窦管交界(STJ)宽度}:
    \begin{itemize}
        \item \textbf{宽STJ}:TAV-in-TAV可行性差(即使用短支架THV)
        \item \textbf{窄STJ}:TAV-in-TAV可行性好
    \end{itemize}
\end{enumerate}

\textbf{临床意义}:
\begin{itemize}
    \item 初次TAVR时应评估未来TAV-in-TAV的可行性
    \item 冠状动脉开口高度是关键决定因素
    \item STJ宽度影响瓣膜支架的展开和密封
    \item 短支架瓣膜在TAV-in-TAV场景中优势明显
\end{itemize}

\subsubsection{解剖学考虑:瓣叶修饰技术}

Dvir D et al. European Heart Journal (2024) 00, 1–11介绍了新兴的瓣叶修饰技术:

\textbf{技术目的}:
\begin{itemize}
    \item 在TAV-in-TAV前修饰原有瓣膜瓣叶
    \item 优化冠状动脉开口通路
    \item 改善新瓣膜定位和功能
\end{itemize}

\textbf{技术类型}(根据图示):
\begin{enumerate}
    \item 瓣叶切割/撕裂
    \item 瓣叶锚定
    \item 瓣叶劈开和支架放置
\end{enumerate}

这些技术仍在研发阶段,可能为未来TAV-in-TAV提供更多选择。

\subsubsection{重做TAVR的可行性和安全性}

Makkar R....Gupta A et al. Lancet 2023; 402: 1529–40报告的STS-TVT注册研究倾向匹配分析:

\textbf{研究设计}:
\begin{itemize}
    \item 重做TAVR组(n=1320)vs 原生TAVR组(n=1320)
    \item 倾向评分匹配
\end{itemize}

\textbf{主要结局}:

\begin{table}[h]
\centering
\caption{重做TAVR vs 原生TAVR的1年结局}
\label{tab:redo_tavr_outcomes}
\begin{tabular}{lcc}
\toprule
\textbf{结局指标} & \textbf{HR (95\% CI)} & \textbf{P值} \\
\midrule
全因死亡 & 0.94 (0.77-1.16) & 0.57 \\
卒中 & 0.94 (0.60-1.49) & 0.80 \\
\bottomrule
\end{tabular}
\end{table}

\begin{table}[h]
\centering
\caption{重做TAVR vs 原生TAVR的手术结局}
\label{tab:redo_tavr_procedural}
\begin{tabular}{lccc}
\toprule
\textbf{手术结局} & \textbf{重做TAVR} & \textbf{原生TAVR} & \textbf{P值} \\
\midrule
术中死亡 & 0.6\% (8/1320) & 0.2\% (3/1320) & 0.23 \\
需要心肺旁路 & 0.9\% (11/1275) & 0.6\% (8/1282) & 0.48 \\
转为开心手术 & 0.5\% (6/1319) & 0.2\% (2/1320) & 0.18 \\
环形撕裂 & 0.2\% (2/1320) & 0.1\% (1/1320) & 1.00 \\
主动脉夹层 & 0.2\% (3/1320) & 0.1\% (2/1320) & 1.00 \\
冠状动脉压迫/阻塞 & 0.3\% (4/1320) & 0.1\% (1/1320) & 0.37 \\
器械栓塞 & 0.1\% (1/1320) & 0.2\% (3/1320) & 0.62 \\
穿孔伴/不伴填塞 & 0.5\% (7/1320) & 0.4\% (5/1320) & 0.56 \\
\bottomrule
\end{tabular}
\end{table}

\textbf{关键发现}:
\begin{itemize}
    \item 重做TAVR的1年死亡率和卒中率与原生TAVR相似(19.0\% vs 17.5\%死亡;3.5\% vs 3.2\%卒中)
    \item 所有手术并发症发生率在两组间无显著差异
    \item 冠状动脉压迫/阻塞风险略高但无统计学差异(0.3\% vs 0.1\%, P=0.37)
    \item 重做TAVR在现实世界中是可行和安全的
\end{itemize}

\subsubsection{患者偏好(Behavioral因素)}

\textbf{选择TAVR-first策略的患者理由}:

\begin{enumerate}
    \item \textbf{需要快速恢复}:
    \begin{itemize}
        \item "我必须照顾我的丈夫,需要快速恢复"
        \item TAVR恢复期更短
    \end{itemize}

    \item \textbf{拒绝手术}:
    \begin{itemize}
        \item "无论如何,我绝不想要手术!"
        \item 对开心手术的恐惧或拒绝
    \end{itemize}

    \item \textbf{有限的预期寿命期望}:
    \begin{itemize}
        \item "我不想活到100岁,能活到80岁就很好"
        \item 对长期耐久性要求较低
    \end{itemize}

    \item \textbf{技术进步的期望}:
    \begin{itemize}
        \item "技术会在未来十年发展,我可以再做2次TAVR!"
        \item 对未来治疗选择持乐观态度
    \end{itemize}

    \item \textbf{职业/生活需求}:
    \begin{itemize}
        \item "我需要回去拍我的电影"
        \item 需要尽快恢复工作或活动
    \end{itemize}
\end{enumerate}

\textbf{选择SAVR-first策略的患者理由}:

\begin{enumerate}
    \item \textbf{"一次性解决"心态}:
    \begin{itemize}
        \item "如果我迟早需要手术,不如现在就做"
        \item 希望避免未来多次干预
    \end{itemize}

    \item \textbf{对长期数据的信心}:
    \begin{itemize}
        \item "SAVR已经存在更长时间,耐久性的确定性更高"
        \item 更倾向于成熟技术
    \end{itemize}
\end{enumerate}

\textbf{临床启示}:
\begin{itemize}
    \item 患者偏好在决策中至关重要
    \item 需要充分的共同决策(shared decision-making)
    \item 医生应了解患者的价值观、生活目标和期望
    \item 没有"一刀切"的策略,需要个体化
\end{itemize}

\subsubsection{临床因素:多瓣膜病变}

Windecker et al, Eur Heart J. 2022;43(29):2729-2750综述了多瓣膜病变的管理:

\textbf{合并瓣膜病变的患病率和治疗选择}:

\begin{table}[h]
\centering
\caption{主动脉瓣狭窄合并其他瓣膜病变}
\label{tab:multivalvular_disease}
\begin{tabular}{lcc}
\toprule
\textbf{瓣膜病变} & \textbf{患病率} & \textbf{主要治疗选择} \\
\midrule
二尖瓣反流 & 20-30\% & SAVR+二尖瓣修复/置换 \\
 &  & TAVI+经导管二尖瓣缘对缘修复 \\
 &  & TAVI+经导管二尖瓣置换 \\
\midrule
二尖瓣狭窄 & 2-3\% & SAVR+二尖瓣置换(交界切开) \\
 &  & TAVI+经皮二尖瓣交界切开 \\
 &  & TAVI+经导管二尖瓣置换 \\
\midrule
三尖瓣反流 & 10-25\% & SAVR+三尖瓣修复/置换 \\
 &  & TAVI+经导管三尖瓣缘对缘修复 \\
 &  & TAVI+经导管三尖瓣成形术/置换 \\
\bottomrule
\end{tabular}
\end{table}

\textbf{决策考虑}:
\begin{itemize}
    \item \textbf{原发性病变}:SAVR通常作为首选(可同时处理多个瓣膜)
    \item \textbf{继发性病变}:TAVI可能合适,联合或分期经导管治疗
    \item 需要心脏团队讨论
    \item 考虑每个瓣膜病变的严重程度和血流动力学影响
\end{itemize}

\subsubsection{临床因素:合并冠状动脉疾病}

Windecker et al, Eur Heart J. 2022;43(29):2729-2750提供的决策算法:

\begin{table}[h]
\centering
\caption{AS合并CAD的治疗策略}
\label{tab:cad_management}
\begin{tabular}{lccc}
\toprule
\textbf{因素} & \textbf{年龄65岁} & \textbf{年龄70-75岁} & \textbf{年龄>80岁} \\
\midrule
手术风险 & 低 & 中危 & 高危 \\
\midrule
CAD严重程度 & 3支病变\& & 3支病变\& & 1-2支病变 \\
 & SYNTAX>22 & SYNTAX≤22 & SYNTAX≤22 \\
 & 左主干\& & 左主干\& &  \\
 & SYNTAX>32 & SYNTAX≤32 &  \\
\midrule
糖尿病 & 是 & — & 否 \\
\midrule
TAVI后冠状 & 敌对的 & 中等的 & 有利的 \\
动脉通路 &  &  &  \\
\midrule
\textbf{推荐} & \textbf{1st: SAVR+CABG} & \textbf{SAVR+CABG} & \textbf{1st: TAVI+PCI} \\
 & \textbf{2nd: TAVI+PCI} & \textbf{或TAVI+PCI} & \textbf{2nd: SAVR+CABG} \\
\bottomrule
\end{tabular}
\end{table}

\textbf{决策要点}:
\begin{itemize}
    \item 年轻患者(<65岁)+ 复杂CAD(3支病变/左主干)+ 糖尿病 → SAVR+CABG优先
    \item 高龄患者(>80岁)+ 简单CAD → TAVI+PCI优先
    \item 中危患者:需个体化,两种策略均可
    \item TAVI后冠状动脉通路的可行性影响决策
\end{itemize}

\subsubsection{耐久性(Durability)}

\textbf{1. 指南中的假设和现实}

Otto CM, et al. Circulation. 2020;143:e72-e227中的假设:

\begin{table}[h]
\centering
\caption{2020 AHA/ACC指南关于瓣膜耐久性的假设}
\label{tab:durability_assumptions}
\begin{tabular}{ll}
\toprule
\textbf{假设} & \textbf{评论} \\
\midrule
外科瓣膜耐久性至少>10年 & 基于长期随访数据 \\
经导管瓣膜耐久性<10年 & 当时TAVR随访数据有限 \\
所有外科瓣膜耐久性相等 & \textcolor{red}{不正确!不同瓣膜差异显著} \\
所有经导管瓣膜耐久性相等 & \textcolor{red}{不正确!不同瓣膜差异显著} \\
指南规划患者达到约85岁 & 基于平均预期寿命 \\
\bottomrule
\end{tabular}
\end{table}

\textbf{2. PARTNER 3试验7年数据}

Leon MB....Makkar R. N Engl J Med. 2025报告的PARTNER 3低危患者7年随访:

\textbf{血流动力学参数}:

\begin{table}[h]
\centering
\caption{PARTNER 3: TAVR vs SAVR血流动力学参数}
\label{tab:partner3_hemodynamics}
\begin{tabular}{lcccccc}
\toprule
\textbf{参数} & \multicolumn{2}{c}{\textbf{基线}} & \multicolumn{2}{c}{\textbf{1年}} & \multicolumn{2}{c}{\textbf{7年}} \\
 & TAVR & SAVR & TAVR & SAVR & TAVR & SAVR \\
\midrule
平均梯度 & 49.4 & 48.3 & 13.7 & 11.6 & 13.1 & 12.1 \\
(mm Hg) &  &  &  &  &  &  \\
\midrule
瓣膜面积 & 0.8 & 0.8 & 1.7 & 1.8 & 1.9 & 1.8 \\
(cm²) &  &  &  &  &  &  \\
\bottomrule
\end{tabular}
\end{table}

\textbf{生物瓣膜失败(BVF)}:
\begin{itemize}
    \item 7年BVF发生率:TAVR 6.3\% vs SAVR 6.9\%
    \item HR 0.93 (95\% CI: 0.56-1.54)
    \item 两组无显著差异
\end{itemize}

\textbf{主动脉瓣再次干预}:
\begin{itemize}
    \item 7年再次干预率:TAVR 6.7\% vs SAVR 6.0\%
    \item HR 1.11 (95\% CI: 0.63-1.94)
    \item 两组无显著差异
\end{itemize}

\textbf{3. PARTNER 2试验10年数据}

SAPIEN 3试验(低危患者)10年结果:

\textbf{全因死亡率}:
\begin{itemize}
    \item SAPIEN 3 TAVR:83.4\%(10年)
    \item 外科手术:82.3\%(10年)
    \item HR [95\% CI]: 1.01 [0.91, 1.13]; P=0.82
    \item \textbf{无显著差异}
\end{itemize}

\textbf{主动脉瓣再次干预}:
\begin{itemize}
    \item SAPIEN 3 TAVR:3.0\%(10年)
    \item 外科手术:3.2\%(10年)
    \item HR [95\% CI]: 1.39 [0.57, 3.41]; P=0.47
    \item \textbf{无显著差异}
\end{itemize}

\textbf{关键时间点数据}:
\begin{itemize}
    \item 1年死亡率:TAVR 12.4\% vs SAVR 6.8\%
    \item 3年死亡率:TAVR 42.9\% vs SAVR 40.2\%
    \item 5年死亡率:TAVR约50\% vs SAVR约45\%
    \item 10年死亡率:趋同(TAVR 83.4\% vs SAVR 82.3\%)
\end{itemize}

\textbf{4. 并非所有外科瓣膜都相同!}

Abushouk AI, et al. Am J Cardiol. 2021和Thyregod et al. European Heart Journal (2024) 45, 1116–1124的荟萃分析:

\begin{table}[h]
\centering
\caption{不同外科瓣膜的10年结构性瓣膜退化(SVD)累积发生率}
\label{tab:surgical_valve_svd}
\begin{tabular}{lcc}
\toprule
\textbf{瓣膜类型} & \textbf{10年SVD率} & \textbf{瓣膜寿命(年)} \\
\midrule
Perimount(牛心包) & <5\% & >20 \\
Epic(猪) & 约7\% & 15-20 \\
Trifecta(牛心包) & 约7\% & 15-20 \\
Mosaic(猪) & 约17\% & 10-15 \\
Sorin Mitroflow(牛心包) & 约30\% & <10 \\
CoreValve(TAVR)* & 约13\% & 10-15 \\
\bottomrule
\end{tabular}
\end{table}

*注:CoreValve数据来自NOTION试验

\textbf{关键观察}:
\begin{itemize}
    \item Perimount表现最佳(<5\% SVD率)
    \item Sorin Mitroflow表现最差(约30\% SVD率)
    \item \textbf{不同瓣膜间10年SVD率差异达6倍以上}
    \item 瓣膜选择对长期结局有重大影响
\end{itemize}

\textbf{临床启示}:
\begin{itemize}
    \item 不能笼统地说"SAVR比TAVR耐久"
    \item 必须在\textbf{特定瓣膜型号}的背景下讨论耐久性
    \item 对于年轻患者,外科瓣膜选择至关重要
    \item Perimount等高性能瓣膜应优先考虑用于年轻SAVR患者
\end{itemize}

\textbf{5. 并非所有TAVR瓣膜都相同!}

\textbf{ACURATE IDE试验}(Makkar R …. Gupta A et al. The Lancet. 2025):

\begin{table}[h]
\centering
\caption{ACURATE neo2 vs 对照组(SAPIEN 3 + Evolut)1年结局}
\label{tab:acurate_ide_outcomes}
\begin{tabular}{lccc}
\toprule
\textbf{结局} & \textbf{ACURATE neo2} & \textbf{SAPIEN 3} & \textbf{Evolut} \\
 & \textbf{(n=752)} & \textbf{(n=504)} & \textbf{(n=244)} \\
\midrule
\multicolumn{4}{l}{\textit{主要终点}} \\
全因死亡/卒中/ & 108例 & 42例 & 24例 \\
再住院1年 & (14.8\%) & (8.6\%) & (10.0\%) \\
\midrule
\multicolumn{4}{l}{\textit{次要终点}} \\
全因死亡 & 5.0\% & 4.1\% & 3.4\% \\
心血管死亡 & 3.7\% & 1.5\% & 2.5\% \\
非心血管死亡 & 1.3\% & 2.7\% & 0.9\% \\
卒中 & 5.7\% & 2.3\% & 5.8\% \\
致残性卒中 & 2.0\% & 0.4\% & 2.9\% \\
非致残性卒中 & 3.9\% & 1.9\% & 2.9\% \\
再住院 & 5.3\% & 3.4\% & 3.9\% \\
\bottomrule
\end{tabular}
\end{table}

\textbf{关键发现}:
\begin{itemize}
    \item ACURATE neo2\textbf{未达到}vs对照组的非劣效性
    \item ACURATE neo2主要终点事件率(14.8\%)显著高于SAPIEN 3 (8.6\%)
    \item ACURATE neo2卒中率(5.7\%)高于SAPIEN 3 (2.3\%)
    \item 后验概率分布显示ACURATE neo2劣于对照组
\end{itemize}

\textbf{血流动力学对比}:

尽管ACURATE neo2有\textbf{更好的血流动力学}:
\begin{itemize}
    \item 出院时平均梯度:ACURATE neo2 9.0 mmHg vs SAPIEN 3 7.7 mmHg
    \item 1年平均梯度:ACURATE neo2 11.9 mmHg vs SAPIEN 3 8.2 mmHg
    \item 1年瓣膜面积:ACURATE neo2 1.88 cm² vs SAPIEN 3 1.77 cm²
\end{itemize}

但临床结局\textbf{更差},提示:
\begin{itemize}
    \item 血流动力学不是唯一决定因素
    \item 卒中风险、瓣膜设计、植入技术等也很重要
    \item 瓣膜平台的整体性能需综合评估
\end{itemize}

\textbf{6. SAPIEN 3 Ultra RESILIA:下一代瓣膜技术}

Stinis CT, et al. J Am Coll Cardiol Intv. 2024;17(8):1032-1044报告的STS/ACC TVT注册数据(2021年1月-2023年6月,N=20,624):

\textbf{血流动力学表现}:

\begin{table}[h]
\centering
\caption{SAPIEN 3/S3U vs SAPIEN 3 Ultra Resilia (S3UR) 血流动力学比较}
\label{tab:s3ur_hemodynamics}
\begin{tabular}{lcccc}
\toprule
\textbf{瓣膜尺寸} & \multicolumn{2}{c}{\textbf{出院平均梯度(mmHg)}} & \multicolumn{2}{c}{\textbf{出院EOA(cm²)}} \\
 & S3/S3U & S3UR & S3/S3U & S3UR \\
\midrule
20 mm & 17 & 10 & 1.3 & 1.5 \\
23 mm & 12 & 11 & 1.5 & 1.5 \\
26 mm & 13 & 8 & 1.8 & 1.8 \\
29 mm & 10 & 9 & 1.8 & 2.0 \\
 &  & 7 & & 2.3 \\
\bottomrule
\end{tabular}
\end{table}

所有比较 p < 0.0001

\textbf{瓣周漏(PVL)比较}(29mm瓣膜):

\begin{table}[h]
\centering
\caption{29mm瓣膜出院PVL发生率}
\label{tab:s3ur_pvl}
\begin{tabular}{lcc}
\toprule
\textbf{PVL程度} & \textbf{S3/S3U (n=2,034)} & \textbf{S3UR (n=2,064)} \\
\midrule
无 & 90.1\% & 94.5\% \\
轻度 & 9.4\% & 5.3\% \\
中度 & 0.4\% & 0.2\% \\
重度 & 0.0\% & 0.0\% \\
\bottomrule
\end{tabular}
\end{table}

p < 0.0001

\textbf{临床结局}:
\begin{itemize}
    \item 30天死亡率和卒中率:S3UR与S3/S3U无显著差异
    \item 住院再入院率:S3UR较高(8.5\% vs 7.7\%, P=0.04)
\end{itemize}

\textbf{7. RESILIA组织技术的抗钙化特性}

Kaneko T et al. HVS. 2025; Flameng et al. J Thorac Cardiovasc Surg. 2015;149:340-345:

\textbf{RESILIA组织技术原理}:
\begin{itemize}
    \item 永久封闭游离醛基,防止钙沉积
    \item 钙化是组织瓣膜失败的首要原因
\end{itemize}

\textbf{外科瓣膜研究结果}:

\begin{table}[h]
\centering
\caption{RESILIA vs 非RESILIA外科瓣膜的8年无再手术生存率}
\label{tab:resilia_freedom_reoperation}
\begin{tabular}{lc}
\toprule
\textbf{瓣膜类型} & \textbf{8年无SVD再手术率} \\
\midrule
RESILIA组织瓣膜 & 99.2\% \\
非RESILIA组织瓣膜 & 93.9\% \\
\midrule
Log-rank P值 & 0.0003 \\
\bottomrule
\end{tabular}
\end{table}

\textbf{临床意义}:
\begin{itemize}
    \item RESILIA技术在外科瓣膜中显著改善了耐久性
    \item 8年无再手术率提高约5.3\%(绝对值)
    \item 对于年轻患者尤为重要
    \item SAPIEN 3 Ultra RESILIA将这一技术应用于TAVR瓣膜
    \item 需要长期随访验证TAVR瓣膜中的效果
\end{itemize}

\textbf{注意}:目前尚无临床数据评估RESILIA组织在患者体内的长期影响(TAVR瓣膜)。

\textbf{8. SAPIEN瓣膜平台的演进}

从2007年至2022年,SAPIEN瓣膜经历了多代改进:

\begin{table}[h]
\centering
\caption{SAPIEN瓣膜平台演进历程}
\label{tab:sapien_evolution}
\begin{tabular}{lll}
\toprule
\textbf{年份} & \textbf{瓣膜型号} & \textbf{主要改进} \\
\midrule
2007 & SAPIEN & 首次引入TAVR \\
 &  & 为不可手术/高危患者提供治疗选择 \\
\midrule
— & SAPIEN XT & 流线型设计 \\
 &  & 减小French尺寸,降低血管并发症 \\
\midrule
— & SAPIEN 3 & 增加外层PET裙边,减少PVL \\
 &  & 优化细胞尺寸以保证未来冠状动脉通路 \\
 &  & 新输送系统,可预测的释放 \\
 &  & 在低危患者1年时优于外科手术* \\
 &  & 5年时同样有效* \\
\midrule
— & SAPIEN 3 Ultra & 延长PVL裙边高度 \\
 &  & 减少中度和轻度PVL发生 \\
\midrule
2022 & SAPIEN 3 Ultra & 引入RESILIA组织 \\
 & RESILIA & 有效解决钙化(组织瓣膜失败的首要原因)** \\
\bottomrule
\end{tabular}
\end{table}

*PARTNER 3试验数据
**基于动物模型和外科瓣膜数据;TAVR瓣膜的长期临床数据尚未获得

\textbf{关键里程碑}:
\begin{itemize}
    \item 逐步降低PVL发生率(通过增加裙边)
    \item 保持冠状动脉通路(通过优化细胞设计)
    \item 改善血流动力学(通过设计优化)
    \item 引入抗钙化技术(RESILIA)
    \item 短支架设计利于未来TAV-in-TAV
\end{itemize}

\subsubsection{病例示例:年轻BAV患者的终生管理}

\textbf{病例资料}:
\begin{itemize}
    \item 65岁女性
    \item 严重二叶主动脉瓣狭窄
\end{itemize}

\textbf{解剖学测量}:

\begin{table}[h]
\centering
\caption{病例解剖学测量数据}
\label{tab:case_anatomy}
\begin{tabular}{ll}
\toprule
\textbf{解剖结构} & \textbf{测量值} \\
\midrule
瓣环面积 & 892 mm² \\
LVOT面积 & 879 mm² \\
窦管尺寸(SOV) & 40.2 × 42.3 × 43.7 mm \\
主动脉最大径 & 44.1 × 44.3 mm \\
右冠状动脉开口高度(RCA) & 20.8 mm \\
左冠状动脉开口高度(LCA) & 17.5 mm \\
\bottomrule
\end{tabular}
\end{table}

\textbf{治疗过程}:
\begin{itemize}
    \item 使用29mm SAPIEN 3进行TAVR
    \item 手术成功
    \item TAVR后平均梯度:6 mmHg(优秀的血流动力学结果)
\end{itemize}

\textbf{长期规划}:

基于CT重建分析,该患者术后解剖:
\begin{itemize}
    \item 瓣环平面尺寸:26.4 mm(平均)
    \item 左冠状动脉高度:21.7 mm
    \item 右冠状动脉高度:21.2 mm
\end{itemize}

\textbf{未来TAV-in-TAV可行性评估}:
\begin{itemize}
    \item \textbf{冠状动脉高度充足}(>21 mm)
    \item \textbf{窦管尺寸良好}
    \item \textbf{该患者理论上可以进行2次额外的TAV-in-TAV!}
\end{itemize}

\textbf{终生管理策略}:
\begin{enumerate}
    \item 第1次干预(65岁):TAVR(29mm SAPIEN 3)- 已完成
    \item 第2次干预(预计75-80岁):TAV-in-TAV
    \item 第3次干预(预计85-90岁):第二次TAV-in-TAV
    \item 潜在覆盖患者预期寿命
\end{enumerate}

\textbf{病例启示}:
\begin{itemize}
    \item 年轻患者首次TAVR时必须考虑长期解剖学
    \item 冠状动脉高度是关键决定因素
    \item SAPIEN短支架设计为多次TAV-in-TAV提供可能性
    \item 个体化终生管理计划是可行的
    \item TAVR-first策略在精心选择的年轻患者中是合理的
\end{itemize}

\subsection{结论}

\subsubsection{主要结论}

\begin{enumerate}
    \item \textbf{个体化决策至关重要}:
    \begin{itemize}
        \item 在年轻严重AS患者中选择TAVR vs SAVR作为首次手术时,需要基于个体情况做出多方面考虑
        \item 共同决策(Shared decision-making)是关键
        \item 没有"一刀切"的策略
    \end{itemize}

    \item \textbf{需要更长期的随访数据}:
    \begin{itemize}
        \item TAVR vs SAVR在低危患者中的长期对比数据仍然有限
        \item 年轻患者(<65岁)的数据更为缺乏
        \item 二叶主动脉瓣患者需要专门的长期研究
    \end{itemize}

    \item \textbf{解剖学因素影响长期策略}:
    \begin{itemize}
        \item 短支架减轻冠状动脉再通或瓣叶对位问题
        \item 短支架非常适合重做TAVR手术
        \item 初次TAVR时应评估未来TAV-in-TAV的可行性
    \end{itemize}

    \item \textbf{SAPIEN平台的优势}:
    \begin{itemize}
        \item 较低的瓣周漏(PVL)发生率
        \item 较低的起搏器植入需求
        \item 较低的卒中率
        \item 与自膨胀平台相比具有可比的死亡率
    \end{itemize}

    \item \textbf{SAPIEN 3 RESILIA的潜力}:
    \begin{itemize}
        \item 血流动力学表现良好
        \item RESILIA组织抗钙化技术有前景
        \item 需要更长期的前瞻性研究进一步验证
    \end{itemize}

    \item \textbf{瓣膜选择的重要性}:
    \begin{itemize}
        \item 不同TAVR瓣膜平台结局差异显著
        \item 不同SAVR瓣膜耐久性差异可达6倍以上
        \item 必须在特定瓣膜型号的背景下讨论耐久性
        \item 不能笼统比较"TAVR vs SAVR"耐久性
    \end{itemize}
\end{enumerate}

\subsubsection{ABCD决策框架总结}

\textbf{A - Anatomical(解剖学)}:
\begin{itemize}
    \item 二叶瓣形态、主动脉病变风险
    \item 冠状动脉开口高度(影响TAV-in-TAV可行性)
    \item 窦管交界宽度
    \item 瓣环和主动脉根部尺寸
\end{itemize}

\textbf{B - Behavioral(患者偏好)}:
\begin{itemize}
    \item 对手术的接受程度
    \item 预期寿命期望
    \item 恢复时间需求
    \item 对未来技术进步的信心
\end{itemize}

\textbf{C - Clinical(临床因素)}:
\begin{itemize}
    \item 手术风险评分
    \item 合并症(CAD、多瓣膜病变)
    \item 衰弱状态
    \item 症状严重程度
\end{itemize}

\textbf{D - Durability(耐久性)}:
\begin{itemize}
    \item 特定瓣膜型号的长期数据
    \item 10-20年治疗规划
    \item 重做干预的可行性
    \item 新技术(如RESILIA)的潜在优势
\end{itemize}

\subsection{临床启示}

\subsubsection{对临床实践的建议}

\begin{enumerate}
    \item \textbf{建立系统化评估流程}:
    \begin{itemize}
        \item 使用ABCD框架系统评估每位年轻AS患者
        \item 多学科心脏团队讨论
        \item 详细的CT解剖评估
        \item 评估未来TAV-in-TAV可行性
    \end{itemize}

    \item \textbf{充分的患者教育和共同决策}:
    \begin{itemize}
        \item 向患者解释TAVR vs SAVR的利弊
        \item 讨论长期治疗路径(10-20年计划)
        \item 了解患者的价值观和生活目标
        \item 提供决策辅助工具
    \end{itemize}

    \item \textbf{瓣膜选择的个体化}:
    \begin{itemize}
        \item TAVR:优先考虑有长期数据支持的平台(如SAPIEN 3/Ultra RESILIA)
        \item SAVR:年轻患者选择高耐久性瓣膜(如Perimount)
        \item 避免已知耐久性差的瓣膜(如Sorin Mitroflow)
    \end{itemize}

    \item \textbf{特殊情况的处理}:
    \begin{itemize}
        \item BAV患者:评估主动脉病变风险,必要时考虑SAVR联合主动脉手术
        \item 合并CAD:根据SYNTAX评分和年龄选择SAVR+CABG vs TAVI+PCI
        \item 多瓣膜病变:原发性病变倾向SAVR,继发性病变可考虑分期经导管治疗
    \end{itemize}

    \item \textbf{长期随访和监测}:
    \begin{itemize}
        \item 建立系统的超声心动图随访计划
        \item 监测瓣膜血流动力学变化
        \item 早期识别结构性瓣膜退化(SVD)
        \item 及时规划重做干预
    \end{itemize}

    \item \textbf{拥抱新技术}:
    \begin{itemize}
        \item 关注RESILIA等抗钙化技术的长期数据
        \item 了解瓣叶修饰技术的进展
        \item 参与或了解重做TAVR的临床研究
    \end{itemize}
\end{enumerate}

\subsubsection{对研究的启示}

\begin{enumerate}
    \item 需要<65岁患者TAVR vs SAVR的前瞻性随机对照试验
    \item 需要BAV患者的专门长期研究(>10年)
    \item 评估RESILIA组织在TAVR中的长期耐久性
    \item 研究重做TAVR的最佳时机和技术
    \item 开发预测瓣膜耐久性的生物标志物
    \item 优化AI辅助的治疗策略选择
\end{enumerate}

\subsection{研究局限性}

\begin{enumerate}
    \item \textbf{演讲性质的局限}:
    \begin{itemize}
        \item 本文献为会议演讲,非原始研究论文
        \item 数据来自多项不同研究,综合分析的系统性有限
        \item 部分数据为演讲者个人经验和病例
    \end{itemize}

    \item \textbf{随访时间的局限}:
    \begin{itemize}
        \item TAVR最长随访数据仅10年(PARTNER 2)
        \item 对于年轻患者(如65岁),10年数据远不足以指导终生管理
        \item SAPIEN 3 Ultra RESILIA刚上市,无长期临床数据
    \end{itemize}

    \item \textbf{研究设计的局限}:
    \begin{itemize}
        \item 许多数据来自注册研究(如STS-TVT Registry),非随机对照试验
        \item BAV患者的TAVR vs SAVR对比来自不同数据库,存在选择偏倚
        \item 倾向匹配分析无法完全控制混杂因素
    \end{itemize}

    \item \textbf{外推性的局限}:
    \begin{itemize}
        \item 不同瓣膜平台的数据不能相互外推
        \item 美国数据可能不适用于其他国家/地区
        \item 病例示例(65岁女性)的经验不能推广到所有患者
    \end{itemize}

    \item \textbf{未解决的问题}:
    \begin{itemize}
        \item TAV-in-TAV-in-TAV(第三次干预)的可行性和安全性未知
        \item 冠状动脉通路在多次TAV-in-TAV后的可行性需要验证
        \item 血流动力学与临床结局的关系仍不完全清楚(如ACURATE neo2的悖论)
    \end{itemize}

    \item \textbf{利益冲突}:
    \begin{itemize}
        \item 演讲者为Edwards Lifesciences和Boston Scientific的顾问/讲者
        \item 可能存在对SAPIEN平台的偏向
        \item 需要独立研究验证结论
    \end{itemize}
\end{enumerate}

\subsection{个人笔记}

\subsubsection{关键数字记忆}

\textbf{TAVR趋势}:
\begin{itemize}
    \item 2021年美国<65岁患者接受TAVR:约50\%
\end{itemize}

\textbf{BAV患者TAVR结局}:
\begin{itemize}
    \item 1年死亡率:4.6\%(vs TAV 6.7\%)
    \item 1年卒中率:1.8\%(vs TAV 2.2\%)
    \item 死亡HR:0.75 (0.55-1.02), P=0.06
\end{itemize}

\textbf{TAVR vs SAVR在BAV}:
\begin{itemize}
    \item TAVR新发房颤:1.0\% vs SAVR 36.6\%(巨大差异!)
    \item TAVR 1年死亡率:4.6\% vs SAVR 3.2\%
\end{itemize}

\textbf{重做TAVR安全性}:
\begin{itemize}
    \item 死亡HR:0.94 (0.77-1.16), P=0.57(与原生TAVR相似)
    \item 冠状动脉压迫/阻塞:0.3\% vs 0.1\%, P=0.37(低风险)
\end{itemize}

\textbf{PARTNER长期数据}:
\begin{itemize}
    \item PARTNER 3(7年):BVF率 TAVR 6.3\% vs SAVR 6.9\%
    \item PARTNER 2(10年):全因死亡 TAVR 83.4\% vs SAVR 82.3\%,P=0.82
\end{itemize}

\textbf{外科瓣膜SVD率(10年)}:
\begin{itemize}
    \item Perimount:<5\%(最佳)
    \item Sorin Mitroflow:约30\%(最差)
    \item 6倍差异!
\end{itemize}

\textbf{ACURATE IDE结局}:
\begin{itemize}
    \item ACURATE neo2主要终点:14.8\%
    \item SAPIEN 3:8.6\%
    \item 未达非劣效性
\end{itemize}

\textbf{S3UR血流动力学}:
\begin{itemize}
    \item 29mm瓣膜无PVL率:94.5\%(vs S3/S3U 90.1\%)
    \item 出院平均梯度:7-10 mmHg(vs S3/S3U 10-13 mmHg)
\end{itemize}

\textbf{RESILIA外科瓣膜}:
\begin{itemize}
    \item 8年无SVD再手术率:99.2\% vs 非RESILIA 93.9\%
    \item P=0.0003
\end{itemize}

\subsubsection{重要概念}

\begin{description}
    \item[Therapy Sequencing] 治疗策略排序 - 为年轻AS患者规划10-20年的多次瓣膜干预顺序,而非仅考虑单次治疗

    \item[ABCD Framework] 决策框架 - Anatomical(解剖)、Behavioral(患者偏好)、Clinical(临床)、Durability(耐久性)四维度综合评估

    \item[TAV-in-TAV Feasibility] TAV-in-TAV可行性 - 取决于冠状动脉高度、窦管交界宽度、初次瓣膜支架设计(短支架优于高支架)

    \item[SVD (Structural Valve Deterioration)] 结构性瓣膜退化 - 瓣膜失败的主要模式,不同瓣膜型号差异巨大(10年SVD率从<5\%到30\%)

    \item[RESILIA Tissue Technology] RESILIA组织技术 - 永久封闭游离醛基以防止钙化,动物实验和外科瓣膜数据显示显著改善耐久性

    \item[Short-Frame THV] 短支架经导管心脏瓣膜 - SAPIEN平台的优势,减少冠状动脉阻塞风险,便于未来TAV-in-TAV和冠状动脉通路

    \item[Leaflet Modification] 瓣叶修饰技术 - 新兴技术,在TAV-in-TAV前处理原有瓣叶以优化冠状动脉通路

    \item[Hemodynamics ≠ Clinical Outcomes] 血流动力学≠临床结局 - ACURATE neo2虽有更好血流动力学但临床结局更差,提示需综合评估瓣膜性能
\end{description}

\subsubsection{临床实践要点}

\begin{enumerate}
    \item \textbf{年轻患者初次TAVR时的"Must-Do"}:
    \begin{itemize}
        \item 详细CT评估冠状动脉高度(>20mm为佳)
        \item 测量窦管交界尺寸
        \item 评估至少2次TAV-in-TAV的可行性
        \item 选择短支架平台(如SAPIEN)
        \item 考虑RESILIA组织瓣膜
    \end{itemize}

    \item \textbf{BAV患者的特殊考虑}:
    \begin{itemize}
        \item 评估主动脉扩张(>45mm考虑SAVR+主动脉手术)
        \item 评估瓣叶钙化模式(影响预后)
        \item TAVR可行但需谨慎选择(NOTION二叶瓣队列结果较差)
        \item 长期随访主动脉尺寸变化
    \end{itemize}

    \item \textbf{何时推荐TAVR-first}:
    \begin{itemize}
        \item 患者强烈偏好微创
        \item 冠状动脉高度>20mm
        \item 窦管交界宽度适中
        \item 无需主动脉手术
        \item 无复杂多瓣膜病变
        \item 无需CABG或仅需简单PCI
    \end{itemize}

    \item \textbf{何时推荐SAVR-first}:
    \begin{itemize}
        \item 主动脉扩张需要同期手术
        \item 复杂多瓣膜病变
        \item 需要CABG(尤其3支病变/左主干)
        \item 冠状动脉高度<15mm(TAV-in-TAV风险高)
        \item 患者偏好"一次性解决"
        \item 患者非常年轻(<60岁)且手术风险低
    \end{itemize}

    \item \textbf{瓣膜选择的优先顺序}:
    \begin{itemize}
        \item TAVR首选:SAPIEN 3 Ultra RESILIA(短支架+抗钙化)
        \item TAVR备选:SAPIEN 3 Ultra(短支架,已证实低PVL)
        \item SAVR首选:Perimount(最佳10年耐久性)
        \item SAVR避免:Sorin Mitroflow(高SVD率)
    \end{itemize}
\end{enumerate}

\subsubsection{争议性问题和思考}

\begin{enumerate}
    \item \textbf{TAVR-first vs SAVR-first:真的有定论吗?}
    \begin{itemize}
        \item 目前数据显示10年死亡率相似
        \item 但10年对于65岁患者只是"人生的一半"
        \item 20-30年数据才能真正回答这个问题
        \item 个体化可能永远优于"一刀切"策略
    \end{itemize}

    \item \textbf{血流动力学的重要性被高估了吗?}
    \begin{itemize}
        \item ACURATE neo2有更好的梯度和瓣膜面积,但临床结局更差
        \item 卒中、PVL、瓣膜设计等因素可能同样或更重要
        \item 需要重新审视血流动力学在瓣膜评估中的权重
    \end{itemize}

    \item \textbf{RESILIA技术是"game-changer"吗?}
    \begin{itemize}
        \item 外科瓣膜数据非常promising(8年99.2\% vs 93.9\%)
        \item 但TAVR环境与SAVR不同(血流动力学、应力分布)
        \item TAVR中的RESILIA长期效果仍未知
        \item 需要至少10年随访才能下结论
    \end{itemize}

    \item \textbf{TAV-in-TAV-in-TAV真的可行吗?}
    \begin{itemize}
        \item 病例显示解剖上可能支持3次干预
        \item 但每次干预后瓣膜有效面积递减
        \item 第三次TAV可能面临严重梯度升高
        \item 患者梯度耐受性随年龄增长可能改变
    \end{itemize}

    \item \textbf{如何平衡创新与谨慎?}
    \begin{itemize}
        \item 新技术(如RESILIA)令人兴奋,但缺乏长期数据
        \item 年轻患者可能从新技术获益最大,但也承担最大不确定性
        \item 是否应该在年轻患者中使用最新技术?
        \item 还是应该等待更多数据后再推广?
    \end{itemize}
\end{enumerate}

\subsubsection{与中国实践的关联}

\begin{enumerate}
    \item \textbf{中国特色的考虑}:
    \begin{itemize}
        \item 中国风湿性心脏病比例可能更高(与美国退行性AS不同)
        \item 二叶瓣的形态学可能存在种族差异
        \item 经济因素在中国瓣膜选择中权重更大
        \item 医保覆盖政策影响TAVR vs SAVR选择
    \end{itemize}

    \item \textbf{可借鉴的经验}:
    \begin{itemize}
        \item ABCD框架适用于中国患者
        \item 共同决策的理念值得推广
        \item 长期规划(10-20年)的思维方式
        \item 重视解剖学评估(尤其冠状动脉高度)
    \end{itemize}

    \item \textbf{需要中国自己的数据}:
    \begin{itemize}
        \item 中国人群的瓣膜耐久性数据
        \item 中国BAV患者的形态学和预后特点
        \item 不同TAVR瓣膜在中国人群中的表现
        \item 中国患者的价值观和偏好调查
    \end{itemize}
\end{enumerate}

\subsubsection{学习要点总结}

\begin{enumerate}
    \item 年轻AS患者的管理是"终生管理",不是"一次性治疗"
    \item ABCD框架提供了系统化的决策工具
    \item 解剖学(尤其冠状动脉高度)决定了TAV-in-TAV的可行性
    \item 患者偏好在决策中的权重与临床因素同等重要
    \item 不能笼统比较"TAVR vs SAVR",必须具体到瓣膜型号
    \item 短支架TAVR(如SAPIEN)在重做场景中有明显优势
    \item RESILIA技术有前景但需要长期验证
    \item 血流动力学不是唯一决定因素,需综合评估
    \item 10年数据不足以指导年轻患者的终生管理,需要20-30年数据
    \item 共同决策和个体化是核心原则
\end{enumerate}

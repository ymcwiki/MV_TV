\section{再次SAVR vs VinV TAVR治疗生物瓣衰败:我们需要随机对照试验}
\label{sec:12_003_tav_in_savr_outcomes}

% ============================================
% 文献信息
% ============================================
\subsection{文献信息}

\begin{itemize}
    \item \textbf{标题}: Redo SAVR vs TAVI VinV for Degenerated Bioprostheses: Time For a Trial
    \item \textbf{作者}: Michael A. Borger, MD, PhD
    \item \textbf{机构}: Leipzig Heart Center (Director of Cardiac Surgery and Medical Director); Helios Health Institute; Universität Leipzig
    \item \textbf{会议}: 学术演讲/综述
    \item \textbf{PDF文件名}: long-term-outcomes-after-tav-in-savr-do-we-need-a-randomized-trial.pdf
    \item \textbf{文献类型}: 学术演讲/综述性文献
    \item \textbf{利益冲突}: 演讲者所在医院接受来自Edwards Lifesciences, Abbott, Medtronic, Artivion的演讲费/咨询费
\end{itemize}

\subsection{研究背景}

\subsubsection{生物瓣膜的核心问题}

\textbf{阿喀琉斯之踵:结构性瓣膜衰败(SVD)}

生物瓣膜的主要局限性在于结构性瓣膜衰败(Structural Valve Deterioration, SVD),表现为:
\begin{itemize}
    \item 瓣叶钙化
    \item 瓣叶撕裂
    \item 瓣叶纤维化和增厚
    \item 导致瓣膜狭窄和/或反流
\end{itemize}

\subsubsection{VinV TAVR vs Redo SAVR的趋势变化}

根据Braasch等人(JAMA Cardiol 2025; Sep 24:e253224)的研究数据:

\textbf{VinV TAVR手术量变化趋势}:

\begin{table}[h]
\centering
\caption{VinV TAVR手术量年度变化}
\label{tab:vinv_tavr_trends}
\begin{tabular}{lrr}
\toprule
\textbf{年份} & \textbf{VinV手术量(例)} & \textbf{年增长率(\%)} \\
\midrule
2015 & \textasciitilde 80 & - \\
2016 & \textasciitilde 200 & - \\
2017 & \textasciitilde 300 & - \\
2018 & \textasciitilde 500 & - \\
2019(FDA批准) & \textasciitilde 600 & - \\
2020 & \textasciitilde 800 & - \\
2021 & \textasciitilde 950 & - \\
2022 & \textasciitilde 1000 & - \\
2023 & \textasciitilde 950 & - \\
2024 & \textasciitilde 1150 & - \\
\bottomrule
\end{tabular}
\end{table}

\textbf{关键观察}:
\begin{itemize}
    \item VinV FDA批准标志着转折点
    \item VinV手术量从2015年的约80例增长到2024年的约1150例
    \item 年度VinV占总TAVR手术的比例从0\%增长到约2.5\%
\end{itemize}

\textbf{Redo SAVR手术量变化趋势}:

\begin{table}[h]
\centering
\caption{Redo SAVR手术量年度变化}
\label{tab:redo_savr_trends}
\begin{tabular}{lrr}
\toprule
\textbf{年份} & \textbf{Redo SAVR手术量(例)} & \textbf{年SAVR占比(\%)} \\
\midrule
2015 & 300 & 0.7 \\
2016 & 350 & 1.0 \\
2017 & 400 & 1.3 \\
2018 & 400 & 1.8 \\
2019 & 450 & 2.1 \\
2020 & 450 & 2.8 \\
2021 & 400 & 2.8 \\
2022 & 400 & 2.8 \\
2023 & 400 & 2.8 \\
2024 & 350 & - \\
\bottomrule
\end{tabular}
\end{table}

\textbf{关键趋势}:
\begin{itemize}
    \item Redo SAVR手术量在2015-2019年间增长,随后保持稳定
    \item 2019年后Redo SAVR手术量开始下降,可能与VinV应用增加有关
    \item Redo SAVR占所有SAVR手术的比例稳定在约2.8\%
\end{itemize}

\subsection{主要研究发现}

\subsubsection{1. Redo SAVR的当代结果}

\textbf{STS数据库研究(2011-2013)}

Kaneko等人(Ann Thorac Surg 2015;100:1298-304)报告:

研究纳入:
\begin{itemize}
    \item Redo SAVR患者:n = 3,380(STS PROM 5.4\%)
    \item 首次SAVR患者:n = 54,183(STS PROM 2.7\%)
    \item 研究期间:2011年7月至2013年9月
\end{itemize}

\textbf{患者特征比较}:

\begin{table}[h]
\centering
\caption{Redo SAVR vs 首次SAVR患者特征}
\label{tab:redo_vs_primary_savr_characteristics}
\begin{tabular}{lrrl}
\toprule
\textbf{特征} & \textbf{Redo SAVR} & \textbf{首次SAVR} & \textbf{P值} \\
\midrule
年龄(岁) & 66 (56-75) & 70 (61-78) & <0.0001 \\
男性(\%) & 67.0 & 57.4 & <0.0001 \\
白人(\%) & 84.1 & 85.6 & 0.004 \\
射血分数 & 0.57 (0.50-0.62) & 0.60 (0.55-0.65) & <0.0001 \\
慢性肺病-中度(\%) & 6.8 & 6.3 & <0.0001 \\
慢性肺病-重度(\%) & 6.1 & 4.2 & - \\
既往心肌梗死(\%) & 14.9 & 8.9 & <0.0001 \\
心律失常(\%) & 32.8 & 19.4 & <0.0001 \\
充血性心衰(\%) & 53.9 & 38.8 & <0.0001 \\
NYHA III级(\%) & 46.8 & 44.1 & <0.0001 \\
NYHA IV级(\%) & 25.7 & 16.3 & - \\
主动脉狭窄(\%) & 62.4 & 88.1 & <0.0001 \\
重度主动脉反流(\%) & 37.3 & 16.5 & <0.0001 \\
活动性感染性心内膜炎(\%) & 13.1 & 3.0 & <0.0001 \\
急诊手术(\%) & 38.8 & 19.9 & <0.0001 \\
STS PROM(\%) & 5.4 & 2.7 & <0.0001 \\
\bottomrule
\end{tabular}
\end{table}

\textbf{手术结果}:

\begin{table}[h]
\centering
\caption{Redo SAVR vs 首次SAVR手术结果}
\label{tab:redo_vs_primary_savr_outcomes}
\begin{tabular}{lrrl}
\toprule
\textbf{结果指标} & \textbf{Redo SAVR} & \textbf{首次SAVR} & \textbf{P值} \\
\midrule
手术死亡率(\%) & 4.6 & 2.2 & <0.0001 \\
复合手术死亡率及主要并发症(\%) & 21.6 & 11.8 & <0.0001 \\
综合结局(\%) & 21.9 & 13.9 & <0.0001 \\
卒中(\%) & 4.7 & 1.8 & <0.0001 \\
血管并发症(\%) & - & - & - \\
起搏器植入(\%) & 11.5 & - & - \\
术后主动脉反流(\%) & - & - & - \\
\bottomrule
\end{tabular}
\end{table}

\textbf{关键发现}:
\begin{itemize}
    \item \textbf{Redo SAVR手术死亡率为4.6\%},高于首次SAVR的2.2\%(p<0.0001)
    \item 复合终点发生率显著更高(21.6\% vs 11.8\%, p<0.0001)
    \item 卒中率更高(4.7\% vs 1.8\%, p<0.0001)
    \item Redo SAVR患者更年轻(66岁 vs 70岁)
    \item 活动性心内膜炎比例更高(13.1\% vs 3.0\%)
\end{itemize}

\textbf{Leipzig Heart Center数据(2011-2022)}

Raschpichler等人(EJCTS 2024;66:ezae353)报告:

研究设计:
\begin{itemize}
    \item 首次SAVR:n = 2,446
    \item Redo SAVR:n = 174
    \item 研究期间:2011-2022年
    \item 排除标准:联合手术、心内膜炎
\end{itemize}

\textbf{主要结果}:

\begin{table}[h]
\centering
\caption{Leipzig单中心Redo SAVR vs 首次SAVR结果}
\label{tab:leipzig_redo_vs_primary}
\begin{tabular}{lcc}
\toprule
\textbf{结果指标} & \textbf{Redo SAVR} & \textbf{首次SAVR (匹配)} \\
\midrule
死亡或卒中率(匹配队列) & 4.8\% & 4.8\% \\
趋势变化 & 2011-2022年下降 & 2011-2022年下降 \\
\bottomrule
\end{tabular}
\end{table}

\textbf{关键发现}:
\begin{itemize}
    \item 在排除心内膜炎且匹配后,Redo SAVR与首次SAVR的死亡率相似(均为4.8\%)
    \item 从2011到2022年,死亡或卒中率持续下降
    \item 表明在选择性患者中,Redo SAVR可以达到与首次SAVR相似的手术效果
\end{itemize}

\textbf{手术瓣膜尺寸的影响}

Thourani等人(Ann Thorac Surg 2015;99:55-61)研究:

研究纳入:
\begin{itemize}
    \item 样本量:N = 141,905例SAVR
    \item 低风险患者:63,751例
    \item 中危患者:58,385例
    \item 高危患者:19,769例
    \item 数据来源:STS数据库(2002-2010)
\end{itemize}

\textbf{瓣膜尺寸分布}:

\begin{table}[h]
\centering
\caption{SAVR瓣膜尺寸分布}
\label{tab:savr_valve_sizes}
\begin{tabular}{lr}
\toprule
\textbf{瓣膜尺寸} & \textbf{比例(\%)} \\
\midrule
19 mm & \textasciitilde 8\% \\
21 mm & \textasciitilde 35\% \\
23 mm & \textasciitilde 35\% \\
25 mm & \textasciitilde 15\% \\
≥27 mm & \textasciitilde 7\% \\
\bottomrule
\end{tabular}
\end{table}

\textbf{关键发现}:
\begin{itemize}
    \item 小瓣膜(≤21mm)占约43\%
    \item 瓣膜尺寸影响长期结果
    \item 小瓣膜组1年生存率较差
\end{itemize}

\subsubsection{2. Redo SAVR vs VinV TAVR的比较研究}

\textbf{Meta分析(Raschpichler et al, JAHA 2022)}

这是迄今为止最全面的meta分析,比较了Redo SAVR与VinV TAVR的结果。

\textbf{短期生存率(30天或院内死亡率)}:

\begin{table}[h]
\centering
\caption{Redo SAVR vs VinV短期死亡率Meta分析}
\label{tab:meta_short_term_mortality}
\begin{tabular}{lccccl}
\toprule
\textbf{研究} & \textbf{VinV死亡} & \textbf{VinV总数} & \textbf{Redo死亡} & \textbf{Redo总数} & \textbf{RR (95\% CI)} \\
\midrule
Hirji 2020 & 61 & 2181 & 109 & 2181 & 0.56 [0.41; 0.76] \\
Deharo 2020 & 26 & 717 & 52 & 717 & 0.50 [0.32; 0.79] \\
Malik 2020 & 7 & 710 & 35 & 710 & 0.20 [0.09; 0.45] \\
Patel 2020 & 3 & 187 & 1 & 86 & 1.38 [0.15; 13.07] \\
Woitek 2020 & 7 & 147 & 5 & 111 & 1.06 [0.34; 3.24] \\
Sedeek 2019 & 2 & 90 & 7 & 260 & 0.83 [0.17; 3.90] \\
Spaziano 2017 & 3 & 78 & 5 & 78 & 0.60 [0.15; 2.42] \\
Cizmic 2021 & 0 & 73 & 3 & 17 & 0.06 [0.01; 0.64] \\
Silaschi 2017 & 3 & 71 & 3 & 59 & 0.83 [0.17; 3.96] \\
Stankowski 2020 & 1 & 30 & 3 & 30 & 0.33 [0.04; 3.03] \\
Erlebach 2015 & 2 & 50 & 0 & 52 & 5.08 [0.26; 100.82] \\
Dokollari 2021 & 0 & 31 & 4 & 57 & 0.14 [0.00; 4.16] \\
Grubitzsch 2017 & 3 & 27 & 2 & 25 & 1.39 [0.25; 7.64] \\
Ejiofor 2016 & 0 & 22 & 1 & 22 & 0.33 [0.01; 7.75] \\
\midrule
\textbf{总计} & \textbf{118} & \textbf{4414} & \textbf{230} & \textbf{4405} & \textbf{0.55 [0.34; 0.91]} \\
\bottomrule
\end{tabular}
\end{table}

\textbf{关键发现}:
\begin{itemize}
    \item \textbf{VinV短期死亡率显著低于Redo SAVR}
    \item 合并相对风险(RR)= 0.55 (95\% CI: 0.34-0.91, p=0.02)
    \item 异质性:I² = 20\%
    \item VinV短期死亡率约2.7\%,Redo SAVR约5.2\%
\end{itemize}

\textbf{中期生存率(1-3年随访)}:

\begin{table}[h]
\centering
\caption{Redo SAVR vs VinV中期死亡率Meta分析}
\label{tab:meta_mid_term_mortality}
\begin{tabular}{lccccl}
\toprule
\textbf{研究} & \textbf{VinV死亡} & \textbf{VinV总数} & \textbf{Redo死亡} & \textbf{Redo总数} & \textbf{HR (95\% CI)} \\
\midrule
Deharo 2020 & 170 & 717 & 147 & 717 & 1.22 [1.01; 1.47] \\
Patel 2020 & 6 & 187 & 3 & 86 & 0.70 [0.19; 2.60] \\
Woitek 2020 & 13 & 147 & 11 & 111 & 0.88 [0.40; 1.96] \\
Sedeek 2019 & 19 & 90 & 49 & 260 & 1.18 [0.62; 2.23] \\
Spaziano 2017 & 9 & 78 & 10 & 78 & 0.89 [0.36; 2.19] \\
Erlebach 2015 & 7 & 50 & 2 & 52 & 8.97 [2.43; 33.13] \\
Silaschi 2017 & 5 & 46 & 4 & 51 & - \\
Dokollari 2021 & 5 & 31 & 4 & 57 & 2.99 [0.81; 11.06] \\
Stankowski 2020 & 14 & 30 & 14 & 30 & 0.67 [0.32; 1.40] \\
Grubitzsch 2017 & 5 & 27 & 4 & 25 & 1.23 [0.33; 4.53] \\
\midrule
\textbf{总计} & \textbf{253} & \textbf{1403} & \textbf{248} & \textbf{1467} & \textbf{1.27 [0.72; 2.25]} \\
\bottomrule
\end{tabular}
\end{table}

\textbf{关键发现}:
\begin{itemize}
    \item \textbf{中期死亡率无显著差异}
    \item 合并危险比(HR)= 1.27 (95\% CI: 0.72-2.25, p=0.37)
    \item 异质性较大:I² = 47\%
    \item 趋势显示VinV中期死亡率可能略高,但未达统计学意义
\end{itemize}

\textbf{血流动力学结果:瓣周漏}:

\begin{table}[h]
\centering
\caption{瓣周漏发生率比较}
\label{tab:meta_pvl}
\begin{tabular}{lccccl}
\toprule
\textbf{研究} & \textbf{VinV PVL} & \textbf{VinV总数} & \textbf{Redo PVL} & \textbf{Redo总数} & \textbf{RR (95\% CI)} \\
\midrule
Patel 2020 & 22 & 187 & 2 & 86 & 5.06 [1.22; 21.03] \\
Woitek 2020 & 50 & 147 & 20 & 111 & 1.89 [1.20; 2.98] \\
Sedeek 2019 & 1 & 90 & 3 & 260 & 0.96 [0.10; 9.14] \\
Cizmic 2021 & 38 & 73 & 0 & 17 & 47.85 [0.54; 4264.63] \\
Silaschi 2017 & 17 & 71 & 8 & 59 & 1.77 [0.82; 3.80] \\
Stankowski 2020 & 10 & 30 & 1 & 30 & 10.00 [1.36; 73.33] \\
Erlebach 2015 & 10 & 50 & 3 & 52 & 3.47 [1.01; 11.87] \\
Dokollari 2021 & 14 & 31 & 0 & 57 & 40.74 [3.51; 472.77] \\
Grubitzsch 2017 & 5 & 27 & 0 & 25 & 10.63 [0.59; 192.72] \\
Ejiofor 2016 & 5 & 22 & 0 & 22 & 11.00 [0.65; 187.42] \\
Santarpino 2016 & 0 & 6 & 0 & 8 & - \\
\midrule
\textbf{总计} & \textbf{172} & \textbf{734} & \textbf{37} & \textbf{727} & \textbf{4.18 [1.88; 9.30]} \\
\bottomrule
\end{tabular}
\end{table}

\textbf{关键发现}:
\begin{itemize}
    \item \textbf{VinV瓣周漏发生率显著高于Redo SAVR}
    \item 合并相对风险(RR)= 4.18 (95\% CI: 1.88-9.30, p=0.003)
    \item VinV瓣周漏率约23.4\%,Redo SAVR约5.1\%
    \item 这是VinV的重要劣势
\end{itemize}

\textbf{血流动力学结果:患者-瓣膜不匹配(PPM)}:

\begin{table}[h]
\centering
\caption{患者-瓣膜不匹配发生率比较}
\label{tab:meta_ppm}
\begin{tabular}{lccccl}
\toprule
\textbf{研究} & \textbf{VinV PPM} & \textbf{VinV总数} & \textbf{Redo PPM} & \textbf{Redo总数} & \textbf{RR (95\% CI)} \\
\midrule
Woitek 2020 & 33 & 147 & 9 & 111 & 2.77 [1.38; 5.55] \\
Sedeek 2019 & 40 & 90 & 31 & 260 & 3.73 [2.49; 5.58] \\
Silaschi 2017 & 10 & 71 & 2 & 59 & 4.15 [0.95; 18.22] \\
Dokollari 2021 & 12 & 31 & 10 & 57 & 2.21 [1.08; 4.52] \\
Grubitzsch 2017 & 2 & 27 & 1 & 25 & 1.85 [0.18; 19.19] \\
Santarpino 2016 & 2 & 6 & 0 & 8 & 5.67 [0.38; 83.83] \\
\midrule
\textbf{总计} & \textbf{99} & \textbf{372} & \textbf{53} & \textbf{520} & \textbf{3.12 [2.35; 4.14]} \\
\bottomrule
\end{tabular}
\end{table}

\textbf{关键发现}:
\begin{itemize}
    \item \textbf{VinV患者-瓣膜不匹配发生率显著高于Redo SAVR}
    \item 合并相对风险(RR)= 3.12 (95\% CI: 2.35-4.14, p<0.001)
    \item 异质性低:I² = 0\%
    \item VinV PPM率约26.6\%,Redo SAVR约10.2\%
\end{itemize}

\textbf{血流动力学结果:跨瓣压差}:

Meta分析显示VinV的跨瓣压差显著高于Redo SAVR:
\begin{itemize}
    \item 标准化均数差(SMD)= 0.44 (95\% CI: 0.15-0.72, p=0.008)
    \item 异质性高:I² = 77\%
    \item VinV平均跨瓣压差约为15-20 mmHg,Redo SAVR约为10-15 mmHg
\end{itemize}

\subsubsection{3. 倾向性匹配分析研究}

\textbf{Sa等人研究(Int J Cardiol 2023)}

研究设计:
\begin{itemize}
    \item VinV-TAVI组:n = 1,676
    \item Redo SAVR组:n = 1,669
    \item 倾向性匹配
\end{itemize}

\textbf{主要结果}:

全因死亡率:
\begin{itemize}
    \item 假设比例风险:HR = 1.02 (95\% CI: 0.87-1.21, P=0.785)
    \item 5年无事件生存率相似
    \item 时间-变化风险比分析显示:早期有利于Redo SAVR,长期趋势不明显
\end{itemize}

\textbf{Deharo等人研究(JACC 2020)}

这是一项重要的倾向性匹配研究,纳入法国FRANCE-TAVI注册研究数据。

研究设计:
\begin{itemize}
    \item VinV-TAVR组:n = 717
    \item Redo SAVR组:n = 717
    \item 1:1倾向性匹配
    \item 中位随访时间:2.3 (1.1-4.0)年
\end{itemize}

\textbf{主要结果}:

\begin{table}[h]
\centering
\caption{Deharo研究长期复合终点结果}
\label{tab:deharo_long_term}
\begin{tabular}{lrrrr}
\toprule
\textbf{随访时间(年)} & \textbf{SAVR风险人数} & \textbf{TAVR风险人数} & \textbf{SAVR累积率(\%)} & \textbf{TAVR累积率(\%)} \\
\midrule
0 & 717 & 717 & 0 & 0 \\
1 & 407 & 474 & \textasciitilde 20 & \textasciitilde 18 \\
2 & 345 & 399 & \textasciitilde 35 & \textasciitilde 33 \\
3 & 291 & 341 & \textasciitilde 45 & \textasciitilde 48 \\
4 & 252 & 284 & \textasciitilde 55 & \textasciitilde 80 \\
\bottomrule
\end{tabular}
\end{table}

\textbf{复合终点}(死亡、卒中、MI、心衰住院、瓣膜再干预):
\begin{itemize}
    \item VinV:18.6\%/年
    \item Redo SAVR:21.9\%/年
    \item P = 0.34(无显著差异)
    \item \textbf{但长期趋势显示Redo SAVR结果可能更优}
\end{itemize}

\textbf{Tran等人研究(JAMA Cardiol 2024)}

这是最新的大型多中心倾向性匹配研究。

研究设计:
\begin{itemize}
    \item 研究期间:2015年1月至2020年12月
    \item 数据来源:加州、纽约州、新泽西州医疗数据库
    \item 总纳入患者:1,771例
    \item 倾向性匹配后:VinV-TAVR组375例,Redo SAVR组375例
    \item 中位随访:2.3 (1.1-4.0)年
\end{itemize}

\textbf{纳入标准}:
\begin{itemize}
    \item 既往SAVR后接受VinV-TAVR或Redo SAVR
    \item 排除:初次SAVR后5年内再干预、合并感染性心内膜炎、其他心脏手术、离开州
\end{itemize}

\textbf{患者特征}:
\begin{itemize}
    \item 女性:36.9\%
    \item 平均年龄:74 (11.3)岁
    \item 倾向性匹配实现了良好的基线平衡
\end{itemize}

\textbf{主要结果 - 全因死亡率}:

\begin{table}[h]
\centering
\caption{Tran研究全因死亡率结果}
\label{tab:tran_mortality}
\begin{tabular}{lcc}
\toprule
\textbf{死亡率指标} & \textbf{风险比 (95\% CI)} & \textbf{P值} \\
\midrule
5年全因死亡率 & HR 1.03 (0.59-1.78) & 0.86 \\
2年前死亡率 & HR 1.03 (0.59-1.78) & 0.86 \\
2年后死亡率 & HR 2.97 (1.18-7.47) & 0.02 \\
\bottomrule
\end{tabular}
\end{table}

\textbf{关键发现}:
\begin{itemize}
    \item 整体5年死亡率无显著差异
    \item \textbf{但存在非比例风险}:
    \begin{itemize}
        \item 前2年:两组死亡率相似
        \item 2年后:VinV死亡率显著升高(HR 2.97, 95\% CI: 1.18-7.47, p=0.02)
    \end{itemize}
    \item 这一发现支持VinV长期结果可能较差
\end{itemize}

\textbf{次要结果 - 心衰住院}:

\begin{table}[h]
\centering
\caption{Tran研究心衰住院率结果}
\label{tab:tran_hf_hospitalization}
\begin{tabular}{lcc}
\toprule
\textbf{心衰住院指标} & \textbf{风险比 (95\% CI)} & \textbf{P值} \\
\midrule
前2年心衰住院 & HR 1.13 (0.76-1.69) & 0.53 \\
2年后心衰住院 & HR 3.81 (1.57-9.22) & 0.003 \\
\bottomrule
\end{tabular}
\end{table}

\textbf{关键发现}:
\begin{itemize}
    \item 前2年心衰住院率相似
    \item \textbf{2年后VinV心衰住院率显著升高}(HR 3.81, 95\% CI: 1.57-9.22, p=0.003)
    \item 5年累积心衰住院率:VinV约25\%,Redo SAVR约10\%
\end{itemize}

\textbf{围手术期并发症}:

VinV组围手术期并发症更少:
\begin{itemize}
    \item 大出血:VinV 2.4\% vs Redo SAVR 5.1\% (P=0.05)
    \item 急性肾损伤:VinV 1.3\% vs Redo SAVR 7.2\% (P<0.001)
    \item 新起搏器植入:VinV 3.5\% vs Redo SAVR 10.9\% (P<0.001)
\end{itemize}

\textbf{研究结论}:

Tran等人总结:
\begin{itemize}
    \item VinV与Redo SAVR相比,围手术期并发症更少
    \item 2年内死亡率相似
    \item \textbf{但2年后,VinV与更高的死亡率和心衰住院率相关}
    \item 这些发现可能受残余混淆影响,需要随机对照试验验证
\end{itemize}

\subsection{为什么需要随机对照试验?}

\subsubsection{现有证据的局限性}

Borger等人在JAMA Cardiology 2022社论中指出:

\begin{quote}
"当两种治疗选择存在明显不同(即非比例)的风险函数时,设计合理的前瞻性随机试验对于指导临床决策是强制性的。"
\end{quote}

\textbf{非比例风险的含义}:
\begin{itemize}
    \item VinV:短期风险低,长期风险高
    \item Redo SAVR:短期风险高,长期风险低
    \item 最佳选择取决于患者预期寿命和治疗目标
\end{itemize}

\textbf{观察性研究的固有局限}:
\begin{itemize}
    \item \textbf{选择偏倚}:医生根据患者特征选择治疗方式
    \item \textbf{残余混淆}:即使倾向性匹配,仍可能存在未测量的混淆因素
    \item \textbf{测量偏倚}:不同治疗组的随访强度可能不同
    \item \textbf{缺失数据}:注册研究通常存在大量缺失数据
\end{itemize}

\subsubsection{临床决策的复杂性}

对于年轻、低风险的生物瓣衰败患者,治疗选择面临困境:

\textbf{VinV的优势}:
\begin{itemize}
    \item 围手术期并发症少
    \item 短期死亡率低
    \item 恢复快
    \item 避免再次开胸
\end{itemize}

\textbf{VinV的劣势}:
\begin{itemize}
    \item 血流动力学性能差(高梯度、高PPM率、高瓣周漏率)
    \item 长期死亡率可能更高
    \item 心衰住院率更高
    \item 未来再次干预的选择受限(valve-in-valve-in-valve?)
\end{itemize}

\textbf{Redo SAVR的优势}:
\begin{itemize}
    \item 优异的血流动力学性能
    \item 长期死亡率可能更低
    \item 可以植入更大尺寸的瓣膜
    \item 如果需要,未来仍可选择VinV
\end{itemize}

\textbf{Redo SAVR的劣势}:
\begin{itemize}
    \item 围手术期风险较高
    \item 恢复时间长
    \item 再次开胸的风险
    \item 手术复杂性
\end{itemize}

\subsection{REPEAT试验:设计合理的随机对照试验}

\subsubsection{试验设计}

\textbf{试验全称}:REpeat Intervention For Failed Surgical BioProsthEtic AorTic Valves (REPEAT)

\textbf{试验设计}:多中心随机对照试验,比较VinV TAVR与Redo SAVR在低风险患者中的疗效

\textbf{主要研究者}:
\begin{itemize}
    \item Michael A. Borger, MD, PhD (Leipzig Heart Center)
    \item Raj Makkar, MD(合作研究者)
\end{itemize}

\textbf{资金支持}:
\begin{itemize}
    \item 德国研究基金会(DFG)- 已批准
    \item 英国心脏基金会(BHF)- 第二轮评审中
    \item 寻求北美和澳大利亚合作
\end{itemize}

\subsubsection{纳入标准}

\begin{enumerate}
    \item \textbf{因结构性瓣膜衰败(SVD)导致的生物瓣膜功能衰竭,有再次干预指征}
    \item \textbf{低手术风险}:STS PROM <8\%
    \item \textbf{年龄}:>18岁且<75岁
    \item \textbf{心脏团队评估}:经当地心脏团队评估,Redo SAVR和VinV均为合理选择
    \begin{itemize}
        \item 包括冠状动脉解剖评估
        \item 包括既往植入瓣膜特征评估
    \end{itemize}
\end{enumerate}

\subsubsection{排除标准}

关键排除标准(部分):
\begin{enumerate}
    \item \textbf{需要冠状动脉旁路移植术(CABG)}的患者:
    \begin{itemize}
        \item 左主干狭窄>50\%,或
        \item 近段3支血管病变,或
        \item 近段2支血管病变伴Syntax评分>32,或
        \item 复杂CAD需要血运重建但无法通过PCI完成
    \end{itemize}
    \item \textbf{瓣膜衰败为非最佳状态}:
    \begin{itemize}
        \item 瓣周漏(主要≥中度)或
        \item 重度患者-瓣膜不匹配(EOA <0.65 cm²/m²)或
        \item 瓣膜内血栓
    \end{itemize}
    \item \textbf{活动性感染性心内膜炎}
    \item \textbf{需要同期处理的其他瓣膜病变}(中-重度或重度)
    \item \textbf{机械瓣膜失效}
    \item \textbf{选择接受机械瓣膜的患者}
\end{enumerate}

\subsubsection{干预措施}

\textbf{试验干预}:

患者随机分配至以下两组之一:
\begin{enumerate}
    \item \textbf{VinV组}:采用经股动脉入路的VinV TAVR
    \item \textbf{Redo SAVR组}:采用常规开胸入路的Redo SAVR,植入新的生物瓣膜
\end{enumerate}

\textbf{随访计划}:
\begin{itemize}
    \item 围手术期:第1天和第30天(±30天)后随访
    \item 长期随访:每6个月随访(±30天)至出院后随访
    \item 随访持续时间:120-240个月,取决于入组时间
\end{itemize}

\subsubsection{主要终点}

\textbf{主要终点}:

5年无下列事件生存率(复合终点):
\begin{enumerate}
    \item \textbf{全因死亡}
    \item \textbf{卒中}
    \item \textbf{心肌梗死}
    \item \textbf{心衰再住院}
    \item \textbf{主动脉瓣再干预}
\end{enumerate}

这一复合终点能够全面评估两种治疗策略的长期效果。

\subsubsection{样本量计算}

基于两项关键研究的数据进行样本量计算:

\begin{table}[h]
\centering
\caption{REPEAT试验样本量计算}
\label{tab:repeat_sample_size}
\begin{tabular}{lccccc}
\toprule
\textbf{参考研究} & \textbf{组别} & \textbf{5年无事件} & \textbf{差值} & \textbf{预期事件数} & \textbf{总样本量} \\
 &  & \textbf{生存率} &  &  & \textbf{(含10\%脱落)} \\
\midrule
Deharo 2020 & Redo SAVR & 39.14\% & 26.72\% & 120 & 412 \\
 & VinV & 12.42\% &  & 171 &  \\
\midrule
Tran 2023 & Redo SAVR & 77.25\% & 18.42\% & 91 & \textbf{890} \\
 & VinV & 58.83\% &  & 186 &  \\
\bottomrule
\end{tabular}
\end{table}

\textbf{最终选择}:
\begin{itemize}
    \item 基于Tran等人2023年研究的更乐观估计
    \item \textbf{总样本量:890例}(含10\%脱落)
    \item 预期总事件数:277例
    \item 检验效能:80\%
    \item 显著性水平:α = 0.05(双侧)
\end{itemize}

\subsubsection{试验可行性}

\textbf{德国参与中心}:

已有15个德国中心书面承诺参与,预期共招募485例患者:

\begin{table}[h]
\centering
\caption{REPEAT试验德国参与中心(部分)}
\label{tab:repeat_german_centers}
\begin{tabular}{clcc}
\toprule
\textbf{编号} & \textbf{中心名称} & \textbf{过去12个月} & \textbf{预期年招募} \\
 &  & \textbf{符合条件患者数} & \textbf{患者数} \\
\midrule
1 & Leipzig Herzzentrum & 71 & 114-128 \\
2 & Universitätsklinikum Schleswig-Holstein (Lübeck) & 40 & 70 \\
3 & Universitätsklinikum Düsseldorf & 25 & 40 \\
4 & Deutsches Herzzentrum München & 37 & 40 \\
5 & Universitätsklinikum Bonn & 25 & 40 \\
12 & Bad Oeynhausen (Herz- und Diabeteszentrum NRW) & 10 & 15 \\
13 & Freiburg & 20 & 12 \\
14 & Westdeutsches Herz- und Gefäßzentrum Essen & 23 & 10-14 \\
15 & Berlin Deutsches Herzzentrum & 5 & 5 \\
\midrule
 & \textbf{所有中心总计} &  & \textbf{485} \\
\bottomrule
\end{tabular}
\end{table}

\textbf{国际合作}:
\begin{itemize}
    \item 英国心脏基金会(BHF):进入第二轮评审
    \item 北美中心:已有口头和书面承诺参与
    \item 澳大利亚中心:已有口头和书面承诺参与
\end{itemize}

\subsection{结论}

\subsubsection{主要结论}

\begin{enumerate}
    \item \textbf{生物瓣膜使用率增加将导致再次干预需求增加}
    \begin{itemize}
        \item 越来越多年轻患者接受生物瓣膜
        \item 随着患者寿命延长,SVD发生率增加
        \item 未来10-20年,生物瓣膜衰败将成为重要临床问题
    \end{itemize}

    \item \textbf{Redo SAVR术后死亡率持续下降}
    \begin{itemize}
        \item 从2011-2013年的4.6\%下降
        \item 在排除心内膜炎的选择性患者中,死亡率约4.8\%
        \item 与首次SAVR死亡率相近
    \end{itemize}

    \item \textbf{VinV在缺乏随机证据的情况下成为主流治疗}
    \begin{itemize}
        \item 2015年VinV手术量约80例
        \item 2024年增长至约1150例
        \item 大部分基于观察性研究和注册数据
    \end{itemize}

    \item \textbf{VinV与Redo SAVR相比有不同的风险-收益特征}
    \begin{itemize}
        \item VinV:短期死亡率低,但血流动力学性能差,长期预后可能较差
        \item Redo SAVR:短期风险高,但血流动力学性能好,长期预后可能更优
        \item 存在非比例风险,最佳选择取决于患者预期寿命
    \end{itemize}

    \item \textbf{年轻、低风险患者迫切需要随机对照试验}
    \begin{itemize}
        \item REPEAT试验设计合理,针对最需要证据的患者群体
        \item 预期样本量890例,5年主要终点
        \item 已获德国DFG资助,国际合作正在扩展
    \end{itemize}
\end{enumerate}

\subsubsection{临床决策建议}

在REPEAT试验结果公布之前,对于生物瓣衰败患者的治疗选择应考虑:

\textbf{倾向选择VinV的情况}:
\begin{itemize}
    \item 高龄患者(>75岁)
    \item 高手术风险(STS PROM >8\%)
    \item 预期寿命有限
    \item 存在再次开胸禁忌
    \item 患者强烈偏好微创治疗
\end{itemize}

\textbf{倾向选择Redo SAVR的情况}:
\begin{itemize}
    \item 年轻患者(<65岁)
    \item 低手术风险(STS PROM <4\%)
    \item 预期寿命长(>10年)
    \item 小瓣膜衰败(植入VinV后会显著PPM)
    \item 需要同期处理其他心脏问题(如CABG、其他瓣膜)
    \item 既往VinV后再次衰败
\end{itemize}

\textbf{需要个体化决策的情况}:
\begin{itemize}
    \item 年龄65-75岁
    \item 中等手术风险(STS PROM 4-8\%)
    \item 中等预期寿命(5-10年)
    \item 应充分知情同意,讨论两种治疗的优劣势
    \item 考虑参加REPEAT试验
\end{itemize}

\subsection{临床启示}

\subsubsection{对临床实践的启示}

\begin{enumerate}
    \item \textbf{生物瓣膜选择的前瞻性思考}
    \begin{itemize}
        \item 对于可能需要再次干预的年轻患者,初次SAVR时应植入较大尺寸瓣膜
        \item 考虑未来VinV的可行性(避免小于21mm的瓣膜)
        \item 选择有利于VinV的瓣膜类型和位置
    \end{itemize}

    \item \textbf{心脏团队决策至关重要}
    \begin{itemize}
        \item 所有生物瓣衰败患者应经心脏团队评估
        \item 需要考虑冠状动脉解剖、其他瓣膜病变、预期寿命等多因素
        \item 充分告知患者两种治疗的短期和长期风险收益
    \end{itemize}

    \item \textbf{随访策略}
    \begin{itemize}
        \item VinV术后需密切监测血流动力学参数
        \item 特别关注瓣周漏、PPM、心衰症状
        \item Redo SAVR术后也需长期随访,监测新瓣膜功能
    \end{itemize}

    \item \textbf{支持随机对照试验}
    \begin{itemize}
        \item 符合条件的患者应考虑参加REPEAT试验
        \item 避免在缺乏充分证据的情况下过度使用VinV
        \item 特别是年轻、低风险患者
    \end{itemize}
\end{enumerate}

\subsubsection{对未来研究的启示}

\begin{enumerate}
    \item \textbf{需要更长期的随访数据}
    \begin{itemize}
        \item 目前大多数研究随访时间<5年
        \item 对于年轻患者,需要10年甚至更长期的结果
        \item REPEAT试验的10年随访数据将非常有价值
    \end{itemize}

    \item \textbf{血流动力学对长期预后的影响}
    \begin{itemize}
        \item VinV的高梯度、高PPM率是否会转化为长期不良结局?
        \item 需要研究血流动力学参数与临床事件的关系
        \item 生活质量评估
    \end{itemize}

    \item \textbf{新一代TAVR瓣膜的评估}
    \begin{itemize}
        \item VinV专用瓣膜是否能改善血流动力学?
        \item 瓣周漏率是否能降低?
        \item 需要持续评估技术改进的效果
    \end{itemize}

    \item \textbf{Valve-in-valve-in-valve的可行性}
    \begin{itemize}
        \item VinV后再次衰败如何处理?
        \item 第三次干预的可行性和效果?
        \item 这将影响初次治疗选择
    \end{itemize}
\end{enumerate}

\subsection{研究局限性}

\subsubsection{当前证据的局限性}

\begin{enumerate}
    \item \textbf{缺乏随机对照试验数据}
    \begin{itemize}
        \item 所有现有证据来自观察性研究
        \item 存在选择偏倚和残余混淆
        \item 因果推断受限
    \end{itemize}

    \item \textbf{异质性高}
    \begin{itemize}
        \item 不同研究的患者特征差异大
        \item 手术技术和瓣膜类型不同
        \item Meta分析异质性较高
    \end{itemize}

    \item \textbf{随访时间有限}
    \begin{itemize}
        \item 大多数研究中位随访<5年
        \item 对于年轻患者,需要更长期数据
        \item 长期并发症(如第二次SVD)尚未充分评估
    \end{itemize}

    \item \textbf{血流动力学数据不完整}
    \begin{itemize}
        \item 许多注册研究缺乏详细的超声心动图数据
        \item 瓣周漏分级可能不一致
        \item PPM定义和评估方法可能不同
    \end{itemize}

    \item \textbf{临床事件定义不统一}
    \begin{itemize}
        \item 不同研究对心衰住院、瓣膜再干预的定义可能不同
        \item 事件裁定方法可能不同
        \item 影响结果的可比性
    \end{itemize}
\end{enumerate}

\subsubsection{本综述的局限性}

\begin{enumerate}
    \item 本文献为会议演讲形式,不是完整的系统综述
    \item 数据主要来自已发表的Meta分析和重点研究
    \item 未进行独立的系统文献检索和质量评估
    \item REPEAT试验尚在筹备阶段,尚无结果数据
\end{enumerate}

\subsection{个人笔记}

\subsubsection{关键数字记忆}

\textbf{手术量趋势}:
\begin{itemize}
    \item VinV增长:2015年\textasciitilde 80例 → 2024年\textasciitilde 1150例
    \item Redo SAVR相对稳定:2015-2024年约300-450例/年
    \item VinV FDA批准年份:2019年
\end{itemize}

\textbf{Redo SAVR死亡率}:
\begin{itemize}
    \item STS数据库(2011-13):4.6\%
    \item Leipzig数据(2011-22,排除心内膜炎):4.8\%
    \item 与首次SAVR相近
\end{itemize}

\textbf{Meta分析关键结果}:
\begin{itemize}
    \item 短期死亡率:VinV优势,RR = 0.55 (0.34-0.91)
    \item 中期死亡率:无差异,HR = 1.27 (0.72-2.25)
    \item 瓣周漏:VinV劣势,RR = 4.18 (1.88-9.30)
    \item PPM:VinV劣势,RR = 3.12 (2.35-4.14)
\end{itemize}

\textbf{Tran研究(2024)关键发现}:
\begin{itemize}
    \item 5年全因死亡率:无差异,HR = 1.03
    \item 2年后死亡率:VinV劣势,HR = 2.97 (1.18-7.47)
    \item 2年后心衰住院:VinV劣势,HR = 3.81 (1.57-9.22)
\end{itemize}

\textbf{REPEAT试验关键参数}:
\begin{itemize}
    \item 样本量:890例
    \item 年龄:18-75岁
    \item STS PROM:<8\%
    \item 主要终点:5年无MACE及心衰住院/瓣膜再干预
    \item 德国预期入组:485例
\end{itemize}

\subsubsection{重要概念}

\begin{description}
    \item[SVD (Structural Valve Deterioration)] 结构性瓣膜衰败 - 生物瓣膜的阿喀琉斯之踵,包括瓣叶钙化、撕裂、纤维化等,导致瓣膜功能衰竭

    \item[VinV (Valve-in-Valve)] 瓣中瓣 - 在既往外科生物瓣膜内植入TAVR瓣膜的技术

    \item[Redo SAVR] 再次外科主动脉瓣置换 - 移除衰败的生物瓣膜,植入新的瓣膜

    \item[PPM (Patient-Prosthesis Mismatch)] 患者-瓣膜不匹配 - 植入的瓣膜有效瓣口面积相对于患者体表面积过小,导致残余梯度升高

    \item[非比例风险 (Non-proportional Hazards)] VinV和Redo SAVR的风险函数随时间变化不同:VinV短期风险低、长期风险高;Redo SAVR短期风险高、长期风险低

    \item[时间-变化风险比 (Time-Varying Hazard Ratio)] 风险比随时间变化,不符合Cox比例风险假设,需要特殊统计方法分析
\end{description}

\subsubsection{值得深思的问题}

\begin{enumerate}
    \item \textbf{为什么VinV的短期优势没有转化为长期优势?}
    \begin{itemize}
        \item 可能原因1:血流动力学性能差(高梯度、PPM、瓣周漏)导致心衰加重
        \item 可能原因2:VinV瓣膜耐久性可能不如外科瓣膜
        \item 可能原因3:选择偏倚 - VinV组患者基线合并症可能更多
        \item 需要REPEAT试验来验证真正原因
    \end{itemize}

    \item \textbf{对于55岁的生物瓣衰败患者,应该选择哪种治疗?}
    \begin{itemize}
        \item 如果选择VinV:短期风险低,但可能在60-65岁时再次衰败,第三次干预困难
        \item 如果选择Redo SAVR:短期风险高,但如果顺利,可能在70-75岁时才需要再次干预,届时可选VinV
        \item Redo SAVR可能是更好的长期策略,但需要承担短期风险
        \item 这正是REPEAT试验要回答的核心问题
    \end{itemize}

    \item \textbf{VinV的血流动力学劣势是否可以接受?}
    \begin{itemize}
        \item PPM率高达26.6\%,这会导致什么长期后果?
        \item 瓣周漏率高达23.4\%,轻度瓣周漏是否影响预后?
        \item 梯度升高是否会加速左心室肥厚和功能恶化?
        \item 需要更多研究探讨血流动力学与临床预后的关系
    \end{itemize}

    \item \textbf{未来的Valve-in-valve-in-valve策略是否可行?}
    \begin{itemize}
        \item 如果首次VinV,10年后再次衰败,是否还能再次VinV?
        \item 连续VinV会导致瓣膜有效开口越来越小
        \item 可能最终仍需Redo SAVR
        \item 这影响初次治疗选择的决策
    \end{itemize}

    \item \textbf{如何平衡患者偏好与最佳医学证据?}
    \begin{itemize}
        \item 许多患者强烈偏好微创VinV,不愿再次开胸
        \item 但证据提示年轻患者Redo SAVR可能长期更优
        \item 如何充分知情同意,帮助患者做出最佳决策?
        \item 共同决策(Shared Decision Making)的重要性
    \end{itemize}
\end{enumerate}

\subsubsection{对中国的启示}

\begin{enumerate}
    \item \textbf{中国生物瓣使用趋势}
    \begin{itemize}
        \item 随着TAVR在中国的普及,越来越多患者接受生物瓣膜
        \item 10-15年后,中国也将面临大量生物瓣衰败患者
        \item 需要提前规划和准备
    \end{itemize}

    \item \textbf{建立VinV和Redo SAVR的质量控制体系}
    \begin{itemize}
        \item 规范化治疗流程
        \item 建立注册研究,收集长期随访数据
        \item 培训足够的外科和介入团队
    \end{itemize}

    \item \textbf{参与国际多中心研究}
    \begin{itemize}
        \item 考虑中国中心参与REPEAT试验
        \item 获得高质量循证证据
        \item 提升中国在国际心脏瓣膜领域的影响力
    \end{itemize}

    \item \textbf{建立中国自己的数据库}
    \begin{itemize}
        \item 中国患者特征可能与欧美不同(年龄、体型、合并症等)
        \item 需要基于中国数据的决策支持
        \item 建立类似STS或TVT Registry的中国注册研究
    \end{itemize}
\end{enumerate}

\subsubsection{延伸阅读建议}

\begin{enumerate}
    \item Raschpichler M, et al. Valve-in-valve transcatheter aortic valve replacement versus redo surgical aortic valve replacement: a systematic review and meta-analysis. J Am Heart Assoc. 2022;11(3):e022392.

    \item Tran JH, et al. Transcatheter or surgical replacement for failed bioprosthetic aortic valves. JAMA Cardiol. 2024. [最新大型倾向性匹配研究]

    \item Deharo P, et al. Long-term prognosis value of paravalvular leak and patient-prosthesis mismatch following transcatheter aortic valve replacement: insight from the France TAVI registry. JACC Cardiovasc Interv. 2020;13(19):2196-2206.

    \item Borger MA, Raschpichler M, Makkar R. When an aortic bioprosthesis fails in a low-risk patient, randomize. JAMA Cardiol. 2022;7(5):473-474. [重要社论]

    \item Kaneko T, et al. Contemporary outcomes of repeat aortic valve replacement: a benchmark for valve-in-valve procedures. Ann Thorac Surg. 2015;100(4):1298-1304.
\end{enumerate}

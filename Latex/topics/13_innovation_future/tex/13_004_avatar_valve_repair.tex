\section{AVaTAR MedTech:革新外科主动脉瓣修复技术}
\label{sec:13_004_avatar_valve_repair}

% ============================================
% 文献信息
% ============================================
\subsection{文献信息}

\begin{itemize}
    \item \textbf{标题}: Revolutionizing Surgical Aortic Valve Repair
    \item \textbf{作者}: Ignacio Lugones, MD PhD
    \item \textbf{机构}: AVaTAR MedTech; Long Island University (Brooklyn, NY, USA); Hospital de Niños Dr. Pedro de Elizalde (Buenos Aires, Argentina)
    \item \textbf{会议}: TCT (Transcatheter Cardiovascular Therapeutics)
    \item \textbf{PDF文件名}: avatar-medtech-revolutionizing-surgical-aortic-valve-repair.pdf
    \item \textbf{文献类型}: 会议演讲/技术介绍
\end{itemize}

\subsection{研究背景}

\subsubsection{健康主动脉瓣的特征}

人类健康主动脉瓣具有以下理想特征:

\begin{itemize}
    \item \textbf{三叶结构}(Trileaflet)
    \item \textbf{对称性}(Symmetrical)
    \item \textbf{功能完整}(Competent)
    \item \textbf{非狭窄性}(Non-stenotic)
    \item \textbf{可随生长}(Grows)
    \item \textbf{自体活组织}(Autologous living tissue)
\end{itemize}

\textbf{进化学意义}:

哺乳动物、鸟类、爬行动物甚至恐龙都共享相同的瓣膜形态学,这表明这种三叶瓣膜结构在进化上具有高度保守性和优越性。

\subsubsection{现有治疗方案的局限性}

\textbf{成人患者的次优治疗选择}:

\begin{table}[h]
\centering
\caption{成人主动脉瓣疾病治疗方案及局限性}
\label{tab:adult_av_treatments}
\begin{tabular}{lp{8cm}}
\toprule
\textbf{治疗方案} & \textbf{主要局限性} \\
\midrule
机械瓣膜 & 终身抗凝治疗;活动受限 \\
生物瓣膜 & 耐久性有限 \\
TAVI & 主要适用于老年患者 \\
AV Neo(Ozaki术式) & 可重复性有限 \\
\bottomrule
\end{tabular}
\end{table}

\textbf{儿童患者面临极大挑战}:

\begin{table}[h]
\centering
\caption{儿童主动脉瓣疾病治疗方案及局限性}
\label{tab:pediatric_av_treatments}
\begin{tabular}{lp{8cm}}
\toprule
\textbf{治疗方案} & \textbf{主要局限性} \\
\midrule
机械瓣膜 & 终身抗凝;不能生长;小尺寸不可用 \\
生物瓣膜 & 耐久性有限;不能生长;小尺寸不可用 \\
瓣膜成形术 & 效果不佳且困难 \\
AV Neo(Ozaki) & 非专为儿童设计 \\
Ross手术 & 技术要求高、风险大 \\
\bottomrule
\end{tabular}
\end{table}

\textbf{核心问题}:

演讲指出:"现有人工瓣膜之所以存在,是因为我们从未找到一种\textbf{可重复的方法},使用\textbf{自体活组织}创建功能良好且\textbf{能够适应躯体生长}的新瓣膜。"

\subsection{研究方法}

\subsubsection{AVaTAR瓣膜的设计理念}

AVaTAR技术的核心思想是\textbf{模仿自然}(mimicking Mother Nature),创建具有天然主动脉瓣所有优良特性的新瓣膜。

\textbf{AVaTAR瓣膜的关键特征}:

\begin{itemize}
    \item ✓ 三叶结构
    \item ✓ 对称性
    \item ✓ 功能完整(无反流)
    \item ✓ 非狭窄性
    \item ✓ \textbf{适应生长能力}
    \item ✓ 自体活组织
\end{itemize}

\subsubsection{技术实现方法}

\textbf{1. 一次性手术器械套装}:

AVaTAR MedTech开发了专用的一次性手术工具套装,使得任何外科医生都能以\textbf{简便和可重复}的方式完成手术。

\begin{itemize}
    \item \textbf{知识产权}:已提交专利(WIPO PCT国际专利体系)
    \item \textbf{监管分类}:预期为FDA Class I类器械
    \item \textbf{审批途径}:510(k)豁免
    \item \textbf{报销}:可使用现有CPT编码报销
\end{itemize}

\textbf{2. 材料来源}:

使用患者\textbf{自体新鲜心包}构建瓣膜叶片,无需化学处理(如戊二醛固定)。

\subsubsection{体外验证(In Vitro Test)}

\textbf{测试设置}(Carlson Hanse et al - ICVTS 2022):

使用脉冲流体力学模拟系统对AVaTAR瓣膜进行测试,并与天然瓣膜对比。

\textbf{关键观察指标}:

\begin{itemize}
    \item \textbf{纤维束(Fiber bundles)}:AVaTAR瓣膜显示出类似天然瓣膜的纤维束结构
    \item \textbf{无狭窄}:彩色多普勒显示无压力梯度
    \item \textbf{无反流}:舒张期完全闭合,无反流信号
\end{itemize}

\subsubsection{体内验证(In Vivo Test)}

\textbf{动物实验}(Carlson Hanse et al - WJPCHS 2023):

在猪模型中植入\textbf{超大尺寸}(oversized)AVaTAR瓣膜,验证其生长适应性。

\textbf{超声心动图结果}:

\begin{itemize}
    \item 无狭窄
    \item 无反流
    \item 新瓣膜功能正常
\end{itemize}

\textbf{生长适应性验证}:

通过在生长中的猪体内植入超大瓣膜,观察到:

\begin{enumerate}
    \item \textbf{风车形状}(Windmill shape):早期童年阶段
    \item \textbf{增加的对位}(Increased coaptation):贯穿所有生长阶段
    \item \textbf{负向膨出}(Negative billow):防止反流
    \item 随时间推移,瓣叶形态从童年早期、中期到青春期逐渐演变,\textbf{适应主动脉环的扩张}
\end{enumerate}

这证明AVaTAR瓣膜在12mm(早期童年)到更大尺寸(青春期)的过程中能够适应生长。

\subsection{主要研究发现}

\subsubsection{临床病例1:6岁儿童}

\textbf{患者信息}:
\begin{itemize}
    \item 年龄:6岁
    \item 诊断:严重主动脉瓣反流(瓣膜成形术后)
\end{itemize}

\textbf{术后1周超声心动图结果}:

\begin{table}[h]
\centering
\caption{6岁患者术后1周超声心动图评估}
\label{tab:case1_echo}
\begin{tabular}{lc}
\toprule
\textbf{评估指标} & \textbf{结果} \\
\midrule
风车形状 & 存在 \\
增加的对位 & 显著 \\
负向膨出 & 存在 \\
狭窄程度 & 无狭窄 \\
反流程度 & 无反流 \\
\bottomrule
\end{tabular}
\end{table}

患者术后1周照片显示恢复良好,活动正常。

\subsubsection{临床病例2:Gala(3岁女童)}

\textbf{最新病例详细记录}:

\textbf{患者基本信息}:
\begin{itemize}
    \item 姓名:Gala
    \item 年龄:3岁
    \item 诊断:严重主动脉瓣狭窄和反流
\end{itemize}

\textbf{手术详情}:
\begin{itemize}
    \item 使用AVaTAR技术
    \item 材料:自体新鲜心包
    \item 原生瓣膜切除,构建新瓣膜
\end{itemize}

\textbf{术后恢复时间线}:

\begin{table}[h]
\centering
\caption{Gala术后恢复时间线}
\label{tab:gala_recovery}
\begin{tabular}{lp{10cm}}
\toprule
\textbf{时间点} & \textbf{临床状态} \\
\midrule
术后第2天 & 在床上进食早餐,状态良好 \\
术后第3天 & 在医院内走动 \\
术后第5天 & 出院回家,挥手告别医生 \\
\bottomrule
\end{tabular}
\end{table}

\textbf{术后超声心动图表现}:

\begin{itemize}
    \item \textbf{风车形状}:明显可见
    \item \textbf{增加的对位}:瓣叶闭合良好
    \item \textbf{负向膨出}:防止反流
    \item \textbf{无狭窄}:彩色多普勒无压力梯度
    \item \textbf{无反流}:完全无反流信号
\end{itemize}

这个病例展示了AVaTAR技术在儿童严重瓣膜病变中的卓越效果和快速恢复能力。

\subsection{结论}

\subsubsection{技术创新性}

AVaTAR技术代表了主动脉瓣修复领域的重大突破:

\begin{enumerate}
    \item \textbf{首次实现}使用自体活组织创建功能完整的新瓣膜
    \item \textbf{可重复性高}:通过专用器械套装,任何外科医生都能掌握
    \item \textbf{生长适应性}:特别适合儿童患者,随躯体生长而适应
    \item \textbf{无需抗凝}:自体活组织,无血栓形成风险
    \item \textbf{监管优势}:Class I器械,510(k)豁免,审批快速
    \item \textbf{经济可行性}:使用现有CPT编码报销
\end{enumerate}

\subsubsection{与现有技术的对比优势}

\begin{table}[h]
\centering
\caption{AVaTAR瓣膜 vs 现有治疗方案对比}
\label{tab:avatar_comparison}
\begin{tabular}{lccccc}
\toprule
\textbf{特征} & \textbf{AVaTAR} & \textbf{机械瓣} & \textbf{生物瓣} & \textbf{Ross} & \textbf{Ozaki} \\
\midrule
自体组织 & ✓ & ✗ & ✗ & ✓ & ✗ \\
无需抗凝 & ✓ & ✗ & ✓ & ✓ & ✓ \\
可生长 & ✓ & ✗ & ✗ & ✓ & ✗ \\
可重复性 & ✓ & ✓ & ✓ & ✗ & △ \\
适用儿童 & ✓ & △ & △ & △ & ✗ \\
手术风险 & 低-中 & 中 & 中 & 高 & 中 \\
\bottomrule
\end{tabular}
\end{table}

注:✓=优势;✗=劣势;△=有限

\subsection{临床启示}

\subsubsection{对儿童心脏外科的革命性意义}

\textbf{1. 解决长期困扰的难题}:

儿童主动脉瓣疾病一直是心脏外科最具挑战性的领域之一,AVaTAR技术提供了突破性解决方案:

\begin{itemize}
    \item \textbf{避免多次手术}:传统治疗中儿童患者需要随生长进行多次瓣膜置换,AVaTAR瓣膜的生长适应性可能大幅减少再次手术需求
    \item \textbf{避免终身抗凝}:儿童使用机械瓣需终身抗凝,严重影响生活质量和安全性
    \item \textbf{保留正常解剖}:不同于Ross手术需要移位肺动脉瓣,AVaTAR在原位重建瓣膜
    \item \textbf{小尺寸可用}:可为婴幼儿制作合适尺寸的瓣膜
\end{itemize}

\textbf{2. 成人患者的新选择}:

对于年轻成人和中年患者,AVaTAR同样具有优势:

\begin{itemize}
    \item 避免抗凝相关并发症
    \item 延长瓣膜使用寿命(活组织可能具有更好的耐久性)
    \item 保持正常血流动力学
    \item 无异物感
\end{itemize}

\subsubsection{对临床实践的影响}

\textbf{1. 技术普及性}:

\begin{itemize}
    \item 专用器械套装降低了技术门槛
    \item 不需要像Ross手术那样的高度专业化技能
    \item 可重复性确保了质量一致性
\end{itemize}

\textbf{2. 手术流程优化}:

\begin{itemize}
    \item 使用自体心包,无需准备同种异体或异种材料
    \item 新鲜组织,无需预处理
    \item 器械标准化,减少手术时间
\end{itemize}

\textbf{3. 患者选择考虑}:

AVaTAR技术的\textbf{理想适应症}:

\begin{enumerate}
    \item \textbf{儿童患者}(首选):
    \begin{itemize}
        \item 先天性主动脉瓣畸形
        \item 瓣膜成形术后反流
        \item 二叶式主动脉瓣合并狭窄/反流
    \end{itemize}

    \item \textbf{年轻成人}(<50岁):
    \begin{itemize}
        \item 不适合或拒绝抗凝治疗者
        \item 有生育计划的女性
        \item 活动量大的患者
    \end{itemize}

    \item \textbf{瓣膜反流为主}的病变:
    \begin{itemize}
        \item 可保留部分原生瓣环结构
        \item 心包质量良好
    \end{itemize}
\end{enumerate}

\textbf{可能的相对禁忌症}:

\begin{itemize}
    \item 心包质量不佳(既往心包炎、放疗后等)
    \item 严重主动脉根部扩张需同时处理
    \item 急性感染性心内膜炎活动期
\end{itemize}

\subsubsection{对心血管外科未来的启示}

AVaTAR技术体现了心血管外科发展的重要趋势:

\begin{enumerate}
    \item \textbf{回归自然}:使用自体组织而非人工材料
    \item \textbf{再生医学整合}:利用机体自身修复和适应能力
    \item \textbf{技术标准化}:通过器械创新实现复杂手术的标准化
    \item \textbf{生物力学优化}:模仿天然瓣膜的几何结构和功能
    \item \textbf{患者中心}:关注长期生活质量而非仅关注短期结果
\end{enumerate}

\subsection{研究局限性}

\subsubsection{当前阶段的局限性}

\textbf{1. 临床数据有限}:

\begin{itemize}
    \item 仅展示了\textbf{2例临床病例}(6岁和3岁儿童)
    \item 随访时间短(仅展示术后1周至5天的数据)
    \item 缺乏长期预后数据(如5年、10年生存率)
    \item 未提供详细的血流动力学参数
    \item 样本量太小,无法评估统计学意义
\end{itemize}

\textbf{2. 缺乏对照研究}:

\begin{itemize}
    \item 无随机对照试验(RCT)数据
    \item 未与标准治疗方案进行系统比较
    \item 缺乏多中心验证
    \item 未报告失败病例或并发症
\end{itemize}

\textbf{3. 技术细节不完整}:

\begin{itemize}
    \item 未详细说明瓣叶尺寸的精确测量方法
    \item 心包处理的具体步骤不明确
    \item 缝合技术的细节未充分展示
    \item 器械的具体工作原理未完全公开(专利保护)
    \item 手术适应症和禁忌症标准未明确定义
\end{itemize}

\textbf{4. 生长适应性证据不足}:

\begin{itemize}
    \item 动物实验数据有限,未提供完整的生长曲线
    \item 人类儿童的实际生长适应性尚待长期观察
    \item 不同年龄段的适应能力可能存在差异
    \item 超大尺寸瓣膜在儿童体内的长期表现未知
\end{itemize}

\textbf{5. 并发症数据缺失}:

\begin{itemize}
    \item 未报告术中并发症
    \item 未提供再手术率
    \item 感染、血栓、瓣膜退化等风险未评估
    \item 缺乏失败病例分析
\end{itemize}

\subsubsection{演讲本身的局限性}

\textbf{1. 利益冲突}:

\begin{itemize}
    \item 演讲者是AVaTAR MedTech的联合创始人和首席科学官
    \item 可能存在对技术优势的过度强调
    \item 商业利益可能影响数据呈现的客观性
\end{itemize}

\textbf{2. 信息披露不完整}:

\begin{itemize}
    \item 未提供完整的文献引用
    \item 动物实验的详细方法学未公开
    \item 临床病例的完整病历资料未展示
    \item 监管审批的具体进展不明确
\end{itemize}

\textbf{3. 缺乏同行评审}:

\begin{itemize}
    \item 会议演讲形式,非正式发表的研究论文
    \item 未经过严格的同行评审过程
    \item 数据可靠性和可重复性待验证
\end{itemize}

\subsubsection{未来需要解决的问题}

\begin{enumerate}
    \item \textbf{长期随访}:至少需要5-10年的随访数据
    \item \textbf{大样本临床试验}:需要多中心RCT验证安全性和有效性
    \item \textbf{不同病因的适用性}:先天性 vs 获得性病变
    \item \textbf{年龄分层分析}:新生儿、婴儿、儿童、青少年、成人的不同表现
    \item \textbf{与Ozaki技术的直接比较}:评估相对优劣
    \item \textbf{成本效益分析}:与现有治疗方案的经济学比较
    \item \textbf{学习曲线研究}:外科医生掌握技术所需的病例数
    \item \textbf{失败模式分析}:技术失败的原因和预防措施
\end{enumerate>

\subsection{个人笔记}

\subsubsection{关键数字和数据点}

\textbf{核心技术参数}:
\begin{itemize}
    \item \textbf{监管分类}:FDA Class I(预期)
    \item \textbf{审批途径}:510(k)豁免
    \item \textbf{专利状态}:已提交WIPO PCT国际专利
    \item \textbf{报销编码}:使用现有CPT编码
    \item \textbf{最小瓣膜尺寸}:12mm(可用于早期儿童)
\end{itemize}

\textbf{临床病例数据}:
\begin{itemize}
    \item \textbf{病例1}:6岁,术后1周,无狭窄/无反流
    \item \textbf{病例2(Gala)}:3岁,术后5天出院
    \item \textbf{住院时间}:5天(Gala病例)
    \item \textbf{术后恢复}:第2天进食,第3天下床活动
\end{itemize}

\textbf{研究发表}:
\begin{itemize}
    \item Carlson Hanse et al - ICVTS 2022(体外测试)
    \item Carlson Hanse et al - WJPCHS 2023(体内测试、生长适应性)
\end{itemize}

\subsubsection{重要概念与技术特点}

\begin{description}
    \item[风车形状(Windmill shape)] AVaTAR瓣膜的特征性超声心动图表现,瓣叶呈风车状排列,类似天然瓣膜的三叶对称结构

    \item[负向膨出(Negative billow)] 舒张期瓣叶向心室侧轻微凹陷,增加瓣叶对位面积,有效防止反流

    \item[增加的对位(Increased coaptation)] 瓣叶闭合时的接触面积增大,确保完全闭合,这是AVaTAR设计的核心优势之一

    \item[纤维束(Fiber bundles)] 体外测试显示AVaTAR瓣膜可见类似天然瓣膜的纤维束结构,提示组织排列接近生理状态

    \item[生长适应性(Accommodates growth)] 最关键的创新点,瓣膜可随儿童主动脉环扩张而适应,从早期儿童(12mm)到青春期均保持功能

    \item[自体新鲜心包(Autologous fresh pericardium)] 使用患者自身心包组织,无需化学处理(如戊二醛固定),保留组织活性

    \item[超大尺寸策略(Oversized)] 在儿童体内植入略大于当前主动脉环的瓣膜,利用负向膨出和增加对位机制,确保即刻功能和长期适应性

    \item[一次性器械套装(Disposable set of surgical tools)] 标准化手术流程的关键,降低技术门槛,提高可重复性
\end{description}

\subsubsection{技术创新的关键点}

\textbf{1. 生物力学设计}:

AVaTAR的成功在于精确模仿了天然瓣膜的几何结构:
\begin{itemize}
    \item 三叶对称布局
    \item 每个瓣叶的曲率和厚度优化
    \item 风车状开放,最大化有效开口面积
    \item 负向膨出增加安全边际
\end{itemize}

\textbf{2. 材料选择的智慧}:

使用自体新鲜心包而非固定心包的优势:
\begin{itemize}
    \item 保留组织活性和细胞成分
    \item 避免钙化(固定组织的主要问题)
    \item 更好的生物相容性
    \item 潜在的重塑和修复能力
    \item 可能随生长而适应
\end{itemize}

\textbf{3. 工程化解决方案}:

通过专用器械实现:
\begin{itemize}
    \item 精确的瓣叶裁剪
    \item 标准化的缝合定位
    \item 对称性的保证
    \item 可重复的手术质量
\end{itemize}

\subsubsection{与Ozaki技术的对比思考}

AVaTAR技术与Ozaki主动脉瓣新生术(AV Neo)有相似之处,都使用自体心包重建三叶瓣膜,但关键区别可能包括:

\begin{table}[h]
\centering
\caption{AVaTAR vs Ozaki技术推测性对比}
\label{tab:avatar_vs_ozaki}
\begin{tabular}{lp{5.5cm}p{5.5cm}}
\toprule
\textbf{特征} & \textbf{AVaTAR} & \textbf{Ozaki} \\
\midrule
心包处理 & 新鲜心包,无化学处理 & 戊二醛固定6分钟 \\
专用器械 & 有标准化器械套装 & 需Ozaki模板,但技术依赖性更强 \\
生长适应性 & 明确强调,有实验证据 & 未专门设计,主要用于成人 \\
儿科应用 & 明确针对儿童优化 & 主要用于成人,儿童经验有限 \\
可重复性 & 强调任何外科医生可掌握 & 需要显著学习曲线 \\
超大尺寸策略 & 明确采用 & 未强调 \\
\bottomrule
\end{tabular}
\end{table}

注:以上对比基于演讲内容推测,实际差异需要直接比较研究验证。

\subsubsection{批判性思考}

\textbf{1. 需要警惕的问题}:

\begin{itemize}
    \item \textbf{选择偏倚}:展示的病例可能是最成功的案例
    \item \textbf{随访不足}:术后5天-1周的数据无法预测长期结果
    \item \textbf{技术成熟度}:作为新技术,可能仍在演进中
    \item \textbf{学习曲线}:虽声称易于掌握,但实际推广中可能面临挑战
\end{itemize}

\textbf{2. 需要更多证据的问题}:

\begin{itemize}
    \item 新鲜心包的长期耐久性如何?会否钙化?
    \item 生长适应性的极限在哪里?能适应多大的主动脉环增长?
    \item 不同年龄段(新生儿、婴儿、青少年、成人)的效果是否一致?
    \item 二叶瓣、单叶瓣等复杂畸形是否适用?
    \item 主动脉环扩张患者如何处理?
    \item 再手术时的技术挑战如何?
\end{itemize}

\textbf{3. 与经导管技术的关系}:

有趣的是,这是在TCT(经导管心血管治疗)会议上展示的外科技术,提示:
\begin{itemize}
    \item 未来可能发展经导管植入版本?
    \item 外科与介入的融合趋势
    \item 为未来"valve-in-valve"提供基础?
\end{itemize}

\subsubsection{对中国临床实践的启示}

\textbf{1. 适用人群}:

中国儿童先天性心脏病患者众多,AVaTAR技术如果得到验证,可能特别适合:
\begin{itemize}
    \item 风湿性心脏病导致的主动脉瓣病变(仍在某些地区存在)
    \item 先天性主动脉瓣畸形
    \item 不适合瓣膜成形术的病例
    \item 经济条件限制无法多次置换的家庭
\end{itemize}

\textbf{2. 技术引进考虑}:

\begin{itemize}
    \item 专利状态和授权问题
    \item 器械的进口或国产化
    \item 外科医生的培训
    \item 临床试验的监管要求
    \item 医保报销政策
\end{itemize}

\textbf{3. 本土创新机会}:

\begin{itemize}
    \item 可否开发类似但不侵权的技术?
    \item 结合中国患者特点进行优化
    \item 开展多中心临床研究
    \item 与Ozaki等现有技术对比
\end{itemize}

\subsubsection{值得关注的未来发展}

\textbf{1. 短期(1-2年)}:
\begin{itemize}
    \item FDA审批进展
    \item 首个大规模临床试验结果
    \item 在美国和欧洲的商业化推广
    \item 更多临床病例报告
\end{itemize}

\textbf{2. 中期(3-5年)}:
\begin{itemize}
    \item 5年随访数据发表
    \item 与标准治疗的RCT结果
    \item 技术改进和第二代产品
    \item 适应症扩展(如二尖瓣、肺动脉瓣)
\end{itemize}

\textbf{3. 长期(5-10年)}:
\begin{itemize}
    \item 儿童患者的生长适应性验证
    \item 长期耐久性数据
    \item 可能的经导管版本开发
    \item 组织工程和再生医学的整合
\end{itemize}

\subsubsection{总结性思考}

AVaTAR技术体现了\textbf{回归自然、模仿生理}的理念,这可能是瓣膜外科未来的重要方向。然而,作为临床医生,我们需要:

\begin{enumerate}
    \item \textbf{保持科学严谨}:等待充分的临床证据
    \item \textbf{批判性评估}:不被初步成功迷惑
    \item \textbf{关注长期结果}:瓣膜手术是终身性决定
    \item \textbf{个体化选择}:技术再好也不是适用于所有患者
    \item \textbf{持续学习}:跟踪技术发展和证据积累
\end{enumerate}

\textbf{最令人兴奋的一点}:如果AVaTAR的生长适应性得到验证,这将是儿童瓣膜外科的\textbf{范式转变}(paradigm shift),从"终身面对人工瓣膜的各种问题"转向"一次手术重建接近天然的瓣膜"。

\textbf{最需要谨慎的一点}:目前的证据极其有限,需要至少5-10年的大规模临床试验才能确定其真正的临床价值。

\subsubsection{联系信息}

如需进一步了解AVaTAR技术:

\begin{itemize}
    \item \textbf{联系人}:Ignacio Lugones, MD PhD
    \item \textbf{职位}:Chief Scientific Officer, AVaTAR MedTech
    \item \textbf{地点}:Buenos Aires, Argentina (GMT -3:00); Brooklyn, NY, USA
    \item \textbf{电话}:+54 9 221 525 6264
    \item \textbf{邮箱}:ignaciolugones@avatarmedtech.co
\end{itemize}

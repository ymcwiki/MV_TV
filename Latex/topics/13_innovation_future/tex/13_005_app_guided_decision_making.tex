\section{使用APP指导决策应对TAVR失败}
\label{sec:13_005_app_guided_decision_making}

% ============================================
% 文献信息
% ============================================
\subsection{文献信息}

\begin{itemize}
    \item \textbf{标题}: Navigating TAVR Failure Using App-Guided Decision Making
    \item \textbf{作者}: Miho Fukui, MD, PhD
    \item \textbf{机构}: Minneapolis Heart Institute Foundation
    \item \textbf{会议}: TCT (Transcatheter Cardiovascular Therapeutics)
    \item \textbf{PDF文件名}: navigating-tavr-failure-using-app-guided-decision-making.pdf
    \item \textbf{文献类型}: 会议演讲/技术介绍
    \item \textbf{利益冲突}: 研究支持:ANTERIS;顾问费/酬金:Medtronic, Edwards
\end{itemize}

\subsection{研究背景}

\subsubsection{TAVR失败的挑战}

随着TAVR技术的广泛应用和患者生存期的延长,TAVR瓣膜失败(valve failure)已成为一个日益重要的临床问题。面对TAVR失败,临床医生需要在以下治疗策略中做出选择:

\begin{itemize}
    \item \textbf{Redo-TAV}(TAV-in-TAV):在失败的TAVR瓣膜内再次植入经导管主动脉瓣
    \item \textbf{外科TAV取出}(TAV Explant):外科手术取出失败的TAVR瓣膜并进行SAVR
    \item \textbf{保守治疗}:对于高危患者
\end{itemize}

\subsubsection{标准化决策的必要性}

Redo-TAV手术的复杂性在于:

\begin{enumerate}
    \item \textbf{解剖学评估复杂}:
    \begin{itemize}
        \item 需要精确评估第一个TAV的位置、大小和状态
        \item 需要评估第二个TAV与第一个TAV的兼容性
        \item 需要评估冠状动脉阻塞风险
    \end{itemize}

    \item \textbf{技术决策复杂}:
    \begin{itemize}
        \item 第二个TAV的尺寸选择
        \item 植入深度的选择(NSP层面)
        \item 冠状动脉保护策略
    \end{itemize}

    \item \textbf{缺乏统一标准}:
    \begin{itemize}
        \item 不同中心使用不同的评估方法
        \item 缺乏标准化的术语和流程
        \item 学习曲线陡峭
    \end{itemize}
\end{enumerate}

\subsubsection{Redo TAV APP的开发}

为了应对这些挑战,由Minneapolis Heart Institute Foundation领导的国际团队开发了\textbf{Redo TAV APP}:

\begin{itemize}
    \item \textbf{平台}:iOS(App Store)和Android(Google Play)
    \item \textbf{目标}:提供从可行性评估到手术实施的标准化路径
    \item \textbf{特点}:免费、易用、基于循证医学和专家共识
\end{itemize}

\subsection{Redo TAV APP的主要功能}

\subsubsection{APP功能模块概览}

Redo TAV APP包含以下主要模块:

\begin{table}[h]
\centering
\caption{Redo TAV APP功能模块}
\label{tab:app_modules}
\begin{tabular}{llp{8cm}}
\toprule
\textbf{序号} & \textbf{模块名称} & \textbf{功能描述} \\
\midrule
1 & Procedural Guide & 手术指南,提供分步骤的手术决策流程 \\
2 & Redo-TAV CT Planning & CT规划工具,评估可行性和冠状动脉风险 \\
3 & Procedure Data \& Outcome & 手术数据和结果记录工具 \\
4 & Blank CT Summary Report & 可下载的CT总结报告模板 \\
5 & Terminology & 术语解释(NSP、CRP、VTA等) \\
6 & Coronary Access after Redo-TAV & Redo-TAV后冠状动脉通路的教育内容 \\
7 & Valve-Specific Resources & 各种TAVR瓣膜的特异性资源和信息 \\
8 & TAV Explant & TAV取出手术的技术指导 \\
9 & Case of the Month & 每月病例分享和学习 \\
\bottomrule
\end{tabular}
\end{table}

\subsubsection{CT规划:可行性评估的核心}

CT规划是Redo-TAV决策的核心环节,APP提供了\textbf{4个关键评估要素}:

\begin{enumerate}
    \item \textbf{第二个TAV的兼容性(2\textsuperscript{nd} TAV Compatibility)}
    \begin{itemize}
        \item 评估不同TAV品牌和型号之间的兼容性
        \item 基于Index TAV的设计特点选择合适的Second TAV
        \item 考虑支架框架设计、扩张特性等因素
    \end{itemize}

    \item \textbf{植入位置(Implant Position)}
    \begin{itemize}
        \item 确定第二个TAV的理想植入深度
        \item 定义NSP(Neoskirt Plane)层面
        \item 选择Node 3、4、5或6作为目标植入位置
        \item 平衡血流动力学和冠状动脉风险
    \end{itemize}

    \item \textbf{冠状动脉风险(Coronary Risk)}
    \begin{itemize}
        \item 评估冠状动脉阻塞(CAO)的风险
        \item 测量VTA(Virtual Transcatheter Aortic valve to coronary ostium)距离
        \item 分为高风险、中等风险、低风险三个等级
        \item 提供冠状动脉保护建议
    \end{itemize}

    \item \textbf{第二个TAV的尺寸选择(2\textsuperscript{nd} TAV Sizing)}
    \begin{itemize}
        \item 基于Index TAV的内径(inner diameter)
        \item 使用算法计算最佳Second TAV尺寸
        \item 考虑面积和周长匹配
        \item 避免尺寸过大(冠状动脉风险)或过小(反流、移位)
    \end{itemize}
\end{enumerate}

\subsubsection{标准化CT规划流程}

APP提供了一个\textbf{标准化的CT规划路径},包括以下步骤:

\textbf{步骤1:确认Index TAV信息}
\begin{itemize}
    \item 输入Index TAV的品牌和型号(如Evolut R)
    \item 输入Index TAV的尺寸(如29mm)
\end{itemize}

\textbf{步骤2:识别冠状动脉风险平面(CRP)}
\begin{itemize}
    \item CRP定义:低于Index TAV某一Node的平面
    \item 不同的Index TAV有不同的CRP参考Node
    \item CRP的位置影响Second TAV的选择和植入策略
\end{itemize}

\textbf{步骤3:选择Second TAV}
\begin{itemize}
    \item 基于Index TAV的类型选择兼容的Second TAV
    \item 示例:Evolut R 29mm + SAPIEN 3 Ultra 23mm
    \item APP自动计算面积和周长匹配度
\end{itemize}

\textbf{步骤4:评估可接受的NSP水平}
\begin{itemize}
    \item 对于特定的TAV组合,确定哪些NSP Node是可行的
    \item 示例流程图显示:
    \begin{itemize}
        \item 如果CRP高于Node 6 → 所有Node(3-6)均可接受
        \item 如果CRP在Node 6 → Node 5及以下可接受
        \item 如果CRP在Node 5 → Node 4及以下可接受
        \item 如果CRP在Node 4 → 仅Node 3可接受(部分瓣膜)
    \end{itemize}
\end{itemize}

\textbf{步骤5:Second TAV尺寸选择}
\begin{itemize}
    \item 在确定NSP Node后,选择合适的Second TAV尺寸
    \item 使用平均面积作为主要依据
    \item APP提供尺寸选择表格
\end{itemize}

\textbf{步骤6:冠状动脉风险评估(所有相关Node)}
\begin{itemize}
    \item 测量VTA距离(从模拟Second TAV支架到冠状动脉口的距离)
    \item 分别评估左右冠状动脉
    \item APP生成可视化总结,标注风险等级
\end{itemize}

\subsubsection{冠状动脉风险分级}

APP根据VTA测量值将冠状动脉阻塞风险分为三个等级:

\begin{table}[h]
\centering
\caption{冠状动脉阻塞风险分级}
\label{tab:coronary_risk}
\begin{tabular}{lcp{8cm}}
\toprule
\textbf{风险等级} & \textbf{标识颜色} & \textbf{建议} \\
\midrule
\textbf{高风险} & 红色 &
\begin{itemize}[leftmargin=*,nosep]
    \item RCA或LCA的VTA距离极短
    \item 强烈建议冠状动脉保护
    \item 考虑其他NSP Node或外科手术
\end{itemize} \\
\midrule
\textbf{中等风险} & 黄色 &
\begin{itemize}[leftmargin=*,nosep]
    \item 如有疑虑,考虑冠状动脉保护
    \item 密切监测
    \item 准备紧急冠状动脉干预设备
\end{itemize} \\
\midrule
\textbf{低风险} & 绿色 &
\begin{itemize}[leftmargin=*,nosep]
    \item VTA距离充足
    \item 必要时考虑冠状动脉保护
    \item 常规监测即可
\end{itemize} \\
\bottomrule
\end{tabular}
\end{table}

\textbf{VTA阈值示例}(具体数值因不同TAV组合而异):
\begin{itemize}
    \item \textbf{RCA}:1.1mm、2.2mm等
    \item \textbf{LCA}:2.2mm、2.8mm、3.3mm等
    \item \textbf{注意}:APP中"N/A"表示VTA测量不必要(风险极低)
\end{itemize}

\subsubsection{动画总结和可视化}

APP的一大特色是能够\textbf{生成动画总结},包括:

\begin{enumerate}
    \item \textbf{瓣膜组合示意图}:
    \begin{itemize}
        \item 显示Index TAV和Second TAV的相对位置
        \item 标注NSP Node位置
        \item 显示冠状动脉位置关系
    \end{itemize}

    \item \textbf{最窄VTA值}:
    \begin{itemize}
        \item RCA和LCA的最短距离
        \item 用颜色编码标识风险等级
    \end{itemize}

    \item \textbf{瓣膜对位(Commissure Alignment)}:
    \begin{itemize}
        \item 评估Index TAV的交界对位
        \item 分为4个等级:Aligned、Mild、Moderate、Severe misalignment
        \item 提供瓣膜旋转角度的可视化参考
    \end{itemize}

    \item \textbf{风险总结}:
    \begin{itemize}
        \item 显示"High risk to coronaries"(高风险)
        \item 或"Intermediate risk to coronaries"(中等风险)
        \item 或"Low risk to coronaries"(低风险)
    \end{itemize}
\end{enumerate}

\subsubsection{流程图和CT规划图表}

APP提供了\textbf{一页流程图}(One-page Flow Chart),概述了整个决策过程:

\textbf{针对S3-in-Evolut和MyVal-in-Evolut的示例流程}:
\begin{enumerate}
    \item \textbf{步骤1}:确认Index TAV
    \item \textbf{步骤2}:识别CRP相对于Index TAV的关系
    \item \textbf{步骤3}:选择Second TAV
    \item \textbf{步骤4}:评估可接受的NSP水平
    \item \textbf{步骤5}:评估CRP与NSP的关系
    \item \textbf{步骤6}:Second TAV尺寸选择
    \item \textbf{步骤7}:所有相关Node的冠状动脉风险评估
    \item \textbf{步骤8}:决策和手术计划
\end{enumerate}

此外,APP还提供\textbf{CT规划图表}(CT Planning Charts),包括:
\begin{itemize}
    \item 针对不同TAV组合的专门流程图
    \item 详细的测量标志点
    \item 尺寸选择表格
    \item 风险评估决策树
\end{itemize}

\subsection{手术指南功能}

\subsubsection{分步骤手术指导}

\textbf{Procedural Guide}模块提供了从CT分析到手术实施的完整指导:

\textbf{步骤1:选择Index TAV和尺寸}
\begin{itemize}
    \item 选择瓣膜类型(如Evolut R)
    \item 选择尺寸(如29mm)
\end{itemize}

\textbf{步骤2:选择Second TAV和尺寸}
\begin{itemize}
    \item 基于CT分析选择Second TAV(如SAPIEN 3 Ultra)
    \item 选择尺寸(如23mm)
    \item APP提示:"根据CT分析选择Second TAV的类型和尺寸"
\end{itemize}

\textbf{步骤3:Second TAV的植入水平}
\begin{itemize}
    \item APP显示不同NSP Node的植入选项
    \item 可视化显示:
    \begin{itemize}
        \item Node 6(最高位置)
        \item Node 5
        \item Node 4
        \item Node 3(仅用于AR,某些瓣膜)
    \end{itemize}
    \item 提供关键信息:如"S3流出在Node 6和4之间"
    \item 选择最佳NSP层面(如Node 5)
\end{itemize}

\textbf{步骤4:Second TAV实施}

APP为每个NSP Node提供了详细的实施指导,以\textbf{Node 5}为例:

\begin{table}[h]
\centering
\caption{Node 5植入参数示例(Evolut 29 + S3/3Ultra 23)}
\label{tab:node5_implantation}
\begin{tabular}{lp{10cm}}
\toprule
\textbf{参数} & \textbf{数值/说明} \\
\midrule
Index TAV & Evolut 29 \\
Second TAV & S3/3Ultra 23 \\
NSP level & Node 5 \\
\midrule
\textbf{流入到NSP的距离} & 21 mm \\
\textbf{S3/3Ultra 23的高度} & 18 mm \\
\textbf{S3流入在Node间的位置} & Node 1和80之间,深度3mm \\
\midrule
\multicolumn{2}{l}{\textit{注:不同NSP Node有不同的参数}} \\
\bottomrule
\end{tabular}
\end{table}

对于其他NSP Node:
\begin{itemize}
    \item \textbf{Node 6}:流入到NSP 21mm,S3/3Ultra 23高度18mm
    \item \textbf{Node 4}:流入到NSP 17mm,S3/3Ultra 23高度18mm,深度-1mm
    \item \textbf{Node 3}(仅AR):流入到NSP 14mm,S3/3Ultra 23高度18mm,深度-4mm
\end{itemize}

\subsubsection{术中可视化指导}

APP提供术中可视化参考:
\begin{itemize}
    \item 透视下的瓣膜位置示意图
    \item 关键解剖标志点的标注
    \item 植入深度的测量参考
    \item 实时调整建议
\end{itemize}

\subsection{手术数据和结果记录}

\subsubsection{手术数据记录(第1页)}

APP提供了详细的\textbf{手术数据表单},包括:

\textbf{基本信息}:
\begin{itemize}
    \item Index TAV:瓣膜类型和尺寸
    \item Second TAV:瓣膜类型和尺寸
\end{itemize}

\textbf{球囊预扩张}:
\begin{itemize}
    \item 是否进行(Yes/No)
    \item 球囊尺寸(mm)
\end{itemize}

\textbf{Second TAV部署}:
\begin{itemize}
    \item 充盈容量(Nominal/其他)
\end{itemize}

\textbf{球囊后扩张}:
\begin{itemize}
    \item 是否进行(Yes/No)
    \item 是否使用输送系统(Yes/No)
    \item 容量添加(cc)
\end{itemize}

\textbf{冠状动脉保护}:
\begin{itemize}
    \item 是否进行(Yes/No)
    \item 保护侧别(Right/Left/Both)
\end{itemize}

\textbf{冠状动脉支架植入}:
\begin{itemize}
    \item 是否进行(Yes/No)
\end{itemize}

\textbf{小叶修饰}(Leaflet Modification):
\begin{itemize}
    \item 是否进行(Yes/No)
\end{itemize}

\subsubsection{结果记录(第2页)}

\textbf{植入后NSP}:
\begin{itemize}
    \item 记录实际NSP位置(如Node 5)
\end{itemize}

\textbf{血流动力学结果}:
\begin{itemize}
    \item 最终平均跨瓣压差(导管测量):\_\_\_ mmHg
    \item 最终平均跨瓣压差(超声测量):\_\_\_ mmHg
\end{itemize}

\textbf{反流评估}:
\begin{itemize}
    \item 经瓣反流(Transvalvular AR):None/Trace/Mild/Moderate/Severe
    \item 瓣周反流(Paravalvular AR):None/Trace/Mild/Moderate/Severe
\end{itemize}

\textbf{主要并发症}:
\begin{itemize}
    \item \textbf{术中死亡}(Intraprocedural Death):Yes/No
    \item \textbf{转外科手术}(Conversion to Surgery):Yes/No
    \item \textbf{瓣膜栓塞}(Valve Embolization):Yes/No
    \item \textbf{需要另一个TAV}(Another TAV Needed):Yes/No
    \item \textbf{环破裂}(Annulus Injury):Yes/No
    \item \textbf{急性冠状动脉阻塞}(Acute Coronary Obstruction):Yes/No
    \begin{itemize}
        \item 阻塞位置(Obstruction):Right/Left/Both
        \item 疑似机制(Suspected Mechanism):下拉菜单
        \item 是否需要PCI(PCI Needed):下拉菜单
    \end{itemize}
\end{itemize}

\subsection{教育和资源模块}

\subsubsection{Redo-TAV后冠状动脉通路}

\textbf{Coronary Access after Redo-TAV}模块提供以下教育内容:

\begin{enumerate}
    \item \textbf{通路和导管}(Access and Catheters)
    \begin{itemize}
        \item 传统的冠状动脉插管技术在Redo-TAV后可能不可行
        \item 通路选择和导管选择在简化该问题中起重要作用
        \item 讨论桡动脉vs股动脉通路
        \item 讨论不同类型的导管
        \item 包含视频教学
    \end{itemize}

    \item \textbf{透视和Redo-TAV}(Fluoroscopy \& Redo-TAV)

    \item \textbf{窦隔离}(Sinus Sequestration)

    \item \textbf{小叶悬垂}(Leaflet Overhang)

    \item \textbf{交界对位与细胞对齐}(Commissural \& Cell Alignment)

    \item \textbf{冠状动脉阻塞}(Coronary Obstruction)
\end{enumerate}

\textbf{贡献者}:来自多国的专家团队(见致谢部分)

\subsubsection{TAV取出手术}

\textbf{TAV Explant}模块包括:

\begin{enumerate}
    \item \textbf{TAV设备}(TAV Devices)
    \begin{itemize}
        \item 不同TAVR瓣膜的设计特点
        \item 影响取出手术的结构因素
    \end{itemize}

    \item \textbf{CT扫描评估}(CT Scan Assessment)
    \begin{itemize}
        \item 术前CT评估要点
        \item 瓣膜位置、钙化、主动脉根部解剖
    \end{itemize}

    \item \textbf{手术步骤}(Procedural Steps)
    \begin{itemize}
        \item 插管和交叉钳夹
        \item 主动脉切开
        \item 心肌保护
        \item 从周围结构剥离装置
    \end{itemize}

    \textbf{关键学习要点}:
    \begin{enumerate}
        \item 插管和交叉钳夹
        \item 主动脉切开
        \item 心肌保护
        \item 从周围结构剥离装置
        \begin{itemize}
            \item 高瓣膜(Tall devices)
            \item 短瓣膜(Short devices)
        \end{itemize}
        \item 取出
    \end{enumerate}

    \item \textbf{瓣膜取出技术}(Valve Explant Techniques)

    \item \textbf{高级注意事项}(Advance Considerations)
\end{enumerate}

\textbf{视频资源}:
\begin{itemize}
    \item Evolut R TAV explant after 5 years for degeneration stenosis and regurgitation
    \item Evolut R TAV explant after 2 years for severe PV leak and mitral surgery
    \item Tourniquet Technique Evolut R
    \item Sapien 3 S3 explant tips
\end{itemize}

\subsubsection{术语解释}

\textbf{Terminology}模块提供了关键术语的详细定义:

\textbf{1. Neoskirt和Neoskirt Plane(NSP)}

\begin{description}
    \item[定义] NSP定义为一旦选择了redo-TAV组合,Neoskirt顶部的平面。NSP对于redo-TAV组合是唯一的,可能位于单个或多个水平。在多个水平可行的组合中,水平由Second TAV在Index TAV内的植入位置决定。NSP与天然解剖的关系(即冠状动脉口、窦管交界等)将根据Index TAV的深度而变化。

    \item[可视化] 提供Short-in-Short和Tall-in-Tall等不同组合的示意图
\end{description}

\textbf{2. 冠状动脉风险平面(Coronary Risk Plane, CRP)}

\begin{description}
    \item[定义] CRP是Index TAV上某个特定Node下方的平面
    \item[意义] CRP的位置决定了哪些NSP Node是安全可行的
\end{description}

\textbf{3. VTAoS, VTC和VTSTJ}

\begin{description}
    \item[VTAoS] Virtual Transcatheter Aortic valve to Aortic ostium distance(虚拟经导管主动脉瓣到主动脉口的距离)
    \item[VTC] Virtual valve to Coronary ostium(虚拟瓣膜到冠状动脉口)
    \item[VTSTJ] Virtual valve to Sinotubular Junction(虚拟瓣膜到窦管交界)
\end{description}

\textbf{4. 小叶悬垂(Leaflet Overhang)}

\textbf{5. 交界对位(Commissure Alignment)}

\begin{description}
    \item[分级]
    \begin{itemize}
        \item Aligned(对齐):0-15度
        \item Mild(轻度错位):15-30度
        \item Moderate(中度错位):30-45度
        \item Severe(重度错位):45-60度
    \end{itemize}
    \item[临床意义] 交界对位影响冠状动脉通路和血流动力学
\end{description}

\textbf{6. 冠状动脉保护(Coronary Protection)}

\subsubsection{瓣膜特异性资源}

APP提供了主流TAVR瓣膜的详细信息:

\begin{table}[h]
\centering
\caption{APP中包含的TAVR瓣膜}
\label{tab:tav_devices}
\begin{tabular}{ll}
\toprule
\textbf{制造商} & \textbf{瓣膜型号} \\
\midrule
Boston Scientific & ACURATE neo/neo2 \\
Abbott & Allegra \\
Medtronic & Evolut R/PRO/PRO+/FX \\
Boston Scientific & Lotus \\
Medtronic & MyVal \\
Abbott & Portico/Navitor \\
Edwards Lifesciences & SAPIEN 3/SAPIEN 3 Ultra \\
Abbott & SAPIEN XT \\
\bottomrule
\end{tabular}
\end{table}

对于每种瓣膜,APP提供:

\textbf{以Portico/Navitor为例}:

\begin{enumerate}
    \item \textbf{瓣膜设计}(Valve Design)
    \begin{itemize}
        \item 设计特点:自扩张、镍钛金属支架框架、高瓣膜
        \item 迭代版本:Portico, Navitor
        \item 环内/环上植入
    \end{itemize}

    \item \textbf{瓣膜尺寸}(Valve Dimensions)
    \begin{itemize}
        \item 可用尺寸:4种(23, 25, 27, 29)
        \item 形状:所有尺寸形状相同
    \end{itemize}

    \item \textbf{Second TAV选项}(Second TAV Options)
    \begin{itemize}
        \item 短瓣膜:SAPIEN 3家族
        \item 高瓣膜:Evolut家族
    \end{itemize}

    \item \textbf{NSP水平}(NSP Levels)
    \begin{itemize}
        \item 列出可用的Node位置
    \end{itemize}

    \item \textbf{CT分析示例}(CT Analysis Example)

    \item \textbf{尺寸表}(Sizing Table)
    \begin{itemize}
        \item 不同Second TAV的尺寸匹配表
        \item 基于面积和周长的计算
    \end{itemize}

    \item \textbf{视频部分}(Video Section)
    \begin{itemize}
        \item 手术演示视频
        \item 专家讲解
    \end{itemize}
\end{enumerate}

\textbf{重要CT和透视标志点}:
\begin{itemize}
    \item NSP位置(不同Node)
    \item 小叶最低点:Node 1
    \item 小叶顶部:交界片高度(leaflet height)
\end{itemize}

\textbf{Second TAV尺寸选择的测量}:
\begin{itemize}
    \item 短瓣膜:NSP处的平均面积和3 nodes以下(用于collar-to-collar跟踪)
    \item 高瓣膜:NSP的相同尺寸或更小尺寸的Evolut
\end{itemize}

\subsection{全球合作与专家贡献}

\subsubsection{国际专家团队}

Redo TAV APP的开发得到了来自\textbf{全球15个以上中心}的专家支持:

\begin{table}[h]
\centering
\caption{主要贡献者(部分)}
\label{tab:contributors}
\begin{tabular}{llll}
\toprule
\textbf{姓名} & \textbf{机构} & \textbf{城市/国家} \\
\midrule
Vinayak (Vinnie) Bapat & Minneapolis Heart Institute Foundation & Minneapolis, USA \\
Miho Fukui & Minneapolis Heart Institute Foundation & Minneapolis, USA \\
Atsushi Okada & Minneapolis Heart Institute Foundation & Minneapolis, USA \\
Mady Olson & Minneapolis Heart Institute Foundation & Minneapolis, USA \\
\midrule
Uri Landes & Rabin Medical Center & Israel \\
Janar Sathananthan & St. Paul's Hospital & Vancouver, Canada \\
Ole De Backer & Rigshopsitalet & Copenhagen, Denmark \\
Syed Zaid & Baylor College of Medicine & Houston, USA \\
Gilbert Tang & Mount Sinai Hospital & New York, USA \\
\midrule
Tsuyoshi Kaneko & Washington University & St. Louis, USA \\
Shinichi Fukuhara & University of Michigan & Ann Arbor, USA \\
Kiahitone Ronald Thao & Minneapolis Heart Institute Foundation & Minneapolis, USA \\
Ross Garberich & Minneapolis Heart Institute Foundation & Minneapolis, USA \\
Dariusz Dudek & Jagiellonian University Medical College & Poland \\
\midrule
Hasan Jilaihawi & Cedar Sinai Hospital & Los Angeles, USA \\
Daniel Blackman & Leeds Teaching Hospital & Leeds, UK \\
John Lesser & Minneapolis Heart Institute & Minneapolis, USA \\
Mohamed Abdel-Wahab & Heart Center Leipzig - University of Leipzig & Leipzig, Germany \\
Michael Reardon & Baylor College of Medicine & Houston, USA \\
\midrule
Arif Khokhar & Hammersmith Hospital, Imperial College Healthcare NHS Trust & London, UK \\
Alessandro Beneduce & IRCCS San Raffaele Scientific Institute & Milan, Italy \\
Martin Leon & Columbia University Medical Center & New York, NY \\
Michael Mack & Baylor Scott \& White Health System, Baylor Plano Research Center & Dallas, Texas \\
\bottomrule
\end{tabular}
\end{table}

\subsection{主要结论和核心信息}

\subsubsection{Take-home Message}

演讲总结了以下核心信息:

\begin{enumerate}
    \item \textbf{全球合作的成果}
    \begin{itemize}
        \item 该APP是通过全球合作创建的
        \item 汇集了来自美国、以色列、加拿大、丹麦、德国、意大利、英国、波兰等多国专家的智慧
        \item 代表了当前Redo-TAV领域的最佳实践
    \end{itemize}

    \item \textbf{这不是终点,而是起点}
    \begin{itemize}
        \item APP不是最终版本
        \item 它是持续学习和改进的起点
        \item 随着经验积累,将不断更新和完善
    \end{itemize}

    \item \textbf{目标:简化、标准化、优化}
    \begin{itemize}
        \item \textbf{简化}(Simpler):使复杂的决策过程变得简单易行
        \item \textbf{标准化}(Standardized):提供统一的术语、流程和评估方法
        \item \textbf{优化}(Optimal):基于循证医学和专家共识,实现最佳临床结果
    \end{itemize}

    \item \textbf{持续改进的承诺}
    \begin{itemize}
        \item 需要继续完善,正如我们对原生AS的TAVR所做的那样
        \item 从早期的TAVR到现在,经历了持续的技术改进和标准化
        \item Redo-TAV也将遵循类似的发展轨迹
    \end{itemize}
\end{enumerate}

\subsection{临床启示}

\subsubsection{对临床实践的意义}

\begin{enumerate}
    \item \textbf{提高Redo-TAV的可及性和安全性}
    \begin{itemize}
        \item 通过标准化流程,降低Redo-TAV的技术门槛
        \item 使更多中心能够安全开展Redo-TAV手术
        \item 减少学习曲线,提高手术成功率
    \end{itemize}

    \item \textbf{改善决策质量}
    \begin{itemize}
        \item CT规划模块提供系统的可行性评估
        \item 冠状动脉风险分层帮助识别高危患者
        \item 基于数据的尺寸选择和植入策略
        \item 减少主观判断导致的差异
    \end{itemize}

    \item \textbf{促进多学科团队沟通}
    \begin{itemize}
        \item 统一的术语和可视化报告
        \item 便于心脏内科、心外科、影像科之间的交流
        \item 促进Heart Team的协作决策
    \end{itemize}

    \item \textbf{教育和培训工具}
    \begin{itemize}
        \item 丰富的教育内容和视频资源
        \item 病例分享和学习(Case of the Month)
        \item 新手和经验丰富的术者都能从中受益
    \end{itemize}

    \item \textbf{数据收集和质量改进}
    \begin{itemize}
        \item 标准化的数据记录表单
        \item 便于开展注册研究和质量评估
        \item 为未来的指南制定提供证据
    \end{itemize}
\end{enumerate}

\subsubsection{应用场景}

\textbf{场景1:可行性评估}
\begin{itemize}
    \item 患者:TAVR术后5年,出现瓣膜衰败
    \item 使用APP的CT规划模块
    \item 输入Index TAV信息(如Evolut R 29)
    \item 评估不同Second TAV选项的可行性
    \item 识别冠状动脉高危患者,建议外科手术
\end{itemize}

\textbf{场景2:术前规划}
\begin{itemize}
    \item 确定进行Redo-TAV后
    \item 使用APP选择最佳Second TAV和尺寸
    \item 确定目标NSP Node
    \item 生成动画总结报告,与团队讨论
    \item 制定冠状动脉保护策略
\end{itemize}

\textbf{场景3:术中指导}
\begin{itemize}
    \item 术中参考APP的手术指南
    \item 根据选定的NSP Node,查看具体植入参数
    \item 使用可视化示意图辅助透视定位
    \item 记录手术数据和即刻结果
\end{itemize}

\textbf{场景4:教育和培训}
\begin{itemize}
    \item 新术者学习Redo-TAV的概念和术语
    \item 观看教学视频,了解不同技术
    \item 查阅瓣膜特异性资源,熟悉不同瓣膜的特点
    \item 学习TAV explant的外科技术
\end{itemize}

\subsubsection{未来方向}

\begin{enumerate}
    \item \textbf{APP的持续更新}
    \begin{itemize}
        \item 纳入新的TAVR瓣膜(如新一代设备)
        \item 更新冠状动脉风险评估算法
        \item 增加更多TAV-in-TAV组合的数据
    \end{itemize}

    \item \textbf{循证医学研究}
    \begin{itemize}
        \item 开展多中心注册研究
        \item 验证APP推荐策略的临床结果
        \item 识别最佳实践和改进领域
    \end{itemize}

    \item \textbf{人工智能整合}
    \begin{itemize}
        \item 自动化CT测量和分析
        \item AI辅助风险预测
        \item 个体化治疗推荐
    \end{itemize}

    \item \textbf{扩展到其他领域}
    \begin{itemize}
        \item 借鉴Redo-TAV APP的经验
        \item 开发类似的工具用于其他复杂介入手术
        \item 如TMVR-in-TMVR、TTVR等
    \end{itemize}
\end{enumerate}

\subsection{研究局限性}

\begin{enumerate}
    \item \textbf{缺乏长期循证数据}
    \begin{itemize}
        \item APP的推荐基于专家共识和有限的临床数据
        \item Redo-TAV是一个相对新兴的领域,长期结果数据有限
        \item 不同TAV组合的最佳策略仍在探索中
    \end{itemize}

    \item \textbf{个体化因素}
    \begin{itemize}
        \item APP提供标准化建议,但每个患者的解剖和临床情况独特
        \item 某些特殊情况(如严重钙化、主动脉根部扩张)可能需要偏离标准流程
        \item 临床医生的经验和判断仍然至关重要
    \end{itemize}

    \item \textbf{技术依赖}
    \begin{itemize}
        \item 需要高质量的CT扫描
        \item 需要准确的CT测量和分析
        \item 测量误差可能影响决策
    \end{itemize}

    \item \textbf{瓣膜组合的覆盖范围}
    \begin{itemize}
        \item 虽然APP涵盖主流TAVR瓣膜,但某些组合数据仍有限
        \item 新瓣膜上市后需要时间纳入APP
    \end{itemize}

    \item \textbf{地区差异}
    \begin{itemize}
        \item 不同国家和地区可用的TAVR瓣膜可能不同
        \item 某些推荐的瓣膜组合在特定地区可能不可用
    \end{itemize}

    \item \textbf{外科手术对比}
    \begin{itemize}
        \item APP主要聚焦Redo-TAV
        \item 对于何时选择外科TAV explant vs Redo-TAV,缺乏明确的循证标准
        \item 需要更多比较研究
    \end{itemize}
\end{enumerate}

\subsection{个人笔记}

\subsubsection{关键数字和概念}

\textbf{CT规划的4个关键要素}(核心记忆点):
\begin{enumerate}
    \item 2\textsuperscript{nd} TAV Compatibility(兼容性)
    \item Implant Position(植入位置 - NSP Node)
    \item Coronary Risk(冠状动脉风险 - VTA测量)
    \item 2\textsuperscript{nd} TAV Sizing(尺寸选择)
\end{enumerate}

\textbf{NSP Node编号}:
\begin{itemize}
    \item Node 6:最高位置
    \item Node 5:常用位置
    \item Node 4:较低位置
    \item Node 3:仅用于AR,某些瓣膜
\end{itemize}

\textbf{冠状动脉风险等级}:
\begin{itemize}
    \item 高风险(红色):VTA距离极短,强烈建议冠状动脉保护
    \item 中等风险(黄色):如有疑虑,考虑冠状动脉保护
    \item 低风险(绿色):VTA距离充足,必要时考虑
\end{itemize}

\textbf{交界对位分级}:
\begin{itemize}
    \item Aligned:0-15度
    \item Mild:15-30度
    \item Moderate:30-45度
    \item Severe:45-60度
\end{itemize}

\subsubsection{重要术语}

\begin{description}
    \item[Redo-TAV] 也称TAV-in-TAV,在失败的TAVR瓣膜内再次植入TAVR瓣膜
    \item[NSP] Neoskirt Plane,新裙边平面,是redo-TAV组合的关键参考平面
    \item[CRP] Coronary Risk Plane,冠状动脉风险平面,决定NSP Node的可行性
    \item[VTA] Virtual Transcatheter Aortic valve to coronary ostium,虚拟瓣膜到冠状动脉口的距离
    \item[Index TAV] 第一个(失败的)TAVR瓣膜
    \item[Second TAV] 第二个(新植入的)TAVR瓣膜
    \item[Node] TAV支架框架上的特定位置标记
\end{description}

\subsubsection{临床实践要点}

\begin{enumerate}
    \item \textbf{Redo-TAV vs TAV Explant的选择}
    \begin{itemize}
        \item Redo-TAV适用于手术高危、解剖合适的患者
        \item TAV Explant适用于外科低危、解剖不适合Redo-TAV(如高冠状动脉风险)的患者
        \item APP主要帮助评估Redo-TAV的可行性
    \end{itemize}

    \item \textbf{CT规划的重要性}
    \begin{itemize}
        \item CT是Redo-TAV规划的基石
        \item 需要高质量的心脏CT(最好是心电门控)
        \item 关键测量:Index TAV尺寸、位置、VTA距离、主动脉根部解剖
    \end{itemize}

    \item \textbf{冠状动脉保护策略}
    \begin{itemize}
        \item 对于高风险患者,强烈建议预防性冠状动脉保护
        \item 方法包括:导引导丝保护、预防性支架、BASILICA等
        \item 术中应备好紧急冠状动脉干预设备
    \end{itemize}

    \item \textbf{瓣膜选择原则}
    \begin{itemize}
        \item Short-in-Short vs Tall-in-Tall vs Short-in-Tall等组合
        \item 不同组合有不同的优缺点
        \item 需要根据Index TAV类型、患者解剖选择
    \end{itemize}
\end{enumerate}

\subsubsection{APP的独特价值}

\begin{enumerate}
    \item \textbf{一站式平台}
    \begin{itemize}
        \item 整合了CT规划、手术指南、教育资源、数据记录
        \item 避免需要查阅多个文献和指南
        \item 随时随地可访问(手机APP)
    \end{itemize}

    \item \textbf{标准化术语}
    \begin{itemize}
        \item 统一了Redo-TAV领域的术语
        \item NSP、CRP、VTA等概念的标准化定义
        \item 促进全球交流和合作
    \end{itemize}

    \item \textbf{可视化工具}
    \begin{itemize}
        \item 动画总结、流程图、示意图
        \item 帮助理解复杂的空间关系
        \item 便于与患者和团队沟通
    \end{itemize}

    \item \textbf{全球专家的集体智慧}
    \begin{itemize}
        \item 汇集了20多位国际顶尖专家的经验
        \item 代表了当前领域的最佳实践
        \item 持续更新和改进
    \end{itemize}
\end{enumerate}

\subsubsection{对中国的启示}

\begin{enumerate}
    \item \textbf{Redo-TAV时代即将到来}
    \begin{itemize}
        \item 中国TAVR起步较晚,但发展迅速
        \item 未来5-10年将面临越来越多的TAVR失败病例
        \item 需要提前准备,建立标准化流程
    \end{itemize}

    \item \textbf{借鉴国际经验}
    \begin{itemize}
        \item Redo TAV APP提供了很好的参考模板
        \item 可以借鉴其标准化思路和决策框架
        \item 结合中国实际情况(瓣膜类型、患者特点)进行本土化
    \end{itemize}

    \item \textbf{多学科团队建设}
    \begin{itemize}
        \item Redo-TAV需要心内科、心外科、影像科的紧密合作
        \item Heart Team模式在中国需要进一步推广
        \item CT分析能力是关键,需要培训影像医生
    \end{itemize}

    \item \textbf{数据收集和研究}
    \begin{itemize}
        \item 建立中国的Redo-TAV注册研究
        \item 收集本土数据,了解中国患者的特点
        \item 参与国际合作,贡献中国经验
    \end{itemize}
\end{enumerate}

\subsubsection{值得进一步探讨的问题}

\begin{enumerate}
    \item \textbf{最佳瓣膜组合}
    \begin{itemize}
        \item 不同TAV-in-TAV组合的长期结果如何?
        \item Short-in-Short vs Tall-in-Tall,哪个更优?
        \item 是否有某些组合应该避免?
    \end{itemize}

    \item \textbf{冠状动脉保护的适应证}
    \begin{itemize}
        \item VTA多少才是真正的高危阈值?
        \item 预防性冠状动脉保护的获益-风险比如何?
        \item 哪些患者真正需要BASILICA等技术?
    \end{itemize}

    \item \textbf{Redo-TAV vs TAV Explant}
    \begin{itemize}
        \item 如何平衡两者的选择?
        \item 年龄、外科风险、解剖因素如何权衡?
        \item 长期结果对比如何?
    \end{itemize}

    \item \textbf{第三次干预}
    \begin{itemize}
        \item Redo-TAV失败后怎么办?
        \item 是否可能进行TAV-in-TAV-in-TAV?
        \item 还是应该早期转向外科手术?
    \end{itemize}

    \item \textbf{预防TAVR失败}
    \begin{itemize}
        \item 如何在初次TAVR时就考虑未来的Redo-TAV可行性?
        \item 瓣膜选择、植入位置是否应该为未来留有余地?
        \item "Redo-friendly" TAVR的概念是否可行?
    \end{itemize}
\end{enumerate}

\subsubsection{学习资源}

\textbf{如何使用Redo TAV APP}:
\begin{enumerate}
    \item 下载APP:在App Store(iOS)或Google Play(Android)搜索"Redo TAV"
    \item 熟悉界面:浏览各个功能模块
    \item 学习术语:从Terminology模块开始
    \item 实践CT规划:使用实际病例进行CT分析
    \item 观看视频:学习手术技术和专家经验
    \item 使用手术指南:术前规划和术中参考
\end{enumerate}

\textbf{相关文献}:
\begin{itemize}
    \item "A Guide to Transcatheter Aortic Valve Design and Systematic Planning for a Redo-TAV (TAV-in-TAV) Procedure"(Vinayak N. Bapat等,文中提到的配套文章)
    \item 建议查阅相关的Redo-TAV综述和指南
\end{itemize}

\textbf{继续学习方向}:
\begin{itemize}
    \item 深入学习各种TAVR瓣膜的设计特点
    \item 掌握CT测量和分析技术
    \item 了解冠状动脉保护技术(BASILICA、chimney stenting等)
    \item 学习TAV explant的外科技术
    \item 关注Redo-TAV领域的最新进展和研究
\end{itemize}

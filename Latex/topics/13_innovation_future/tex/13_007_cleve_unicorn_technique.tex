\section{CLEVE-UNICORN技术预防TAVR后冠状动脉阻塞:需谨慎应用}
\label{sec:13_007_cleve_unicorn_technique}

% ============================================
% 文献信息
% ============================================
\subsection{文献信息}

\begin{itemize}
    \item \textbf{标题}: CLEVE-UNICORN Technique to Prevent Coronary Obstruction After TAVR in Native Valves: A Word of Caution
    \item \textbf{作者}: Jean-Benoît Veillette, MD; Anthony Poulin, MD; Siamak Mohammadi, MD; Erwan Salaun, MD; Pierre-Yves Turgeon, MD; Jean-Michel Paradis, MD
    \item \textbf{机构}: Quebec Heart and Lung Institute (Institut Universitaire de Cardiologie et de Pneumologie de Québec, Université Laval)
    \item \textbf{会议}: TCT (Transcatheter Cardiovascular Therapeutics)
    \item \textbf{PDF文件名}: tct-1446-cleve-unicorn-technique-to-prevent-coronary-obstruction-after-tavr.pdf
    \item \textbf{文献类型}: 会议演讲/病例报告
    \item \textbf{利益冲突}: 第一作者Jean-Benoît Veillette声明无财务关系需要披露
\end{itemize}

\subsection{研究背景}

\subsubsection{TAVR后冠状动脉阻塞的风险}

经导管主动脉瓣置换术(TAVR)后冠状动脉阻塞是一种罕见但严重的并发症,特别是在以下高危情况下:

\begin{itemize}
    \item 冠状动脉开口高度较低
    \item 虚拟瓣膜到冠状动脉距离(VTC distance)过小
    \item 主动脉窦狭小
    \item 瓣叶大量钙化
    \item 瓣膜内瓣膜(Valve-in-Valve)手术
\end{itemize}

\subsubsection{CLEVE-UNICORN技术简介}

CLEVE-UNICORN(Coronary Leaflet Electrosurgical Laceration followed by Valve-IN-valve)技术最初用于瓣膜内瓣膜(ViV)手术,通过电外科方式撕裂原瓣膜瓣叶,防止其阻塞冠状动脉开口。

本病例报告探讨了将该技术应用于\textbf{原生主动脉瓣}TAVR的经验和注意事项。

\subsection{病例报告}

\subsubsection{患者基本信息}

\textbf{人口学特征}:
\begin{itemize}
    \item \textbf{年龄}: 84岁
    \item \textbf{性别}: 女性
    \item \textbf{主要诊断}: 已知的严重原生主动脉瓣狭窄
\end{itemize}

\textbf{既往病史}:
\begin{itemize}
    \item 心房颤动(AF)
    \item 高血压(HTN)
    \item 血脂异常(DLP)
    \item 类风湿性关节炎
    \item 慢性肾脏病IIIa期(CKD IIIa)
\end{itemize}

\textbf{入院原因}:急性失代偿性心力衰竭

\subsubsection{术前评估数据}

\textbf{超声心动图检查结果}:

\begin{table}[h]
\centering
\caption{术前超声心动图关键参数}
\label{tab:preop_echo}
\begin{tabular}{lc}
\toprule
\textbf{参数} & \textbf{数值} \\
\midrule
左室射血分数 & 保留 \\
主动脉瓣口面积(AVA) & 0.87 cm² \\
主动脉瓣平均压力梯度 & 40 mmHg \\
主动脉瓣反流(AR) & 中度 \\
二尖瓣反流(MR) & 轻度 \\
三尖瓣反流(TR) & 轻度 \\
\bottomrule
\end{tabular}
\end{table}

\textbf{心脏CT扫描关键测量}:

\begin{table}[h]
\centering
\caption{术前CT测量 - 冠状动脉阻塞风险评估}
\label{tab:preop_ct}
\begin{tabular}{lc}
\toprule
\textbf{测量参数} & \textbf{数值} \\
\midrule
右冠状动脉开口高度 & 14 mm \\
左冠状动脉开口高度 & 10 mm \\
虚拟瓣膜到左主干距离(VTC) & \textbf{2 mm} \\
\bottomrule
\end{tabular}
\end{table}

\textbf{风险评估}:
\begin{itemize}
    \item \textcolor{red}{\textbf{高危特征}}:左主干VTC距离仅2 mm,存在TAVR后冠状动脉阻塞的显著风险
    \item 决策:采用CLEVE-UNICORN技术预防冠状动脉阻塞
\end{itemize}

\subsubsection{手术过程}

\textbf{CLEVE-UNICORN技术步骤}:

\begin{enumerate}
    \item \textbf{瓣叶穿刺}
    \begin{itemize}
        \item 使用Astato 20电外科导管
        \item 穿刺目标瓣叶(对应左冠状动脉开口的瓣叶)
    \end{itemize}

    \item \textbf{瓣叶扩张}
    \begin{itemize}
        \item 首先使用3 mm球囊扩张穿刺部位
        \item 然后使用10 mm球囊进一步扩张
        \item 目的:在瓣叶上创建裂口,使其在THV部署后向外翻转,避免阻塞冠状动脉
    \end{itemize}

    \item \textbf{第一次经导管心脏瓣膜(THV)部署}
    \begin{itemize}
        \item \textbf{问题}:尽管努力在部署过程中将THV向主动脉侧移动,但无法像标准TAVR程序那样重新定位THV
        \item \textbf{结果}:主动脉造影显示\textcolor{red}{\textbf{严重主动脉瓣反流}}
        \item \textbf{分析}:瓣膜定位偏向心室侧,导致瓣周漏
    \end{itemize}

    \item \textbf{第二次THV部署(瓣膜内瓣膜)}
    \begin{itemize}
        \item 决策:在第一个瓣膜内再次部署第二个瓣膜
        \item \textbf{观察}:尽管采用非常缓慢的充盈,THV在部署过程中始终被推向心室侧
        \item \textbf{结果}:主动脉造影显示轻度主动脉瓣反流
        \item 最终瓣膜位置可接受
    \end{itemize}

    \item \textbf{瓣周组织反应}
    \begin{itemize}
        \item 术中观察到瓣周组织反应
        \item 超声心动图可见瓣周强回声结构
        \item CT影像测量显示瓣周组织厚度约0.47 cm
    \end{itemize}
\end{enumerate}

\subsubsection{术后结果}

\textbf{即刻术后超声心动图}:

\begin{table}[h]
\centering
\caption{术后超声心动图结果}
\label{tab:postop_echo}
\begin{tabular}{lc}
\toprule
\textbf{参数} & \textbf{数值} \\
\midrule
主动脉瓣平均压力梯度 & 12 mmHg \\
主动脉瓣反流 & 微量 \\
心包积液 & 无 \\
\bottomrule
\end{tabular}
\end{table}

\textbf{术后并发症}:
\begin{itemize}
    \item \textbf{传导系统异常}:发生孤立性左束支传导阻滞(LBBB)
    \item \textbf{无其他主要并发症}
\end{itemize}

\textbf{临床转归}:
\begin{itemize}
    \item 患者临床过程顺利
    \item 术后2天出院
    \item 血流动力学改善满意
\end{itemize}

\subsection{主要研究发现}

\subsubsection{1. CLEVE-UNICORN技术改变瓣膜部署行为}

\textbf{关键观察}:

\begin{itemize}
    \item 在原生主动脉瓣上应用CLEVE-UNICORN技术后,THV部署行为与标准TAVR显著不同
    \item \textbf{向心室侧的推力}:两次部署均观察到THV持续被推向心室侧
    \item \textbf{定位困难}:无法像标准TAVR那样在部署过程中精细调整瓣膜位置
    \item \textbf{可能机制}:
    \begin{itemize}
        \item 瓣叶撕裂改变了瓣膜环的力学特性
        \item 瓣周组织反应可能影响THV的扩张和定位
        \item 撕裂的瓣叶可能产生不对称的径向力
    \end{itemize}
\end{itemize}

\subsubsection{2. 瓣周组织反应不可预测}

\textbf{病例中的发现}:

\begin{itemize}
    \item 术中发现明显的瓣周组织反应
    \item \textbf{影像学表现}:
    \begin{itemize}
        \item 超声心动图:瓣周强回声团块
        \item CT:瓣周组织厚度约4.7 mm
    \end{itemize}
    \item \textbf{临床意义}:
    \begin{itemize}
        \item 增加THV定位的难度
        \item 术者必须实时调整策略
        \item 可能影响最终的血流动力学结果
    \end{itemize}
    \item \textbf{组织反应的可能来源}:
    \begin{itemize}
        \item 电外科能量导致的局部组织损伤
        \item 球囊扩张引起的组织撕裂和出血
        \item 炎症反应和血栓形成
    \end{itemize}
\end{itemize}

\subsubsection{3. 主动脉夹层的潜在风险}

\textbf{理论风险}:

本病例提出了在原生主动脉瓣上应用CLEVE-UNICORN技术可能导致主动脉夹层的风险:

\begin{itemize}
    \item \textbf{机制}:
    \begin{itemize}
        \item 瓣叶电外科撕裂可能延伸至主动脉壁
        \item 球囊扩张产生的张力可能撕裂主动脉内膜
        \item 原生瓣叶解剖比生物瓣更接近主动脉壁
    \end{itemize}
    \item \textbf{风险因素}:
    \begin{itemize}
        \item 高龄患者主动脉壁脆性增加
        \item 钙化延伸至主动脉壁
        \item 主动脉窦解剖异常
        \item 结缔组织疾病(本例:类风湿性关节炎)
    \end{itemize}
    \item \textbf{注意事项}:
    \begin{itemize}
        \item 必须在心脏团队决策中充分讨论此风险
        \item 术中影像监测至关重要
        \item 需要准备应急处理方案
    \end{itemize}
\end{itemize}

\subsection{结论}

\subsubsection{主要结论}

\begin{enumerate}
    \item \textbf{技术可行性}:CLEVE-UNICORN技术可应用于原生主动脉瓣TAVR以预防冠状动脉阻塞,本例患者最终获得满意结果

    \item \textbf{技术挑战}:该技术显著改变瓣膜部署行为,使精确定位更加困难,可能需要多次瓣膜部署

    \item \textbf{安全性考虑}:存在主动脉夹层的潜在风险,必须在决策过程中充分评估

    \item \textbf{谨慎应用}:标题"A Word of Caution"强调了该技术在原生瓣膜上应用需要极其谨慎
\end{enumerate}

\subsubsection{成功的关键因素}

本例成功的可能因素:
\begin{itemize}
    \item 经验丰富的术者团队
    \item 充分的术前规划和风险评估
    \item 术中实时影像监测(透视 + TEE + CT融合)
    \item 准备多个瓣膜以应对可能的需求
    \item 术中灵活的决策能力
\end{itemize}

\subsection{临床启示}

\subsubsection{适应证选择}

\textbf{可能适合CLEVE-UNICORN技术的情况}:

\begin{itemize}
    \item VTC距离<4 mm的高危患者
    \item 外科手术风险极高的患者
    \item 无其他替代治疗选择
    \item 患者充分知情同意
\end{itemize}

\textbf{相对禁忌证}:

\begin{itemize}
    \item 严重主动脉壁钙化
    \item 已知的主动脉病变(如动脉瘤)
    \item 结缔组织疾病导致的主动脉壁脆弱
    \item 术者经验不足
\end{itemize}

\subsubsection{术前准备要点}

\begin{enumerate}
    \item \textbf{详细的影像评估}
    \begin{itemize}
        \item 高质量心脏CT扫描
        \item 精确测量VTC距离
        \item 评估主动脉壁完整性
        \item 模拟瓣膜部署位置
    \end{itemize}

    \item \textbf{多学科团队讨论}
    \begin{itemize}
        \item 介入心脏病专家
        \item 心脏外科医生
        \item 影像专家
        \item 麻醉团队
        \item 充分评估风险/获益比
    \end{itemize}

    \item \textbf{技术准备}
    \begin{itemize}
        \item 准备多个尺寸的THV
        \item 备用球囊
        \item 主动脉夹层的应急设备
        \item 外科备台(如需紧急转化)
    \end{itemize}

    \item \textbf{患者沟通}
    \begin{itemize}
        \item 详细解释技术的创新性
        \item 明确告知可能的风险
        \item 讨论替代方案
        \item 获得充分知情同意
    \end{itemize}
\end{enumerate}

\subsubsection{术中注意事项}

\begin{enumerate}
    \item \textbf{瓣叶撕裂阶段}
    \begin{itemize}
        \item 精确定位穿刺点
        \item 控制电外科能量
        \item 避免损伤过深
        \item 实时影像监测
    \end{itemize}

    \item \textbf{球囊扩张阶段}
    \begin{itemize}
        \item 逐步增加球囊尺寸(本例:3 mm → 10 mm)
        \item 低压缓慢充盈
        \item 观察主动脉根部有无异常
        \item 注意患者血流动力学变化
    \end{itemize}

    \item \textbf{瓣膜部署阶段}
    \begin{itemize}
        \item \textbf{预期向心室侧的推力}
        \item 可能需要初始定位偏向主动脉侧
        \item 非常缓慢的部署速度
        \item 准备第二个瓣膜(ViV)的可能性
        \item 持续的TEE和透视监测
    \end{itemize}

    \item \textbf{并发症监测}
    \begin{itemize}
        \item 主动脉夹层征象
        \item 心包积液
        \item 冠状动脉血流
        \item 瓣周漏程度
        \item 心律失常
    \end{itemize}
\end{enumerate}

\subsubsection{术后管理}

\begin{itemize}
    \item 密切血流动力学监测
    \item 连续心电监测(传导阻滞风险)
    \item 术后超声心动图评估
    \item 必要时考虑术后CT扫描排除主动脉并发症
    \item 抗血小板/抗凝治疗
    \item 瓣周组织反应的随访
\end{itemize}

\subsubsection{对未来研究的启示}

\begin{enumerate}
    \item \textbf{技术改进方向}
    \begin{itemize}
        \item 优化瓣叶撕裂的能量设置
        \item 开发更精确的撕裂工具
        \item 改进THV设计以适应这种特殊应用
        \item 研究预防瓣周组织反应的方法
    \end{itemize}

    \item \textbf{临床研究需求}
    \begin{itemize}
        \item 前瞻性注册研究评估安全性和有效性
        \item 确定最佳适应证
        \item 建立标准化操作流程
        \item 与其他预防冠状动脉阻塞技术的比较(如BASILICA、chimney stenting)
    \end{itemize}

    \item \textbf{教育培训}
    \begin{itemize}
        \item 建立培训课程
        \item 模拟器训练
        \item 经验中心的指导
        \item 建立质量控制标准
    \end{itemize}
\end{enumerate}

\subsection{研究局限性}

\begin{enumerate}
    \item \textbf{病例报告性质}
    \begin{itemize}
        \item 单一病例,不能代表所有情况
        \item 无法评估技术的总体成功率和并发症率
        \item 缺乏对照组比较
        \item 无长期随访数据
    \end{itemize}

    \item \textbf{技术相关局限}
    \begin{itemize}
        \item 本例需要两个瓣膜,增加了成本和复杂性
        \item 瓣周组织反应的长期影响未知
        \item 左束支传导阻滞的临床意义需要随访
        \item 未评估与其他技术的比较优劣
    \end{itemize}

    \item \textbf{可推广性问题}
    \begin{itemize}
        \item 需要高水平的术者技能和经验
        \item 需要高级影像设备(CT融合、TEE)
        \item 不是所有中心都具备条件
        \item 特定设备的可获得性(Astato 20)
    \end{itemize}

    \item \textbf{未解答的问题}
    \begin{itemize}
        \item 主动脉夹层的实际发生率
        \item 最佳的瓣叶撕裂程度
        \item 不同THV平台的表现差异
        \item 瓣周组织反应的预测因素
    \end{itemize}
\end{enumerate}

\subsection{个人笔记}

\subsubsection{关键数字记忆}

\begin{table}[h]
\centering
\caption{关键临床数据速记}
\label{tab:key_numbers}
\begin{tabular}{ll}
\toprule
\textbf{参数} & \textbf{数值} \\
\midrule
\multicolumn{2}{l}{\textit{患者特征}} \\
年龄 & 84岁 \\
CKD分期 & IIIa期 \\
\midrule
\multicolumn{2}{l}{\textit{术前血流动力学}} \\
AVA & 0.87 cm² \\
平均梯度 & 40 mmHg \\
\midrule
\multicolumn{2}{l}{\textit{解剖测量}} \\
右冠高度 & 14 mm \\
左冠高度 & 10 mm \\
\textcolor{red}{VTC距离(左主干)} & \textcolor{red}{\textbf{2 mm}} \\
\midrule
\multicolumn{2}{l}{\textit{技术细节}} \\
球囊尺寸 & 3 mm → 10 mm \\
使用THV数量 & 2个(ViV) \\
瓣周组织厚度 & 4.7 mm \\
\midrule
\multicolumn{2}{l}{\textit{术后结果}} \\
术后平均梯度 & 12 mmHg \\
术后AR & 微量 \\
住院时间 & 2天 \\
\bottomrule
\end{tabular}
\end{table}

\subsubsection{重要概念解析}

\begin{description}
    \item[CLEVE-UNICORN] Coronary Leaflet Electrosurgical Laceration followed by Valve-IN-valve的缩写。是一种通过电外科撕裂瓣叶来预防TAVR后冠状动脉阻塞的创新技术。

    \item[VTC距离] Valve-to-Coronary distance,虚拟瓣膜到冠状动脉距离。<4 mm被认为是冠状动脉阻塞的高危因素。本例仅2 mm,风险极高。

    \item[瓣周组织反应] 瓣叶撕裂和球囊扩张后在主动脉根部产生的组织反应,包括出血、血栓、炎症等。可能影响THV定位和最终结果。

    \item[向心室侧推力] 本例中观察到的特殊现象:在瓣叶撕裂后,THV部署时持续被推向心室侧,导致定位困难。可能与瓣膜环力学改变有关。

    \item[A Word of Caution] 标题中的"警示"强调了该技术的潜在风险,特别是在原生瓣膜上应用时。提示临床医生必须谨慎评估和应用。

    \item[Astato 20] 电外科导管,用于瓣叶穿刺和撕裂。利用射频能量切割组织。
\end{description}

\subsubsection{与其他预防冠状动脉阻塞技术的比较}

\begin{table}[h]
\centering
\caption{预防TAVR后冠状动脉阻塞的技术比较}
\label{tab:co_prevention_techniques}
\begin{tabular}{p{3cm}p{4cm}p{4cm}p{3cm}}
\toprule
\textbf{技术} & \textbf{原理} & \textbf{优势} & \textbf{局限性} \\
\midrule
BASILICA & 瓣叶电外科撕裂(单纯撕裂,无球囊扩张) &
\begin{itemize}[leftmargin=*,nosep]
    \item 技术相对成熟
    \item 不改变瓣环结构
\end{itemize} &
\begin{itemize}[leftmargin=*,nosep]
    \item 主要用于ViV
    \item 需要特殊设备
\end{itemize} \\
\midrule
CLEVE-UNICORN & 瓣叶电外科撕裂 + 球囊扩张 &
\begin{itemize}[leftmargin=*,nosep]
    \item 更彻底的瓣叶移位
    \item 可能降低CO风险
\end{itemize} &
\begin{itemize}[leftmargin=*,nosep]
    \item 改变瓣膜部署行为
    \item 主动脉夹层风险
    \item 定位困难
\end{itemize} \\
\midrule
Chimney Stenting & 在冠状动脉内预置支架 &
\begin{itemize}[leftmargin=*,nosep]
    \item 直接保护冠状动脉
    \item 技术标准化
\end{itemize} &
\begin{itemize}[leftmargin=*,nosep]
    \item 长期支架问题
    \item 限制未来冠脉介入
\end{itemize} \\
\midrule
外科AVR & 直接切除瓣叶 &
\begin{itemize}[leftmargin=*,nosep]
    \item 金标准
    \item 无CO风险
\end{itemize} &
\begin{itemize}[leftmargin=*,nosep]
    \item 手术风险高
    \item 恢复时间长
\end{itemize} \\
\bottomrule
\end{tabular}
\end{table}

\subsubsection{临床决策流程图}

对于VTC距离<4 mm的TAVR患者,建议决策流程:

\begin{enumerate}
    \item \textbf{评估手术风险}
    \begin{itemize}
        \item 如果外科AVR风险可接受 → 优先考虑外科手术
        \item 如果外科风险极高 → 进入下一步
    \end{itemize}

    \item \textbf{评估解剖特征}
    \begin{itemize}
        \item VTC距离、窦部尺寸、瓣叶长度、钙化程度
        \item 主动脉壁完整性
    \end{itemize}

    \item \textbf{选择预防策略}
    \begin{itemize}
        \item ViV手术:BASILICA或CLEVE-UNICORN
        \item 原生瓣膜:
        \begin{itemize}
            \item VTC 2-4 mm:考虑chimney stenting或CLEVE-UNICORN(需充分讨论风险)
            \item VTC <2 mm:CLEVE-UNICORN或chimney stenting(需MDT充分讨论)
        \end{itemize}
    \end{itemize}

    \item \textbf{多学科团队决策}
    \begin{itemize}
        \item 充分讨论各种方案的风险/获益
        \item 评估中心经验和资源
        \item 患者偏好和知情同意
    \end{itemize}
\end{enumerate}

\subsubsection{值得思考的问题}

\begin{enumerate}
    \item \textbf{为什么瓣膜会持续被推向心室侧?}
    \begin{itemize}
        \item 可能的机制:
        \begin{itemize}
            \item 撕裂的瓣叶失去了对THV的对称性支撑
            \item 瓣周组织反应改变了局部解剖
            \item 球囊扩张导致瓣环形态改变
            \item THV扩张时的径向力分布不均
        \end{itemize}
        \item 需要进一步的力学研究和影像分析
    \end{itemize}

    \item \textbf{瓣周组织反应是否可以预防?}
    \begin{itemize}
        \item 可能的策略:
        \begin{itemize}
            \item 优化电外科能量参数
            \item 改进球囊扩张技术
            \item 使用药物涂层球囊
            \item 术前抗炎预处理
        \end{itemize}
        \item 需要实验研究验证
    \end{itemize}

    \item \textbf{如何预测主动脉夹层风险?}
    \begin{itemize}
        \item 可能的风险标志物:
        \begin{itemize}
            \item 主动脉壁厚度
            \item 钙化模式
            \item 结缔组织疾病
            \item 高龄
            \item 主动脉壁应力分析(CT)
        \end{itemize}
        \item 需要建立风险评分系统
    \end{itemize}

    \item \textbf{长期随访会发现什么?}
    \begin{itemize}
        \item 关注点:
        \begin{itemize}
            \item 瓣周组织反应的演变
            \item 左束支传导阻滞的影响
            \item 瓣膜耐久性(2个瓣膜的ViV配置)
            \item 冠状动脉再通的可行性
        \end{itemize}
        \item 需要系统的随访计划
    \end{itemize}

    \item \textbf{该技术在原生瓣膜上是否应该推广?}
    \begin{itemize}
        \item 支持推广的理由:
        \begin{itemize}
            \item 为高危患者提供了治疗选择
            \item 本例获得了成功
            \item 随着经验积累可能改进
        \end{itemize}
        \item 反对推广的理由:
        \begin{itemize}
            \item 主动脉夹层的潜在风险
            \item 定位困难,可能需要多个瓣膜
            \item 缺乏大样本数据
            \item 存在其他替代方案
        \end{itemize}
        \item 当前建议:\textbf{仅在高度选择的病例中、经验丰富的中心、充分知情同意后应用}
    \end{itemize}
\end{enumerate}

\subsubsection{对中国TAVR实践的启示}

\begin{enumerate}
    \item \textbf{技术储备}
    \begin{itemize}
        \item 中国TAVR中心应了解各种预防冠状动脉阻塞的技术
        \item 建立高危病例的MDT讨论机制
        \item 选择性开展新技术培训
    \end{itemize}

    \item \textbf{设备准备}
    \begin{itemize}
        \item 评估Astato等电外科设备在国内的可获得性
        \item 准备多种预防策略的设备
        \item 建立应急预案
    \end{itemize}

    \item \textbf{经验积累}
    \begin{itemize}
        \item 从ViV手术中积累瓣叶撕裂经验
        \item 建立病例注册和经验分享机制
        \item 谨慎地将技术扩展到原生瓣膜
    \end{itemize}

    \item \textbf{患者教育}
    \begin{itemize}
        \item 向患者充分解释创新技术的风险和获益
        \item 强调与标准TAVR的区别
        \item 确保真正的知情同意
    \end{itemize}
\end{enumerate}

\subsubsection{Take-Home Messages(带回家的信息)}

\begin{tcolorbox}[colback=yellow!10, colframe=orange!75!black, title=核心要点]
\begin{enumerate}
    \item \textbf{CLEVE-UNICORN技术可能改变瓣膜部署行为},使定位更加困难,术者必须有充分准备和应对策略。

    \item \textbf{瓣周组织反应不可预测},给术者带来挑战,需要术中实时调整,可能需要部署多个瓣膜。

    \item \textbf{主动脉夹层风险必须在决策中充分考虑},特别是在原生主动脉瓣上应用该技术时,心脏团队需要权衡风险/获益。

    \item \textbf{"A Word of Caution"} - 谨慎应用是关键,该技术应限于:
    \begin{itemize}
        \item 冠状动脉阻塞风险极高的患者(VTC <4 mm,特别是<2 mm)
        \item 外科手术风险极高或禁忌
        \item 经验丰富的术者和中心
        \item 充分的术前规划和设备准备
        \item 患者充分知情同意
    \end{itemize}

    \item 本例虽然成功,但需要两个瓣膜,并出现了左束支传导阻滞,提示技术仍需优化。

    \item 长期随访数据和前瞻性研究对于确定该技术在原生瓣膜上的地位至关重要。
\end{enumerate}
\end{tcolorbox}

\section{经腔静脉入路在严重髂股动脉迂曲情况下行TAVR联合PCI}
\label{sec:14_004_transcaval_tavr_pci}

% ============================================
% 文献信息
% ============================================
\subsection{文献信息}

\begin{itemize}
    \item \textbf{标题}: Transcaval approach for combined TAVR and PCI in the setting of prohibitive iliofemoral tortuosity
    \item \textbf{作者}: Andrea Mariani, MD; Nicolas M Van Mieghem, MD, PhD
    \item \textbf{机构}: Erasmus MC (鹿特丹伊拉斯姆斯医学中心)
    \item \textbf{会议}: TCT 2025 (Transcatheter Cardiovascular Therapeutics)
    \item \textbf{PDF文件名}: tct-1398-transcaval-approach-for-combined-tavr-and-pci-in-the-setting-of-pro.pdf
    \item \textbf{文献类型}: 病例报告/会议演讲
\end{itemize}

\subsection{研究背景}

\subsubsection{经腔静脉入路的发展}

经腔静脉(transcaval)入路最初被设计作为经股动脉TAVR的备用路径,主要用于存在严重外周动脉疾病(PAD)、髂股动脉解剖禁忌或其他经股入路不可行的患者。该技术通过在下腔静脉(IVC)和腹主动脉之间建立临时通道,为大口径器械的输送提供可行路径。

\subsubsection{临床挑战}

严重主动脉瓣狭窄(AS)患者常合并冠状动脉疾病(CAD),需要联合瓣膜置换和冠脉血运重建。当患者同时存在:
\begin{itemize}
    \item 严重髂股动脉迂曲和/或动脉瘤样扩张
    \item 复杂冠脉病变需要PCI
    \item 传统经股入路不可行
\end{itemize}

这种情况下的治疗策略选择极具挑战性。

\subsubsection{文献缺口}

虽然经腔静脉入路用于单纯TAVR已有较多报道,但在TAVR过程中通过经腔静脉入路同时完成PCI的病例报道极少,尤其是因髂股动脉迂曲(而非PAD严重程度)而选择该入路的病例更为罕见。

\subsection{病例详细信息}

\subsubsection{患者基本信息}

\begin{table}[h]
\centering
\caption{患者人口学特征}
\label{tab:patient_demographics_transcaval}
\begin{tabular}{ll}
\toprule
\textbf{特征} & \textbf{数值} \\
\midrule
性别 & 男性 \\
年龄 & 81岁 \\
体重 & 82 kg \\
身高 & 157 cm \\
BMI & 33.32 kg/m² \\
\bottomrule
\end{tabular}
\end{table}

\subsubsection{心脏病史}

\begin{itemize}
    \item \textbf{1990年}:动脉高血压
    \item \textbf{2015年}:腹主动脉瘤手术修复
    \item \textbf{2024年12月}:脑血管意外(左顶枕叶栓塞性卒中)
    \item \textbf{2024年12月}:永久性房颤伴快速心室率
    \begin{itemize}
        \item 抗凝治疗:阿哌沙班 2.5 mg BID
        \item 心率控制:美托洛尔 + 地高辛
    \end{itemize}
\end{itemize}

\subsubsection{其他重要病史}

\begin{itemize}
    \item 痛风
    \item 阻塞性睡眠呼吸暂停综合征(OSAS)
    \item \textbf{4期慢性肾脏病(CKD)}:eGFR 26 ml/min/1.73m²
    \item 高胆固醇血症
\end{itemize}

\subsubsection{临床表现}

\begin{itemize}
    \item \textbf{急性肺水肿}:NYHA功能分级IV级
    \item \textbf{心绞痛}:CCS分级III级
    \item \textbf{心电图}:
    \begin{itemize}
        \item 心率121 bpm
        \item 心房颤动
        \item 左室肥厚伴劳损
    \end{itemize}
\end{itemize}

\subsubsection{经胸超声心动图检查}

\begin{table}[h]
\centering
\caption{超声心动图关键参数}
\label{tab:tte_parameters}
\begin{tabular}{ll}
\toprule
\textbf{参数} & \textbf{结果} \\
\midrule
左室功能 & 正常 \\
左室肥厚 & 严重 \\
双房扩大 & 是 \\
左房容积指数(LAVi) & 40 ml/m² \\
主动脉瓣形态 & 钙化瓣叶 \\
主动脉瓣狭窄类型 & 严重保留射血分数低流量低梯度(pLFLG) \\
每搏输出量指数(SVi) & 24 ml/m² \\
主动脉瓣口面积指数(AVAi) & 0.38 cm²/m² \\
\bottomrule
\end{tabular}
\end{table}

\textbf{诊断}:\textbf{严重保留射血分数低流量低梯度主动脉瓣狭窄(severe paradoxical low-flow low-gradient aortic stenosis, pLFLG-AS)}

\subsubsection{冠状动脉造影}

\textbf{冠脉循环类型}:右优势循环

\textbf{病变分布}:
\begin{itemize}
    \item \textbf{右冠状动脉(RCA)}:中段显著钙化狭窄
    \item \textbf{左回旋支(LCX)}:中段显著钙化狭窄
    \item \textbf{左前降支(LAD)}:弥漫性非显著病变
\end{itemize}

\subsubsection{CT血管造影评估}

\textbf{主动脉瓣评估}:

\begin{table}[h]
\centering
\caption{主动脉瓣CT参数}
\label{tab:av_ct_parameters}
\begin{tabular}{ll}
\toprule
\textbf{参数} & \textbf{测量值} \\
\midrule
瓣膜形态 & 三叶瓣 \\
钙化程度 & 中度钙化 \\
Agatston钙化评分 & 1900 \\
瓣环钙化(VBR水平) & 小钙化斑 \\
左室流出道(LVOT)钙化 & 无 \\
膜部间隔(MS)长度 & 6 mm \\
左冠状动脉高度(LCA) & 13.2 mm \\
右冠状动脉高度(RCA) & 16.8 mm \\
\bottomrule
\end{tabular}
\end{table}

\textbf{髂股动脉评估(关键发现)}:

\begin{table}[h]
\centering
\caption{髂股动脉解剖参数}
\label{tab:iliofemoral_anatomy}
\begin{tabular}{ll}
\toprule
\textbf{参数} & \textbf{测量值/描述} \\
\midrule
右侧髂外动脉(EIA)最大直径 & 27.2 mm \\
左侧髂外动脉(EIA)最大直径 & 32.1 mm \\
髂股动脉形态 & 高度迂曲和动脉瘤样扩张 \\
\textbf{经股入路可行性} & \textbf{不可行} \\
\bottomrule
\end{tabular}
\end{table}

\textbf{经腔静脉入路评估}:

\begin{table}[h]
\centering
\caption{经腔静脉入路解剖评估}
\label{tab:transcaval_anatomy}
\begin{tabular}{ll}
\toprule
\textbf{参数} & \textbf{评估结果} \\
\midrule
靶点位置 & L3椎体上缘 \\
靶点钙化 & 无钙化 \\
内脏器官干扰 & 无 \\
与重要动脉分支关系 & 远离 \\
下腔静脉-腹主动脉距离 & 9.4 mm \\
腹主动脉直径 & 22.5 mm \\
\textbf{经腔静脉入路可行性} & \textbf{可行} \\
\bottomrule
\end{tabular}
\end{table}

\subsubsection{心脏团队讨论}

\textbf{综合评估}:

\begin{enumerate}
    \item \textbf{老年医学评估}:
    \begin{itemize}
        \item 虚弱患者
        \item 功能状态受损
        \item 谵妄高风险(既往卒中史)
        \item 卒中高风险
    \end{itemize}

    \item \textbf{心脏外科评估}:
    \begin{itemize}
        \item 患者手术风险过高
        \item \textbf{STS-PROM评分:5.69\%}
    \end{itemize}
\end{enumerate}

\textbf{治疗决策}:

\textbf{共识方案}:先行冠脉血运重建(RCA和LCX的PCI),随后进行经腔静脉TAVR,使用Edwards Sapien 3 Ultra 23 mm球囊扩张瓣膜。

\subsection{手术过程}

\subsubsection{第一步:RCA-PCI(成功)}

\textbf{手术入路}:经股动脉入路

\textbf{手术过程}:
\begin{itemize}
    \item 植入支架:3.50 × 15 mm 药物洗脱支架(EES)
    \item 后扩张:4.00 mm球囊
    \item \textbf{结果}:成功
\end{itemize}

\subsubsection{第二步:LCX-PCI首次尝试(失败)}

\textbf{手术入路}:经股动脉入路

\textbf{所用器械}:
\begin{itemize}
    \item 超支撑导丝(extra-support guidewire)
    \item 6 Fr 导引延长导管(guide extension catheter)
    \item 65 cm 7 Fr 导入鞘(introducer sheath)
\end{itemize}

\textbf{失败原因}:
\begin{enumerate}
    \item 严重髂股动脉迂曲
    \item LCX本身迂曲
    \item LCX钙化
\end{enumerate}

\textbf{结果}:\textbf{尽管使用了所有可用的辅助器械,仍无法完成LCX-PCI}

\subsubsection{第三步:经腔静脉TAVR(成功)}

\textbf{入路建立}:

\begin{itemize}
    \item \textbf{穿刺器械}:电导0.014'' Astato XS20导丝
    \item \textbf{捕获器械}:25 mm圈套器(snare)
    \item \textbf{方法}:标准经腔静脉入路技术
\end{itemize}

\textbf{鞘管输送}:

\begin{itemize}
    \item \textbf{支撑导丝}:超硬Lunderquist导丝
    \item \textbf{输送鞘}:14 Fr eSheath
    \item 成功通过经腔静脉通道推送至升主动脉
\end{itemize}

\textbf{瓣膜植入}:

\begin{itemize}
    \item \textbf{瓣膜型号}:Edwards Sapien 3 Ultra 23 mm
    \item \textbf{结果}:成功植入
\end{itemize}

\subsubsection{第四步:经腔静脉LCX-PCI(成功)}

\textbf{策略转变}:利用已建立的经腔静脉入路完成LCX-PCI

\textbf{手术过程}:
\begin{itemize}
    \item \textbf{入路}:通过经腔静脉通道
    \item \textbf{植入支架}:3.00 × 8 mm 药物洗脱支架(EES)
    \item \textbf{后扩张}:3.50 mm OPN球囊
    \item \textbf{结果}:成功
\end{itemize}

\textbf{通道关闭}:

\begin{itemize}
    \item \textbf{封堵器}:8×10 mm Amplatzer Duct Occluder-1(ADO-1)
    \item \textbf{封堵类型}:1型封堵
    \item \textbf{结果}:成功
\end{itemize}

\subsection{手术详细数据总结}

\begin{table}[h]
\centering
\caption{手术器械和参数汇总}
\label{tab:procedure_summary}
\begin{tabular}{lll}
\toprule
\textbf{手术步骤} & \textbf{关键器械/参数} & \textbf{结果} \\
\midrule
\multirow{2}{*}{RCA-PCI} & 支架:3.50×15 mm EES & \multirow{2}{*}{成功} \\
 & 后扩张:4.00 mm球囊 & \\
\midrule
\multirow{3}{*}{LCX-PCI(经股)} & 超支撑导丝 & \multirow{3}{*}{\textbf{失败}} \\
 & 6 Fr导引延长导管 & \\
 & 65 cm 7 Fr导入鞘 & \\
\midrule
\multirow{3}{*}{经腔静脉TAVR} & 穿刺:0.014'' Astato XS20 & \multirow{3}{*}{成功} \\
 & 输送:14 Fr eSheath & \\
 & 瓣膜:Sapien 3 Ultra 23 mm & \\
\midrule
\multirow{2}{*}{LCX-PCI(经腔静脉)} & 支架:3.00×8 mm EES & \multirow{2}{*}{成功} \\
 & 后扩张:3.50 mm OPN & \\
\midrule
通道关闭 & 8×10 mm ADO-1(1型) & 成功 \\
\bottomrule
\end{tabular}
\end{table}

\subsection{文献回顾}

\subsubsection{已报道的类似病例}

截至目前,文献中仅有\textbf{两例}报道经腔静脉入路联合TAVR和PCI的病例。

\textbf{已报道病例的特点}:

\begin{enumerate}
    \item \textbf{病例来源}:
    \begin{itemize}
        \item ACC.20 World Congress of Cardiology (JACC 2020)
        \item JACC Cardiovascular Interventions 2024
    \end{itemize}

    \item \textbf{手术顺序}:
    \begin{itemize}
        \item 先进行TAVR(瓣膜植入)
        \item 后进行Impella保护下的PCI
    \end{itemize}

    \item \textbf{选择经腔静脉入路的原因}:
    \begin{itemize}
        \item \textbf{严重外周动脉疾病(PAD)}
        \item \textbf{而非髂股动脉迂曲}
    \end{itemize}

    \item \textbf{PCI保护策略}:
    \begin{itemize}
        \item 使用Impella机械循环支持
    \end{itemize}
\end{enumerate}

\subsubsection{本病例的独特性}

\begin{table}[h]
\centering
\caption{本病例与既往文献的对比}
\label{tab:case_comparison}
\begin{tabular}{p{4cm}p{4cm}p{4cm}}
\toprule
\textbf{特征} & \textbf{既往文献报道} & \textbf{本病例} \\
\midrule
经腔静脉入路适应证 & 严重PAD & \textbf{严重髂股动脉迂曲} \\
\midrule
手术顺序 & TAVR → PCI & RCA-PCI → TAVR → LCX-PCI \\
\midrule
PCI机械支持 & Impella保护 & \textbf{无机械支持} \\
\midrule
PCI入路 & 瓣膜植入后经腔静脉 & \textbf{经股(RCA)+ 经腔静脉(LCX)} \\
\midrule
主要挑战 & PAD导致器械输送困难 & \textbf{迂曲导致导管操作困难} \\
\bottomrule
\end{tabular}
\end{table}

\textbf{本病例的创新点}:

\begin{enumerate}
    \item \textbf{首次报道}因髂股动脉迂曲(而非PAD)选择经腔静脉入路同时完成TAVR和PCI
    \item \textbf{灵活的入路策略}:根据病变特点选择不同入路(RCA经股,LCX经腔静脉)
    \item \textbf{无需机械循环支持}:在无Impella保护的情况下安全完成PCI
    \item \textbf{证明了经腔静脉入路的多功能性}:不仅用于大口径器械输送,也可用于需要良好支撑的复杂冠脉介入
\end{enumerate}

\subsection{主要研究发现}

\subsubsection{核心发现}

\begin{enumerate}
    \item \textbf{经腔静脉入路可克服髂股动脉迂曲}:
    \begin{itemize}
        \item 当髂股动脉极度迂曲和动脉瘤样扩张(EIA直径达27-32 mm)时
        \item 经腔静脉入路提供了相对直线的路径
        \item 避免了迂曲路径对导管操作的影响
    \end{itemize}

    \item \textbf{经腔静脉入路支持复杂PCI}:
    \begin{itemize}
        \item 传统观点认为经腔静脉入路仅适用于大口径器械输送(TAVR、EVAR等)
        \item 本病例证明该入路也能为复杂PCI提供足够的支撑力
        \item 特别是在病变钙化、迂曲的情况下
    \end{itemize}

    \item \textbf{联合手术的可行性}:
    \begin{itemize}
        \item 通过单一经腔静脉通道可依次完成TAVR和PCI
        \item 14 Fr eSheath足够大,可容纳TAVR输送系统和PCI导管
        \item 1型封堵(8×10 mm ADO-1)可安全关闭通道
    \end{itemize}

    \item \textbf{入路选择的个体化}:
    \begin{itemize}
        \item 本病例展示了根据不同血管病变特点灵活选择入路的策略
        \item RCA病变相对容易到达,通过经股入路完成
        \item LCX病变因髂股迂曲无法通过经股入路,改用经腔静脉入路成功
    \end{itemize}
\end{enumerate}

\subsubsection{技术要点}

\textbf{经腔静脉入路建立}:

\begin{itemize}
    \item \textbf{关键器械组合}:
    \begin{itemize}
        \item 电导导丝(0.014'' Astato XS20):用于穿刺
        \item 25 mm圈套器:用于在主动脉内捕获导丝
        \item 超硬Lunderquist导丝:提供支撑力
        \item 14 Fr eSheath:大口径输送鞘
    \end{itemize}

    \item \textbf{解剖选择标准}:
    \begin{itemize}
        \item 无钙化靶点区域
        \item 无内脏器官干扰
        \item 远离重要动脉分支
        \item VCI-主动脉距离适中(9.4 mm)
        \item 主动脉直径合适(22.5 mm)
    \end{itemize}
\end{itemize}

\textbf{通道关闭策略}:

\begin{itemize}
    \item 使用Amplatzer Duct Occluder-1(ADO-1)
    \item 型号:8×10 mm
    \item 达到1型封堵(完全封堵,无残余分流)
\end{itemize}

\subsection{结论}

\subsubsection{主要结论}

\begin{enumerate}
    \item \textbf{经腔静脉入路的传统定位}:
    \begin{itemize}
        \item 主要作为非经股TAVR候选者的备用路径
        \item 适应证包括严重PAD、髂股动脉闭塞、既往腹主动脉瘤修复等
    \end{itemize}

    \item \textbf{经腔静脉入路联合PCI的现状}:
    \begin{itemize}
        \item 在TAVR过程中同时进行PCI的报道仍然很少
        \item 尤其是在严重髂股动脉迂曲患者中报道罕见
        \item 既往病例主要因PAD选择该入路,而非迂曲
    \end{itemize}

    \item \textbf{本病例的临床价值}:
    \begin{itemize}
        \item 强调了经腔静脉入路作为\textbf{可行且有效的替代方案}
        \item 适用于髂股动脉解剖禁忌的患者
        \item 不仅用于TAVR,也可促进导管操作和导航
        \item 特别适用于\textbf{复杂冠脉介入治疗}
    \end{itemize}
\end{enumerate}

\subsubsection{技术意义}

\textbf{扩展了经腔静脉入路的适应证}:

\begin{itemize}
    \item 从单纯"器械输送通道" → "全功能介入通道"
    \item 从"PAD备用方案" → "迂曲解剖的首选方案"
    \item 从"单一操作通道" → "多操作联合通道"
\end{itemize}

\subsection{临床启示}

\subsubsection{对临床实践的指导}

\textbf{1. 术前评估要点}:

\begin{enumerate}
    \item \textbf{全面的髂股动脉评估}:
    \begin{itemize}
        \item 不仅评估直径和钙化
        \item 更要重视\textbf{迂曲程度}和\textbf{动脉瘤样扩张}
        \item 使用3D重建评估器械输送路径
    \end{itemize}

    \item \textbf{经腔静脉入路解剖筛选}:
    \begin{itemize}
        \item CT扫描评估下腔静脉-主动脉关系
        \item 理想靶点特征:
        \begin{itemize}
            \item 位置:通常在L3-L4水平
            \item 无钙化区域
            \item 无内脏器官干扰
            \item VCI-主动脉距离:8-12 mm为宜
            \item 远离肾动脉、肠系膜动脉等重要分支
        \end{itemize}
    \end{itemize}

    \item \textbf{冠脉病变评估}:
    \begin{itemize}
        \item 如需联合PCI,评估病变复杂程度
        \item 钙化、迂曲病变可能需要额外支撑
        \item 预判经股入路完成PCI的可能性
    \end{itemize}
\end{enumerate}

\textbf{2. 入路选择策略}:

\begin{table}[h]
\centering
\caption{髂股动脉解剖异常的入路决策}
\label{tab:access_decision}
\begin{tabular}{p{5cm}p{5cm}p{3cm}}
\toprule
\textbf{解剖特点} & \textbf{推荐入路} & \textbf{替代方案} \\
\midrule
轻中度迂曲 & 经股入路 & - \\
\midrule
严重迂曲 + 直径正常 & 考虑经腔静脉或其他替代入路 & 经颈动脉、经锁骨下 \\
\midrule
迂曲 + 动脉瘤样扩张(>25 mm) & \textbf{首选经腔静脉} & 经颈动脉、经心尖 \\
\midrule
严重PAD + 钙化 & 经腔静脉或经锁骨下 & 经颈动脉 \\
\midrule
既往主动脉手术 & 个体化评估 & - \\
\bottomrule
\end{tabular}
\end{table}

\textbf{3. 联合手术的操作建议}:

\begin{enumerate}
    \item \textbf{手术顺序}(根据本病例经验):
    \begin{itemize}
        \item 先尝试经股入路完成部分PCI(如本例RCA)
        \item 建立经腔静脉入路并完成TAVR
        \item 利用已建立的经腔静脉通道完成剩余PCI(如本例LCX)
        \item 最后关闭经腔静脉通道
    \end{itemize}

    \item \textbf{器械准备}:
    \begin{itemize}
        \item 确保有足够大的输送鞘(本例14 Fr)
        \item 准备多种支撑导丝
        \item 准备合适的封堵器(通常ADO-1或AVP)
        \item 备用止血方案
    \end{itemize}

    \item \textbf{团队协作}:
    \begin{itemize}
        \item 需要熟练掌握经腔静脉技术的操作者
        \item 需要复杂PCI经验
        \item 心外科待命以防并发症
    \end{itemize}
\end{enumerate}

\textbf{4. 患者选择}:

\textbf{适合经腔静脉TAVR+PCI的患者特征}:

\begin{itemize}
    \item 严重AS需要TAVR
    \item 合并需要血运重建的CAD
    \item 髂股动脉解剖禁忌(迂曲、动脉瘤、严重PAD)
    \item 经腔静脉入路解剖合适
    \item 外科风险高或患者拒绝外科手术
    \item 凝血功能可接受
\end{itemize}

\textbf{相对禁忌证}:

\begin{itemize}
    \item 腹主动脉或下腔静脉严重钙化
    \item 腹主动脉直径过大(>35 mm)或过小(<18 mm)
    \item VCI-主动脉距离过大(>15 mm)
    \item 重要内脏器官干扰
    \item 严重凝血功能障碍
    \item 既往腹部放疗史
\end{itemize}

\subsubsection{对未来研究的启示}

\begin{enumerate}
    \item \textbf{扩大经腔静脉入路的应用范围}:
    \begin{itemize}
        \item 探索在其他复杂介入治疗中的应用
        \item 如复杂CTO-PCI、左心耳封堵、二尖瓣介入等
    \end{itemize}

    \item \textbf{优化器械和技术}:
    \begin{itemize}
        \item 开发专用的经腔静脉穿刺和闭合装置
        \item 改进封堵器设计以提高成功率
        \item 开发更灵活、支撑力更好的导管系统
    \end{itemize}

    \item \textbf{建立标准化流程}:
    \begin{itemize}
        \item 制定经腔静脉联合手术的标准操作流程
        \item 建立培训和认证体系
        \item 积累多中心经验
    \end{itemize}

    \item \textbf{长期随访研究}:
    \begin{itemize}
        \item 评估经腔静脉通道闭合的长期效果
        \item 监测迟发并发症
        \item 比较不同入路的长期预后
    \end{itemize}
\end{enumerate}

\subsection{研究局限性}

\begin{enumerate}
    \item \textbf{单一病例报告}:
    \begin{itemize}
        \item 本文仅为单一病例报告,缺乏统计学意义
        \item 无法推断整体成功率和并发症发生率
        \item 需要更多病例和系列研究验证
    \end{itemize}

    \item \textbf{缺乏长期随访数据}:
    \begin{itemize}
        \item 会议演讲未提供长期随访结果
        \item 无法评估经腔静脉通道闭合的持久性
        \item 无法评估迟发并发症
        \item 无法比较与其他入路的长期预后
    \end{itemize}

    \item \textbf{患者选择偏倚}:
    \begin{itemize}
        \item 病例来自有丰富经腔静脉经验的高容量中心(Erasmus MC)
        \item 操作者技术水平可能影响成功率
        \item 结果可能无法外推至所有中心
    \end{itemize}

    \item \textbf{未提供详细并发症信息}:
    \begin{itemize}
        \item 未详细报告手术过程中的并发症
        \item 未报告透视时间、对比剂用量等参数
        \item 未报告住院期间和出院后早期并发症
    \end{itemize}

    \item \textbf{缺乏对照}:
    \begin{itemize}
        \item 无法与其他替代入路(如经颈动脉、经锁骨下、经心尖)进行比较
        \item 无法评估该策略相对于分期手术的优劣
    \end{itemize}

    \item \textbf{费用效益分析缺失}:
    \begin{itemize}
        \item 未评估额外器械(封堵器等)的成本
        \item 未比较与传统方法的经济学差异
    \end{itemize}
\end{enumerate}

\subsection{个人笔记}

\subsubsection{关键数字记忆}

\textbf{患者特征}:
\begin{itemize}
    \item 年龄:81岁
    \item BMI:33.32 kg/m²(肥胖)
    \item eGFR:26 ml/min/1.73m²(4期CKD)
    \item STS-PROM:5.69\%(中高危)
\end{itemize}

\textbf{超声心动图}:
\begin{itemize}
    \item SVi = 24 ml/m²(低流量)
    \item AVAi = 0.38 cm²/m²(严重狭窄)
    \item LAVi = 40 ml/m²(双房扩大)
\end{itemize}

\textbf{CT关键数据}:
\begin{itemize}
    \item 主动脉瓣Agatston评分:1900
    \item 右EIA最大直径:27.2 mm(动脉瘤样)
    \item 左EIA最大直径:32.1 mm(动脉瘤样)
    \item VCI-主动脉距离:9.4 mm(适合经腔静脉)
    \item 腹主动脉直径:22.5 mm
    \item LCA高度:13.2 mm,RCA高度:16.8 mm
\end{itemize}

\textbf{手术器械}:
\begin{itemize}
    \item RCA支架:3.50×15 mm EES,后扩4.00 mm
    \item LCX支架:3.00×8 mm EES,后扩3.50 mm OPN
    \item 瓣膜:Edwards Sapien 3 Ultra 23 mm
    \item 输送鞘:14 Fr eSheath
    \item 封堵器:8×10 mm ADO-1(1型封堵)
\end{itemize}

\subsubsection{重要概念}

\begin{description}
    \item[pLFLG-AS] 保留射血分数低流量低梯度主动脉瓣狭窄(paradoxical low-flow low-gradient aortic stenosis)- 一种特殊的AS类型,尽管射血分数正常,但由于严重LVH导致小左室腔、低每搏量,从而呈现低流量低梯度的特点,诊断具有挑战性。

    \item[经腔静脉入路(Transcaval access)] 通过在下腔静脉和腹主动脉之间建立临时通道,为大口径器械输送提供路径的技术。传统用于经股入路禁忌的TAVR患者,本病例扩展了其应用至复杂PCI。

    \item[髂股动脉迂曲(Iliofemoral tortuosity)] 髂动脉和股动脉过度弯曲、扭转,导致导管和器械通过困难。与动脉瘤样扩张结合时(如本例EIA直径达27-32 mm),传统经股入路几乎不可能。

    \item[1型封堵(Type 1 closure)] 经腔静脉通道的完全封堵,无残余分流,代表最理想的闭合结果。通常使用Amplatzer Duct Occluder(ADO)或Amplatzer Vascular Plug(AVP)实现。

    \item[ADO-1] Amplatzer Duct Occluder-1,原本设计用于封堵动脉导管未闭(PDA),现被广泛用于经腔静脉通道的闭合。
\end{description}

\subsubsection{临床思考要点}

\textbf{1. 为什么经股入路失败后经腔静脉入路能成功?}

\begin{itemize}
    \item \textbf{解剖路径差异}:
    \begin{itemize}
        \item 经股入路:需通过极度迂曲的髂股动脉,路径长、弯曲多
        \item 经腔静脉入路:下腔静脉→腹主动脉→胸主动脉,路径相对直
    \end{itemize}

    \item \textbf{支撑力差异}:
    \begin{itemize}
        \item 迂曲路径会"吸收"导管推送力,降低远端支撑
        \item 直线路径能更有效地传递操作力
    \end{itemize}

    \item \textbf{器械干扰}:
    \begin{itemize}
        \item 迂曲路径中,鞘管和导管容易弯折、打折
        \item 直线路径中器械性能更好
    \end{itemize}
\end{itemize}

\textbf{2. pLFLG-AS的诊断要点}

本病例展示了典型的pLFLG-AS特征:
\begin{itemize}
    \item 正常射血分数(EF)
    \item 严重左室肥厚(LVH)→ 小左室腔
    \item 低每搏量指数(SVi = 24 ml/m²,正常>35 ml/m²)
    \item 严重瓣口狭窄(AVAi = 0.38 cm²/m²)
    \item 相对低的跨瓣压差(虽然文中未提供具体数值)
\end{itemize}

这类患者常被误诊为中度AS而延误治疗,强调了综合评估的重要性。

\textbf{3. 联合TAVR和PCI的时机选择}

本病例的手术顺序值得思考:
\begin{itemize}
    \item \textbf{先PCI(RCA)}:稳定冠脉灌注,降低TAVR期间心肌缺血风险
    \item \textbf{再TAVR}:解除AS,改善血流动力学
    \item \textbf{后PCI(LCX)}:利用已建立的经腔静脉通道
\end{itemize}

这一顺序合理,但也可考虑其他策略:
\begin{itemize}
    \item 如果冠脉病变非常严重,可能需要先完成全部血运重建
    \item 如果AS症状占主导,可能先TAVR再PCI
    \item 需要个体化决策
\end{itemize}

\textbf{4. 特殊患者群体的考量}

本患者有多个高危因素:
\begin{itemize}
    \item 高龄(81岁)+ 虚弱
    \item 近期卒中(1个月内)→ 再发卒中和谵妄高风险
    \item 4期CKD → 对比剂肾病风险、出血风险
    \item 房颤 + 抗凝 → 出血风险
    \item 既往主动脉手术 → 解剖改变
\end{itemize}

这些因素都支持选择:
\begin{itemize}
    \item 微创入路(TAVR vs 外科)
    \item 尽可能减少创伤(一次完成vs分期)
    \item 缩短手术时间和对比剂用量(虽然文中未提供数据)
\end{itemize}

\subsubsection{技术细节思考}

\textbf{1. 14 Fr鞘管的选择}

\begin{itemize}
    \item Edwards Sapien 3 Ultra 23 mm通常需要14 Fr鞘管
    \item 同一鞘管也足够进行PCI操作
    \item 如果使用更大的瓣膜,可能需要16 Fr鞘管
    \item 鞘管越大,通道闭合难度越大
\end{itemize}

\textbf{2. 封堵器的选择}

\begin{itemize}
    \item ADO-1 8×10 mm:腰部直径8 mm,盘片直径10 mm
    \item 对于14 Fr(外径约5 mm)的通道,8 mm封堵器是合理的
    \item 需要超选1-2 mm以确保有效封堵
    \item 替代选择:Amplatzer Vascular Plug II或IV
\end{itemize}

\textbf{3. 为什么选择L3水平?}

\begin{itemize}
    \item L3-L4通常是最理想的穿刺水平
    \item 原因:
    \begin{itemize}
        \item 远离肾动脉(L1-L2水平)
        \item 远离肠系膜下动脉(L3水平)
        \item 腹主动脉直径合适
        \item VCI-主动脉距离通常较短
        \item 无肠道干扰
    \end{itemize}
\end{itemize}

\subsubsection{值得关注的问题}

\begin{enumerate}
    \item \textbf{手术时间和辐射剂量?}
    \begin{itemize}
        \item 文中未提供,但可能显著长于单纯TAVR
        \item 对患者和术者都是考验
    \end{itemize}

    \item \textbf{对比剂用量?}
    \begin{itemize}
        \item 患者eGFR仅26 ml/min,对比剂肾病风险极高
        \item 联合手术势必增加对比剂用量
        \item 需要权衡利弊
    \end{itemize}

    \item \textbf{抗凝管理?}
    \begin{itemize}
        \item 患者长期服用阿哌沙班
        \item 手术前如何桥接?
        \item 手术中肝素用量?
        \item 术后何时恢复抗凝?
        \item 双联抗血小板 + 抗凝的出血风险?
    \end{itemize}

    \item \textbf{术后监测?}
    \begin{itemize}
        \item 如何监测通道闭合情况?
        \item 腹膜后出血的监测?
        \item 下肢缺血的风险?
    \end{itemize}

    \item \textbf{备选方案?}
    \begin{itemize}
        \item 如果经腔静脉入路失败,下一步是什么?
        \item 经颈动脉?经锁骨下?经心尖?
        \item 外科开胸手术?
    \end{itemize}
\end{enumerate}

\subsubsection{对中国的启示}

\begin{enumerate}
    \item \textbf{技术可行性}:
    \begin{itemize}
        \item 中国已有多个中心开展经腔静脉TAVR
        \item 但联合PCI的经验仍较少
        \item 需要积累经验和培训
    \end{itemize}

    \item \textbf{患者群体特点}:
    \begin{itemize}
        \item 中国AS患者发病年龄可能更早
        \item 风湿性病变比例可能更高
        \item 二叶瓣畸形比例较高
        \item 需要根据国内患者特点调整策略
    \end{itemize}

    \item \textbf{医保支付}:
    \begin{itemize}
        \item 额外封堵器等器械的费用
        \item 联合手术的收费标准
        \item 需要合理的定价和报销政策
    \end{itemize}

    \item \textbf{中心资质}:
    \begin{itemize}
        \item 应该限制在有经验的高容量中心
        \item 需要多学科团队支持
        \item 建立质控体系
    \end{itemize}
\end{enumerate}

\subsubsection{后续文献追踪}

建议关注:
\begin{itemize}
    \item Erasmus MC团队的后续病例系列报道
    \item 其他中心的类似经验
    \item 长期随访结果
    \item 新型专用器械的开发
    \item 指南或专家共识的更新
\end{itemize}

\subsubsection{个人总结}

这是一个极具教学价值的病例,展示了:
\begin{itemize}
    \item \textbf{创新思维}:突破传统适应证,扩展经腔静脉入路的应用
    \item \textbf{技术整合}:将TAVR和PCI技术有机结合
    \item \textbf{个体化策略}:针对患者特点灵活选择入路
    \item \textbf{多学科协作}:老年医学、心脏病学、心外科、影像学的紧密配合
\end{itemize}

\textbf{关键启示}:当传统方法遇到困难时,不要轻易放弃,创新性地使用现有技术可能开辟新的治疗路径。但同时要注意:
\begin{itemize}
    \item 充分的术前评估和准备
    \item 严格的适应证把握
    \item 完善的备用方案
    \item 细致的术后监测
\end{itemize}

\section{应对主动脉瓣狭窄治疗不足:TARGET AS、DETECT AS及未来展望}
\label{sec:16_002_addressing_undertreatment}

% ============================================
% 文献信息
% ============================================
\subsection{文献信息}

\begin{itemize}
    \item \textbf{标题}: Addressing Undertreatment in Aortic Stenosis: Target AS, Detect AS, and Beyond
    \item \textbf{作者}: Sammy Elmariah, MD, MPH
    \item \textbf{机构}: UCSF Health; Leone-Perkins Family Endowed Professor of Medicine; Chief, Interventional Cardiology; Director, UCSF Cardiac Catheterization Laboratory
    \item \textbf{会议}: TCT (Transcatheter Cardiovascular Therapeutics)
    \item \textbf{PDF文件名}: addressing-undertreatment-in-aortic-stenosis-target-as-detect-as-and-beyond.pdf
    \item \textbf{文献类型}: 会议演讲/综述
    \item \textbf{利益冲突}: 获得Edwards Lifesciences、Medtronic、Abbott的研究支持和顾问费;持有Prospect Health股份
\end{itemize}

\subsection{研究背景}

\subsubsection{主动脉瓣狭窄治疗不足的现状}

症状性严重主动脉瓣狭窄(AS)与高发病率和死亡率密切相关,若不治疗预后极差。主动脉瓣置换术(AVR)具有治愈性,可延长各种AS亚型患者的生命。然而,AS存在显著的治疗不足问题,尤其在以下人群中:

\begin{itemize}
    \item \textbf{女性患者}
    \item \textbf{老年患者}
    \item \textbf{种族/族裔少数群体}
\end{itemize}

\subsubsection{不同血流动力学亚型的治疗不足}

根据Li SX等人的研究(JACC 2022;79:864-77),不同AS亚型的AVR治疗率存在显著差异:

\begin{table}[h]
\centering
\caption{不同血流动力学亚型的AVR治疗率及生存获益}
\label{tab:hemodynamic_subtypes_treatment_mortality}
\begin{tabular}{lcccc}
\toprule
\textbf{血流动力学亚型} & \textbf{指南推荐} & \textbf{AVR治疗率} & \textbf{未治疗率} & \textbf{AVR死亡风险降低} \\
\midrule
高梯度-正常射血分数 (HG-NEF) & Class I & 70\% & 30\% & 2.4倍 \\
高梯度-低射血分数 (HG-LEF) & Class I & 53\% & 47\% & 3.6倍 \\
低梯度-正常射血分数 (LG-NEF) & Class II & 32\% & 68\% & 1.4倍 \\
低梯度-低射血分数 (LG-LEF) & Class II & 38\% & 62\% & 2.1倍 \\
\bottomrule
\end{tabular}
\end{table}

\textbf{关键发现}:
\begin{itemize}
    \item \textbf{即使是Class I指征(HG-NEF、HG-LEF),仍有30-47\%的患者未接受治疗}
    \item 低梯度AS(LG-NEF、LG-LEF)的治疗率极低(<40\%)
    \item 所有亚型的AVR均显著降低死亡风险
    \item \textbf{总体治疗率<50\%},存在严重的治疗缺口
\end{itemize}

\subsubsection{现有质量改进需求}

演讲强调:\textbf{"存在明确且未满足的需求,需要有效、低成本、可扩展的工具来促进严重AS的指南驱动管理。"}

% ============================================
% TARGET AS项目
% ============================================
\subsection{TARGET AS项目:AHA质量改进倡议}

\subsubsection{项目概述}

\textbf{全称}:Target: Aortic Stenosis - An AHA Quality Initiative

\textbf{项目目标}:
\begin{itemize}
    \item \textbf{对医疗系统}:实施基于最新指南的质量措施
    \item \textbf{对医疗提供者}:提供指南导向的最佳护理标准教育
    \item \textbf{对患者}:提高患者认知和参与度
\end{itemize}

\textbf{项目赞助}:Edwards Lifesciences是美国心脏协会Target: Aortic Stenosis项目的国家赞助商

\subsubsection{AS患者护理路径}

Target AS项目覆盖AS管理的全流程:

\begin{enumerate}
    \item \textbf{认知(Awareness)}:提高AS疾病认知
    \item \textbf{检测(Detection)}:早期发现AS患者
    \item \textbf{诊断(Diagnosis)}:准确诊断AS严重程度
    \item \textbf{转诊(Referral)}:转诊至心脏瓣膜团队
    \item \textbf{治疗(Treatment)}:实施AVR治疗
    \item \textbf{监测(Monitoring)}:术后随访和监测
\end{enumerate}

\textbf{Target AS与现有手术注册库的对比}:

\begin{itemize}
    \item \textbf{Target AS}:覆盖全流程(认知、检测、诊断、转诊、治疗、监测)✓
    \item \textbf{现有手术注册库}:仅覆盖治疗和监测阶段 ✗
\end{itemize}

\subsubsection{项目进展数据(截至2025年9月24日)}

\begin{table}[h]
\centering
\caption{Target: Aortic Stenosis项目实施情况}
\label{tab:target_as_implementation}
\begin{tabular}{lc}
\toprule
\textbf{指标} & \textbf{数据} \\
\midrule
签约参与医院 & 75家 \\
录入患者记录 & 12,386例 \\
患者就诊次数 & 47,704+次 \\
\bottomrule
\end{tabular}
\end{table}

\subsubsection{质量测量关系图}

Target AS项目建立了三级质量测量体系:

\textbf{关键指标(Key Metric)}:
\begin{itemize}
    \item 超声心动图提示中-重度或重度AS
    \item 可能或明确重度AS(AV面积≤1.0 cm²,峰值流速≥4 m/s,峰值梯度≥64 mmHg,平均梯度≥40 mmHg)
\end{itemize}

\textbf{支持性指标(Supporting Metric)}:
\begin{itemize}
    \item 超声关键发现
    \item 超声总结中的临床建议
    \item \textbf{及时诊断AS严重程度}:及时评估症状、LVEF、SVI和多模态检查
    \item 及时完成随访超声
\end{itemize}

\textbf{过程步骤(Process Step)}:
\begin{itemize}
    \item Class I指征评估
    \item MDT评估(仅评估接受AVR的患者)
    \item \textbf{重度AS及时治疗}:主要测量指标
\end{itemize}

\subsubsection{2026年认可标准(基于2024年数据)}

\begin{table}[h]
\centering
\caption{Target AS 2026年医院认可标准}
\label{tab:target_as_recognition_criteria}
\begin{tabular}{lcc}
\toprule
\textbf{测量指标} & \textbf{目标值} & \textbf{最低分母要求} \\
\midrule
\textbf{主要测量}:及时治疗重度AS & 75\% & 最少6例患者 \\
\textbf{次要测量}:无缺陷及时诊断 & 50\% & 最少30例超声 \\
\bottomrule
\end{tabular}
\end{table}

\textbf{主要测量定义}:
\begin{itemize}
    \item 有Class I指征的重度AS患者
    \item 在初次诊断后90天内接受确定性治疗(SAVR或TAVI)的比例
\end{itemize}

\textbf{次要测量定义}:
\begin{itemize}
    \item 可能重度AS的超声检查
    \item 完成所有必要评估和检查以明确严重程度并确定是否存在Class I指征的比例
\end{itemize}

\textbf{支持性测量}(必须报告,但无阈值或最低要求):
\begin{itemize}
    \item 超声关键发现报告和总结/结论
    \item MDT评估
    \item 及时完成随访超声
\end{itemize}

\textbf{容量标准}:
\begin{itemize}
    \item 必须在注册库中有40例患者才能获得认可资格
\end{itemize}

% ============================================
% 新质量标准
% ============================================
\subsection{瓣膜性心脏病的新质量标准}

\subsubsection{AHA Target: Aortic Stenosis倡议}

根据Lindman BR等人的研究(Circ Cardiovasc Qual Outcomes. 2023;16(6):e009712),Target AS倡议旨在改善严重AS患者在AVR上游的护理和结局。

\subsubsection{ACC/AHA性能测量标准}

Jneid H等人发表的2024 ACC/AHA临床性能和质量测量标准(J Am Coll Cardiol. 2024;83(16):1579-1613)明确提出:

\textbf{核心性能测量}:
\begin{itemize}
    \item \textbf{"已准备好用于公开报告和按绩效付费项目"}
    \item \textbf{症状性严重AS患者在诊断后90天内接受AVR的比例}
\end{itemize}

这一标准的设立进一步强调了及时治疗AS的重要性。

% ============================================
% DETECT AS试验
% ============================================
\subsection{DETECT AS试验:电子提供者通知系统的效果}

\subsubsection{试验设计}

\textbf{试验全称}:Detection and Evaluation of Critical Aortic Stenosis

\textbf{ClinicalTrials.gov注册号}:NCT05230225

\textbf{研究性质}:
\begin{itemize}
    \item 实用性、单盲、整群随机对照试验
    \item 在多中心MGH学术医疗系统内进行的质量改进倡议
\end{itemize}

\textbf{研究对象}:
\begin{itemize}
    \item \textbf{纳入标准}:经胸超声心动图(TTE)显示主动脉瓣面积(AVA)≤1.0 cm²的患者
    \item \textbf{样本量}:945名患者
    \item \textbf{提供者数量}:285名临床提供者
\end{itemize}

\textbf{随机化方案}:
\begin{itemize}
    \item 1:1随机化临床提供者
    \item 分层分配在后续患者中保持稳定
\end{itemize}

\textbf{干预组 - 电子提供者通知(EPN)}:
\begin{itemize}
    \item 通过电子邮件和EMR收件箱发送个性化EPN
    \item 通知包含患者具体的血流动力学信息和指南推荐
\end{itemize}

\textbf{对照组}:
\begin{itemize}
    \item 常规护理(Usual Care)
\end{itemize}

\textbf{主要终点}:
\begin{itemize}
    \item 指标TTE后1年内接受AVR的患者比例
    \item 随访时间:完整1年
\end{itemize}

\textbf{研究资助}:Edwards Lifesciences资助的研究者发起的研究

\subsubsection{个性化电子提供者通知(EPN)内容}

EPN根据患者的血流动力学特征分为4类:

\begin{table}[h]
\centering
\caption{个性化EPN分类标准}
\label{tab:epn_classification}
\begin{tabular}{lll}
\toprule
\textbf{分类} & \textbf{平均梯度} & \textbf{LVEF} \\
\midrule
1. 高梯度-正常LVEF & mAVG ≥40 mmHg & LVEF ≥50\% \\
2. 高梯度-低LVEF & mAVG ≥40 mmHg & LVEF <50\% \\
3. 低梯度-正常LVEF & mAVG <40 mmHg & LVEF ≥50\% \\
4. 低梯度-低LVEF & mAVG <40 mmHg & LVEF <50\% \\
\bottomrule
\end{tabular}
\end{table}

\textbf{EPN示例内容}(高梯度-正常LVEF患者):

\begin{quote}
\textit{"尊敬的Dr ------}

\textit{您的患者-----最近接受了经胸超声心动图检查,发现存在保留射血分数的严重主动脉瓣狭窄。}

\textit{ACC/AHA瓣膜性心脏病管理指南针对该患者提出以下建议:}

\begin{itemize}
    \item \textit{对于症状性严重AS患者,建议行AVR(Class 1推荐)}
    \item \textit{对于无症状严重AS且手术风险低的患者,当满足以下条件时AVR是合理的:}
    \begin{itemize}
        \item \textit{AS非常严重(定义为主动脉流速≥5 m/s)且手术风险低时,AVR是合理的(Class 2a推荐)}
        \item \textit{运动试验显示运动耐量下降或收缩压从基线到峰值下降≥10 mmHg(Class 2a推荐)}
        \item \textit{血清B型利钠肽(BNP)水平>正常值3倍(Class 2a推荐)}
        \item \textit{连续检查显示主动脉流速增加≥0.3 m/s/年(Class 2a推荐)}
        \item \textit{LVEF在至少3次连续影像学检查中进行性下降至<60\%(Class 2b推荐)}
    \end{itemize}
\end{itemize}

\textit{严重瓣膜性心脏病患者在考虑干预时应由多学科心脏瓣膜团队评估(Class 1推荐)。"}
\end{quote}

\subsubsection{主要研究结果}

\textbf{1. 症状性患者的AVR治疗率}

\begin{table}[h]
\centering
\caption{DETECT AS试验:症状性患者AVR治疗率}
\label{tab:detect_as_primary_endpoint}
\begin{tabular}{lccc}
\toprule
\textbf{组别} & \textbf{90天AVR率} & \textbf{1年AVR率} & \textbf{风险比} \\
\midrule
EPN组 (n=305) & 36.9\% & 60.1\% & \multirow{2}{*}{HR 1.40 (95\%CI 1.06-1.85)} \\
常规护理组 (n=241) & 27.6\% & 47.0\% & \\
\midrule
\textbf{统计学意义} & \multicolumn{3}{c}{\textbf{p=0.02}} \\
\bottomrule
\end{tabular}
\end{table}

\textbf{关键发现}:
\begin{itemize}
    \item EPN显著提高了AVR实施率
    \item \textbf{90天AVR率(符合新质量标准)}:EPN组比常规护理组高9.3个百分点(36.9\% vs 27.6\%)
    \item \textbf{1年AVR率}:EPN组比常规护理组高13.1个百分点(60.1\% vs 47.0\%)
    \item 竞争风险模型调整死亡风险后,结果仍然显著
\end{itemize}

\textbf{2. 性别亚组分析}

\begin{table}[h]
\centering
\caption{DETECT AS试验:按性别分层的AVR治疗率}
\label{tab:detect_as_gender_subgroup}
\begin{tabular}{lcccc}
\toprule
\textbf{亚组} & \textbf{样本量} & \textbf{1年AVR率} & \textbf{风险比} & \textbf{P值} \\
\midrule
\multicolumn{5}{l}{\textit{女性患者}} \\
\quad EPN组 & 248 & 46.1\% & \multirow{2}{*}{HR 2.10 (1.56-2.82)} & \multirow{2}{*}{<0.001} \\
\quad 常规护理组 & 189 & 25.9\% & & \\
\midrule
\multicolumn{5}{l}{\textit{男性患者}} \\
\quad EPN组 & 247 & 48.7\% & \multirow{2}{*}{HR 1.16 (0.89-1.51)} & \multirow{2}{*}{NS} \\
\quad 常规护理组 & 253 & 47.2\% & & \\
\midrule
\multicolumn{5}{l}{\textit{EPN组内性别比较}} \\
\quad 女性 vs 男性 & - & 46.1\% vs 48.7\% & HR 0.90 (0.70-1.16) & 0.43 \\
\midrule
\multicolumn{5}{l}{\textit{常规护理组内性别比较}} \\
\quad 女性 vs 男性 & - & 25.9\% vs 47.2\% & HR 0.46 (0.35-0.62) & <0.001 \\
\bottomrule
\end{tabular}
\end{table}

\textbf{性别差异的关键发现}:
\begin{itemize}
    \item \textbf{EPN对女性的获益显著大于男性}
    \item 在常规护理组中,女性接受AVR的可能性比男性低54\%(HR 0.46)
    \item \textbf{EPN消除了性别差异}:在EPN组中,女性和男性的AVR率相似(46.1\% vs 48.7\%, p=0.43)
    \item 女性从EPN中获得的相对获益:AVR率从25.9\%提升至46.1\%(绝对提升20.2个百分点)
    \item 男性从EPN中获得的相对获益较小:AVR率从47.2\%提升至48.7\%(绝对提升1.5个百分点)
\end{itemize}

\textbf{3. 生存分析}

\begin{table}[h]
\centering
\caption{DETECT AS试验:限制性平均生存时间分析}
\label{tab:detect_as_survival}
\begin{tabular}{lccc}
\toprule
\textbf{分析人群} & \textbf{EPN组} & \textbf{常规护理组} & \textbf{差异} \\
\midrule
\multicolumn{4}{l}{\textit{总体症状性患者}} \\
限制性平均生存时间 & 335天 & 312天 & 23天 \\
统计学意义 & \multicolumn{3}{c}{p=0.01} \\
Log-rank检验 & \multicolumn{3}{c}{p=0.06} \\
\midrule
\multicolumn{4}{l}{\textit{按性别分层(总体Log-rank p=0.05)}} \\
女性EPN获益 & - & - & 26天, p=0.03 \\
男性EPN获益 & - & - & 17天, p=0.16 \\
\bottomrule
\end{tabular}
\end{table}

\textbf{生存分析关键发现}:
\begin{itemize}
    \item EPN延长了症状性AS患者的生存时间
    \item \textbf{限制性平均生存时间延长23天}(p=0.01)
    \item 女性患者从EPN中获得更显著的生存获益(26天 vs 17天)
    \item 生存曲线显示EPN组和常规护理组之间存在持续分离
\end{itemize}

\subsubsection{试验结论}

根据Tanguturi V等人发表在Circulation 2025的研究结果:

\begin{quote}
\textbf{"在AVA≤1.0 cm²的AS患者管理中,EPN带来了:}
\begin{itemize}
    \item \textbf{更高的AVR实施率(90天和1年)}
    \item \textbf{延长生存时间}
    \item \textbf{减少AS管理中的性别和年龄差异"}
\end{itemize}
\end{quote}

\textbf{重要意义}:
\begin{quote}
\textit{"DETECT AS试验证明了基于AI的警报、决策支持和管理工具在提高护理质量方面的潜在影响。"}
\end{quote}

% ============================================
% 2025 ASE标准化指南
% ============================================
\subsection{2025年ASE超声心动图报告标准化指南}

\subsubsection{指南概述}

Taub CC等人发表在JASE 2025;38(9):735-774的\textbf{《成人超声心动图报告标准化指南》}由20个全球超声学会联合背书,代表了超声心动图领域的重大变革。

\textbf{指南参与学会}(部分列举):
\begin{itemize}
    \item American Society of Echocardiography (ASE)
    \item Argentine Federation of Cardiology
    \item Brazilian Society of Cardiology
    \item Chinese Society of Echocardiography
    \item Indian Society of Echocardiography
    \item 等20个学会
\end{itemize}

\subsubsection{关键指南更新}

\textbf{核心理念转变}:

\begin{quote}
\textit{"这些指南可能有助于重新定义超声心动图在患者护理中的角色,\textbf{从被动的、描述性的报告转变为主动的、医生指导的患者管理参与}。"}
\end{quote}

\textbf{关键更新1:关键发现的沟通}

\begin{itemize}
    \item \textbf{关键发现(包括严重AS)应在报告中记录并在数分钟内口头告知开具检查的提供者}
    \item 要求:及时、主动的沟通
    \item 时限:数分钟内(within minutes)
\end{itemize}

\textbf{关键更新2:推荐陈述的纳入}

\begin{itemize}
    \item \textbf{超声心动图医师应在报告中包含针对显著AS的进一步转诊/评估的推荐陈述}
    \item 改变角色:从单纯报告结果到提供临床建议
\end{itemize}

\subsubsection{标准化推荐陈述模板}

指南提供了针对严重AS的标准化报告语言:

\begin{quote}
\textbf{推荐陈述模板:}

\textit{"该患者存在显著的主动脉瓣狭窄,根据当前美国心脏病学会/美国心脏协会/ASE瓣膜性心脏病指南,可能需要治疗。在临床适当的情况下,应考虑进一步评估和/或转诊。"}
\end{quote}

\subsubsection{指南的临床意义}

这一指南更新具有深远影响:

\begin{enumerate}
    \item \textbf{促进早期干预}:通过主动通知和推荐,减少诊断延误
    \item \textbf{标准化临床路径}:为AS患者的后续管理提供清晰指引
    \item \textbf{提高医师责任}:超声医师从诊断者转变为护理协调者
    \item \textbf{支持质量改进}:与Target AS和DETECT AS等倡议协同
    \item \textbf{全球共识}:20个学会背书确保国际影响力
\end{enumerate}

% ============================================
% 主动监测目标
% ============================================
\subsection{主动监测的目标和实施策略}

\subsubsection{主动监测的核心目标}

\begin{quote}
\textbf{"促进规范化监测以及对显著主动脉瓣狭窄的无偏见和及时的评估与管理。"}
\end{quote}

\subsubsection{主动监测的关键组成部分}

\textbf{1. EMR集成的监测超声提示}
\begin{itemize}
    \item 自动化提醒系统
    \item 根据指南推荐的随访时间表触发
    \item 确保患者不会因随访缺失而延误诊断
\end{itemize}

\textbf{2. 便利化的心脏瓣膜团队转诊(带有限时"退出"选项)}
\begin{itemize}
    \item 默认转诊机制
    \item 提供者可在限定时间内选择退出
    \item 需要记录退出原因
\end{itemize}

\textbf{3. 避免随访丢失}
\begin{itemize}
    \item 建立患者追踪系统
    \item 确保连续性护理
    \item 监测未就诊患者
\end{itemize}

\textbf{4. 明确记录不转诊的原因}
\begin{itemize}
    \item 提高决策透明度
    \item 识别系统性障碍
    \item 促进质量改进
\end{itemize}

\subsubsection{EMR提示示例}

\textbf{UCSF实施的自动化提示}:

\begin{quote}
\textit{"患者符合严重主动脉瓣狭窄标准,射血分数≤49\%,在过去90天内没有转诊至UCSF瓣膜门诊或就诊记录。请考虑在下方转诊。"}
\end{quote}

\textbf{提示特点}:
\begin{itemize}
    \item 明确患者符合的临床标准
    \item 检查现有转诊状态
    \item 直接链接转诊选项
    \item 便捷的操作流程
\end{itemize}

% ============================================
% 结论
% ============================================
\subsection{结论}

\subsubsection{主要结论}

\textbf{1. AS治疗不足的证据确凿}
\begin{itemize}
    \item 即使是Class I指征患者,30-47\%仍未接受治疗
    \item 低梯度AS患者治疗率<40\%
    \item 女性、老年人、少数族裔存在显著差异
\end{itemize}

\textbf{2. 电子提供者通知(EPN)的有效性得到证实}
\begin{itemize}
    \item DETECT AS试验证明EPN可提高AVR率(HR 1.40, p=0.02)
    \item 90天AVR率从27.6\%提升至36.9\%
    \item 1年AVR率从47.0\%提升至60.1\%
    \item 延长生存时间(23天,p=0.01)
    \item \textbf{显著减少性别差异}(女性获益最大)
\end{itemize}

\textbf{3. 系统性干预的可行性}
\begin{itemize}
    \item Target AS项目已在75家医院实施
    \item 建立了可测量的质量标准(75\%及时治疗,50\%无缺陷诊断)
    \item 2025 ASE指南支持主动报告和推荐
    \item EMR集成工具可扩展应用
\end{itemize}

\textbf{4. 多层次质量改进框架已建立}
\begin{itemize}
    \item \textbf{指南层面}:2025 ASE指南要求主动沟通和推荐
    \item \textbf{质量标准层面}:ACC/AHA性能测量标准纳入90天治疗率
    \item \textbf{系统层面}:Target AS提供注册平台和认可标准
    \item \textbf{实施层面}:DETECT AS证明EPN的有效性
    \item \textbf{技术层面}:EMR集成和AI辅助工具
\end{itemize}

\subsubsection{演讲的核心信息}

\begin{quote}
\textbf{AI基于的警报、决策支持和管理工具在改善护理质量方面具有巨大潜力。}
\end{quote}

% ============================================
% 临床启示
% ============================================
\subsection{临床启示}

\subsubsection{对临床实践的建议}

\textbf{1. 采用主动监测和通知系统}

\begin{itemize}
    \item \textbf{实施电子提供者通知(EPN)}:
    \begin{itemize}
        \item 为中-重度AS患者自动生成通知
        \item 根据血流动力学亚型个性化通知内容
        \item 包含指南推荐和转诊建议
    \end{itemize}

    \item \textbf{建立EMR集成提示系统}:
    \begin{itemize}
        \item 自动识别符合Class I指征的患者
        \item 提示需要随访超声的患者
        \item 追踪未转诊至心脏瓣膜团队的患者
    \end{itemize}

    \item \textbf{超声报告标准化}:
    \begin{itemize}
        \item 遵循2025 ASE指南
        \item 在报告中明确包含推荐陈述
        \item 关键发现应在数分钟内口头告知
    \end{itemize}
\end{itemize}

\textbf{2. 参与质量改进项目}

\begin{itemize}
    \item \textbf{加入Target: Aortic Stenosis项目}:
    \begin{itemize}
        \item 网站:www.heart.org/TargetAS
        \item 邮箱:TargetAorticStenosis@heart.org
        \item 有限的参与补助名额可用
    \end{itemize}

    \item \textbf{实施质量测量}:
    \begin{itemize}
        \item 追踪90天AVR率(目标≥75\%)
        \item 监测诊断完整性(目标≥50\%无缺陷)
        \item 确保至少40例患者的注册容量
    \end{itemize}

    \item \textbf{建立多学科心脏团队(MDT)}:
    \begin{itemize}
        \item 包含介入心脏病医师、心外科医师、影像医师
        \item 规范化转诊流程
        \item 定期MDT会议讨论复杂病例
    \end{itemize}
\end{itemize}

\textbf{3. 关注易被忽视的患者群体}

\begin{itemize}
    \item \textbf{女性患者}:
    \begin{itemize}
        \item DETECT AS试验显示女性从EPN获益最大
        \item 常规护理下女性AVR率仅为男性的46\%
        \item 需要特别关注和主动干预
    \end{itemize}

    \item \textbf{低梯度AS患者}:
    \begin{itemize}
        \item 治疗率仅32-38\%,但AVR仍可显著降低死亡风险
        \item 需要更全面的评估(DSE、CT钙化评分等)
        \item 考虑MDT讨论以确定治疗适应证
    \end{itemize}

    \item \textbf{老年患者}:
    \begin{itemize}
        \item 不应仅因年龄而排除治疗
        \item TAVR为高龄患者提供了微创选择
        \item 综合评估虚弱度和生活质量
    \end{itemize}
\end{itemize}

\textbf{4. 优化患者教育和共享决策}

\begin{itemize}
    \item 使用Target AS提供的患者教育材料
    \item 解释未治疗严重AS的预后(根据亚型,2-3倍死亡风险)
    \item 讨论AVR的益处、风险和选择(SAVR vs TAVR)
    \item 尊重患者偏好,但确保充分知情
\end{itemize}

\subsubsection{对医疗系统的建议}

\textbf{1. 基础设施建设}

\begin{itemize}
    \item 投资EMR升级以支持自动化通知和提示
    \item 建立AS患者注册库
    \item 实施数据分析工具以监测质量指标
\end{itemize}

\textbf{2. 流程优化}

\begin{itemize}
    \item 简化从超声诊断到MDT评估的转诊流程
    \item 设立AS快速通道门诊
    \item 建立失访患者追踪机制
\end{itemize}

\textbf{3. 绩效考核}

\begin{itemize}
    \item 将90天AVR率纳入科室质量指标
    \item 追踪性别、年龄等亚组的差异
    \item 定期报告和反馈质量数据
\end{itemize}

\subsubsection{对研究的启示}

\textbf{1. 需要进一步研究的问题}

\begin{itemize}
    \item \textbf{为什么女性从EPN获益更大?}
    \begin{itemize}
        \item 是否存在特定的转诊障碍?
        \item 提供者对女性AS患者的认知偏见?
        \item 女性患者的医疗寻求行为差异?
    \end{itemize}

    \item \textbf{如何优化低梯度AS的管理?}
    \begin{itemize}
        \item 哪些检查最有助于确定真性严重AS?
        \item 低梯度AS的最佳治疗时机?
        \item 如何提高低梯度AS的识别率?
    \end{itemize}

    \item \textbf{AI工具的开发和验证}
    \begin{itemize}
        \item AI辅助超声诊断的准确性
        \item 预测模型识别高危未治疗患者
        \item 算法公平性和偏见消除
    \end{itemize}
\end{itemize}

\textbf{2. 推荐的研究方向}

\begin{itemize}
    \item 扩展DETECT AS研究至其他医疗系统
    \item 评估长期随访结果(2-5年)
    \item 成本效益分析
    \item 患者报告结局测量(PROs)
    \item 不同医疗环境下的实施研究
\end{itemize}

% ============================================
% 研究局限性
% ============================================
\subsection{研究局限性}

\textbf{DETECT AS试验的局限性}:

\begin{enumerate}
    \item \textbf{单中心研究}:
    \begin{itemize}
        \item 在MGH多中心学术医疗系统内进行
        \item 可能限制其他医疗环境的推广性
        \item 学术中心的资源和基础设施可能优于社区医院
    \end{itemize}

    \item \textbf{单盲设计}:
    \begin{itemize}
        \item 患者和提供者知晓干预
        \item 可能存在霍桑效应
        \item 但整群随机化降低了偏倚风险
    \end{itemize}

    \item \textbf{纳入标准基于AVA≤1.0 cm²}:
    \begin{itemize}
        \item 可能包含部分中度AS患者
        \item 未针对症状状态进行筛选
        \item 依赖于超声测量的准确性
    \end{itemize}

    \item \textbf{随访时间相对较短}:
    \begin{itemize}
        \item 主要终点为1年
        \item 长期生存获益尚不明确
        \item 需要更长期的随访数据
    \end{itemize}

    \item \textbf{未评估成本效益}:
    \begin{itemize}
        \item 实施EPN系统的成本未报告
        \item 增加的AVR带来的医疗费用增加
        \item 需要卫生经济学评估
    \end{itemize}
\end{enumerate}

\textbf{Target AS项目的局限性}:

\begin{enumerate}
    \item \textbf{自愿参与}:
    \begin{itemize}
        \item 参与医院可能已有较高的AS管理意识
        \item 选择偏倚可能高估实际效果
    \end{itemize}

    \item \textbf{容量要求}:
    \begin{itemize}
        \item 需要至少40例患者才能获得认可
        \item 可能排除小型医院或低容量中心
        \item 限制了广泛推广
    \end{itemize}

    \item \textbf{质量标准的挑战}:
    \begin{itemize}
        \item 75\%的90天AVR率目标可能较高
        \item 未充分考虑患者拒绝治疗的情况
        \item 可能存在文档负担
    \end{itemize}
\end{enumerate}

\textbf{总体局限性}:

\begin{enumerate}
    \item 主要基于美国医疗系统,其他国家的适用性未知
    \item 技术依赖性高(需要成熟的EMR系统)
    \item 资源需求可能限制低收入地区的实施
    \item 患者自主权和过度治疗的平衡
\end{enumerate}

% ============================================
% 个人笔记
% ============================================
\subsection{个人笔记}

\subsubsection{关键数字记忆}

\textbf{治疗不足的程度}:
\begin{itemize}
    \item HG-NEF(Class I)未治疗率:30\%
    \item HG-LEF(Class I)未治疗率:47\%
    \item LG-NEF(Class II)未治疗率:68\%
    \item LG-LEF(Class II)未治疗率:62\%
\end{itemize}

\textbf{DETECT AS试验关键数据}:
\begin{itemize}
    \item 样本量:945名患者,285名提供者
    \item 90天AVR率:EPN 36.9\% vs 常规护理 27.6\%(差异9.3个百分点)
    \item 1年AVR率:EPN 60.1\% vs 常规护理 47.0\%(差异13.1个百分点)
    \item HR 1.40 (95\%CI 1.06-1.85), p=0.02
    \item 生存获益:23天(p=0.01)
    \item 女性获益:HR 2.10 (1.56-2.82), p<0.001
    \item 男性获益:HR 1.16 (0.89-1.51), NS
\end{itemize}

\textbf{Target AS项目数据}:
\begin{itemize}
    \item 75家签约医院
    \item 12,386名患者记录
    \item 47,704+次就诊
    \item 2026年认可标准:75\%及时治疗,50\%无缺陷诊断
    \item 容量要求:40例患者
\end{itemize}

\textbf{死亡风险降低(AVR vs 未治疗)}:
\begin{itemize}
    \item HG-NEF:2.4倍
    \item HG-LEF:3.6倍
    \item LG-NEF:1.4倍
    \item LG-LEF:2.1倍
\end{itemize}

\subsubsection{重要概念}

\begin{description}
    \item[电子提供者通知(EPN)] 一种基于EMR的自动化通知系统,当超声检测到严重AS时,向临床提供者发送个性化的指南推荐和转诊建议。DETECT AS试验证明其可提高AVR率40\%并减少性别差异。

    \item[Target: Aortic Stenosis] AHA发起的质量改进倡议,旨在改善AS患者在AVR上游的护理。覆盖从认知、检测、诊断、转诊、治疗到监测的全流程,建立了可测量的质量标准。

    \item[90天AVR率] ACC/AHA 2024年性能测量标准的核心指标,要求症状性严重AS患者在诊断后90天内接受AVR。该指标"已准备好用于公开报告和按绩效付费项目"。

    \item[血流动力学亚型] 根据平均梯度(mAVG)和射血分数(LVEF)将AS分为4类:高梯度-正常LVEF、高梯度-低LVEF、低梯度-正常LVEF、低梯度-低LVEF。不同亚型的治疗率和预后差异显著。

    \item[2025 ASE指南范式转变] 超声心动图从"被动的、描述性的报告"转变为"主动的、医生指导的患者管理参与"。要求关键发现在数分钟内口头告知,并包含转诊推荐陈述。

    \item[主动监测(Active Surveillance)] 通过EMR集成的自动化系统,主动提示随访超声、促进心脏瓣膜团队转诊、避免随访丢失、记录不转诊原因,以实现规范化和无偏见的AS管理。

    \item[性别差异] DETECT AS试验揭示,在常规护理下,女性接受AVR的可能性仅为男性的46\%。EPN可完全消除这一差异,女性从EPN获得的绝对获益(20.2个百分点)远大于男性(1.5个百分点)。

    \item[低梯度AS治疗不足] 尽管低梯度AS患者从AVR中仍可获得显著生存获益(死亡风险降低1.4-2.1倍),但治疗率仅32-38\%,远低于高梯度AS。需要更积极的识别和评估策略。
\end{description}

\subsubsection{值得思考的问题}

\textbf{1. 为什么女性从EPN获益显著大于男性?}

可能的解释:
\begin{itemize}
    \item \textbf{提供者偏见}:医生可能低估女性AS的严重性或治疗需求
    \item \textbf{症状归因偏差}:女性的呼吸困难等症状可能被误归因于其他疾病
    \item \textbf{转诊障碍}:女性可能在传统护理流程中面临更多隐性障碍
    \item \textbf{系统性提示的无偏见性}:EPN提供了标准化、无偏见的推荐,消除了人为决策中的性别偏见
\end{itemize}

临床启示:
\begin{itemize}
    \item 系统性、自动化的干预可能是消除健康不平等的有效工具
    \item 需要特别关注女性AS患者的识别和转诊
    \item 培训提高医生对性别偏见的认识
\end{itemize}

\textbf{2. 低梯度AS为何治疗率如此低?如何改善?}

障碍因素:
\begin{itemize}
    \item \textbf{诊断不确定性}:需要DSE、CT钙化评分等额外检查
    \item \textbf{指南推荐等级较低}:Class IIa vs Class I
    \item \textbf{临床医生认知不足}:对低梯度AS的预后认识不够
    \item \textbf{假性严重AS的担忧}:担心过度治疗
    \item \textbf{症状不典型}:低梯度常伴LVEF下降,症状可能归因于心衰
\end{itemize}

改善策略:
\begin{itemize}
    \item EPN中针对低梯度AS提供更详细的评估建议
    \item 强调低梯度AS的AVR获益数据(死亡风险降低1.4-2.1倍)
    \item MDT讨论复杂低梯度病例
    \item 推广多模态评估(超声+CT+生物标志物)
\end{itemize}

\textbf{3. 90天AVR率作为质量指标是否合适?}

支持论据:
\begin{itemize}
    \item 症状性严重AS预后极差,及时治疗至关重要
    \item 90天给予充分时间进行评估和患者决策
    \item DETECT AS试验显示EPN可达到36.9\%的90天AVR率
    \item 有明确的文献支持及时治疗的生存获益
\end{itemize}

潜在问题:
\begin{itemize}
    \item 是否充分考虑患者自主选择?
    \item 某些患者可能需要更多时间考虑
    \item 可能存在地区差异(农村vs城市)
    \item 老年、虚弱患者的评估可能需要更长时间
\end{itemize}

平衡点:
\begin{itemize}
    \item 90天可能是合理的目标,但需要有例外情况的记录
    \item 应区分"系统延误"和"患者选择"
    \item 可考虑分层目标(不同患者群体不同标准)
\end{itemize}

\textbf{4. AI在AS管理中的角色:机遇与风险}

机遇:
\begin{itemize}
    \item \textbf{早期识别}:AI分析超声图像自动检测AS
    \item \textbf{风险分层}:预测哪些患者最可能从早期干预中获益
    \item \textbf{消除偏见}:标准化的算法决策减少人为偏见
    \item \textbf{提高效率}:自动化工作流程,减轻医生负担
    \item \textbf{个性化推荐}:基于患者特征的精准建议
\end{itemize}

风险:
\begin{itemize}
    \item \textbf{算法偏见}:如果训练数据存在偏见,可能固化或加剧不平等
    \item \textbf{过度依赖}:医生批判性思维的削弱
    \item \textbf{隐私和安全}:患者数据的保护
    \item \textbf{责任归属}:AI错误时的法律责任
    \item \textbf{可解释性}:"黑箱"决策的透明度问题
\end{itemize}

建议:
\begin{itemize}
    \item 确保AI训练数据的多样性和代表性
    \item 定期审计算法的公平性
    \item AI应作为辅助工具,最终决策仍由医生做出
    \item 建立AI应用的伦理框架和监管标准
\end{itemize}

\textbf{5. 如何在中国医疗环境下应用这些策略?}

中国特有挑战:
\begin{itemize}
    \item \textbf{城乡差距}:可能比美国更显著
    \item \textbf{医疗资源分布不均}:TAVR中心主要集中在大城市
    \item \textbf{EMR标准化程度}:各医院系统不统一
    \item \textbf{医保覆盖}:TAVR费用和报销政策差异
    \item \textbf{文化因素}:老年患者对手术的接受度
\end{itemize}

可借鉴的经验:
\begin{itemize}
    \item \textbf{分级诊疗体系}:基层医院识别→三级医院诊断→区域TAVR中心治疗
    \item \textbf{区域AS注册库}:建立省级或市级AS患者数据库
    \item \textbf{远程医疗}:超声远程会诊,减少患者转诊负担
    \item \textbf{医联体内EPN}:在医联体内实施电子通知系统
    \item \textbf{质量控制}:参考Target AS建立中国版质量标准
\end{itemize}

优先行动:
\begin{itemize}
    \item 在大型医疗中心试点EPN系统
    \item 培训超声医师按2025 ASE指南报告
    \item 建立区域MDT协作网络
    \item 开展中国AS治疗现状的流行病学研究
\end{itemize}

\subsubsection{关键信息汇总}

\textbf{核心发现}:
\begin{enumerate}
    \item AS治疗不足是普遍且严重的问题(总体治疗率<50\%)
    \item 电子提供者通知(EPN)可有效提高AVR率、延长生存、减少性别差异
    \item 女性、低梯度AS患者是最被忽视的群体
    \item 系统性、自动化的干预优于依赖个人意识的传统模式
    \item 多层次质量改进框架已建立(指南-标准-项目-技术)
\end{enumerate}

\textbf{可操作的下一步}:
\begin{enumerate}
    \item 在本机构实施EMR集成的AS提示系统
    \item 参考2025 ASE指南标准化超声报告
    \item 考虑加入Target: Aortic Stenosis项目
    \item 建立或优化MDT流程
    \item 特别关注女性和低梯度AS患者
\end{enumerate}

\textbf{对未来的展望}:
\begin{itemize}
    \item AI辅助诊断和决策支持工具的进一步发展
    \item 更大规模、多中心的EPN实施研究
    \item 成本效益评估和医保政策调整
    \item 患者参与和共享决策的优化
    \item 全球范围内的AS管理质量改进
\end{itemize}
